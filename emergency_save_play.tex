\chapter{The Emergency Save Play}

% Updated: 20190529

This will be, I expect, the most controversial chapter in the
book. It's a wild play that can cost you a lot of chips, but at least
keeps you in the tournament. If the alternative is knocking you out of
the tournament completely, then this expensive, stack destroying play
is, in fact, the right way to play the hand.

A lot of players lose in tournaments by check-calling a huge all-in
bet and losing with a powerful hand against a monster. The monster is
either a rung higher than their hand on the poker ladder or sometimes
a higher hand on the same rung. Two pair loses to a set or a straight,
a straight loses to a full house, a flush loses to a higher flush or a
full house and so on. In the most extreme case a full house loses to a
higher full house.

Here's how I lost in a big tourney I played in May 2019. This
tournament was a huge Main Event, \$175 buyin with four Day Ones. Over
1230 people entered, across all those Day 1s, and I made Day Two and
had made the money and was in the final 106 players when this hand
occurred.

I'm first to act with Ace-Ten. The flop comes 987 and I check-call a
decent bet of 50k. We each have about 180k behind, blinds are 5k/10k.
The turn is my hoped-for Jack and I'm nut unless Last-to-Act has
played Queen-Ten very, very inventively. I check hoping for a
check-raise, but Last-to-Act checks behind. The river comes an 8 and
now the board is paired at 987J8 I check and I face an all-in bet from
Last-to-Act. I call and I'm shown 87 and Last-to-Act, agonisingly,
just has me covered.

\section{A Tough Spot}

Some people think that in tough spots like this they can start talking
to the all-in bettor and get a soul read on him. They've watched too
much of Daniel Negreanu or pro players making big correct laydowns on
YouTube.

Know something? The pros get it wrong a lot of the time too. Not as
much as we pub players too, but they make wrong calls and wrong folds
just like the amateurs.

I'm going to suggest a different approach on the river, which I call
The Emergency Save Play. Bet 60 to 70\% of your own stack, and fold if
you're raised. Remember from last chapter that nobody bluffs a made
hand? Here you're putting in over half your stack, looking fully
pot-committed, but in your own mind you know that the only hands that
beat you, raise you.

Sometimes you'll make this same big river bet when you're Nut, and it
looks like you're value betting your monster for as much as you think
you can sell it; giving a bit of a discount to Last-to-Act who at
least doesn't have to double all of your remaining chips on the
river. So the bet size you make here isn't always the Emergency Save
Play.

Let's look at the actions of Last-to-Act facing an Emergency Save Play
bet from me on the river; given a variety of his hands. The pot is
170k and I have 164k and he has 170k. The final board is 987J8 and
I've bet 100k on the river.

\begin{tabular}{|l|l|l|l|} \hline
HOLDING & HAND & ACTION & RESULT\\ \hline
AJ      & JJ88A & fold  & saves bluff chips\\ \hline
65      & low straight  & fold or call & lose or lose huge\\ \hline
Tx      & good straight & call & split pot\\ \hline
QT      & best straight & call & win huge\\ \hline
8-nonfull & trips       & fold or call & lose or lose huge\\ \hline
J9      & top two       & fold & lose\\ \hline
AA,KK,QQ & topper two   & fold or call  & lose or lose huge\\ \hline
87,77 & low full house  & call or raise & win huge\\ \hline
8J,89 & full house      & raise & win huge\\ \hline
99,JJ & mega full house & raise & win huge\\ \hline
88      & quads         & raise & win huge\\ \hline
other   & playing board & fold  & lose\\ \hline
\end{tabular}

Look at this table really carefully. Do you agree with the ACTION
column? I sure do. By betting this big at him, on the wettest of
boards, I've taken away his bluff-raising and bluffing power. The only
hands he can raise with are the hands that beat me. He might sometimes
just call with hands that beat me (QT overstraight, 78 and 77 low full
houses) but the only hands he can raise with are the hands that beat
me.

This is poker's most expensive Information Bet. At pub level, nobody
will be bluff-raising you here, because you've put in so much of your
stack already that you look fully pot-committed.

While this is mostly a First-to-Act play, it sometimes crops up when
you're in position. You've been betting strongly the whole hand and
suddenly on the river you get check-raised. There's a great hand on
YouTube with Phil Laak against Luke Schwartz worth looking at here.
For the rest of this discussion, I'm First-to-Act.

As I said at the top of this chapter, this is a wild play. Only use it
for your hands that scream ``I'm so strong I will have to call an
all-in even though the board looks so scary.'' Don't routinely give
half your stack away just because your hand beats two-pair. But try to
have this play in your arsenal.

You might be shaking your head so hard your neck hurts right now,
asking ``why didn't you just shove the Turn?''. Great question. I
angled for the check-raise, it didn't come off, and I paid the
ultimate price when a scungy two-pair got a four-out full house on the
river. A turn shove would have folded bottom two-pair, and been called
only by sets or top two-pair, and won me a handy enough pot. I do wish
I'd shoved the turn. Or at least put in a bet big enough that two-pair
hands weren't getting the right odds to beat me, and even sets would
know they were behind.

Remember last chapter the speech I got from the guy that helped me
correctly fold 33 to the board 993T6? If I'd tried the Emergency Save
Play in this hand I might've got the same speech. ``Do you really have
the Ten, Stuart? Turned the straight on me?? I was saying to myself
``No Jack, no Six.. No Jack, no Six. What a horrible turn card. Oh
well.. if you've got it you've got it. I'm All-in.'' If I get that
much of a speech, I'll find a fold, and with the table I've come up
with I should find the fold anyway.

Let's listen to what the bets are saying on the river.
MY BIG BET: ``My, this is a wet board, and I've got a big piece of
it. I can beat two pair, I can beat trips. Aces no good here. I'm high
up on the poker ladder in this hand''.
HIS RAISE: ``Yes, I can see there's big hands possible on this board,
and that's great because I've got the biggest. So you'll have to give
me the last of your chips too. Nothing personal, but I'm in this
tournament to win and hands this huge don't come along that
often. Even if you've got the straight you're no good here. I can see
from your bet that your hand's too big to lay down. Bad luck that I
rivered you, that's poker. Let's have the rest of your chips,
thanks.''
MY FOLD: ``Wow. I mean, look, I've got a big made hand. You know
that. I've shoved in most of my chips, you can see I'm pot committed,
and you're still raising me? You've just got to have the full house. I
hate doing this, but I fold. See.. there's my Ten. I'm folding my
straight.''

The Emergency Save Play is used only in Emergencies. It's like the
glass you have to break in an office building when there's a fire. In
case of Emergency, break glass.

In my cash Pot-Limit Omaha writing, I've developed a first-to-act bet
I call The Crying Bet. This is a bet you make on the river when you
know, in your heart of hearts, that if you check and someone puts you
all in, you'll call. The Crying Bet takes away the check-behind option
for the later players. In Omaha you don't check to induce a bluff as
often as you do in Holdem, people who make bets normally have a
super-strong hand. Now sometimes you Crying Bet smack-bang into the
Nuts and you look a bit foolish. But remember you would've called
anyway. The Crying Bet helps you win bigger on your winners, by
stopping later players from checking behind on the river.

The Omaha Crying Bet takes away Check-behind, and the Holdem
Emergency Save Play takes away Bluff-raise-behind and Bluff-behind. It
purifies his betting, so that an all-in raise from him can only be
from a better hand.

You don't smash the glass unless there's a fire and you don't use the
Emergency Save Play except in tournament decision emergencies. Good
luck with this powerful play.
