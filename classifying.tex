\chapter{Classifying your opponents}

% Last updated: 20181030

\section{The two dimensions}

In poker you classify your opponents on two dimensions.
The first is called Loose/Tight and is a measure of the quality
of starting cards that a player will take a flop with. Someone
who folds nearly all their starting hands, and
only enters the pot with JJ,QQ,KK,AA,AK,AKs or AQs is ultratight.
Someone who regularly plays J3s and KTo is loose.

The second dimension is called Passive/Aggressive and indicates
whether players bet and raise their hands, or are more likely
to check and call. This dimension used to be called ``Weak/Strong''
but nowadays is called Passive/Aggressive.

Four combinations of these qualities are possible. \\
Loose/Passive, Loose/Aggressive, Tight/Passive and Tight/Aggressive. \\
There are nicknames for each style of play: \\
Loose/Passive: ``Calling Station'' \\
Loose/Aggressive: ``Maniac'' \\
Tight/Passive: ``Rock'' \\
Tight/Aggressive: ``Expert''

% FIXME: Cite the Schoonmaker book properly

An entire book has been written on just this two-dimensions view
of player characterisation. It's called \textbf{The Psychology of Poker}
by Al Schoonmaker and it's excellent, though it concentrates
on Limit poker betting structure and has a lot of Seven Card Stud
content. Phil Hellmuth's \textbf{Play Poker like the Pros} has a
similar system, again focusing on Limit betting, though Holdem only.
Phil uses animals to characterise the player types, his animals are
as follows: \\
Loose/Passive: ``The Mouse'' \\
Loose/Aggressive: ``The Jackal'' \\
Tight/Passive: ``The Elephant'' \\
Tight/Aggressive: ``The Lion''

Phil can't help but add a fifth animal: ``The Eagle'', one of the
top 100 players in the world who occasionally comes down from
the sky and swoops on all your chips. You are unlikely to play
any of the top 100 players in the world in a Fast Pub game, they
have their own big games to play in. Although Jeff Lisandro
(tournament winning of around four million dollars, and
probably our biggest ever live cash player) played in
a free APL game in Sydney once, so anything's possible! I doubt
Jeff brought his best game to the APL! Many
of the bigger Sydney players will play in \$100 to \$500 buyin
games when they're offered, so keep an eye out for the experts
when you play in these games.

You can see from the nicknames and the animals that the style
of play admired is Tight/Aggressive. Indeed,
it's the style of play you want to be playing in all forms of
poker, and the style you must play to bring your game to higher
levels.\footnote{In recent times, the Loose Aggressive style
(``LAG'') is coming to the fore. When you feel more expert
at your game you can try broadening the hands you play, which
will make you harder to read, but start with fundamentally
tight aggressive play}

Hellmuth advises new players to start tight, only playing the
top 10 or 15 starting hands. He offers this excellent
advice to his readers: ``After all, how much fun can it be to
play so many pots and lose most of them''?

In Pub Poker there some other factors you might want to assess
players on, such as their board reading skills and their general
awareness of the game. ``Card sense'' I'll call it.  I'll cover more
dimensions at the end of this chapter. For now we'll concentrate on these
two dimensions and how to use them.

Online players are looking for the same information (and they
can't see the other player so there's no point looking for Mike Caro
tells). Automated stats programs tell online players how many pots an
opponent enters and whether he's limping into them or raising. This
gives a player a feel for whether an unseen player is Tight or Loose
(is he entering a high percentage of pots preflop?) and whether he's
Passive or Aggressive (is he raising the pots which he's entering?).

% FIXME: mention VPIP in the Online World

\section{Four player types}

Let's look at how each of the four combinations play their poker.

\subsection{Loose/Passive: The Calling Station}

These players are the easiest to play against and win chips from.
You get a good hand and you put in value bet after value bet and
get paid off--- often by having top pair with the better kicker or
just top pair in the first place. You have KQ, the flop is Q84, and
you get called, called, called by Q5. The player's in the pot in the
first place with Q5 because he's loose, and letting you keep the
betting lead by just calling, calling, calling. At no stage in
the hand are you put under pressure with a raise.

Many complete beginners play poker this way. If someone makes
a few beginner's errors (such as never knowing when the action
is on him) and is staying in lots of pots, he's a calling station.


% FIXME: move this next stuff to the stereotypes on women section?

Some women are calling stations. They
need a huge hand to raise preflop. I've seen
(and paid off) many ladies who limp in late position holding
Ace-King suited. If they won't raise a pot with that, a preflop
raise from them pretty much means QQ, KK or AA. Be very wary
of a preflop raise from a genuinely tight player. She isn't
loosening her play, she's woken up with a very powerful starting
hand. Fold your QT as soon as etiquette will allow.

Some of the more aggressive guys jump into a contest with a
preflop raising woman, and pay off her very strong hand. I've
done this myself once or twice, but grown out of the habit. This
isn't smart play at all. Try and be ego-less, and use your
superior player classification skills to get out of the way
of a preflop raise from a very passive player. Even the most
passive player in the tournament is dealt pocket Aces now and
again.

Very friendly, social people are often calling stations.
They feel as though raising is rude, and a good hand deserves
to win a few more chips, so folding seems rude also. Since
they're winning less with their monsters (by not raising)
and losing more with their losers (by calling) they are
the epitome of Win Small/Lose Big players. Nice Guys
Finish Last in poker, because they call too much, fold
too little and raise too little.

\subsubsection{Playing against the Calling Station}

Calling stations can drive you mad when they river two pairs,
or hit some raggedy low flush with a garbage holding like 94 suited.
Because they're passive, they don't themselves bet or raise unless
they've got a pretty strong hand. Once you've picked them as
Calling Stations, you can beware of their raises, and get out of
their way unless you're very strong yourself.

Don't try and bluff calling stations. Go for the value bet,
and even the overbet. Charge them too much for their loose draws.

Even when a garbage holding does luck a good hand, it's often
not a strong enough hand to bet. 94 suited makes only a Nine-high
flush and loses to a higher flush. If 94 makes two pairs,
it's never top two pairs. On a board like T964Q the two pair 9944
loses to any other two pair holding except for 64. So the garbage
holding ends up being ``good enough to call with'', which is a
recipe for small pot wins when ahead and large pot losses when
behind.

\subsection{Tight/Passive: The Rock}

After busting out in early rounds of their first half dozen
tournaments, people soon figure out that it's best to take flops
with good starting cards. They still don't bet or raise without
great cards, but they're now in the pot with better starting hands.

True rocks will only enter 10\% of pots. Apart from when they're
in the big blind in a limped pot, they're only taking one
flop per round, unless they hit a run of unusually good starting
cards. If the other players are bad and haven't noticed
he's a rock, he'll still get paid off on his ultra-rare bets.

%% Some of the ``make money playing poker online'' schemes use this approach.
%% Play ultratight, and get paid off when the good cards come. In ``Rounders'',
%% Matt Damon's voiceover suggests early on that the target for a grinding
%% Limit player is two big bets per hour.

Rocks enter so few pots that they're hardly noticed
at the table. Many of them are quiet anyhow,
a persona which helps them look innocuous and get paid off by the
more aggressive players. You might hear comment from the table
when a rock finally enters a pot or (still rarer) raises. Recognize
the raise from a rock as what it is: an indication of a very very
strong hand.

Older players are likely to be rocks.

Rocks have the ``Lose small'' part of gambling down pat. It can
be an effective (if dull) style of playing cash poker. It's not a
good style for Fast Tournament Poker because they don't win enough
on their few good hands and the fast rising blinds put them
under pressure.

\subsubsection{Playing against the Rock}

Rocks can frustrate you because they stay at the table for a long time
and you can't get rid of them quickly because they play so few hands.
It takes some time for the blinds to eat them up (although it's
pretty quick in fast tournaments). Rocks avoid big confrontations
and tough decisions by simply being out of the hand preflop so often.

If maniacs try to break them quickly,
it's like they're smashing their own heads against a rock.

You want to blind steal a lot on the Rock's Small Blind and Big Blind.
If the Rock calls a decent flop bet or even worse raises, just get
out of his way.

\subsection{Loose/Aggressive: The Maniac}

These are the most noticeable players at the table. They're constantly
drawing attention to themselves with their bets, raises, table chatter
and behaviour. If someone gets under your skin at the table, chances
are excellent that he's a Maniac.

Drunk players are invariably maniacs. They lack the patience to be
tight preflop and they also convince themselves midway through a
hand that winning \textbf{this} pot is the only thing that counts, so
they'll make a big bluff at it.

If maniacs get good cards, look out! They can go on rushes like
nobody else. Their aggression can steal them pots, and when
they have the goods and someone takes a stand against them, their
chip stacks rocket up even faster.

Following my advice in this book, you'll be projecting a Loose/Aggressive
image. This is great both for stealing pots and for getting paid off.
I've had tournaments where for five hands in a row I've either folded
the table to my bluff, or had two stacks bet all in to my nut full house.
The strategy in this book is to play Tight/Aggressive poker, but
have a Loose/Aggressive image.

The best movie example of a player pretending to be drunk but playing
smart is Paul Newman in ``The Sting''. In one of the best poker scenes
before ``Rounders'', he sets up Robert Shaw's Chicago mobster,
Doyle Lonnegan, in a game of five card draw played on a train.

Maniacs are good at the ``Win Big'' aspect of gambling. They splash
their chips around that everyone at the table realises that they
can't always have a monster hand, and their big hands often get
paid off and win big. Unless they've got great poker judgement, they're
also good at losing big.

\subsubsection{Playing against the Maniac}

The first thing to realise is that you'll have to show down a winning
hand against the maniac. They're not easily bluffed. You want to set up a
trap against them, have a very strong hand yourself and let the maniac
bet into you the whole way. Don't raise or check raise until
the turn, by which time the maniac will probably be pot committed.

You don't win regular small pots against Maniacs, you look for a single big
pot, even an all-in, that will put you and the Maniac up against each
other with you having a big advantage.

When you've got a good starting hand it's a good idea to Isolate
re-raise the
Maniac preflop so that it's just the two of you in the hand. You don't
want a third player sticking around and catching a big flop when your
aim for the hand is to get all the Maniac's chips.
You want to play this hand Heads Up against the Maniac,
giving you an exclusive shot at winning his chips.

\subsection{Tight/Aggressive: The Expert}

Tight/Aggressive is the style of poker the pros all move to. They're
in the pot with quality cards, and their bets and raises put the
other player to the tough decision, and give them information on
the other player's hand, helping define their own hand. They're great
at sensing weakness and being the player who bets all-in and gets
the weaker player to fold.

Expert players are also great blind stealers and card readers.
They have complete understanding on
the concepts of position, chips and bet sizing. They're working on
their game continuously, taking notes, staying alert and playing
their best game of poker.

To do well at pub poker, and to take your game to the next level,
you have to be aggressive in poker. Passive just won't get it done.

By folding more rather than calling, the Tight/Aggressive player
has the ``lose small'' quality of the Rock. And by raising more than
calling, he has the ``win big'' quality of the Maniac. Lose small, Win big.
It's what poker is all about.

\subsubsection{Playing against the Expert}

The best way to play against experts
is to treat them with respect. There's plenty of other chips
from worse players to be mopped up before you have to run into
confrontations with the experts. Of course once you're at the final
table you'll have to confront the expert, and if you hold a nut hand
early on you'll extract chips from the expert. Outside of those
situations, stay out of the experts way and take the easier chips
on offer from the rocks, calling stations and maniacs.

One trick worth trying against the expert is an overbet. When you
hold a power hand on the turn or river, instead of betting half the pot
to the pot, try overbetting to one and a half times the pot. This can
throw the expert's hand reading off-kilter, he'd correctly fold to
the normal bet, but he might get overly suspicious of this strangely
sized bet and think that it's a big bluff and he'll pay you off.
The Expert's thought process runs something like this: ``Wouldn't he
be trying for a small surefire payoff if he had a strong hand?''.

You can also try ``honest tells'' against him: Strong means Strong and
Weak means Weak. Since there's not much trapping involved in fast
tournaments, honest betting and fast playing works best anyhow, but
if the expert is looking for Mike Caro tells and using
``Weak means Strong'' theory, you can try reversing that and maybe
the expert will hang himself on his own tell-searching.

\section{Players expect you to play like them}

``Players expect you to play like them''.\footnote{Except the Expert, who
has figured out your style of play, and has made the adjustments
needed to exploit it} This is powerful advice.
Once you've got that figured you
can go a long way to manipulating other players into doing what
you want them to.

A rock gets pushed off hands because he expects other players to
have strong hands also. He raises preflop with pocket Jacks,
you call him from later position with 93. Flop is AT6. He checks, you
bet half the pot. He says ``I know you've got the Ace'' and he
folds the Jacks face up. You fold face down, say ``Good read''
and try not to smile as you pull in the pot.

If the rock check-raises you, Hollywood it for three seconds then fold.
You're facing a strong move from an extremely passive player. It
doesn't get much stronger than this.

Maniacs expect other players to be maniacs. So they will call the
very rare bets from rocks and calling stations because ``he could
be bluffing here''.

Because a maniac is always suspicious of bluffs, he calls way too much.
This behaviour is at the heart of the poker saying ``You can't bluff
a bluffer''.

Some ``Experts'' figure out the Maths of drawing hands and conclude
that someone who called against the odds couldn't have made a longshot
draw such as a straight. These Experts are often overbetting top pair
good kicker and get upset when somebody draws out on them.

Calling stations call the bluffs of maniacs because they're calling
stations. A terrible mistake maniacs
make again and again is to bluff and bluff at calling
stations. Remember the way to play them is to value bet and overbet
your made hands. Throwing big bluffs at a calling station is a
fast track to disaster.

\section{Some other dimensions}

\subsection{Card sense}

Does a player use poker terminology correctly? I hear some hilarious
mistakes at the tables: someone called a C bet a ``continuous bet''
and someone else told me once they had 20 outs with a low straight
flush draw. I love players who think they're giants of the game
and actually aren't. They're always good for a chuckle.

Extreme beginners will have no card sense at all, which will be
obvious in the first few hands when they don't know when it's
their turn to act, don't know about posting the blinds and show
similar inexperience with live poker. Some will call on a flop
of AQ4 rainbow with 87, holding no pair and no draw. These people
are welcome at any poker game in the world.

\subsection{Risk}

Is a player cautious or wild? Wild players take big risks and
enjoy the gamble in poker. They're happy to gamble all their chips
in one hand where they've got a decent draw. Some players happily
go against the odds or make crazy preflop plays with little upside,
such as shoving all in when deepstacked holding their favourite
hand, pocket fours.

Risky players will be winning and losing abnormally large pots
early in tournaments. Cautious players will lose steadily and be
more patient until their stacks are small relative to the blinds.
Which isn't a bad way to begin a tournament.
