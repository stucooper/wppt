\chapter{Basics}

In case you're completely new to poker, and to put my mind at ease
that this book covers Texas Holdem Tournaments as completely as
possible, this chapter covers some absolute basics of the game.

\section{The Deck of Cards}

Poker is played with a standard 52 card deck of cards. There are no
jokers used. A 53rd full-coloured card, the cut card, is used to cut
the cards onto and is the base card of the deck, to stop anyone
seeing the bottom card of the deck and also to prevent cheats from
dealing the bottom cards of the deck.

Each card has a rank and a suit. There are thirteen ranks. From
highest to lowest they are Ace, King, Queen, Jack, Ten, Nine, Eight,
Seven, Six, Five, Four, Three, Two. The lowest two ranks are also
called Trey and Deuce. Ace can be used as a One to make
the lowest straight 5432A, and also as the highest card of the highest
straight AKQJT. Besides its role in the 5432A straight the Ace is
always high.

There are four suits: Clubs, Diamonds, Hearts and Spades. The suits
are used to make flushes (five cards of the same suit) and for no
other reason. There is no ordering between the suits and because
Holdem uses five shared cards and two unique hole cards, a flush is
only possible in one suit. You can't get a heart flush versus a
Diamond flush in the same hand.

Every card is different. There are thirteen clubs (Ace of clubs, King
of clubs, Queen of Clubs down to the Two of Clubs), thirteen hearts
(Ace of hearts, King of hearts down to the Two of Hearts) and
thirteen Spades and Thirteen Diamonds, making Fifty-Two cards total.

\section{Looking at your cards}

Put your two hole cards together and peel up the corners and read the
rank and suit of each card. Cover lines of sight with your other hand,
the same way you cover your pin number with your other hand when you
use an Automatic Teller Machine.

Sometimes a plyer won't be looking at his cards properly and without
too much effort you can actually see his cards! If
you can actually see his hole cards, it's polite to give him a
warning but it's not required. This is the same protocol in cricket
when the non-striker batsman is backing up too far and the bowler is
considering running him out instead of bowling the ball; the
``Mankad'' dismissal.

\section{Poker rankings}

Poker hands are on a ladder with nine rungs. The hand on
the highest rung wins, if two hands are on the same rung the hand with
the higher determining cards wins. The same-rung determination
sometimes goes against poker rarity: it's rarer to make a seven-high
flush than a King-high flush, but the King-high flush wins.

Because we're looking at Texas Holdem, I've added a column to the
poker rungs table called ``Board Needs''. This shows what needs to be
on the board for a hand on that poker rung to be possible. For a
straight flush to be possible the board must have 3+ (three or more)
same-suit cards that are close together. Close means three in a
continuum of 5: three of (A2345), (23456), (34567) through to
(TJQKA). For quads and Full Houses to be possible, the board must have
two or more cards of the same rank; which is a fancy way of saying
that the board must itself have a pair, trips or quads.

% FIXME: Find some LaTeX magic so this table isn't too far over on
% the right.

\begin{tabular}{|l|l|l|l|} \hline
Hand            & Example   & Board Needs    & Who wins \\ \hline
Straight Flush  & KQJT9 samesuit     & 3+ close suited &
Higher card \\ \hline
Quads           & 7777A     & 2+ of a rank   & Higher quads then
1 kicker \\ \hline
Full House      & JJJ55     & 2-3 of a rank  & Higher trips then
higher pair \\ \hline
Flush           & KT852 samesuit & 3+ suited & Highest unique flushcard
\\ \hline
Straight        & 87654     & 3+ close       & Highest card \\ \hline
Trips           & AAA75     & any            & Highest trips then 2
kickers \\ \hline
Two Pair        & JJ773     & any            & Highest pair then
second pair then 1 kicker \\ \hline
One Pair        & JJT75     & any            & Highest pair then 3
kickers \\ \hline
High Card       & AJ986     & no pairs       & Highest card then 4
kickers \\ \hline
\end{tabular}

A quick word on ``any'' for the Board Needs column. In
Holdem you can ``play the board'', use the five community cards as
your final poker hand. For a straight to be somebody's hand, not only
must the board have at least three close cards, but the board can't
itself be a flush or a full house or quads. For High Card to be
someone's hand, the board must have no straights, flushes or even a pair
on it. You have to play the highest rung hand you can.

\section{Kickers}

Kickers are used to find the winners for the low-rung poker hands of
trips, two pair, one pair and high card. Full houses, flushes and
straights use all five cards and don't have kickers. A two pair hand
has one kicker (the highest card not involved in the two pair), trips
has two kickers (the best two cards not in the trips), one pair has
three kickers (the best ones that don't make the pair) and high card
is effectively the five highest cards. You count down the kickers
until one is higher. If the kickers are the same, both hands win and
it's a split pot.

You have to use higher kickers from the board if there are some.
If I've got J8 against J6 and the board is AKJ93 my best hand is JJAK9
and the player holding J6 also has the best hand JJAK9. Both our hands
are a pair on the poker ladder, the kickers in this case are AK9 for
both of us so it's a split pot. If I've got JT against J6 on the board
of AKJ93 my final hand is JJAKT and I beat J6 whose best hand is
JJAK9. If I've got K5 against JT I win pair of Kings higher than pair
of Jacks and kickers don't come into it. If I've got AQ against 93 I
lose because the enemy makes two pair, nines and threes, so I'm
out-rungged.

You can think of kickers as ``tie-breaking'' cards that are used when
the main part of the poker hand is the same.

There are no kickers in Straights. AJ against KJ on a board of
T9872 sees both players making the JT987 straight and it's a split
pot. AJ against QJ on the T9872 board sees QJ make a higher
straight, QJT98 and win the pot. AJ against JT on T9872 sees another
split pot, the fact that JT can make a pair of tens doesn't count, he
has to play his highest rung poker hand which is the JT987 straight.

If two players have flushes, the five highest suited flush cards are
counted down high to low and the person with the highest unique flush
card wins.
% FIXME give an example of this.

\section{Cards Speak}

With all these rules to determine who has the better five-card poker
hand, it's easy to make a mistake. The safest thing to do, when you're
new to poker, is turn your hand over onto the table at showdown when
it's your turn. Once you do that, ``Cards Speak'' and you play your
best possible poker hand, even if you're a bit confused and you think
you're playing a lesser hand. One of the worst mistakes in poker is to
lose a pot when all you needed to do to win it, or split it, was to
turn your cards over at the showdown.

\section{Counterfeiting}

Counterfeiting is when your two pair hand loses most of its value when
the next board card pairs the wrong card. You've got J3 in the big
blind, you get to see a flop, and it comes J83. Great, you've got two
pair! Then the turn card comes another 8. On the turnboard of J838
your hand is now JJ883 and someone with JT has now got you outkicked
and has the hand JJ88T.

Counterfeiting also puts overpairs in front. Up against KK, I'm ahead
on J83 but once that 8 repeats the enemy has KK88J and his two pair
are much better than mine.

Counterfeiting is really, really annoying. You want the board pair to
bring you a full house, not to wipe out the power of your flopped
two-pair and put the enemy in front.

One of the suckiest ways to lose at a final table is to get all in
with a pocket pair against two higher cards and lose on kicker on a
double counterfeit. With 88 against AQ, the flop comes T99,
the turn is a 4 and the river comes a T. With two overpairs to your
eights, the enemy wins with TT99A while the best you can make is
TT998. There is no ``three pairs'' hand in Holdem poker, it's always
your best Five-card poker hand.
