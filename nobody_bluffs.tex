\chapter{Nobody bluffs a made hand}

% Updated: 20190527

One really tough situation on the river is when you have a
non-nut power hand that you fully expect to be the winner.
You value bet it, expecting either a fold or a crying call,
but much to your surprise you get raised.

One of the Old Poker Sayings is ``Nobody bluffs a made hand''.
You really need to stop and think if your powerful hand could
be behind to an even better one.

I've said a few times that the Dream situation in
Poker is Nut hand versus Great hand; here there's a chance
that you've got the Great hand and the raiser's got the Nut hand.
You could be on the Nightmare side of this matchup.

Take a fresh look at the board and see what the nut hands are.
Look carefully for sneaky straights. How about runner-runner flushes?
What are the hole cards that beat you? Could the enemy
have those cards? Did the river complete an unlikely draw, or is it
possible you were behind the whole way?

\section{A Surprise River Raise}

This hand is one of my favourites. It's the biggest
laydown I've made in a pub poker game.

I have 33 and the flop comes 993, two hearts, a fantastic
flop for me, a made full house. I check and the Button puts in a
suspiciously low bet and I call. It's Heads Up now.

The Turn is the Ten of hearts, making the flush. I check and, to my
surprise, the Button also checks.

The River is the 6 of spades and I bet 800 into the 1,500 pot. The Button
checked behind on the turn and there's no way I'm letting him
see a showdown for free when I have the full house. Maybe he'll fold
and his small flop bet was a cheap attempt to pick up the pot.

Instead of a quick call, the Button starts mumbling to himself and the
table, and looks pained as though he's been drawn out on.
``Stupid turn card. I think you've got the flush...two hearts right?
800 to call...I think you're going to roll over that flush on me...
stupid turn...this hand is too big to lay down...oh well do or die...
I have to raise with these cards...I'm all in...2,300 more''.

Wow. What a speech. What kind of player spends a full minute agonising
over a decision to call on a board of 993T6 three hearts, and keeps
wondering to himself and the table whether I have the flush, yet still
raises?

A player with a full house. Full houses love being called by flushes
and straights. Interesting in this speech is that he never asked
if I had a full house. He's umming and aahing over me having
hit a flush on the turn. He's calling my hand a flush, over and over
again, pretending to be scared of the flush, and still raising!

If I'm genuinely afraid of a bigger hand than mine, but I've
got a strong showdown-worthy hand, I just call the river bet.
If I hold say A9 and the final board is 993T6 and I face an 800 call into
a 2,300 pot, I'll just call the 800. I don't ``have to raise''. Sure 999AT
looks nice but there's a lot of hands, including that flush, that beat me.
Even a sneaky 87 makes a straight on me here. There's actually not many
hands my opponent can have that he'd raise on the river that I beat here- maybe
he's excited by A9 and he has 999AT but I'll just be calling with that hand,
not making long speeches and raising all in.

Laying down a Two-card full house is never easy. I'll take a final look
at the board 993T6 and see what hands can beat me, remembering the
action on every betting round.

\subsection{Replaying the hand}

What holdings beat my full house? Quite a few as it turns out, since
the full house is the lowest possible:

99,TT,T9,96,93,66

I'll eliminate some holdings I don't think he has. 93 is too raggedy
for the preflop call, plus I think he'd slowplay the flop of 993.
I have 33 myself so there's only one 3 unaccounted for. While it's
still possible he has that fourth and final 3, it's very unlikely.
99 makes quads which is again unlikely, plus he put a small bet in
on the flop and everybody slowplays quads! So I'm left with the following
hands that beat me:

TT,T9,96,66

TT is possible but I think the Button would have raised preflop with TT.
Most of the time you want to get Heads Up with TT and 4 people limped
into this pot. TT would make that small flop bet on 993, to try and see
if he's ahead with the overpair and get out of the way if there's a big
check-raise.

Actually I think 96 is a bit raggedy for the preflop call, so I'll
discount that too.

T9,66

66 makes sense from the betting, the flop bet could've been a probe
and he could have slowed down on the turn and caught lucky on the river.
T9 also fits the betting action.

It turned out that the Button had T9 on this hand.
He built the pot with a small bet on the flop, called by me.
The Ten of hearts on the turn was his money card, the
Full House of Flush. He checked behind
to try and let me catch up with a four-flush if the river was
another heart, or catch a lucky straight.
When I came out betting on the black six on
the river, he figures me for the flush (based on my quick
flop call, and perhaps I was shooting for a check-raise on the turn)
so he chats for a while about the flush, pretends to be scared of it,
and then puts in his raise.

His long speech about the flush helped me here. About
the only honest thing he said in that whole speech was
``this hand is too big to lay down'' and ``I have to raise here''.

I folded my 33 in this hand, a huge laydown but one
justified by the betting actions, my reasoning and the result.
Luckily you don't have to do this hand analysis too often, but
when you're facing a big decision in a big pot, take your time to
work things through and give yourself every chance of making your best
decision.

It's hard to lay down a hand when your natural poker instincts are
screaming ``I've got a powerful hand, I have to call here!''. One of
the differences between the amateurs and the Pros is the Pros don't
fall in love with their hands and can make big correct laydowns when
their instinct tells them that their power hand is still losing.
You can be correct to call with middle pair bad kicker in a huge pot
and correct to fold a two-card full house in a huge pot. Try to see
the situation as accurately as you can, and make the big laydown when
you think you're losing.

\section{Behind the whole way}

Here's a hand where I doubled somebody up. I held J9 and raised from
under the gun and got two callers. Flop was
Q95 rainbow, turn was a 9 and river was a 3. I put in a good bet
with my trip 999 Jack kicker and got raised and I called and paid off
a guy who had QQ and slowplayed it nicely, flat calling all my bets
except the river one. Nobody bluffs a made hand.

