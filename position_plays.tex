\chapter{Position Plays}

% Last updated: 20181029

Here's a short poker hand.
Blinds are 50/100, there's a raise to 300 and one caller, the blinds fold.
Two players see the flop, there's 750 in the pot. The flop is \Qh\sevs\fivec.
The first player checks, the second player bets 500, the first player folds.

This hand plays out over and over and over again in poker games all
around the world. This is the poker equivalent to a 1-0 score in soccer.
If I watched an average poker table for ten hands, I would expect
about three of them to play out like this.

This chapter is all about position. It's one of the
most powerful concepts in poker.

\section{Fundamental law of position}

The later your position, the more power you have.

Poker is a game of incomplete information, unlike Chess where you
can see all the pieces all the time.\footnote{In Chess the player
with the first move, White, has an advantage. In Poker the player
who acts last has the advantage.} If you're in last position
the other players must bet,check,raise and fold in front of you. By
the time it's your turn to act you have all the information on
what every live player has done in this betting round. This will help you
put them on hands, evaluate your chances and make better decisions.

Also you can control the pot size a lot better from late position.
Often you can see a cheap turn and river. If you want to bet
you'll have a better sense of the right amount to bet to achieve
your goal. Controlling the pot size helps you win big and lose small.
When you want the pot small, you keep it small. When you want the pot
big, you grow it. It's beautiful. Sometimes an early player
bets big and you have to fold your draw, but all you've done is
lost a small pot, and again you've kept the pot small.

Mike Caro has a saying ``your profit comes from the right'' and another
saying ``Chips flow clockwise''. These mean
that the chips you win usually come from the players on your right,
who act before
you. This is entirely because of position. Players to your left win
your chips, when they have position on you.

Position is the one aspect of poker that comes around every few hands.
You can't materialise extra chips into your stack by magic, and you
can't wave your hands and turn your 33 into T9 to a flop of 678.
But every few hands, you'll be on the Button, and next hand the Cutoff,
and be in powerful late position.

\section{Position Play 1: Blind stealing in Position}

The simplest position play is called ``Blind stealing'' and involves
winning the small blind and the big blind with a
standard raise from late position. Blinds 50/100, four players fold in
front of me, I make it 400 and the small blind folds, the big
blind grumbles slightly with a hand he'd like to have seen a flop with
(87 suited) and folds. I win 150 chips with this late position play.

Now see what happens when the big blind does take the flop. There's
850 in the pot ( 2 x 400 plus 50 small blind) and the flop comes AT5
rainbow. The big blind checks. He wants to see if I have an Ace.
He's drawing to pretty much nothing, he's a small chance of runner-runner
flush and if a 9 turns he's up and down to a straight, but really he's
folding to any bet with these cards. He'll take a free turn if he can
get one, to see if he gets the straight draw or maybe a flush draw.

On the big blind's check, I bet 500. This is a bit more than half
the pot, and the big blind has nothing. He folds.

What were my cards in this hand?

73 offsuit. If I'm asked I might make up a good but non nut
hand that would have been a good preflop raise, and a good post-flop bet,
like ``Ace Queen'' or ``Ace Jack''. But I wasn't playing my cards. I was
playing my position. I was betting that the other player wouldn't hit a flop
that he liked enough to continue in the hand with. And most of the time
in poker, you don't hit a flop that you like.

This is a key point so I'll repeat it. I'm not value-betting that my hand is
stronger than his. I'm betting \textbf{against} the possibility
that my opponent likes the flop enough to continue in the hand, in the
face of all my betting.

So my bet has nothing to do with my own cards. What my
flop bet is saying is ``I bet you didn't like that flop enough to
continue in this hand''. That's a high probability bet in Holdem. Since I'm
the preflop raiser my bet is also backed up with perceived hole card
strength.

When the big blind took the flop, I win 450 just from having late position
(I've subtracted my own chips in counting my winnings from this pot---
I've won 50 from the small blind and 400 from the big blind).
Recall that when the big blind didn't take the flop, I won 150 from
winning the blinds.

Here's a third scenario. The big blind, not with 87 suited this time,
has called and the flop comes \Ac\tend\fives. The big blind checks, I bet 500.
He raises to 1,200.

What kind of hand would he not raise with preflop, take a flop with,
and then check-raise me with?

Pocket fives. This is a classic play when hitting your set. Once you hit
your set, you hope your opponent has the Ace, and is drawing almost dead.
You check it, wait for the continuation bet (which you hope is from a
real Ace-great hand) and raise it.

Here's the beauty of the position play. I've got 73 offsuit.
I've just been check raised on a board of \Ac\tend\fives. I fold!

Much of the time I win 150. Sometimes I win 450. Sometimes I'm check-raised
and I lose 900. Losing 900 isn't great, but overall the play is a moneymaker,
and I've got a simple fold to the check-raise
because I haven't even got a hand! I'm in a lot more trouble having
a real AQ hand here. I should probably respect the check
raise as a set or two-pair, but AQ is hard to throw away
on a board of AT5 rainbow in a fast tournament.
The most likely thing here is that I'm a long way behind,
and I need the discipline to fold the AQ. Whereas I can easily
fold my position-only 73 hand, I certainly know that my hand
is behind.

Bets can be bluffs in Pub Poker, but check-raises are nearly always strong.
Unless I pick the player as one of the more tricky players I've seen
in a pub game, I expect the check-raise to be strong. Check-raisers
expect to be called and usually have strong hands that are about
80 to 95 percent favourite. Unless you're up against a super tricky player,
expect the check-raise to be super strong and fold to it.

Position plays don't knock you out of tournaments, good but vulnerable
hands knock you out of tournaments. How many times have you lost
a Poker tournament with Pocket Kings, Pocket Queens, Pocket Jacks,
Ace-King?

% I once had three flopped sets
% all drawn out on by straight and flushes. One of them was JJ to a flop
% of QJ8 and my opponent already had the T9! You can't get away from sets
% in a fast tournament, you have to call and hope, praying your 
% 34\%, 1 in 3 chance of making a full house occurs this time.

Here's another hand that shows it's your strong hands that knock you out of tournaments.
I'm in late position with QQ and make a standard raise from the Button,
getting two callers, from the big blind and a limper.

Flop: 456. I bet and get a call from the big blind and the limper folds. \\
Turn: Ace.  Check and check here. I'm scared my opponent hit his Ace \\
River: Q.  The big blind leads out for about 60\% of my stack, and I raise all
in. He beats me to the pot and shows 87 for the nut straight.

Great but non-nut hands knock you out of tournaments. Position play
bluffs that get raised don't.

\section{Position Play 2: Calling a raise in better position}

My second position play is to call an early position raiser from better
position. Again, I'm looking for the he checks/I bet half the pot/he folds
pattern on any flop. If the early raiser bets about half the
pot and seems a solid player, this could be a continuation bet. I might
try a raise here and see how strong his hand is, or if I think he's a
one-shot bluffer I can call again and look for the check/bet/fold
pattern on the turn.\footnote{A call looking to bluff on a later betting
street is called a \textbf{float}} If he fires out a bigger bet again on the turn I
can fold. If he fires out the same bet on the turn I can consider if
I'm being milked or if he's weak. If he's a very passive player,
he just doesn't like betting big, but he's still got a power hand.

There's a lot of possibilities even with just two players left in the hand.
I want a good sense of what my opponent will do, and whether I can
get him to fold. If my opponent is a rock and plays solid cards
and calls big bets, I would probably fold preflop and avoid calling his raise
and trying to make this play at all.

Sometimes the early raiser is a rock, who likes his QQ preflop.
A flop comes AK6 and he checks. I bet half the pot, he grumbles and
folds. Remember that rocks expect you to be rocks as well, and think
to themselves ``he wouldn't be doing that without the Ace or the King''.

This is why continuation bluffs are so important. If the same rock bets
preflop and I call and the flop comes AK6 and he bets half the pot
I put him on Ace-Queen. I'm folding here. But by checking, he's giving
up the lead in the betting and allowing me to represent an Ace. After
all, I called him with something, right?

\section{Position Play 3: Checked to twice}

You won't always bet on the flop when you're checked to. You might
be afraid of the check-raise or you might have flopped a good looking
draw that could win you lots of chips if it gets there. The turn comes,
missing your draw, and again it's checked to you. Now it's time to bet.

Very few players will check on the turn after checking the flop,
looking for the check-raise. The turn is the second-last betting
street and if you've got a huge hand, you have to start building the pot.
There's no point winning a 5BB pot with a flopped set; you have to
build the pot up with this big hand according to Big Hand/Big Pot.
Checking the flop and checking the turn keeps the flop small.

So if you're the in-position player and it's been checked to you twice,
it's a great time to bluff at it. Your opponent has slowed right down
and this pot looks ripe for the taking.

% FIXME: cite Dan Harrington on this idea

\section{Playing from out of position}

This is one of the hardest things to do in poker, and is one of
the reasons why the small blind and big blind are
terrible positions in poker (it's not just because you had to post
a blind). You expect to lose money from these
positions overall, from being unable to defend your blinds and secondly
from having to fold to aggression from better-positioned opponents when you
are in a hand.

The best hands to play in the blinds are pocket pairs which might become
sets. You can often take flops cheaply with speculative hands, but even
when they hit you can't control the pot size from the small blind. If you
overbet it you'll probably get folds, if you bet it you might get one or
two callers but you're winning a small pot (unless you can check-raise
a set). Whereas if you're in late position, sometimes you get early
position players betting into your made hands. You can decide to raise
if you think the better is a maniac who likes his hand, call and look
like you're on a draw, or take it down then and there. Playing in late
position gives you two powerful things: more information and better control
over the size of the pot.

I sometimes like to lead the betting from out of position, especially
if I've called a better position's bet preflop. Remember that
it's easy for him to miss the flop as well. I do this
with sets every now and again, instead of making the classic check-raise
move. But every time I'm out of position, I find it an
uncomfortable test of my poker skills.

This is the reason I'm folding A9 suited under the gun but raising
with A6 from the cutoff preflop. I'll be playing my A9 out of position the
whole way, and I can get a flop of AJ4 and be outkicked by AQ the
whole way. Whereas with the A6 in the cutoff, and everyone folded to me,
I'm only up against three remaining hands and I figure mine is
the strongest right now. I've got a chance to steal the blinds with a
bit to back it up, and more often than not I'll get those blinds. If I get
one caller, I hope for the check/bet/fold pattern on any flop.

Having position on your opponents is a huge advantage in poker. Think
about your position in every pot you play.

\section{Position hits}

Sometimes I'll call a preflop raiser with position with any two cards
and that trash will hit a monster hand. I have J3 and called
the early raiser and the flop is J33. Sweet! The preflop raiser
bets half the pot.

I just call this bet. I act like nobody could have hit that flop, and
I'm confident that your opponent couldn't. I don't try and act like
I'm on a draw because there are no draws to J33. I'm representing an
OK hand on that flop, like QJ, JT, J9. Of course if I did have a hand
like this I'd probably put in an information raise to find out where
I was at, and if I was up against AK, AQ or an overpair like QQ or KK.
But few people in Pub Poker will be good enough to reason that the
absence of my raise here proves I don't have a JT style hand.

The turn card is a 7. Again the bettor bets and I call him.
No reason to raise here. He's doing all the betting into me. Again,
I don't waste time pretending I'm on a draw.

The river is a 4. There's no way I had 65 the whole way. He bets a third
time and I raise him all in and he calls with KK.

Now's the time I'll get asked the question ``How could you call that
preflop raise with J3?''. The true answer is I wasn't calling with
cards, I was calling with position. I was setting the hand up for
a check/bet/fold situation if my opponent missed the flop. In this
particular hand my opponent didn't need to hit the flop (he's got pocket
Kings the whole way) and I lucked a monster hand.
It happens.\footnote{Joe Hachem's famous 73 winning hand in the 2005
WSOP Main Event was a Position Hit. He raised in position with 73, so
he could steal a bigger pot from position later in the hand, and caught
the miracle flop 654. Must be nice.}

These are the hands and showdowns that give me my wild/loose image.
I never tell the table I'm a smart position player who can
win from the Button with any two cards. In this particular hand I
won with two crazy cards, in the strangest way possible. The table
thinks I'm a moron, calling a decent raise with J3! Two hands later I get
pocket Queens and I get one call from a decent stack preflop who is all-in.
The table groans that an idiot like me got dealt a good
starting hand, a starting hand that they deserve, because they'd never
throw money into a pot holding J3.

I'm playing Tight/Aggressive poker, with positional awareness,
and I've got an image of a maniac. Watch as my good hands get paid
off over and over again.

I've won whole tournaments playing like this.

\section{Cards still count}

You still need to make winning hands in poker. No matter how good you
are at reading hands and bluffing the timid players, you'll have to
hit some hands and win some showdowns. If you get dealt
\tenh\twoc\ and \Qh\fivec\ all night you'll be able to stay in
the tournament for a while with position plays and blind steals,
but eventually you'll be out of the tournament.

This unhappy state of affairs is called being ``card dead'' and is
the bane of poker players everywhere. Even when card dead, you should always
be on the lookout for position plays that can win you pots with nothing,
and keep you in the game for as long as possible.

Here's what happens to a rock who only plays cards and not position.
He finally gets a proper hand, \tenh\tenc\ after an hour. He raises all-in
with his tiny stack and gets called by a much bigger stack
holding \Kd\Qd. The flop comes \Kh\sixc\trec\ and that's the
end of the rock's tournament. He busts out of tournaments again and
again like this. He picks up a new bad beat story each time, but he
ignores all the opportunities he passed on earlier in the tournament
to make position plays and grow his stack, even when he didn't have
the cards to back it up with.

Many Pub Players have card sense only and no appreciation of position.
Some are never stealing blinds and they'll even fold the
Button after everyone else folds and leave the hand a battle between the
two blinds! They're certainly never calling a raiser from position and
trying to use the power of position to steal the pot.

\section{Finding out more}

\textbf{The Poker Tournament Formula} by Arnold Snyder
(Cardoza Publishing, 2006) really opened my eyes to the power of position.
Snyder talks about
Tournament Holdem as being a game of scissors/paper/rock, only the
three weapons are cards, chips and position. Position beats cards, as we've
seen in the examples in this chapter. Cards beat chips, because if your
cards are strong enough you're calling any bet. And chips beat position,
because if you're sitting behind a huge stack, you're calling
from out of position and you're harder to bully with ``I bet you
must have missed that flop'' bluffs.

Snyder considers position such a force he suggests playing a small
tournament without even looking at your cards, only pretending to. You make
your checks, bets and raises entirely on position (and any tells you can
get from the other players). ``After all, how often do you win when you
DO look at the cards?'' he asks. I haven't tried playing a whole tournament
blind yet, but one night in a \$10 game I'll give it a go.

Snyder's book, like this one, concentrates on fast tournaments. It's
the most revolutionary Holdem poker book I've read, and also one of
the funniest. The chapter on Player types (Flush Masters, Pair Masters,
Show 'n' Tellers etc) is one of the best things I've read in poker.

Snyder's written a sequel, \textbf{The Poker Tournament Formula 2}
(Cardoza Publishing, 2008) which
concentrates on bigger tournaments. A key idea in the second book
is Chip Utility--- the importance of having enough chips to do
everything you need chips for in poker.

Arnold Snyder is a pseudonym. He's also published some books on the
game of Blackjack.
