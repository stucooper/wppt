\chapter{Skill and Luck}

How much of poker is skill and how much is luck? This is a question
that has interested poker players, and poker writers, for decades.
In many countries and states the judgement of poker as either a
skill game or a luck game determines whether games are legal or
illegal.

\section{Stuart's view}

Tournament Poker is a game of skill where you also need more than your
fair share of card luck.

Consider a single day \$220 tournament of 100 players. The top 10
players make money, 11th is the bubble. The buyin of
\$220 keeps away total beginners and the first place \$10,000
guarantee attracts professionals and casino-grade poker players.
This tournament has ballpark 80 regular players and 20 professional
players, with more experience and poker skill.

In the final 10 players there will be on average 6 pros and 4
regulars. The pros made up 20\% of the starting field but 60\% of
the paid players. 80\% of the 100 starters were capable regulars, but
only 40\% of those who are paid are non-pros. That's the skill
advantage of poker. Luck of the cards, bad beats and suckouts will
determine which 6 of the 20 pros get paid today, but their poker skill
gives them a greater chance than the amateurs.

I play in a game like this three or four times a year in the far
south of Sydney, at Engadine RSL. Over the last few years I've played
in it 10 times and I've got two second places and a fifth. Of the
seven times I didn't cash my best finish was 23rd. This tournament has
been a big moneymaker for me; for \$2200 in buyins I've got back
\$11,000. The game has weeknight pub poker players in it who win their
ticket in weeknight games, so the player split isn't 80 regulars/20 pros.
From my own casual judgement the game has 70 capable regulars, 20 dead
money players and 10 professionals.

My swapmate plays this game also whenever he can and has had similar
results to mine. One game we both finished top 4, one time he paid
me on his good finish and another time I paid him on my good finish.

\section{Dan Harrington's Lottery}

\textbf{At the WSOP} Dan Harrington made the WSOP Main Event final
table in both 2003 and 2004, on top of winning it in 1995. He
did a short interview for the WSOP 2005 coverage. He explained that
winning the WSOP was like winning a lottery, but he had more tickets
than the other entrants.

``Think of it as a lottery. I'm being given 4 or 5 tickets in the
lottery, and some other people are being given a quarter of a ticket
or whatever. You get the idea. Four tickets out of last year 2,600
entries.. it's very nice, but it doesn't exactly make for a sure
cashing of the money.''

In much smaller fields, Doyle Brunson won back-to-back in 1976/1977,
Stu Ungar in 1980/1981 and Johnny Chan 1987/1988. Chan just missed out
on three-in-a-row, coming second in 1989 to Phil Hellmuth.

\section{Pros versus the Pub Players}

In Sydney in 2010 there was a huge \$2,200 tournament of 250 players
called the NPL 500. Full Tilt Poker sponsored the event and although
it was ``Qualifiers only'' for months, in the final weeks direct
buyins were allowed and then some Full Tilt Pros played in the
tournament.  Full Tilt Australasian stars Mark Vos, Van Marcus, Simon
Watt, Aaron Benton and Jason Gray were some of the pros who took to
the felt against the pub qualifiers.

Amateur player Terry Tserdanis won the event, second was Jason Gray and
third was Aaron Benton. Two of the top three finishers were pros. With
12 or so pros, the experts made up 0.5\% of the starting field and
66\% of the final three.

\section{It's a skill game}

If you need to explain poker skill to your partner or your family, you
can try what Matt Damon tells his girlfriend in Rounders.

\textit{Why..why does this still seem like gambling to you? Why do you
  think the same five guys make it to the final table at the World
  Series of Poker every single year? What are they, the luckiest guys
  in Las Vegas?? It's a skill game, Jo}
