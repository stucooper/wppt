\chapter{Decisions and Outcomes}

% Last updated: 20200507

You play good poker by making the best decisions you can,
using the information you have. You don't control the cards
you get dealt, nor the cards that appear on the flop, turn and river.
You don't control what your opponents will do -- you try to influence
them but in the end it's up to them what they do.

% A poker pot is awarded either to the last man standing in the
% betting, or the best hand at a showdown. It's not awarded to the person
% who made the best decisions on the hand on every occasion.

\section{Decisions}

Things under your control I'll call decisions. During
every poker hand you face several decisions.

Do I stay in this hand? Check or bet? Facing a bet, do I
Raise, Fold or Call? If I'm betting or raising, by how much?

These are the decisions you make at the table, every hand
of poker. If you make it to Showdown, you've made decisions on four
betting streets. This book is about making the best
decisions you can based on your position, your cards, your chips,
the tournament stage, the pot odds and implied odds, and the number
and types of opponents you have and their chips, position and
expected cards.

Apart from the obvious betting decisions you can make at the table,
here's some other things you can decide.

Will you stay patient tonight, or will you take early chances and risk
a quick bust out?

Will you going to try a new idea tonight? Which idea will
you try, and how will you measure its success?

There's a whole lot of other scenarios, my
point is you want to have the best attitude
you possibly can.

\section{Outcomes}

At times your tournament life rests entirely on the outcome of
a 50-50 event (called a ``coinflip'').
The classic coinflip is AK vs QQ, all-in preflop. QQ is a slight
favourite, 52\% to 48\%, but the numbers are so close that it's 50-50.
A coinflip where the chips are in preflop is called a race.

An Old Poker Saying teaches us ``To win a
Poker Tournament, you have to win \textbf{with} Ace-King and
you have to win \textbf{against} Ace-King''. Which says that you're
going to have to win at least two big races.

It's maddening when your strong hand loses to a 20\% or less
hand on a bad runout. Pros shrug this off with
the thought ``I got my money in when I was the favourite'' and they're
correct. If they're getting their opponents to pay too much, calling
with bad pot odds, then they're playing good poker, regardless of
the outcome.

Bad players are supposed to draw out in poker. If they never drew
out, the game wouldn't be worth playing, because the bad players
would be forced to improve to good players, or leave the game
altogether. Bad players are what makes poker worthwhile.

\textbf{At the tables} You're going to suffer bad beats but you'll
also give bad beats to others. Here's four big matchups from a game
I won \$1,000 in once. My hand is first, my opponents hand is second
and the chips are all-in preflop each time. In the hands that I
lose, I have chips remaining and stay in the tournament even
though I've just been bad beated.

My Suckouts: AK vs KK, AQ vs JJ

My Bad Beats: 77 vs 55, AK vs A5

AK losing to A5 was tough, but AK beating KK was an enormous suckout.

Everybody remembers and complains about their bad beats but few
players admit to their suckouts. If they get lucky in a big pot, they
feel they deserved this pot, because of their superior poker skill;
even when they made an obvious mistake like calling all-in preflop
with 66 against an obvious high pair and they lucked their set.

% FIXME: \footnote{} on the term rabbit hunting

The more you focus on decisions and the less on outcomes, the
better player you'll be. This is why rabbit hunting is so bad. If
you fold a drawing hand on the turn, that's a good or bad decision
based on pot odds, implied odds, your tournament situation
and your reads on the other hands in play. Whatever the river card is
doesn't change the quality of your turn decision. Your fold on the
turn is a decision. The river card is an outcome.

People who must see the river card are outcome-focussed. Looking
at the river card can only tilt them, if their folded hand would
have hit its draw. It's not a good fold because ``it would have
missed'' and it's not a bad fold because ``it would have hit''. It's
a good or bad fold, on the turn, based on the information you
had at the time, not on the outcome of the river card.

I'm so strong against rabbit hunting these days that if I'm in the
Big Blind and everyone folds preflop (I get a walk) I don't even look
at my own hole cards. Because if I do and they're pocket Kings I'll be
annoyed that nobody gave me action on this hand and I only won my big
blind back plus the small blind. I don't want to annoy myself at the
poker tables when I don't have to, so I fold my hand unseen.

%% In home cash games I do often ask for a rabbit hunt. I ask the
%% dealer ``Could you run it out please?'' and I tip the dealer a Big
%% Blind chip for doing this for me, \$1 in 1/1 games and \$5 in 2/5
%% games. Tipping each time I get a rabbit hunt is polite to the dealer
%% and also means I only ask for rabbit cards when I really need them. In
%% casino games and tournaments I never rabbit hunt.

%% I should probably get myself out of this habit for the reasons I've
%% already said, my fold was a good or bad fold based on the information
%% I had, not on the outcome of the river card.

One pleasure good players get from poker is the satisfaction
of making the right decisions, over and over. Pots and prizemoney
are awarded on outcomes, but they're really out of your control.

Make the best decisions as often as you can. You'll find the pots
and prizemoney will follow, over time.
