\chapter{Heads Up play}

% Last updated: 20190529

Heads Up play is very important and is where the biggest money
difference usually is. For a tournament with
a first prize of \$150 and a second prize of \$100, once
you reach the final two you've both won \$100 and now you're
playing a Heads Up game for \$50.

%% You can break your tournament into three tasks, each of
%% which requires different skills

%% Making the top 50 percent \\
%% Making the final table \\
%% Making the final four  \\
%% Winning the tournament

Some players aren't comfortable with Heads Up play
and will try to do a fity-fifty split straight away (in the prize-pool
above, both players get \$125). They give away the
fact that they're not happy playing Heads Up. In this case
you should have a big advantage and you should play on.

If however you respect the other player's game, or he
has a big chip lead, you should take the split. If you've
got a good read on his style of play and think you
can exploit this, keep playing.

Something I'll often propose is flattening the prizemoney.
Ask yourself if you'd be comfortable playing a Heads Up
game for the prize difference. One game I played had
a first prize of \$900 and a second prize of \$650. I wasn't
comfortable playing for \$250, chip stacks were about even,
so we changed the prizemoney to \$800 first and \$750 second.
Instead of playing Heads Up for \$250 we were now playing
Heads Up for \$50, and second place was now \$750
not \$650. That \$50 difference between
First and Second was still worth playing for.

My opponent had won at the same venue the night before
so seemed in excellent form. Chips were even and we both
had about 80 big bets. All in all the deal looked
pretty good. I was patting myself on the back for a smart
deal as we took a short break.

I was kicking myself five hands into the
Heads Up as my opponent played like a complete chump.
If he checked I'd steal, if he bet he always had it.
How on earth had he won here the night before?
Three hands later he pushed all in on a board of K99K.
I was happy to call with my K3, expecting a split.
He rolls over QJ and said he bet so big because he'd
wanted to get me off my draw. What draw could I be
on to a board of KK99?

This particular tournament was a very deepstack \$150
game, without too many players. So a \$900 first prize
was a \$750 win. By flattening first prize to \$800 I still won \$650 but
I could have won \$750 if I hadn't done the deal. And we should
try to win big in poker.

So I've changed my deal-making strategy. Now
if the stacks are more than 20 BB deep
I play ten hands Heads Up against someone,
and then start talking about deals. This way I know for sure if someone
is comfortable Heads Up or not. If the stacks are 15 BB or less then
there's only a few hands left in the tournament and immediate short term
card luck will determine the winner not poker skill.\footnote{Poker
players call this situation a \textbf{crapshoot}.} When stacks are
this short I'll look for a fair split.

One man I got Heads Up against in a sit and go once hated
Heads Up so much that he pushed all-in no-look every hand and
I got to look at my cards and decide if I wanted to play or not.
Strictly speaking he's out of turn every 2nd hand preflop but
if that's the way he wanted to play Heads Up I was happy to
ignore that rule.

What an advantage! Remember that Q7 is the median hand in Holdem?
The first hand he did this I found 32 offsuit
and folded, the second hand I had AQ and called extremely quickly.
He found JT as his hand, a flop of 986 gave him a swag of
outs twice but the turn and river blanked off and
I won that sit and go.

\section{When you get Heads Up}

Once the third player busts out, propose a short
break in play. Unless you've just come back from
a long break, it's a good chance to take a 5 minute
pause. Don't propose any deals just yet, because
your opponent could turn out to be a Heads Up chump.
Take a break, stretch your legs, gather your thoughts and have a quick
breather before the final battle.

Congratulate yourself that you're in the real money
and the worst you can do is come second. But
recognize that there's still work to do.

When you start playing you should do your own
cuts. That keeps it friendly and says to your
opponent that you trust him not to cheat. You
still both use the cut card, you keep it in
front of yourself and cut your own cards onto it.
If someone offers to deal for the two of you, you
should usually allow it (see the chapter on
Dealing).

\section{Heads Up Play}

Heads Up hands differ from hands with more players in them.
You don't have to raise with QQ and hands that play well
against only one opponent- because you're already down
to only one opponent on every hand. Hands that play
well Heads Up are high card hands and pocket pairs.
Connectors and suited cards go down in value because
they just don't hit very often.

Position is still important Heads Up. So is blind stealing.
So is keeping track of the stack sizes and how many big bets
you have left and how many big bets your opponent has left.

Some players (myself included) have a habit of doing min-raises
Heads Up. With blinds of 1,000/2,000, the preflop raise will usually
be just to 4,000 not to the usual 2.5 to 3x big bet of 5,000 to 6,000.

Try and get a good feel for how your opponent plays
preflop and on the flop. There's usually not enough
chips to worry about subtle play on the turn and the river.
Someone with a flush or straight draw will usually be
pot-committed by the turn and won't be pushed off a
good hand. Someone with middle pair shouldn't be folding
very often with two cards to come.

In a fast tournament you may not have many big bets left by
the time it comes down to Heads Up. An extreme example would
be a 10 player sit and go, you get 2,000 chips to start with
but the blinds go up fast and when you get to Heads Up
you each have 10,000 chips but the blinds are 1,000/2,000.
You've each only got 5 big bets! This is a complete
crapshoot and you should make a prize deal split here. Poker skill
isn't going to decide this game, it's whoever gets better
cards over the next 4 or 5 hands because that's how long
it's going to take to finish the game one way or the other.

% FIXME I said this earlier in the chapter

Remember the final tables you see on World Poker Tour and
World Series of Poker events are of multi-day tournaments that had
hundreds or even thousands of entrants to begin with. So Heads Up the
players still have 50 BB or so. In a sit and go, or
a fast tournament with 30 runners, the deep chipstacks
just aren't there.

Look for tells. You've only got one opponent to study so
you shouldn't miss anything. Don't be shy in staring.
Think about what style of player he is, and what style
plays well against that player. Because there aren't
that many chips in play, now's the time to steal from
the rocks. Even if you make a mistake and push all in
and get called by a better hand, much of the time you'll
have two live cards and still be a 35\% chance of doubling
up or winning.

Any two cards can win Heads Up, especially if you continue
to play aggressively and push your advantages. Play any
pair strongly. If you have 44 and the flop is JJ3 that's
a dream flop for you. If your opponent bets he's likely
to have K3 and be a long way behind. If the flop is 77T
you're still likely to have the best hand. Remember,
from position, that it's hard for your opponent to
hit good flops.

One of my favourite starting hands Heads Up is AQ.
This is because it dominates a lot of other hands
that look good as Heads Up starting hands.
It's a huge favourite against AJ, AT, A9 etc and also
against KQ and QJ which look like great Heads Up hands
and usually are until they run into AQ. AQ is only in
real trouble against AK or AA, KK, QQ and you'll often
sense your opponent is that strong preflop. You should
take your chances that an opponent doesn't have a monster
hand preflop against you when you get AQ. Against JJ, TT
and other great heads-up hands you're an OK coinflip.

In a game where we have 15 big bets behind us, I'm
taking AQ to war.

\section{Some Heads Up moves}

Once in a while I'll be on a steal-a-thon against
a rock. I might have stolen 4 pots preflop in a row.
Rather than press the accelerator fully to the floor,
I'll fold a hand as dealer preflop.

This move looks insanely tight. Having won several
pots in a row I'm now folding for the cost of a small bet?
Yes I am. For a very small cost I've made all my other
preflop steals (and my future ones) look genuine. I can go
back to preflop stealing next hand.

Whenever I make this move I always like to see what
my opponent had, and guess what he would have done had
I limped in. Often I'll show my hand and say ``Can't play
Jack three Heads Up''. Even if I fold face down my opponent
will usually take a peek at his starting cards. Once
in a while he'll find KK or AK suited and be tilted
that he only won a small bet. Folding Heads Up preflop
like this can be one of the most enjoyable moves in poker.

One suggestion. If you win a pot from the big blind in a walk,
don't even look at your own holding. This hand is already over. By
checking what your holding was, all you're doing is giving yourself
the chance to get upset by finding AA or KK and realising you're not
winning anything but the small blind with this monster holding. If you
check your holding when you don't have to, you're Rabbit Hunting your
holding, with no upside. All it can do is make you angry.

\begin{description}

\item[Shove quickly] If I flop something good but vulnerable,
but I'm confident I'm ahead, I'll shove quickly. You have
to be prepared to beat your opponent out of the pot, and
give him no chance to catch and no time to think. Try to
rush him into making a mistake. If I have 87 and get a flop
of Q87, I'm shoving. I might get a passive opponent
off TT which a Queen on the river would have won for him.

Fast poker tournaments aren't about trapping, and fast
Heads Up games are even less about trapping.

A quick shove can look stronger than it is, and fold
your opponents hand. It can also bully your opponent
into calling when way behind, giving you the best
chance of winning the game here and now.

\item[Keep something behind] Even when a preflop call obviously
pot commits me, I like to keep some chips behind to bet
any flop with. I'm not getting my opponent out of this
pot preflop with a raise, but if he gets a scary flop
then my last few chips might get him out of the pot.

Say we have 10k each, blinds are 1k/2k (crapshoot time!).
Preflop my opponent raises it to 6k. Now he's already
committed so a raise to 10k does nothing for bluff value.
I have KT offsuit- decent Heads Up but possibly dominated.
I call the 6k.

The flop comes AQ9 and my opponent checks. I announce
all in for my remaining 4k. My opponent grumbles
and folds his pocket 44 faceup. ``Queen Ten'' I say
as I fold face down and pass him the deck for the next deal.

What flop looks bad to pocket fours? Any flop with three
separate overcards (which is most of them). You're betting
your last 4k into a 12k pot, but he might not feel like calling
it. But that same little 4k wasn't scaring him preflop,
when he still wants to see the flop and is prepared to take
his 44 to war. If he's all in preflop he gets the full Five-Card
runout; if you can bluff him off on the flop you save your tournament
life.

The ability of your remaining chips to get your opponent to
fold is called ``fold equity''.

\item[Show preflop folds] I like to show my
preflop folds in Heads Up. This will encourage my
opponent to do the same. I want to get a feel for his
limping range and raising range. If he raises from the
Button and I have J3, I'll fold face up saying ``I hate
Jack three''. He'll quite often show his own raising hand,
like Ace Queen or Ace Jack. If he rolls over trash like
83 then I know he's tricky and will try a few steals.
If he rolls over 76 suited then I know he overvalues
his speculative hands Heads Up. If he rolls over KK
then I'll say ``nice hand'' and be genuinely thankful
he pushed me off the hand preflop.

If my opponent doesn't show his hands as well, I'll give
up showing my folds.

Remember more than ever not to look at your own
starting cards until it's your turn to act. You
don't want to be giving away telegraphs when your
opponent is trying to decide whether to limp, raise
or fold!

\item[Call to disguise] If you have a big pocket
pair like JJ or better then a lot of the time
you should just limp or check and hope that the
flop is tempting enough to get your opponent to hang himself
while still leaving you a mile in front. If you're
raised preflop, even better. I love having JJ
and I call a raise and the flop comes T72 and my
opponent goes all in! He's usually got AT (sometimes
KT, QT or JT) and I'm a mile in front. If he's hit
a set well that's poker and I wouldn't have got him
off the hand with a preflop raise anyhow. If my
opponent has the AT, he'll be shoving
all his chips in when he's 20\% to win, and I'm calling
that every day.

Once in a while call to disguise can hang yourself. Once
I had QQ and was very happy with the flop of 952 until I
saw my opponent roll over 95! This time a preflop raise
would have won me a small pot, and my greed to win as much
as I could with a monster hand crippled me in the tournament.
You don't get hands as big as QQ very often Heads Up, so
you want to win the maximum with them, even at risk
of losing the tournament. So overall I think that my QQ call
preflop was the right play.

\item[Pot Committment] Keep an eye out on whether your opponent is pot committed
(which is now tournament committed). If you have 90\% of the chips and
the blinds are big your opponent might push all-in no look.
Now is not the time to double him up with complete trash like
52. If he just limps, you can raise him all in if you think
he'll fold, but in this instance I'll let him take a flop,
hopefully miss, and then bet him all in on any flop.

\end{description}
