\chapter{Glossary}

% Last updated: 20200510

% FIXME: Add Semibluff
% Crapshoot, Ante, Showdown Value, Card dead
% Pot Committed

Poker has its own language, it applies special meaning to some
common English words and also has some terms all of its own. You
don't need to know all the jargon but I've used the following terms
in the book and a brief explanation here should be useful.

\begin{description}

\item[Bad Beat] A player losing the pot on the turn or the river when
he was the clear favourite.

\item[Bettor] The person making the bet. In this book I spell it ``Bettor''
and use the spelling ``better'' to mean ``preferable''.

\item[Big Blind] Both the player two left of the dealer and the amount
he has to bet before he gets any cards. It will be obvious from the
context which meaning I'm using. In the sentence ``the big blind bets
1,000'' I'm talking about a bet from that player, when I say ``he had
a stack of 5 big blinds'' I'm taking about the amount. Sometimes I
call the amount the Big Bet.

\item[Blocking Bet] A bet made by a player to try to forestall a
bigger bet. It's an attempt to set the bet size for this betting
street.

\item[Board] Another name for the Community Cards. The first three
cards make the Flop, adding the Turn makes the four card Community the
Turnboard and adding the River makes the Finalboard.

\item[Button] Both the player in the dealing position and a small
piece of plastic with ``Dealer'' written on it that indicates the
dealing position. This is the best playing position in Holdem.

\item[Cbet] A Continuation Bet, a bet made on the flop by the same Bettor
who raised preflop.

\item[Cbluff] A Continuation Bet you make when you're bluffing, you're
hoping to win the pot right now.

\item[Cooler] A big hand where a player loses a lot of his chips with
a big hand, to an even bigger hand. These hands often see a Go-Broke
hand run into the Nuts.

\item[Cooler Card] A board card, on the turn or river, that makes a
single card straight or flush possible. This card usually cools down
the betting from Made Hands, who are now afraid they've been outdrawn.

\item[Counterfeited] When your two pair hand is ruined by an unrelated
board pair. If you hold 76 on a board of AJ76 and the river comes a J,
you've been counterfeited.

\item[Crying Call] A call you make on the river where you expect to
lose the pot (and start crying!). Just to be 100\% sure, you make the
crying call.

\item[Current Nuts] The Current Nuts is the best possible hand, at a
stage of a deal before the Finalboard. Preflop the current nuts is
Pocket Aces, on the flop, top set is often the current nuts.

\item[Cutoff] The player who acts immediately before the Button.

\item[Dominated] A player with the same high hole card as his opponent
but a lower side card is said to be dominated and will usually be
only a 30\% chance to win a showdown against that hand.

\item[Downstream] Players who act after you are downstream.

\item[Finalboard] The Five-card Board once the River has been
produced.

\item[Flop] The first three Board cards.

\item[Flopstreet] The round of betting immediately after the flop is
produced. This is the second of four betting rounds. I made this word
up but I like it a lot. Some veteran players who've played a lot of
Seven Card stud call this betting round ``Third Steet''

\item[Heads Up] A hand with just two players remaining. When a poker
tournament is down to the last two players, all remaining hands are
Heads Up.

\item[Hole Cards] The two private cards unique to each player.

\item[Hollywood] To put on a deliberate act, like you're in a
Hollywood film. I sometimes Hollywood my folds but rarely anything
else.

\item[Live Cards] A player who can win a showdown by pairing one of
his hole cards is said to have Live Cards. In an AK v J7 matchup, the
player with J7 can win by pairing his Jack or his Seven.

\item[River] The fifth and final Board card. Once the River has been
shown, the five-card Board is the Finalboard and the final round of
betting takes place.

\item[Set] A set is a three of a kind hand where you hold two of the
three in your own hand. Your hole cards are a pocket pair which
matches one of the board cards. When you hold 66 and the flop is A63,
you've flopped a set of sixes. See \textbf{Trips}.

\item[Sit and Go] A single table No Limit Holdem tournament, usually
with 8 to 10 players starting and the top 2 or 3 players
paid. Sometimes a Sit and Go is a 10 player winner-take-all
first-prize-only satellite to a bigger game, with an entry fee of a
bit over 10\% of the bigger prize. For example you play a 10-person
Sit and Go for \$60, the only prize of which is an entry to a \$550
tournament.

\item[Scare Card] A card on the turn or river that makes a draw hit.
With two diamonds on the flop, a diamond on the river is a scare card.
A Coolear Card is an even more scary Scare Card, making powerful
Single-Hand hands possible.

\item[Small Blind] Both the player one left of the dealer and the
amount of money he has to bet before he gets any cards. See
\textbf{Big Blind}. Sometimes because of a player bustout or player
move there's no small blind posted; in every hand of poker there is at
least a Big Blind preflop.

\item[Tank] To tank is to take a long time over the decision
(you give it a lot of consideration, you go into the Thinktank).
Once you make your decision, you've tank-folded, tank-called, tank-bet,
tank-raised or tank-reraised.

\item[Trips] Trips is a three of a kind hand where you hold one of the
three in your hole cards. If you have A6 and the flop is 766, you've
flopped trip sixes. See \textbf{Set}.

\item[Turn] The fourth board card. It is produced after the Flopstreet
betting round is complete.

\item[Turnboard] The four-card Board after the Turn has been shown.

\item[Turnstreet] The round of betting immediately after the turn is
produced. This is the third of four betting rounds. This is one
of my invented terms. Some veteran players who've played a lot of
Seven Card stud call this betting round ``Fourth Steet''

\item[Upstream] Players who act before you. Remember to watch them
preflop before you even look at your own hole cards.

\end{description}
