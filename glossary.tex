\chapter{Glossary}

% Last updated: 20200609

% FIXME: Pot Committed

Poker has its own language, it applies special meaning to some
common English words and also has some terms all of its own. You
don't need to know all the jargon but I've used the following terms
in the book and a brief explanation here should be useful.

I read \citep{AllanMackay} while preparing this glossary; not to steal
their work but to make sure I'd left nothing out. There's some good
Gambling glossaries and dictionaries on the Internet; use ``poker
terms'' in a Google search and you'll find plenty of them.

\begin{description}

\item[Ante] A small forced bet each player puts in before receiving
cards, in the late rounds of a tournament. Not all tournaments use
Antes, and some games use the Big-Blind Ante format.

\item[Ace Magnets] Pocket Kings. When you have this holding, an Ace is
magically attracted to the flop, to make this hand tough to play for
you.

\item[All-in] When you've bet all of your chips, you're All-in. The
phrase All-in is now used in writing to mean ``fully committed''.

\item[Backdoor] A hand made using both the turn and river cards. Also
called Runner-Runner.

\item[Bad Beat] A player losing the pot on the turn or the river when
he was the clear favourite.

\item[Bettor] The person making the bet. In this book I spell it ``Bettor''
and use the spelling ``better'' to mean ``preferable''.

\item[Big Blind] Both the player two left of the dealer and the amount
he has to bet before he gets any cards. It will be obvious from the
context which meaning I'm using. In the sentence ``the big blind bets
1,000'' I'm talking about a bet from that player, when I say ``he had
a stack of 5 big blinds'' I'm taking about the amount. Sometimes I
call the amount the Big Bet.

\item[Big Game] A game with a sizeable entry fee; \$80 or more.

\item[Big Slick] The hole cards Ace-King.

\item[Blank] A Blank is a turn card or river card that doesn't
complete any drawing hands. Also called a Brick or a Rag. On a flop of
JT6 two diamonds, a 3 of clubs on the turn is a Blank.

\item[Blocker] A card you can see in your own holding, that lessens
the chance that someone else makes a powerful hand. If you've got TT
on a QJ9 flop, you don't think anybody has a straight because you've
got two of the required Tens yourself.

\item[Blocking Bet] A bet made by a drawing player to try to forestall
a  bigger bet. It's an attempt to set the bet size for this betting
street.

\item[Board] Another name for the Community Cards. The first three
cards make the Flop, adding the Turn makes the four card Community the
Turnboard and adding the River makes the Finalboard.

\item[Board Pair] A turn or river card that makes a pair on the
board. Also called a Repeater.

\item[Broadway] The highest possible straight, AKQJT.

\item[Bubble] The finishing position just outside the money payments.
If seven places are paid, the eight place finisher has ended On the
Bubble, and gets the same payout as the first busted out player. Zero.

\item[Button] Both the player in the dealing position and a small
piece of plastic with ``Dealer'' written on it that indicates the
dealing position. This is the best playing position in Holdem.

\item[Card Dead] A player who is being dealt trash holdings in every
hand is said to be Card Dead.

\item[Cards Speak] The stipulation that a tabled hand at showdown
plays for the best possible hand that can be made from it. If a player
mistakently declared his hand as a pair but it can actually make a
straight, it plays as the straight.

\item[Case Card] The fourth and final card of a rank in the deck. If
you've got AA and you're all in on a flop of KK6 and the Enemy has AK,
you will only win if the turn or the river is the Case Ace.

\item[Cbet] A Continuation Bet, a bet made on the flop by the same Bettor
who raised preflop.

\item[Cbluff] A Continuation Bet you make when you're bluffing, you're
hoping to win the pot right now.

\item[Cheapstack] A Deepstack tournament with a low entry fee. This
sounds more exciting than it really is; it can take a long time for
players to lose and there's not much prizemoney at the end.

\item[Check] The action of not betting, in a betting round where
nobody else has bet and there's no bet to match to stay in the hand.

\item[Checking it Down] The behaviour of active players when another
player is all-in. The active players often check the later betting
streets, to keep everyone in the pot and maximise the chances that the
all-in player will lose the showdown and be eliminated.

\item[Complete Hand] A complete hand is a five-card poker hand where
you're using all five of your cards. Your hand doesn't have a kicker
in it. The Complete Hands are Full Houses, Flushes and Straights.

\item[Cooler] A big hand where a player loses a lot of his chips with
a big hand, to an even bigger hand. These hands often see a Go-Broke
hand run into the Nuts.

\item[Cooler Card] A board card, on the turn or river, that makes a
single card straight or flush possible. This card usually cools down
the betting from Made Hands, who are now afraid they've been outdrawn.

\item[Counterfeited] When your two pair hand is ruined by an unrelated
board pair. If you hold 76 on a board of AJ76 and the river comes a J,
you've been counterfeited.

\item[Crapshoot] A no-skill luck situation. This is said either
because the blinds are now so big that short term high-card luck will
determine the winner; or you're playing in such a huge field
tournament that you expect someone else to get lucky.

\item[Crying Call] A call you make on the river where you expect to
lose the pot (and start crying!). Just to be 100\% sure, you make the
crying call.

\item[Current Nuts] The Current Nuts is the best possible hand, at a
stage of a deal before the Finalboard. Preflop the current nuts is
Pocket Aces, on the flop, top set is often the current nuts.

\item[Cutoff] The player who acts immediately before the Button.

\item[Dead Money] A player with no chance of winning the tournament.
He's paid his entry fee and his money is in the prizepool but he's
never going to be winning any money himself.

\item[Deepstack] A tournament that starts with stacks of over 100 Big
Blinds. Stacks of 20,000 and 30,000 with starting blinds of 25/50 or
50/100 are Deepstack tournaments.

\item[Dominated] A player with the same high hole card as his opponent
but a lower side card is said to be dominated and will usually be
only a 30\% chance to win a showdown against that hand.

\item[Downstream] Players who act after you are downstream.

\item[Drawing Hand] A hand on the flopstreet or turnstreet that has
the potential to develop to a complete hand, such as a Full House,
Flush or Straight.

\item[Dry Flop] A widely spaced Rainbow flop where top set is the best
possible hand and there's no open-ended straight draws or flush
draws.

\item[Equity] Equity is an estimation of how much of the pot belongs
to you, before the pot is over. If you're a 60\% chance of winning a
pot of 3,000 chips, your equity is 1,800 chips; plus any Fold Equity
you have.

\item[Fifth Street] A Seven Card stud name for the round of betting
after the river. All five board cards are now showing. I call this betting
round the Finalstreet.

\item[Finalboard] The Five-card Board once the River has been
produced.

\item[Fixed Limit Poker] A Poker game where the betting size is
fixed. This used to be called simply Limit Poker but because so much
of Poker is No Limit; I now call it Fixed Limit Poker to really
emphasise that I'm talking about a Fixed Limit game.

\item[Flop] The first three Board cards.

\item[Flop Game] A poker game involving shared Community Cards that
are revealed in three phases; a three card Flop, a fourth card Turn
and a final card River. Holdem, Pineapple and Omaha are all Flop
Games.

\item[Flopstreet] The round of betting immediately after the flop is
produced. This is the second of four betting rounds. I made this word
up but I like it a lot. Some veteran players who've played a lot of
Seven Card stud call this betting round ``Third Steet''

\item[Fold] To give up your cards and finish your involvement in this
hand. You lose whatever chips you've contributed to the pot so far.

\item[Fold Equity] Fold Equity is the additional value you have in a
hand through being able to get the other players to fold with a bet or
a bluff.

\item[Fourth Street] A Seven Card stud name for the round of betting
after the turn. Four board cards are now showing. I call this betting
round the Turnstreet.

\item[Freeroll] A game with no buyin fee.

\item[Free Card] A turn or river card that active players get to see
for free, because everyone checked on the betting round before it.

\item[Freezeout] Originally this term meant a poker tournament with
rising blinds that continued until someone had all the chips. A
Satellite where the top three players win Quarterly Final tickets and
the fourth player wins \$150 is not a Freezeout, action stops once the
tournament is three-handed though the final three can play on to find
a third/second/first place if they really want to. These days
Freezeout means a single-entry-fee Poker tournament; with no Rebuy or
Add-ons.

% In Rounders the first game between Mike and Teddy KGB is simply a Cash
% Game; once Teddy busts Mike the game is over because nobody else in
% the game has many chips and Mike can't rebuy for the minimum
% \$25,000. The final game between Mike and Teddy is a Cash Freezeout;
% we don't stop until one of us has it all.

\item[Full Bet Rule] The rule that an all-in bet that is less than a
full raise does not re-open betting to people who have already bet to
the old amount.

\item[Full House] A Five Card poker hand featuring both
three-of-a-kind and a pair. In a matchup between two Full Houses, the
higher three of a kind wins; if they're the same then the higher pair
wins. Consider the finalboard KKQ77 and players with the holdings K5,
KQ and J7. All of the players have Full Houses. KKKQQ beats KKK77
beats 777KK.

\item[Go to War] Bet and raise aggressively.

\item[Heads Up] A hand with just two players remaining. When a poker
tournament is down to the last two players, all remaining hands are
Heads Up.

\item[Heater] A hand where you make lots of chips. The opposite of a
Cooler.

\item[Holding] Your two private cards you get to hold as long as
you're involved in a hand. Another name for your hole cards.

\item[Hole Cards] The two private cards unique to each player.

\item[Hollywood] To put on a deliberate act, like you're in a
Hollywood film. I sometimes Hollywood my folds but rarely anything
else.

\item[Limit Poker] A Poker game where the betting size is
fixed. Because so much of Poker is now No Limit; I use the term
Fixed Limit Poker to really emphasise the point that I'm talking about
a Limit game.

\item[Limp] To match the big blind amount in the preflop betting
round.

\item[Live Cards] A player who can win a showdown by pairing one of
his hole cards is said to have Live Cards. In an AK v J7 matchup, the
player with J7 can win by pairing his Jack or his Seven.

\item[Lock] Old-time name for the Nuts. If you have a lock on a pot,
there's no way you can lose.

\item[Miscall] To declare the wrong hand at the showdown. This should
only ever be done accidentally.

\item[Misdeal] A mistake in the dealing, that forces the cards to be
reshuffled and a new hand dealt.

\item[No Limit Poker] A Poker game where a player can bet all of his
chips on the table when the action is on him, subject only to the Full
Bet Rule.

\item[One time!] A plea from a player to get lucky on this hand.

\item[Overcard] A hole card of yours that's bigger than any of the
cards on the community. When you hold A8 and the flop comes T75,
you're holding one Overcard. If you've got KJ on that flop of T75,
you've got two overcards, or ``two overs''.

\item[Preflop] The first betting street, before the flop is produced.
All you've got to go on is your own two hole cards and your estimation
of the strength of the other player's hole cards.

\item[Pot Builder, Pot Sweetener] A preflop bet that you expect will
be called by a few people; making a bigger pot to bet at on the
Flopstreet.

\item[Rainbow] All different suits; there is no flush
draw going to the next card. A flop is rainbow when there's three
different suits. A Turnboard is rainbow when there are four different
suits on it. A Finalboard is rainbow when there is not
three or more of any suit on it. If a turnboad has two diamonds, a
spade and a heart it isn't rainbow, but when the river comes a spade,
the Finalboard is rainbow.

\item[Repeater] A turn or river card that makes a pair on the
board. Also called a Board Pair.

\item[River] The fifth and final Board card. Once the River has been
shown, the five-card Board is the Finalboard and the final round of
betting takes place.

\item[Satellite] A Poker tournament where the prizes are entries into
another, bigger Poker tournament.

\item[Set] A set is a three of a kind hand where you hold two of the
three in your own hand. Your hole cards are a pocket pair which
matches one of the board cards. When you hold 66 and the flop is A63,
you've flopped a set of sixes. See \textbf{Trips}.

\item[Semibluff] A semibluff is a bet or raise with a drawing hand on
the flopstreet or the turnstreet. You don't have the best hand right
now, so your betting action is a bluff, however if you catch the right
board card your hand will develop into what you expect to be the best
hand. Typically a semibluffing hand is a straight draw, a flush draw,
or two overcards.

\item[Sit and Go] A single table No Limit Holdem tournament, usually
with 8 to 10 players starting and the top 2 or 3 players
paid. Sometimes a Sit and Go is a 10 player winner-take-all
Satellite to a bigger game, with an entry fee of a
bit over 10\% of the prize value. For example you play a 10-person
Sit and Go for \$60, the only prize of which is an entry to a \$550
tournament.

\item[Scare Card] A card on the turn or river that makes a draw hit.
With two diamonds on the flop, a diamond on the river is a scare card.
A Coolear Card is an even more scary Scare Card, making powerful
Single-Hand hands possible.

\item[Showdown Value] A hand on the final board that you'd like to see
a Showdown with. It won't often win (your hand is something like
middle pair decent kicker) but sometimes it will. It's not as strong
as a Value Bet hand.

\item[Small Blind] Both the player one left of the dealer and the
amount of money he has to bet before he gets any cards. See
\textbf{Big Blind}. Sometimes because of a player bustout or player
move there's no small blind posted; in every hand of poker there is at
least a Big Blind preflop.

\item[Small Game] A game with a small entry fee; \$30 or less.

\item[Stud] A different poker game where players get their own hands,
with visible upcards and private downcards. Seven Card Stud is the
enduring stud game, where players finish with three downcards and four
upcards.

\item[Suck bet] A small bet from a powerful hand, trying to look so
weak that you might even bluff raise it. A suck bet is about the same
size as a Blocking Bet.

\item[Take one off] To call a bet on the flop, hoping to immediately
improve your hand on the turn. If the turn doesn't improve your hand,
you expect to fold to any further betting action. Old-time players say
``calling one time'' when doing this; but the phrase one time is now
used by players asking for immediate good luck on this one crucial
hand for them.

\item[Tank] To tank is to take a long time over the decision
(you give it a lot of consideration, you go into the Thinktank).
Once you make your decision, you've tank-folded, tank-called, tank-bet,
tank-raised or tank-reraised.

\item[Third Street] A Seven Card stud name for the round of betting
after the flop. Three board cards are now showing. I call this betting
round the Flopstreet.

\item[Trips] Trips is a three of a kind hand where you hold one of the
three in your hole cards. If you have A6 and the flop is 766, you've
flopped trip sixes. See \textbf{Set}.

\item[Turn] The fourth board card. It is produced after the Flopstreet
betting round is complete.

\item[Turnboard] The four-card Board after the Turn has been shown.

\item[Turnstreet] The round of betting immediately after the turn is
produced. This is the third of four betting rounds. This is one
of my invented terms. Some veteran players who've played a lot of
Seven Card stud call this betting round ``Fourth Steet''

\item[Upstream] Players who act before you. Remember to watch them
preflop before you even look at your own hole cards.

\item[Value Bet] A bet on the river with the probable best hand. You
expect the bet to be called and to be the winner of the pot.

\item[Wet flop] A co-ordinated flop where straights and flush draws
are possible. JT9 two hearts is a Wet flop.

\item[Wheel] The lowest possible straight, 5432A.

\end{description}
