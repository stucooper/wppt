\chapter{Fast Tournaments}

% Last updated: 20190611

What makes a \$10 buyin pub game different from the Main Event of
the World Series of Poker? The buyin is \$10 not \$10,000 and there's 30
players not 5,000. The game will finish in 4 hours, not 6 days. You get
5,000 chips to start with, not 20,000\footnote{The World Series of Poker
used to give you a start stack equal to your buyin. For a \$1500 event,
you got a start stack of 1500 chips. The 2005 WSOP Main Event
was the last Main Event to have a start stack of \$10000. In
the small 1500 chip events the first three levels were 25/25, 25/50
and 50/50 so people got a few rounds of play even though they're
starting with just a 60 BB stack.}. The blinds increase every 15
minutes, not every 2 hours.

\section{Stack size}

Always look at your stack as a multiple of the big blind.
At the start of a tournament you have 5,000 in chips with blinds
at 25/50. That's 100 big blinds. You could take two flops, call a raise to
400 with \sixh\sixs\ , miss the flop and fold to action and your
chip stack is now 4,600, 92 big blinds. No need to do the exact
maths, 4500/90 Big Bets is close enough. You can still take some flops
and play some poker.

But the blinds are going up every 15 minutes. You fold \As\eigh\ under
the gun (good play) then the next hand blinds are up to 50/100.
Suddenly your stack of 4,500 has been halved from 90 big blinds to 45
big blinds. And there's one guy at the table with a stack of 11,000 after
busting out another player by flopping a low set. He's got 110 big blinds.

To repeat myself, count your stack in relation to the size of
the big blind. If the blinds are 100/200 and you have 4,800 chips, you
have 24 big bets. The more big bets you have, the better off you are,
and the more deep stack poker you can play.

Your stack is getting eaten away and devalued, both by pots you've entered
and lost, and by increasing blinds where your power compared to the big bet
is diminished. Your stack is only replenished when you win pots, so you've
got to take your chances to win pots and stay ahead of the other players.

Remember that the minimum bet is the big blind. In our 5,000 chip
game, you might find yourself with 2,500 chips as the blinds reach 300/600.

You've lost only half your start stack of 5,000, but by this
stage of the game you have just four bets. You've lost 50\% of your
starting chips (5000 to 2500) but 96\% of your chip power (100BB to
4BB). If you push all in preflop it's only just bigger than a standard
raise. The quickly rising blinds have crippled your stack to the point
where you can only just afford to make a standard raise, with an
all-in bet.

These quickly rising blinds are what makes the tournaments so fast.
The start stack doesn't matter so much, whether it's 4,000 or 5,000
or 6,000 (though 10,000 is noticeably big).

Because you don't get dealt many hands per level, and the blinds are
continually increasing to devalue your stack, you have to play super fast
in these poker tournaments.

\section{Small Game Structure, 5,000 start stack}

Here's a pretty standard structure for a small fast pub tournament
in the APW Era.

Game: No Limit Holdem. Freezeout. \\
Start stack: 5,000 in chips. \\
Registration: 6:30pm for 7:30pm start. \\
Buyin: \$10 \\
Guarantee: \$300, First: \$150 Second: \$100 Third: \$50. \\
Tournament pays down to 10\% + 1 if more than 30 players.

\begin{tabular}{|l|l|l|} \hline
BLINDS  &  DURATION   & TIME \\ \hline
25/50   &  15 minutes & 7:30 \\ \hline
50/100  &  15 minutes & 7:45 \\ \hline
100/200 &  15 minutes & 8:00 \\ \hline
15 MIN BREAK &        & 8:15 \\ \hline
200/400 &  15 minutes & 8:30 \\ \hline
300/600 &  15 minutes & 8:45 \\ \hline
400/800 &  15 minutes & 9:00 \\ \hline
500/1,000 & 15 minutes & 9:15 \\ \hline
15 MIN BREAK  &        & 9:30 \\ \hline
1,000/2,000 & 15 minutes & 9:45 \\ \hline
2,000/4,000 & 15 minutes & 10:00 \\ \hline
3,000/6,000 & 15 minutes & 10:15 \\ \hline
4,000/8,000 & 15 minutes & 10:30 \\ \hline
5,000/10,000 & FIXED & 10:45 \\ \hline
\end{tabular}

A quick word on the two 15 minute breaks. During the first break,
the Tournament Director moves to the chip stacks and removes
everyone's 25 chips. He rounds up players 25 chips to an amount
in 100 chips. Now the 25 chips are out of play, and bets will be
in multiples of 100. At 200/400, the smallest bigger-than-minimum
raise possible to under the gun is 900.

In the second break, the TD removes everyone's 100 chips, rounding up
to the nearest 500. Now the 100 chips are out of play, and bets
will be in multiples of 500. At 1000/2000, the smallest
bigger-than-minimum raise possible to under the gun in 4500.

Tonight 33 people entered this tournament. That's 33 x 5,000 = 165,000
chips in play total. With \$330 in the prize pool, the payouts have been
adjusted to \$160/\$100/\$50/\$20.

When it gets to Heads Up there is a 90,000 stack
playing a 75,000 stack, but the blinds are already at 2,000/4,000.
Looking at stacks in terms of big bets as we always do,
it's 22 BB against 16 BB. Not a lot of room for deepstack
poker here.

This is how a pub can start a tournament at 7:30pm and have it over
by 11pm, consistently.

Some games have a 150/300 level, and I've also seen 75/150. It'd be
a generous tournament where both these levels are played. Most games
have a 700/1,400 or 800/1,600 level between 500/1,000 and 1,000/2,000. Feel
free to ask the tournament director for the tournament structure before
the game commences.

Because the blinds go up so fast and your chips are becoming less and
less powerful, you need to take chances to win chips early and often.
The best way to do this is to avoid slowplaying your strong but
vulnerable hands. Bet your sets all-in in the face of flushing boards.
If you're called and drawn out on, so be it, you were a good favourite
and the cards were against you this time. Most of the time you'll double
up and you'll be the one with the big stack.

Rocks can't survive in a fast tournament. The good hands don't come
around often enough. When they do get \Kh\Kc\ their stack is down to 2,200 at
the 300/600 level and they get called by \As\Qs\ who gets an Ace on the flop.
It happens. Even when their hand holds they've now got 5,000 and only have
8 BB. They're still under extreme pressure.

Once you see the effects of the rising blinds on your stack size
(expressed in terms of the big bet) you'll see why you have to move
fast always. The early stages of the tournament are not about survival,
they're about getting more chips. You want the stack of 40 big bets facing
the stack of 5 big bets, you don't want the stack of 5 big blinds. You
have to win big and lose small, steal pots and stay ahead of the others.
It's a tough ask.

This is why you've got to make Position Plays over and over again.
You just won't get good hands often enough. You'll get \Js\Jd\ or better
maybe 1 in 50 hands, and those Jacks are no guarantee of a double up.
You'll get dealt maybe 8 hands in 15 minutes.
By the time those Jacks arrive, you've at 200/400 and your 2,200 stack is
a slightly oversized bet. No wonder everyone's always grumbling that
they haven't got a hand all night! They started with 100 big blinds,
they've seen only 25 hands in two hours and their best hand all
night has been \Js\nines\ that completely missed the flop.
They finally get \Js\Jd\ and the best they can hope for is a double
up to eight big bets.

What you do get often enough is position. Every 8 hands (at a round table)
and every 10 hands (at an oval table) you're on the Button. Position
is power in Holdem. You've got to use it.

\section{White Chip Poker}

Here's a game I've invented to help you think about stack sizes.
In this game, there's only one chip colour: white. The game is
still played No Limit, but when you start you simply get 50 white chips.
There's two blinds: the small blinds is 1 white chip and the
big blind is 2 white chips.

Every 15 minutes the Tournament Director comes
around and takes either half, a third, or a quarter of everyone's
chips (rounded down). People grumble good naturedly when this happens.
The TD takes these chips back to the registration area and locks them away
in the chip case. The small blind and big blind stay at 1 chip/2 chips.

In White Chip Poker, you've never got that many chips to play with.
Your small chip stack gets eaten away by losing pots and by the
Tournament Director taxing the stacks every so often. You have to
get in there and fight for more chips before they get taxed again.
It's war.

You can see where I'm going with this. The TD taking half of
everyone's chips in White Chip Poker is a blind double in a normal game.
Taking a third or a quarter of a stack is a more gentle blind increase,
such as the blinds moving from 200/400 to 300/600 in a normal chip game.
Using only white chips helps you count your stack in terms of the big bet.
Stacks are big when they are a big multiple of the big bet, not when
they're some big number.

If you tell me you have a stack of 16,000 in a tournament, you've
told me nothing. If you have 16,000 from a starting stack of 5,000
and the blinds are at 100/200, that's a fantastic start for you, you
have 80 big bets. If you have 16,000 and there's twelve
players left and the blinds are at 10,000/20,000, you're probably
not going to win this tournament, you can't even complete the big blind.

Not all blind increases are doubles. An increase from 200/400 to 300/600
devalues a stack by 1/3. The increase from 300/600 to 400/800 devalues
a stack by 1/4. The increase from 400/800 to 500/1,000 is the gentlest
in my structure, and devalues a stack by 1/5.

You don't need to know these numbers (although I found it mathematically
pleasing to figure them out), the key issue is that the blinds are going up
and up and you need to keep winning chips as fast as you can.

%% \section{Pretend Deepstack Games}

%% Some poker games call themselves Deepstack but have very fast blind
%% increases and end up just as fast as a 25/50, 4,000 start stack game.
%% I've seen games start with 20,000 chips get short in the fifth level.
%% These games look Deepstack but don't be fooled by them.
