\chapter{Gambling Theory}

% Last updated: 20200425

% FIXME: Shorten chapter just a bit as just over 4 pages

This chapter covers some gambling maths that is well worth knowing.

\section{Expected Value}

The Expected Value (``EV'') of a bet is how much you expect to win or lose
each time you make that bet. This quantity is also called the
\textbf{expectation} but we'll simply use EV (everyone else in poker
calls it EV so we will too). It's calculated as the probability of
winning times amount won, minus the probability of losing times amount
lost. Probability in Maths is a decimal between 0.00 (no chance)
and 1.00 (certain). A lot of the probabilties we will see in Poker
are expressed as percentages, it is trivial to turn these into a decimal
50\% is 0.5, 18\% is 0.18 and so on. Divide a
percentage by 100 to get the decimal, multiply the decimal by 100 to
get the percentage.

The Probability of winning I call P(win) and the Probability of losing
I'll call P(loss). Expected Value is defined by the equation \\
EV = P(win) x (amount won) - P(loss) x (amount lost)

The amount won doesn't count your stake; if you bet \$10 at \$1.70
then the amount won to use in the formula is \$7.00. The amount lost
is your stake.

Consider flipping a coin against your friend
for \$5. Half the time you win and make \$5. The other half of the
time you lose and lose \$5. You multiply the probability of winning
by how much you'll win, and subtract the probability of losing times
how much you'll lose.

The EV of the \$5 cointoss is (0.5 x \$5) - (0.5 x \$5)
= \$2.50 - \$2.50 = 0. A bet with an expected value of 0 is
paying true odds. A bet with an expected value less than 0 is
paying under the odds, and a bet with an expected value more than
0 is paying over the odds. Under the odds bets are said to be
``negative expectation'' and are written -EV (pronounced ``minus Eee Vee'').
Over the odds bets are ``positive expectation'' and are written +EV.
(``plus Eee Vee'').

The higher your EV number is above zero, the better the bet is for
you. The lower your number is below zero, the worse the bet is for
you. If you make a bet or a call with a negative EV, you're making an
\textbf{EV Mistake}.

\subsection*{Sports Bets}
Sports bets pay under the odds because the betting agency take out a
percentage of the pool as their profit for running the game. A
football bet market (Rugby League, Rugby Union or AFL) gives 6.5
points start to one team and have both teams paying \$1.90. If two
people each bet \$100 on different sides in this game, one wins and
the other loses. One player gets \$190, winning \$90 and the other
player lost \$100. The betting agency held \$200 on the game and kept
\$10 of it, paying the \$190 out to the winning player. If they can
balance the bets, they win no matter what. This is a house cut of 5\%,
because the \$10 that they keep is 5\% of the \$200 that they held from
the two players. From the player's point of view:

EV = P(win) x (amount won) - P(loss) x (amount lost) \\
   = (1/2) x \$90 - (1/2) x \$100  = \$45 - \$50  \\
   = -\$5

In the bad old days
the TAB paid \$1.85 on fifty-fifty bets, keeping a sizeable 7.5\% for
themselves. They've improved with greater competition and now pay
\$1.87 to \$1.95 for fifty-fifty bets. Online betting sites pay a bit
more and return \$1.90 to \$1.95 or more for 50-50 bets. Sometimes as
a promotion an Online site will pay \$2.00 on both sides of a
fifty-fifty bet. This is normally done to attract new customers to the
site; who then find that their future fifty-fifty bets only get odds
of \$1.87 to \$1.95. Making a fifty-fifty bet that returns \$2.00 feels like
playing in a rake-free poker tournament.

% FIXME: check the numbers for even money sports bets these days

Notice the winning bettor at \$1.90 wins less than his stake,
on a footy game considered a 50-50 chance. If he makes 10 of these
bets over time and wins five and loses five, he's not even. He's lost
\$500 on his losing bets and won \$90 x 5 = \$450 on his winning bets.
Overall he's down \$50- that \$50 hasn't been paid to winning players,
it's the house cut that the betting site takes for running the game.
He's effectively paying \$5 on every \$100 he bets, win or lose.
Five percent.\footnote{Another way of seeing this is imagining that the
site removes \$5 from his \$100 bet, five percent, and he is now
playing with \$95 at the true fifty-fifty odds of \$2.00 to make \$190.}

The amount of money a casino or betting site keeps for running the
game is called the ``rake'' or the ``house cut''.
In small poker tournaments you normally pay a rake of 10\%, with the
rest of your buyin going into the prize pool.

Consider Roulette, ``pick the exact number'' bet. I'm in a casino,
I don't want to play too big but I'll try a number at Roulette for a few
spins, betting \$10 each time. If a different number comes up I lose \$10,
if I score the right number I get \$360 back, winning \$350. There's two
zeroes on the table and the numbers 1 to 36, so I have a 1 in 38 chance of
winning.

EV = P(win) x (amount won) - P(loss) x (amount lost) \\
   = (1/38) x \$350 - (37/38) x \$10 \\
   = (350/38 - 370/38)  \\
   = \$9.21 - \$9.74 \\
   = -\$0.53

Every \$10 bet at roulette I expect to cost me 53 cents. This is a 5.3\% house
edge-- they are paying under the true odds of picking the right
number.

Another way of looking at this is seeing the cost if you backed every
outcome. It costs you \$380 to put \$10 on each of the 38 numbers, and
you know without even watching the ball drop that you will have \$360
afterwards. You've lost \$20 of the \$380 you put on, which as a
fraction is 1/19 which is 0.053 which as a percentage is 5.3\%.
Returning to that 1.90 50\% chance sports bet, if I back both sides for
\$100 each, I'm guaranteed a \$190 return on my \$200, costing me \$10
of the \$200 I put on. Each of my \$100 bets cost me \$5, I'm down 5\%
on my total.

% We'll be using exactly the same Expected Value calculation in poker. In
% those calculations the amount won will usually be the pot we are calling a bet
% into and the amount lost will usually be the amount we have to call.

EV is a hard concept to grasp at first, because when you actually
play, you'll find that every bet at Roulette costs you \$10, not
53 cents. You can't sit at a Roulette Table and
have Ten \$10 shots at picking the right number and walk out \$5.30
down. You walk out \$100 down.\footnote{Trust me on this.} But if you
play thousands and thousands of spins, and get your fair share of
wins, every \$10 spin (including the winning spins) costs you 53
cents.

%% That 5.3\% is the house edge, and it works on every spin. The little 5.3\%
%% edges add up over time and over spins and they pay for the staff, the
%% security, the equipment and the nice casino profits. They pay for the
%% subsidised meals at the cafe and the lucky member's draw. That plus
%% all the Poker Machines.

In a typical spin on a roulette table there's a few thousand dollars
spread across the table. The casino's making 5.3\% on the exact number
bets and various other percentages on the other kinds of Roulette Bets.

Run a few spins in an hour, sit back and watch the money roll in.

\section{Variance}

In gambling you win or lose on a single outcome of an event,
not the long term EV. The short term luck of a single outcome
is what counts.

Online poker players talk about results over thousands of hands. Because the
dealing online is fast and flawless, and because they can play at
more than one table at once, active online players can rack up
20,000 hands of poker a month. At these volumes they expects their
results to be pretty close to the EV of their bets.

Variance is the difference between the short term result and the
expected long term outcome.

Imagine a rich deluded friend of mine is convinced \nineh\nined\ is a better
starting hand in Holdem than \Jc\Js\ . ``Black Jacks never win'' he grumbles
``and Red nines always hits their set''. He wants to play me one hand, he
gets the \nineh\nined\ and I get the \Jc\Js\, we shuffle and deal out a flop,
turn and river and we both put \$1,000 into the pot and the best
poker hand at the end gets the full \$2,000.

Would I play him? No. I'm not comfortable playing one hand of poker
for \$1,000, even as an 80\% favourite. There's still a chance he can get
lucky on just this one hand.

My rich friend won't let me go, so I make him a counter offer.
``Let's play it this way. We'll play for \$10 each per pot, and do
it 100 times. You have the Red nines and I'll have the Black Jacks.''

Now this is a game I'll play for \$1,000. What I'm doing here is smoothing
out the variance of one single hand, by playing a smaller bet over
100 hands. Anyone can beat \Jc\Js\ with \nineh\nined\ in the course of one
hand. Nobody can do it often enough over 100 hands. Laws of mathematics
catch you.

Played over 100 hands I now expect to make close to \$600. 80 times I'll
make \$10, 20 times I'll lose \$10. I should end up \$540 to \$660 ahead
at the end of the 100 hands.

Let's look at the EV:
EV = P(win) x amount won - P(loss) x amount lost \\
   = (80/100) x \$10 - (20/100) x \$10 \\
   = \$8 - \$2 \\
   = \$6.

Every time we play this \$10 JJ versus 99 game, I expect to win \$6. That's a
massive 60\% on my investment, more than 10 times the 5.3\% casino makes on
pick the number in roulette. I just sit back and let the money roll in.

But I didn't want to play this game for one \$1,000 hand, because then
I'm at the mercy of short term blind luck. Likewise the casino won't
want to play against James Bond turning up and plonking \$10 million
on number 23 in Roulette. They protect themselves against variance
with table maximums, and they encourage a group of people to be playing,
which leads to a fairly even spread of bets across the numbers, limiting
their exposure to any particular single outcome. With a roughly even spread
of money on a roulette table, the casino makes money no matter what
number comes up.

Besides, 007's a poker player these days. Best in the Service.
He busted Le Chiffre, Felix Leiter, the giant African guy and
the sneaky Japanese guy in the big game in ``Casino Royale''.

Casino profit does actually grow and shrink based on the recent luck
of the highest of high rollers. The whales don't always lose and when
they get a good run it can cost the casino a few million dollars.

You can do EV calculations on all kinds of bets in casino games.
Strangely enough, the house has the edge every time. Some bets
are quite fair but some (e.g pick one number in Club Keno
where 20 numbers are drawn out of 80 and they pay
3 to 1 on a 4 to 1 chance) are daylight robbery!

%% Michael Konik
%% in his book ``The Man with the \$100,000 breasts'' is very
%% entertaining on the different house edges in various casino games. The
%% dice game craps (very popular in America but not in Australia)
%% is said to be the fairest Casino game with the smallest house edges.

%% Who was the man with the \$100,000 breasts? A guy who took a bet from
%% a friend that he'd get breast implants and keep them for at least a
%% year. He took that bet on, and won his hundred thousand. There's a
%% picture of him in the book with his creepy-looking breasts proudly
%% exposed. I wonder if it's non-pornographic to share that picture on
%% Facebook?

When sizing your bets in No Limit Holdem, you get to decide how much
the chasing player will be paid. He might have a 1 in 5 chance
but you've sized your bet so that the pot is only paying him 3 to 1
if he wins. In this way you get to act like a casino, with the edge
on your side. The pot will still be decided on the short term luck
of a single outcome, but you have the edge.

\section{Exercises}

% Notice I get the chapter number from a LaTeX variable, not hard coded.

\arabic{chapter}.1 There are 18 Red, 18 Black and 2 Green numbers on
a Roulette Wheel in Star City Casino. If you make a \$10 bet on Red
and win, you double your money, winning \$10. (a) What is the Expected
Value of this bet? (b) Is this bet fairer than Pick the Exact Number?

\arabic{chapter}.2 If Club Keno had a -5\% EV, how much would a \$10
exact bet return? \textit{Hint: Do the calculation in dollars not
cents, the EV to use in the equation is -0.5}

\arabic{chapter}.3 (Gambler's Ruin). A Roulette player thinks he
can make \$10 by continually doubling his \$10 bet on Red until it
comes up, then quitting \$10 ahead. If his second bet of \$20 loses,
he's down \$30 but will bet \$40 on his third bet. The table maximum
for this kind of bet is \$200. What is the chance that he will go bust
by losing his fifth bet of \$160? \textit{Hint: The probability of
losing five bets in a row is the probability of losing it once (20/38)
multiplied by itself five times, ie. P(loss) to the power of
five. Use a calculator (physical or online) and round to two
decimal places}.
