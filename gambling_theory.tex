\chapter{Gambling Theory}

% Last updated: 20181012

This chapter covers some statistical ideas from gambling in general that 
are worth knowing.

\section{Expected Value}

The Expected Value (``EV'') of a bet is how much you expect to win or lose
each time you make that bet. It's calculated as the probability of
winning times amount won, minus the probability of losing times amount
lost. Probability in Maths is a fraction between 0 (no chance)
and 1 (certain). A lot of the probabilties we will see in Poker
are expressed as percentages, it is trivial to turn these into a number
between 0 and 1. 50\% is 0.5, 18\% is 0.18 and so on.

Consider flipping a coin against your friend
for \$5. Half the time you win and make \$5. The other half of the
time you lose and lose \$5. You multiply the probability of winning
by how much you'll win, and subtract the probability of losing times
how much you'll lose.

The EV of the \$5 cointoss is (0.5 x \$5) - (0.5 x \$5) 
= \$2.50 - \$2.50 = 0. A bet with an expected value of 0 is 
paying true odds. A bet with an expected value less than 0 is
paying under the odds, and a bet with an expected value more than
0 is paying over the odds. Under the odds bets are said to be
``negative expectation'' and are written -EV (pronounced ``minus Eee Vee'').
Over the odds bets are ``positive expectation'' and are written +EV. 
(``plus Eee Vee''). True odds bets are called ``zero expectation'' 
though you don't see that terminology often.

\subsection*{Sports Bets}
Sports bets pay under the odds because the betting agency take out a percentage
of the pool as their profit for running the game. A football bet market 
(Rugby League, Rugby Union or AFL) gives 6.5 points start 
to one team and have both teams paying \$1.92. If two people each bet \$100 on
different sides in this game, one wins and the other loses. One 
player gets \$192, winning \$92 and the other player lost \$100. The 
betting agency held \$200 on the game and kept \$8 of it, paying
the \$192 out to the winning player. If they can balance the bets, 
they win no matter what.

This is a house cut of 4\%, because \$8 is 4\% of the \$200 the bookmaker
held on the game. In the bad old days
the TAB paid \$1.85 on even money bets, keeping a sizeable 7.5\% for
themselves. They've improved with greater competition and now pay \$1.87 to \$1.90 for
even money bets. Online betting sites pay even more and can pay
\$1.92 or more for 50-50 bets.

% FIXME: check the numbers for even money sports bets these days

Notice the winning bettor at \$1.92 wins slightly less than his stake,
on a footy game considered a 50-50 chance. If he makes 10 of these
bets over time and wins five and loses five, he's not even. He's lost
\$500 on his losing bets and won \$92 x 5 = \$460 on his winning bets.
Overall he's down \$40- that \$40 hasn't been paid to winning players,
it's the house cut that the betting site takes for running the game.
He's effectively paying \$4 on every \$100 he bets, win or lose.

The amount of money a casino or betting site keeps for running the
game is called the ``rake'', the ``house cut'' (or even the ``vig'',
short for ``vigorish'', a word you won't hear used very often). 
In Poker tournaments you normally pay a rake of 10\%, with the rest of  
your buyin going into the prize pool.

Consider Roulette, ``pick the exact number'' bet. I'm in a casino,
I don't want to play too big but I'll try a number at Roulette for a few
spins, betting \$10 each time. If a different number comes up I lost \$10,
if I score the right number I get \$360 back, winning \$350. There's one
zero on the table and the numbers 1 to 36, so I have a 1 in 37 chance of
winning.

EV = Probability of winning x (amount won) - Probability of losing x (amount lost) \\
   = (1/37) x \$350 - (36/37) x \$10 \\
   = (350/37 - 360/37)  \\
   = \$9.46 - \$9.73 \\
   = -\$0.27

Every \$10 bet at roulette I expect to cost me 27 cents. This is a 2.7\% house
edge- they are paying under the true odds of picking the right number.

% We'll be using exactly the same Expected Value calculation in poker. In
% those calculations the amount won will usually be the pot we are calling a bet
% into and the amount lost will usually be the amount we have to call.

EV is a hard concept to understand at first, because when you actually
play, you'll find that every bet at Roulette costs you \$10, not
27 cents. You can't sit dressed up at a Roulette Table and sip free
drinks and have 10 shots at picking the right number and walk out \$2.70
down. You walk out \$100 down. But if you play thousands and thousands
of spins, and get your fair share of wins, every \$10 spin (including 
the winning spins) costs you 27 cents. 

This 2.7\% is the house edge, and it works on every spin. These little 2.7\%
edges add up over time and over spins and they pay for the staff, the
security, the equipment and the nice casino profits. They pay for the
subsidised meals at the cafe and the lucky member's draw. That plus
all the Poker Machines.

In a typical spin on a roulette table there could be a few thousand dollars
spread across the numbers. The casino's making 2.7\% on the exact number
bets and various other percentages on all the other kinds of Roulette Bets, 
some of which have exotic names like ``Manque''.

Run a few spins in an hour, sit back and watch the money roll in.

\section{Variance}

In gambling you win or lose on a single outcome of an event,
not the long term EV. The short term luck of a single outcome
is what counts.

Online poker players talk about results over thousands of hands. Because the
dealing online is fast and flawless, and because they can play at
more than one table at once, active online players can rack up
20,000 hands of poker a month. At these volumes they expects their
results to be pretty close to the EV of their bets.

Variance is the difference between the short term result and the
expected long term outcome.

Imagine a rich deluded friend of mine is convinced \nineh\nined\ is a better
starting hand in Holdem than \Jc\Js\ . ``Black Jacks never win'' he grumbles
``and red nines always hits their set''. He wants to play me one hand, he
gets the \nineh\nined\ and I get the \Jc\Js\, we shuffle and deal out a flop,
turn and river and we both put \$1,000 into the pot and the best
poker hand at the end gets the full \$2,000.

Would I play him? No. I'm not comfortable playing one hand of poker
for \$1,000, even as an 80\% favourite. There's still a chance he can get 
lucky on just this one hand. 

My rich friend won't let me go, so I make him a counter offer.
``Let's play it this way. We'll play for \$10 each per pot, and do
it 100 times. You have the red nines and I'll have the black Jacks.''

Now this is a game I'll play for \$1,000. What I'm doing here is smoothing
out the variance of one single hand, by playing a smaller bet over
100 hands. Anyone can beat \Jc\Js\ with \nineh\nined\ in the course of one 
hand. Nobody can do it often enough over 100 hands. Laws of mathematics 
catch you.

Played over 100 hands I now expect to make close to \$600. 80 times I'll
make \$10, 20 times I'll lose \$10. I should end up \$540 to \$660 ahead
at the end of the 100 hands.

Let's look at the EV:
EV = (Probability of winning) x amount won - (Probability of losing) x amount lost \\
   = (80/100) x \$10 - (20/100) x \$10 \\
   = \$8 - \$2 \\
   = \$6.

Every time we play this \$10 JJ versus 99 game, I expect to win \$6. That's a 
massive 60\% on my investment, more than 20 times the 2.7\% casino makes on
pick the number in roulette. I just sit back and let the money roll in.

But I didn't want to play this game for one \$1,000 hand, because then
I'm at the mercy of short term blind luck. Likewise the casino won't
want to play against James Bond turning up and plonking \$10 million
on number 23 in Roulette. They protect themselves against variance
with table maximums, and they encourage a group of people to be playing,
which leads to a fairly even spread of bets across the numbers, limiting
their exposure to any particular single outcome. With a roughly even spread
of money on a roulette wheel, the casino makes money no matter what
number comes up.

Besides, 007's a poker player these days. Best in the Service.
He busted Le Chiffre, Felix Leiter, the giant African guy and
the sneaky Japanese guy in the big game in ``Casino Royale''.

You can do EV calculations on all kinds of bets in casino games.
Strangely enough, the house has the edge every time. Some bets
are quite fair but some (e.g pick one number in Club Keno 
where 20 numbers are drawn out of 80 and they pay
3 to 1 on a 4 to 1 chance) are daylight robbery! Michael Konik
in his book ``The Man with the \$100,000 breasts'' is very entertaining on the
different house edges in various casino games. The dice game ``craps''
(hugely popular in America but not in Australia) is said to be
the fairest Casino game with the smallest house edges.

When sizing your bets in No Limit Holdem, you get to decide how much
the chasing player will be paid. He might have a 1 in 5 chance
but you've sized your bet so that the pot is only paying him 3 to 1 
if he wins. In this way you get to act like a casino, with the edge
on your side. The pot will still be decided on the short term luck
of a single outcome, but you have the edge.

