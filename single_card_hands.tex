\chapter{Single-Card Hands, Double-Card Hands}

% Last updated: 20200505

In Holdem you can use both, one or none of your hole cards to make your
best Five Card hand. If you use none of your hole cards you're
\textbf{Playing the Board}, and the best you can do is split the
pot, because other players in the pot can match your best hand by also
Playing the Board, or can beat it if they can by using a better hole
card of their own. On a final board of \Ad\nined\eigd\Jd\sixd\ your
\Qc\Qs\ has made a Diamond flush, but so has everyone else in the hand,
and if anybody has a single diamond higher than the \sixd\ they can
beat the board and beat you.

Legend has it that in the early stages of the 2005 WSOP Main Event,
the final board was AKQJT rainbow (no flushes possible). Nobody can
beat that board and anyone who stayed in would split the pot. On the
river betting, one guy bet and a second guy raised. Two inexperienced
Internet guys folded, and the pot was split two ways not four ways!
I'm not sure if this is a true story or Urban Legend, but it sounds
one of those stories that are too good not to be true.

\section{Single Card Hands}

A Single Card Hand is one where you're only using one of your hole
cards to make your best poker hand.

Single Card hands of top pair or two pair are normally pretty weak;
you need your second card kicker to play and be good to have a
confident made hand. A5 on a board of AK9T6 is a single-card to pair;
sometimes you'll win but you lose to anyone else with an Ace unless
they've got the even worse A4, A3 or A2 in which case it's a split.

If you can make trips with a Single Card Hand you're in much more
powerful shape. 83 on board of K88T4 feels like the winner. There's no
straights possible on that board, there's only one more 8 in play and
it's probably still in the deck not in someone else's hand.

\subsection{Single Card Straights}

If the straight is made by a gap card in the board and you've got that
gap card, you should be the winner and only lose to an unlikely Double
Card Straight. On 9873J, your AT should win you the pot, but if
someone has QT you're toast. Someone with the Double Card straight 65
has been outdrawn; remember a Double Card hand loses its Power when
there's a Single Card hand possible.

When the board shows four-to-a-flush or four-to-a-straight, a Single
Card Power hand is possible. It's obvious to everyone in the hand that
the highest single hole card of the flush-suit will win, or someone
with a card of the right straight rank will win.

When the board is this wet, your Double Card hand loses its power and
you've often been outdrawn. Your Double Card hand is only powerful on
boards where a Single Card Hand isn't possible.

A Single Card hand is only possible after the turn or the river. If a
Single Card hand is present on the turn, the flop must have been wet
(you don't get any single card hands after a flop of K93
rainbow). Almost any flop allows a Single Card hand by the river,
given the right runout (K93 can runout K93QT making a Single-Card
straight for the Jack).

\section{Double Card Hands}

A hand where you're using both of your hole cards, in a significant
way, to make your best poker hand is what I call a \textbf{Double Card
Hand}. A Double Card Hand is stronger, and more deceptive, than a
Single Card hand. Double Card straights and Double Card flushes are
strong, Two pairs and Sets are strong. When you hold a good Double
Card hand, expect it to be the winner; except on Final Boards where
there are Single Card Power or Monster hands possible.

If you've got a high pair weak kicker, you're only using one of your
cards and you've got a Single Card weak Made hand. K6 on a KT942
Finalboard is an example; your six isn't really playing and you're
basically playing just your King for top pair no kicker.


\section{Exercises}

\arabic{chapter}.1 AKQJT rainbow is by far the most common unbeatable
board, where everyone still in the hand splits the pot. AAAAK and a
Royal Flush are also unbeatable. Is there any other unbeatable board?
