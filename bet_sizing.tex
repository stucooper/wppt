\chapter{Bet Sizing}

% Last updated: 20190702

Poker is all about betting. In No Limit Holdem, not only do you have to
decide whether to bet or raise, you also must decide by how much.

Every bet or raise should be made with a purpose. A bet
is usually made for one of three reasons: to thin the field of callers,
to extract more money from beaten hands or to get better hands to fold.
Another reason, a bit rarer but still valid, is to make the pot bigger
so if you hit your draw you can get a great payoff with bigger bets called
later in the hand. These bets are called Isolating Bets, Value Bets,
Bluffs and Pot Builders.

You have to consider a range of factors in sizing your bets. The first is
the size of the pot. The second is the size of your chip stack, and
the size of your opponent's chip stack. Think about
what you're trying to achieve in this bet, and bet the best amount
for that purpose.

% FIXME: this example started out as just 6000 stack but I want to
% rewrite it to 16,000 in chips.

Let's say the blinds are 300/600 and you've started with 16,000 in chips
and make it 2,000 to go from the Button after everyone's folded to you.
You're holding AQ. You're trying to make it look like a blind steal, but
you've got a great hand three handed and you're going to call a re-raise
all-in should anybody attempt one.

The small blind folds, the big blind thinks about it, he's already in
for 600 so he calls the 2,000 and the pot
preflop is 4,150 (2,000 x 2 + 150). The big blind has 14,000 in chips, you
have him covered.

The flop comes and it's Q63 all diamonds. You don't have a diamond, but
you have Ace Queen for top pair top kicker and you figure your hand
is the best. The big blind checks and action is on you. You figure you've
got the best hand here so how should you size your bet?

If you bet half the pot (a standard continuation bet) you're giving the
right odds to the big blind to chase a flush draw. A four flush has 9 outs
and a 36\% chance of winning. You need to bet more than 2,000 into a 4,000 pot,
because that is giving the big blind 3 to 1 to make his flush, which is
on the money.

How about a 3,000 bet? Now the flush draw has to call 3,000 into a 7,000 pot,
and is getting below the odds. You're left with 11,000 in chips. Your
opponent calls, the flush misses on the turn and now you've only got
11,000 to bet and an 11,000 pot. Your enemy is still an 18\% chance 
(1 in 5) to hit his flush, if that's the draw that he has.

It's useless leaving yourself 1,000 behind when you make the flop bet.
That 1,000 won't keep you in the action with 300/600 blinds if the
flush does hit. Your move with this chipstack on this pot is to shove
all in. Now your opponent faces a 4,000 call into a 6,000 pot, and is
getting less than 2 to 1 on his call. He's making a mistake if he calls
because he's getting less than the odds on his 36\% chance of winning.
If he doesn't have a diamond he'll be folding to this bet. If the does
have a diamond he might call and 7 out of 10 times you've doubled your chips.
Sometimes he might put you on a flush draw and call with pocket 8s no
diamond! You see some strange things in poker.

If he's already flopped the flush, too bad for you. Blame the poker gods
for giving you top pair, top kicker and your opponent the flush in
the same flop. You should be taking your chances in this hand.

By the way, you only had 6,000 chips to begin with at this 300/600 blind
hand so you only had 10 big blinds. A preflop all-in could well have
been the right move holding AQ with 900 in the blinds to win with a shove.
You'd love a call from a hand like AJ, ATs, KQ, QJ, QT when you're holding
AQ because you have all these hands dominated. Even if your opponent
has JJ or TT you've got a coin toss against him. If he looks down to his
cards to find AK, AA, KK or QQ, good luck to him. You're betting that
he doesn't have a monster hand, and you're hoping to get called by a
dominated hand, or pick up a useful 900 in blinds.

\section{Reverse Bet Sizing}

Some players do perverse reverse bet sizing. They bet tiny with
their monster hands and bet big with their bluffs. They seem terrified
that their nut hands won't make extra chips on the river, and also terrified
that their bluffs will sometimes be called.

Betting like this a surefire way of winning small pots and losing huge pots.
Exactly the opposite of what you should be achieving in poker.

A few players bet small on the flop and small on the turn, pricing every draw
in. Make sure you don't do this. Try to make every bet of yours at least
40\% of the pot, and even 40 is a little on the low side. Betting 700
or less into a 2,000 pot achieves little and costs you chips every
time you're drawn out on.

\section{Betting the nuts on the river}

Sizing your bet when you know you have the best hand is an interesting problem,
and a nice problem to have (it's always nice holding the best hand once all
the cards are out!). It's a bit like selling your car. You're disappointed to
sell it for \$2,500 if you find out later somehow that someone would have paid
\$3,000 for it. But if you advertise your car for \$4,000 you won't hear from
anyone.

Discussion and bargaining aren't available in poker. You can't value
bet 2,200 and see your opponent's going to fold, then reduce
the bet to 1,700. You priced your value bet too high.
You can't bluff 1,700, see he's about to reluctantly call, and raise
your bet to 2,200 to get him to fold. Your priced your bluff too low.

There's a few factors to consider when sizing this bet. The size of the pot, the
size of your stack, the size of your opponent's stack and the hand your
opponent is likely to have are the main elements here. You also want to know
if your opponent will pay you off a big bet-- some calling stations call just
about any pot sized bet or less, so if you're up against one of these players
you should go ahead and bet a good amount. You want to have a good sense
on whether your opponent calls big bets.

% FIXME: Write about Skalnsky Mistakes here.

How much of a lock do you have on this board? If you have KK and the
board is KTT43, you won't get much action from anybody not holding a Ten.
This is why it's good to bet your fullhouse on the turn, because you might
get a call from QJ who still thinks those cards are live, but if you wait
until the river there's no way further chips can be won from QJ.

If you hold the top flush on a four-flush board you can usually get a crying
call from the 2nd, 3rd or 4th highest flush. I'd usually bet 50-70\% of the
pot here. If I held \Jh\tens\ on a board for \Ah\tenh\nines\fiveh\fourh, I'll
make a call of 700 into a 1,700 pot most times.

If your opponent is short stacked, put him all in if you think his hand is
strong, or bet half his stack if you think his hand is marginal. This
gives him a chance to realise a call pot-commits him and throw all his
chips in anyhow.

As a poker player, there's always some hope that the magic words ``all-in''
will cause everyone else to fold, even if the all-in amount is so tiny
that the opponent is getting 8 to 1 or better for his call. So by betting
half your opponent's stack, you give him the chance to call and lose half
his stack, or raise and lose all of it.

Extreme calling stations should be bet their whole stacks. Bet a bit bigger
so it doesn't look like you've carefully counted their stacks down and you're
putting them all-in. If he's got 1,100 left, bet 1,500. Don't waste time
waiting for your change to be given if your opponent calls. Just display
your winning hand and say ``send it''.

If you've made some against-the-odds miracle hand like runner-runner
straight, or runner-runner flush, bet it big. You probably talked
yourself into some questionable calls using ``implied odds'', so now's
the time to get paid for it. An opponent should think ``he wouldn't
have called with just an inside straight draw'' and pay you off. This
is where the longshot draws can really pay off. In No Limit, there's
no limit to what you can get paid off.

The minimum I would bet with the nuts on the river is 20\% of the pot.
My opponent might have nothing and fold. That's OK. He wouldn't have
called a bet of any size. What I do in this case is remember the amount
I bet and see if next time I can bluff a pot for the same amount.

In a fast game, and doubly so in a pub game, you're more likely
to get called. People are here to play poker and see showdowns and
nobody likes folding. If their final hand is at all playable, most
players will call a bet on the river.

Always remember to bet your nut hands on the turn, to charge
the drawing hands and get some money out of them even when they're
drawing dead. You're more likely to get paid off with a 600 bet on the
turn and a 1,000 bet on the river than by checking the turn and putting
in a 1,200 bet on the river.

If I'm against an expert opponent with a lot of chips, I won't bet
20\% of the pot. He's good enough to see this is a value bet screaming
for a call. Since he's got lots of chips I'll overbet the pot and try
to make it look like a steal, maybe he'll even come right over the top
with a big bluff-raise.

Not many opponents will try a bluff-raise so I won't make this move
that often, but if I'm up against an expert with chips it's worth
considering. His hand probably isn't strong enough for a call but if he
reads me as nervous he could try a bold raising steal.

Here's a complete hand of mine from the second level
of a tournament which had start stacks of 5000.

Blinds 50/100 Board \Kd\Kh\Jd\tenh\tres. My hand: \Kc\Jc\ on the button.

Preflop: A min-Raise to 200, 5 callers including blinds, pot 1,000
Flop: \Kd\Kh\Jd. Big blind folds to no bet, 4 checkers. I'm happy to slow play.
Turn: \tenh. Great card for me, makes straights and there's flush
  draws in two suits. To get money out of the draws, I bet 500 and get
  just one caller. Bet your full houses on the turn. Actually there's
  two Royal Flush draws out there. Pot 2,000.
River: \tres, a real blank. I'd bet 500 on the turn so a bet of 400 or
  500 would look like a real milk. Unless he has a strange straight
  (AQ, Q9) I won't get paid much for this pot, although AJ and AT are
  probably going to call. I'm thinking 800 to 1,500 is my bet range
  into this pot of 2,000, I end up betting 1,200. 60 percent.

My opponent decides it's time for some probing.\\
``You got a King, Stu?'' \\
``Both of them. Four Kings is a good hand'' \\
``Sure you do. Show if I fold?'' \\
``I'll show if you call. You might be behind''. \\
``Call'' \\
``Kings full of Jacks'' \\
``Stupid Kings on the flop'' says my prober and flashes \Ah\Jh\ and folds.


Players are annoyed when their hand dominates another hand but they
still lose. Here AJ has lost to KJ. As I stack the pot, the friendly
chat continues. \\
``If the Queen of Hearts comes on the river, I make a Royal Flush''. \\
``It would have won'' \\
``They usually do'' \\
``Most draw heavy full house I've ever flopped. Royal Flush in diamonds.
Royal Flush in Hearts. I was actually pretty surprised that Kings
full of Jacks held up''.

That's enough banter, although I enjoy a good chat at the table
once a big hand is over. A new hand's just been dealt, and
it's time to get back to work.

If the river had been more flushy or straighty I would have bet more
than that 60\%, 1200 chip bet. KJ is a nut hand on a KKJT turn, and a
gobroke hand on any river, should it not still be the nuts.
if someone draws out on me with Aces full or a Royal flush or Quad
Tens they can have the rest of my chips. With the final board of KKJT3
no flush I'm unlikely to get a big bet paid off; although if someone
has a gobroke losing full house like JJ, TT, 33, KT or K3 they'll
raise my bet and I'll get the rest of their chips. If the final board
is KKJTQ or KKJTA or KKJT9 or brings a flush, I'd overbet 2500 at that
pot, and expect a call from a beaten straight and a fold from any
other hand. If the flush comes in, I'll go for a 1600 bet.

% FIXME: Add a section on Bluffing and Bluff sizing
