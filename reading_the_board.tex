\chapter{Reading the Board}

Reading the Board is a key basic Holdem skill. Most players get
pretty good at it through experience. In this chapter I'll talk about
what to think about when reading the board and some questions to ask
yourself while you're doing it.

At each change in the community, you need to know what the current
nuts is and the chances of the next card changing the current nuts.

\section{Reading The Flop}

The first trick to reading the flop is not to watch it when it's
revealed. This isn't a game of snap, there's no prize for the fastest
person to react to the flop. It will still be there five seconds later
for you to look at. Instead of looking at the flop, pick one of the
other players still active in the pot, and watch \textit{his} face as
he watches the flop revealed. Once you've done that, then have a look
at the flop yourself.

The flop is a defining change point in a Holdem hand. 60\% of the
community cards are now present and you've seen 5 of the 7 cards
you'll be able to use to make your best poker hand. Here's some things
to look for in every flop.

\begin{description}

\item[Wet/Dry] How likely is it that the final board will support a
flush or a straight? It is unlikely that anyone can finish with a
flush or straight when the flop comes K72 rainbow. But if the flop is
QJ9 two diamonds, a diamond flush could eventuate and someone could
aready have a straight. A flop that doesn't support straight draws or
flush draws is called a Dry flop, a flop that supports straights or
flushes is a wet flop.

\item[Two pair chances] Look at the three flop cards two at a time and
see if it is likely that anyone has flopped two pair. With a flop of
Q63 it's unlikely someone has two pair because to do so they'd have to
be playing Q6, Q3 or 63. On a QJ9 flop however, a player flops two
pair if his holding is QJ, Q9 or J9 and they are playable hands.
A flop that is Straight-Wet is also Two Pair-Wet. Straights and Two
Pairs often collide in Holdem; the straight wins unless the Two Pair
can improve to a Full House.

\item[Orphan flops] An orphan flop is a flop that looks like it
connected with nobody still in the hand. Orphan flops often have low
pairs on them: 722, 744, 855. If the highest card on the flop is 9 or
higher it starts to look like someone could have flopped top pair but
the low flops look like nobody could connect with them.

These flops are called ``orphan'' flops because they're waiting for a
player to adopt them, by putting in a bet and winning the pot. If
there's been a preflop raise, the preflop raiser can often win the pot
with a standard continuation bet, because nobody else has a piece of
this flop and the preflop raiser could very well have an overpair.

If you're a careful listener, you can often hear a sigh of
disappointment from a folded player when an Orphan flop comes out.
That player correctly folded J2 preflop, but now wishes he hadn't
after the flop came 622.

\item[Action Killing flops] A flop such as AAK, AKK or TTT is a flop
where a player either has a monster hand or trash. There shouldn't be
much more betting action after these flops.

\end{description}
