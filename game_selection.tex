\chapter{Choosing a Game}

% Last updated: 20200428

When I first played poker there were only two places for me to
play, a private club (the legendary APA, Australian Poker Association)
and Star City casino. Star only had four poker tables
back then and games would often wrap up at 6am and not start again
until midday. I used to shake my head at the sections in US poker
books talking about game selection, advising you to choose the games
suitable for you at the amounts you could afford to play. That choice
wasn't available to me.

Today players are spoilt for choice. On any night
of the week there's a game running in each of
the major suburbs of Sydney, and three or four in
the city centre. Some venues run Poker Tournaments five, six or seven
days a week. You should have no trouble finding a game
you can enjoy and do well in. Here are some commonsense
suggestions.

\begin{itemize}

\item Play at a venue that's easy to get to and easy to go home
from. Make sure you're not under time pressure to finish a game.
Always have good safe transport home afterwards.

\item Play at the stakes you're comfortable at. A \$10
game with a \$200 first prize might be peanuts to you.
A \$35 game with a \$600 first prize could be more your size.

It should hurt a bit when you lose a Poker Tournament; not
throw-yourself-off-a-bridge hurt but you don't want to play for
Matchsticks either. Winning a tournament should be a genuine thrill,
both emotionally and financially. To win even the smallest tournament
will take you over three hours. I'd like a first prize of over \$150 for
spending that time winning a poker tournament. I'm happy to play Pub
Trivia for Matchsticks because I enjoy Trivia a lot and I like
the cameraderie with my teammates.

If you're a student, \$100 games might be something
you only play once a month, and a \$10 game twice a week is
a better use of your money and time.

\item Don't play every night. I find that if I play
three nights in a row I hate poker on the third
night. There's some players I see every time I go into a venue,
I don't think they have as much fun with their poker as I do.
There's much more to your life than poker, sometimes the smart
move is not to play.

\item Play where you're interested in the other services
that the venue has to offer. If you like the food or drink or
TAB or sports coverage or other facilities at a Club you can
have a good time there even if you're out of the poker early.

In the winter months, the NRL often has three-in-a-row Super Saturdays
where you can only watch the games on Pay TV; playing a Saturday night
Poker Tournament while watching the games is a great option
for League Fans. I like Pinball Machines and full
sized Snooker Tables. A good beer selection adds greatly to my
enjoyment.

\item Look at the bonuses available for that game. Is there a
guaranteed prize pool? Free food for the players? Sometimes
there's an overlay--- you might have 20 people paying \$10 each
to enter with a guaranteed prize pool of \$300.

\item Bring some friends along or play somewhere that you'll
know a few people. Poker is a lonely game at the best
of times and it's good to have some people even just to share
a bad beat story with during the breaks. That said, I also like
to go to a new venue alone and face the challenge of
reading entirely new players, and being an unknown player.
I'm always on the lookout for new games and new venues.

\item Play somewhere that you respect the tournament directors
and the poker room setup. If the equipment (tables, chairs, chips,
monitors) is excellent you know the venue has made a commitment
to running good poker. Lighting and air conditioning
play a huge part in your comfort at the table.

\item If a venue close to where you live starts
up a poker game, go along and support it. Even if it turns out to
be a bad game or folds after a few weeks, you've met some people
from around your area interested in poker, which might lead
you to other games in your area or some home games.

\item I don't like competitions with a League Table/Points
structure and a big season final. I much prefer a single night poker
tournament where I can win the prizemoney for that tournament, that
night, and not worry about getting qualifying points towards a
huge field Poker Dream game weeks in the future. Because of my
preferences, I'm far more likely to play in an \textbf{NPL} game than
an \textbf{APL} game.

\end{itemize}

In the next chapter ``The Pub Poker Price'' I write about some
aspects of Pub Poker that I'm not so thrilled at.
