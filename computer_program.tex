\chapter{A preflop Computer Program}

% Last updated: 20190716

There are a lot of different ways to think about poker and write
about poker. My day job is in computers, Information Technology (IT),
as both an Operations guy and a programmer. In this chapter I look at
making a single poker decision as being like a computer program. You
can't use a computer program at the tables; your computer is your
brain. Looking at some of the issues a computer programmer would think
about will help me in making better decisions at the table.

People have actually written computer programs to play poker on
the Internet. They're called bots, which is Internet-speak for
robots. They do well at playing a solid game and they never go on tilt.
Occasionally a computer program versus humans poker game is played.
These are the poker equivalents of the famous Chess matches between
champion Garry Kasparov and the IBM Supercomputer programs Deep Blue
and Deeper Blue. Mike Caro was involved in early poker playing
computers, his computer was called Orac (Caro spelled backwards) which
was also the name of the genius computer in the BBC Science Fiction TV
show Blake's Seven.

A computer AI named Pluribus came out ahead after 12-days of playing
six-max No Limit Holdem, according to a story in The Guardian in
July 2019. It came out \$48,000 in front, playing a different five
opponents each day. I don't know how any of the humans went in the
competition, but one feature that the AI used, according to the story,
is donk betting. From the story in The Guardian:

To master Texas Holdem Pluribus adopted some surprising, and
distinctly non-human, strategies which have already been adopted by
the professionals it played. It used wildly differing bet sizes, a
strategy humans seem to find hard to do. And while humans usually
avoid so-called ``donk betting'' – the practice of ending the first
round of betting with a call and opening the next with a bet –
Pluribus embraced the tactic. The received wisdom in poker is that
donk betting is a weak move that rarely makes sense. Pluribus found
otherwise.


\section*{Under the gun Preflop Decision}

I'll sketch out a program which I'll call utgPreflop. It's written
in what IT guys call Pseduocode. Pseudocode isn't a real programming
language like Java, Python or C++, but someone who does know those
programming languages can take the Pseudocode and turn it into a real
program. When I use brackets after a word that's a function call.
A function is a mini-program that returns a small piece of information
that the program uses to make its decisions.

A program has inputs and outputs. The output of utgPreflop will
be one of six words: FOLD, LIMP, RAISE2, RAISE4, RAISE5 and SHOVE.
For my first version, there will just be one input, my two card
holding.

utgPreflop version 0.1

% FIXME: Some LaTeX programming/verbatim effect here

if pocketaa(holding) return RAISE4;
if pocketkk(holding) return RAISE2;
if pocketjj(holding) return RAISE5;
if pocketqq(holding) return RAISE4;
if mediumpair(holding) return RAISE2;
if lowpair(holding) return LIMP;
if aceking(holding) return SHOVE;
if trashHand(holding) return FOLD;
if okHand(holding) return LIMP;
\# if we reach this part we have a better-than-ok hand worth a minraise
return RAISE2;

This program runs from top to bottom and as soon as it it hits a
return statement it stops. I haven't written the functions
trashHand() or okHand() but you can imagine what holdings
meet which criteria. My program raises a different amount
for the high pocket pairs. JJ raises biggest because we'll win the
blinds right now with that hand if we can.

One improvement to utgPreflop is to add randomisation. A good strategy
with KK might be to RAISE2 60\% of the time, RAISE4 20\% of the time,
RAISE2 15\% of the time and LIMP 5\% of the time. With a low pair we
might LIMP 70\%, RAISE2 25\% of the time and even FOLD 5\%. Computer
programs are much better at doing random numbers than humans are. If
you've read Dan Harrington's books, he suggests using the second hand
of your watch to come up with a random number. A computer program will
do a much better job of properly randomising the decisions.

There's a huge problem with utgPreflop. It's not considering enough
inputs. It's just looking at the holding in isolation. Another thing
that needs to be looked at is stack sizes. Shoving all-in with
Ace-King is good when our stack is 10BB or so, but it's poker suicide
doing it with a 50BB stack where it's winning 1.5BB most of the
time but the only hands that call us are now AA or KK and we lose our
tournament. Also, it hasn't taken into account the number of players
at the table. A9 is a fold UTG at an 8-player table, but at a 5-player
table it's a raise.

I won't write up Pseudocode for Version 0.2 of utgPreflop, what I'm
interested in now is what extra inputs should the program consider?
As well as our own cards and the number of players dealt to, what do
we know about those other players? If three of them have been maniacs
so far, we can change our big pocket pair strategy to LIMP and expect
to get a big check raise in.

A full version of utgPreflop would have a whole bunch of inputs, such
as

holding
numDealtPlayers
numActivePlayers
NewRandomNumber
myStackSize()
blindsAboutToGoUp()
seat4StackSize()
seat4Image()
seat4OnTilt()
seat4BadBeatedLately()
seat4Maniac()
seat4Rock()
seat4CallingStation()
seat4Expert()
seat5StackSize()
seat5Image()
...

The logic and thinking that computer programs use to
make their decisions are called Algorithms. They're like a good
cooking recipe that a chef uses to make a great meal. Facebook,
Amazon, Google, Netflix, Instagram, YouTube and big smart Internet
companies use Algorithms to give you what you want on the Internet.
They continually improve and refine these Algorithms, making them
better and better over time. A whole bunch of companies trade Shares
and Foreign Currency using Computer Algorithms; buying and selling
when their algorithms determine the price will go up or down.
Self-driving cars use algorithms, continuously updating their input
from their cameras and also getting wireless updates about traffic and
road conditions.

One thing no computer program can save you from is wrong input.
If utgPreflop has been fed the input that Seat 6 is a maniac,
but Seat 6 is actually a rock, it will come up with the wrong decision
for playing JJ. utgPreflop will bet-reraise against Seat 6; which is
fine against a maniac but terrible against a rock who has KK here.

% FIXME: Genesis of the Daleks Dr Who forced to give away Dalek defeat
% reasons

\section*{A suite of programs}

utgPreflop is just one program, and you'll need four of them to
play a hand to completion from that position. First utgPreflop,
then utgFlop, utgTurn and utgRiver. The later programs of course
have knowledge of the earlier programs and the full play of the hand
up to that point. The complexity of the later programs goes up and up,
as they respond to new information in the hand.

Because of bad beats, utgPreflop and your other poker programs can
function perfectly and you still get trounced out. You can be totally
card dead and just not have enough card power to win a
tournament. Being card dead is a preflop bad beat,
spread over many hands. By normal laws of probability you'd get dealt
some playable hands now and again, but tonight you're just getting
trash hand after trash hand. When Seat 6 is a maniac he incorrectly
calls your JJ shove with Ace-Four suited and flops an Ace. Even with
great decision making and near-perfect play; you still lose most of
the tournaments you play in.

