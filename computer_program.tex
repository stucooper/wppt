\chapter{A preflop Computer Program}

% Last updated: 20190703

There are a lot of different ways to think about poker and write
about poker. My day job is in computers, Information Technology, as
both an Operations guy and a programmer. In this chapter I look at
making a poker decision as being like a computer program. You can't
use a computer program at the tables; your computer is your brain.
Looking at some of the issues a computer programmer would think about
will help me in making better decisions at the table.

People have actually written computer programs to play poker on
the Internet. They're called bots, which is Internet-speak for
robots. They do well at playing a solid game and they never go on tilt.
Occasionally a computer program versus humans poker game is played.
These are the poker equivalents of the famous Chess matches between
champion Garry Kasparov and the IBM Supercomputer programs Deep Blue
and Deeper Blue. 

\section*{Under the gun Preflop Decision}

I'll sketch out a program which I'll call utg_preflop. It's written
in what IT guys call Pseduocode. Pseudocode isn't a real programming
language like Java, Python or C++, but someone who does know those
programming languages can take the Pseudocode and turn it into a real
program. When I use brackets after a word that's a function call.
A function is a mini-program that returns a small piece of information
that the program uses to make its decisions.

A program has inputs and outputs. The output of utg_preflop will
be one of six words: FOLD, LIMP, RAISE2, RAISE4, RAISE5 and SHOVE.
For my first version, there will just be one input, my two card
holding.

utg_preflop version 0.1

if pocketaa(holding) return RAISE4;
if pocketkk(holding) return RAISE2;
if pocketjj(holding) return RAISE5;
if pocketqq(holding) return RAISE4;
if mediumpair(holding) return RAISE2;
if lowpair(holding) return LIMP;
if aceking(holding) return SHOVE;
if trashHand(holding) return FOLD;
if okHand(holding) return LIMP;
\# if we reach this part we have a better-than-ok hand worth a minraise
return RAISE2;

This program runs from top to bottom and as soon as it it hits a
return statement it stops. I haven't written the functions
trashHand() or okHand() but you can imagine what holdings
meet which criteria. My program raises a different amount
for the high pocket pairs. JJ raises biggest because we'll win the
blinds right now with that hand if we can.
