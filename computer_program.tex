\chapter{A preflop Computer Program}

% Last updated: 20190703

There are a lot of different ways to think about poker and write
about poker. My day job is in computers, Information Technology, as
both an Operations guy and a programmer. In this chapter I look at
making a poker decision as being like a computer program. You can't
use a computer program at the tables; your computer is your brain.
Looking at some of the issues a computer programmer would think about
will help me in making better decisions at the table.

People have actually written computer programs to play poker on
the Internet. They're called bots, which is Internet-speak for
robots. They do well at playing a solid game and they never go on tilt.
Occasionally a computer program versus humans poker game is played.
These are the poker equivalents of the famous Chess matches between
champion Garry Kasparov and the IBM Supercomputer programs Deep Blue
and Deeper Blue. 

\section*{Under the gun Preflop Decision}

I'll sketch out a program which I'll call utgPreflop. It's written
in what IT guys call Pseduocode. Pseudocode isn't a real programming
language like Java, Python or C++, but someone who does know those
programming languages can take the Pseudocode and turn it into a real
program. When I use brackets after a word that's a function call.
A function is a mini-program that returns a small piece of information
that the program uses to make its decisions.

A program has inputs and outputs. The output of utgPreflop will
be one of six words: FOLD, LIMP, RAISE2, RAISE4, RAISE5 and SHOVE.
For my first version, there will just be one input, my two card
holding.

utgPreflop version 0.1

if pocketaa(holding) return RAISE4;
if pocketkk(holding) return RAISE2;
if pocketjj(holding) return RAISE5;
if pocketqq(holding) return RAISE4;
if mediumpair(holding) return RAISE2;
if lowpair(holding) return LIMP;
if aceking(holding) return SHOVE;
if trashHand(holding) return FOLD;
if okHand(holding) return LIMP;
\# if we reach this part we have a better-than-ok hand worth a minraise
return RAISE2;

This program runs from top to bottom and as soon as it it hits a
return statement it stops. I haven't written the functions
trashHand() or okHand() but you can imagine what holdings
meet which criteria. My program raises a different amount
for the high pocket pairs. JJ raises biggest because we'll win the
blinds right now with that hand if we can.

There's a huge problem with utgPreflop. It's not considering enough
inputs. It's just looking at the holding in isolation. Another thing
that needs to be looked at is stack sizes. Shoving all-in with
Ace-King is good when our stack is 10BB or so, but it's poker suicide
doing with with a 50BB stack when the only hand that will call us in
AA or KK. Also, it hasn't taken into account the number of players at
the table. A9 is a fold UTG at an 8-player table, but at a 5-player
table it's a raise.

I won't write up Pseudocode for Version 0.2 of utgPreflop, what I'm
interested in now is what extra inputs should the program consider?
As well as our own cards and the number of players dealt to, what do
we know about those other players? If three of them have been maniacs
so far, we can change our big pocket pair strategy to LIMP and expect
to get a big check raise in.

A full version of utgPreflop would have a whole bunch of inputs, such
as

holding
numPlayers
myStackSize()
blindsAboutToGoUp()
seat4StackSize()
seat4Image()
seat4OnTilt()
seat4BadBeatedLately()
seat4Maniac()
seat4Rock()
seat4CallingStation()
seat4Expert()
seat5StackSize()
seat5Image()
...

The logic and thinking that computer programs use to
make their decisions are called Algorithms. They're like a good
cooking recipe that a chef uses to make a great meal. Facebook,
Amazon, Google, Netflix, Instagram, YouTube and big smart Internet
companies use Algorithms to give you what you want on the Internet.
They continually improve and refine these Algorithms, making them
better and better over time. A whole bunch of companies trade Shares
and Foriegn Currency using Computer Algorithms; buying and selling
when their algorithms determine the price will go up or down.

% FIXME: add the fact that a computer program can do properly random
% numbers?

One thing no computer program can save you from is wrong input.
If utgPreflop has been fed the input that Seat 6 is a maniac,
but Seat 6 is actually a rock, it will come up with the wrong decision
for playing JJ. utgPreflop will bet-reraise against Seat 6; which is
fine against a maniac but terrible against a rock who has KK here.

% FIXME: Genesis of the Daleks Dr Who forced to give away Dalek defeat
% reasons

Because of bad beats, utgPreflop and other poker programs can function
perfectly and we will still get trounced out. Also we can be totally
card dead (which is a kind of preflop bad beat over many hands) and
just not get enough card power to win a tournament. Sometimes Seat 6
is a maniac and calls our JJ shove with his Ace-Four suited and flops
an Ace.

