\chapter{Hand Strength}

% Last updated: 20190628

Poker hands have absolute and relative strength.
The absolute strength is defined by the Poker Hands Ladder:
Straight flush, Four of a Kind, Full House, Flush, Straight
and so on. The relative strength is the strength of the
hand relative to the Board.

Aces full of Fives is a powerful hand with high absolute strength.
But holding 55 on a board of AAKQA your AAA55 is weak. It loses
to the final Ace, any King, any Queen, and six other pocket pairs.
But if you have Aces full of Fives by holding AA on a board of AT5J5
you've got a monster Aces full, that only loses to the highly
improbable Quad Fives.

Two card hands, where you use both your hole cards to make a final
hand that's much stronger than the five-card Board, are much
better Holdem hands than one card hands. If the board has
three of a suit and you've got two more of that suit, your flush
should be the winner. If the board has four of the suit, your
flush loses to a single higher card in that suit and the highest
one-card flush wins.

From weakest to strongest, here's a list of seven strengths your hand
can be on the river.

\begin{description}

\item[Cant Win] A Cantwin hand is a hand with literally no chance of
winning the pot.

\item[Bluff Catcher] A Bluff Catcher is a hand that you can call with
that wins only if the other player was bluffing (typically an
Entitlement Bluff with a busted draw).

\item[Showdown Value] A Showdown Value hand is middle pair with a
decent kicker or some hand that will win some showdowns but lose some too.

\item[Value Bet] This is a hand you expect to win the pot, it's strong
enough that you value bet it on the river and expect to be the winner
when you're flat called and the hand goes to Showdown.

\item[Expected Winner] This hand is a little stronger than the Value
Bet hand, it's a hand that you're genuinely surprised to lose
with. Two pair on a dry board is an Expected Winner, but still loses
to sets and strangely played better two-pair or straights.

\item[Go Broke] A Go Broke hand is a non-nut hand that is nevertheless
so strong that you're comfortable raising with it on the river. You've
got the top card of a two-paired board but you still lose to quads or
two card better full houses (eg JT on a J9JK9 board). You make
runner-runner King-high flush on an unpaired board and lose to the
Ace-high flush, a hand that looked impossible given the earlier
betting action.

\item[Nuts] An unbeatable hand. Happy days. Hopefully the other player
in the pot has the Go Broke hand.

\end{description}

% FIXME Put in the "Once all the cards are out" table
% or maybe just have a "Confidence out of 10" column.

I've listend seven hands, with a bit more imagination and writing
I could have come up with ten hand strengths, though I would have been
coming up with very long winded names like
bit-weaker-than-valuebet-but-stronger-than-showdownvalue. Think of
your hand strength as being on a spectrum, between totally
worthless and Stone Cold Nuts. The important thing is to have
a well-developed poker sense so you know where you're at and you
have a good sense where your opponent is at. Have a good poker radar.

Facing a decent bet on the river, here is what you should
usually, sometimes and never do with each hand type.

\begin{tabular}{|l|l|l|l|} \hline
Hand    &  Usually & Sometimes & Never \\ \hline
Cantwin & Fold  & Raise     & Call  \\ \hline
Bluff catcher  & Fold & Call & Raise \\ \hline
Showdown value & Call & Fold & Raise \\ \hline
Value Bet & Call & Fold & Raise \\ \hline
Expected Winner & Raise & Call & Fold \\ \hline
Go Broke & Raise & Call & Fold \\ \hline
Nuts     & Raise & Raise & Call \\ \hline
\end{tabular}

This table is a little simplified but you get the idea. On hand strength
alone, this is how you should respond to a bet on the river.
