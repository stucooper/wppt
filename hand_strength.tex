\chapter{Hand Strength}

% Last updated: 20181119

Poker hands have absolute and relative strength.
The absolute strength is defined by the Poker Hands Ladder:
Straight flush, Four of a Kind, Full House, Flush, Straight
and so on. The relative strength is the strength of the
hand relative to the Board.

Aces full of Fives is a powerful hand with high absolute strength.
But holding 55 on a board of AAKQA is a weak Aces full. It loses
to the final Ace, any King, any Queen, and six other pocket pairs.
But if you have Aces full of Fives by holding AA on a board of AT5J5
you've got a monster Aces full, that only loses to the highly
improbable Quad Fives.

Two card hands, where you use both your hole cards to make a final
hand that's much stronger than the five-card Board, are much
better Holdem hands than one card hands. If the board has
three of a suit and you've got two more of that suit, your flush
should be the winner. If the board has four of the suit, your
flush loses to a single higher card in that suit and the highest
one-card flush wins.


% FIXME Put in the "Once all the cards are out" table
% or maybe just have a "Confidence out of 10" column.

My table has seven rows, with a bit more imagination and writing
I could have produced ten or more rows, though I would have been
coming up with very long winded names like
bit-weaker-than-midstrength-but-stronger-than-weak. Think of
your hand strength as being on a spectrum, between completely
worthless and Stone Cold Nuts. The important thing is to have
a well-developed poker sense so you know where you're at and you
have a good sense where your opponent is at. Have a good poker radar.
