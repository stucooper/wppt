\chapter{Cricket games and Poker structure}

% Last updated: 20181030

Winning a poker tournament is like batting and scoring a hundred in cricket.
You have to build your innings, and take care not to get out early
with a rash shot. The bowling and fielding might be good or it might
be easy. You might be hitting the ball well or be struggling and out
of form. Sometimes you'll get bowled by a swinging ball and
there's nothing you can do about it.

\section{Cricket Structure}

Test cricket and One Day cricket are vastly different games. Then
there's Twenty-20 cricket which is super fast. Ball by ball the
action is much the same, but the different formats mean
different skills are required in each game.

Test cricket is considered the pinnacle of the game by the players
and the sportswriters. Test statistics and records
matter the most to the fans and the players. There are some players who excelled
at Test cricket but not the faster paced One Day matches, while
other players (famously Michael Bevan) were One Day
specialists and never reached a position of comfort in the Test team.

Big buyin, deepstack slow poker is the equivalent of Test Cricket.
Small buyin, shortstack fast poker is One Day Cricket.
And ultra-fast tournaments, such as 30 big blind stacks with 10 minute
blind doubles, are poker's hit and giggle Twenty-20 games.

You won't play in a \$550 entry 50,000 starting stack 1 hour blind double
game very often, but you can play \$10 5,000 starting stack 15 minute
blind double games 5 times a week. The \$10 game is vastly
different to the \$550 game.

A good batsman might average 55 in Test Matches and 40 in One Day
Matches. The fast pace of the One Day game forces him to take
chances earlier and more often than in Tests. The important thing in
Test Matches is often staying in, whereas in one day matches
the focus is on getting runs quickly. There's an extra
measure of how quickly you score runs, Strike Rate, that's mentioned
a lot in short form cricket but not so much in Tests. If you get out,
you're leaving it to other batsmen to score runs, and there's another
one day game in three days time anyhow.

In a Test Match you can wait for the bad balls, or for the bowlers to
get tired. In big stack poker you can wait for good hands, simply folding the
mediocre starting cards and waiting for the better hands. In One Day games
and small stack poker you've got to take chances sooner.

Any cricket fan can tell you the difference between the Test,
One-Day and Twenty-20 formats. But a lot of pub poker players haven't given any
thought to the difference between deepstack and shortstack
tournaments. It's the same game they saw on World Poker Tour,
right?

In all forms of cricket a bowler runs in to a batsmen who
can leave the wide ball, play it and score runs from it, or get out
to it. In some games there's extra restrictions on where the
fielders can be placed and how many overs a bowler
can bowl, but the mechanics of each delivery in cricket is the same.
The mechanics of each hand in No Limit Holdem is the same.
But just as the cricket games are very different from each other, the
poker games are very different also.

There's no shame specialising in fast poker tournaments. Indeed,
the extra thought you put into structure and risk taking can give
you the advantage over more experienced players who are better in
slow deepstack tournaments, but don't understand fast tournaments.

\section{Tennis}

Tennis is another sport with vastly differently paced competitions.
The prized tournaments are the four Majors (Australian Open,
French Open, Wimbledon and US Open) with a full field of 128 players
and men's matches best-of-five sets throughout. These slow tournaments
take two weeks to complete.

Smaller ATP tour events have a field of 32, 16 or 8 players in the draw
and early round matches (and sometimes even the final) are best-of-three
sets. There are a few of these events happening in different cities in
the world at the same time, players pick and choose which events
to play in.

The ultra fast hit-and-giggle Tennis format is called Fast-Four.

A few retired tennis stars have done very well in Poker. Russian
champion Yevgeny Kafelnikov has won a few big Poker tournaments.
Tennis legend Pete Sampras is a keen player. World Poker Tour
commentator Vince Van Patten was a US tennis star in the late
70s and early 80s, reaching 26th in the world in the heady days
of Vitas Geraulitis, John McEnroe and Jimmy Connors

\section{Golf}

Similar to Tennis, Golf has four Major tournaments each
year---the British Open, US Masters, US PGA and US Open. Smaller
tournaments have smaller fields and some are played over
three rounds of 54 holes not 4 rounds of 72 holes.

There's actually a second string golf tour where players pay an
entry fee and compete for each other's money, just like
Tourament Poker players do.

Golf, Tennis and Poker are individual sports, where
the normal competition format is ``every man for himself'' and
you have to come out on top of the field to win the tournament.
In a standard knockout Tennis tournament, you get helped
if star players lose to lesser players and you get to play the
lesser players in later rounds. In Poker tournaments you're
helped if star players lose to lesser players, often on other
tables. In both sports, you can only beat the players you're
up against on the day.
