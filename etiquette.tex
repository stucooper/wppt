\chapter{Etiquette}

% Last updated: 20200126

Imagine if every time you bought a meal at McDonalds the staff
said something insulting to you, like ``You're an idiot''. They'd
soon go out of business. Service and Hospitality staff are trained
to be polite to the customers and poker players should be polite to
each other.

A few people have got it into their heads that nastiness
and putting other players on tilt is a crucial part of the game.
It isn't. If you berate bad players when they get lucky against
you, you'll create a poisionous atmosphere that will, over time,
drive the nice guys and weaker players out of the game. There's plenty
of other things they can do with their time. If the nice
guys all leave the game, Tournament poker will become a small
bunch of jerks snarling over the scraps of a tiny prizepool.
If fields get too small, Pubs and Clubs will stop running Poker
Tournaments altogether. I don't want that, and neither do you.

If you're a winning poker player, the other players are
your customers. Be nice to your customers.

The etiquette at Pub Tournaments is normally great, at a much
better level than Casino cash games, where needling other players
and getting them to make big mistakes \textbf{is} an important
part of the game. Here's a list of etiquette suggestions that
make a Pub Poker game even better. The first two apply at the worst
time of your poker tournament; the moment when you've just been
eliminated.

\begin{description}

\item[Shake the hand of the player who busts you] Look the winner in the
eye, shake his hand and say ``Nice hand, sir'' or ``You got lucky there''.
Wish the winner and the other players at your table good luck for the
rest of the Tournament: ``Good luck everybody'' is the phrase I use.

If you're at the final table and there's only 5 or 6 players left you
might like to shake everyone's hand, starting with the player who
busted you. If you're in the money, move to the side so that play
can continue, and attract the attention of a Tournament Director who
can get you your payout.

\item[Leave the table promptly and take all your stuff] On small
tournament slow nights, some losers like to stay on the final table
after busting out and watch the action from there. This is confusing
for the dealers who often pitch the no-chips player cards in the deal,
which needs to be quickly corrected. Seats at the poker table,
including the poker table, are for players only.

Whenever we're away from home, we should try to keep our personal
belongings close to us at all times. I'm the king of leaving my poker
diary on my seat and be ten metres away from the table before other
players have to shout out to me that ``You forgot your Diary!!''. I'll
try to improve on this in my poker going forward.

\item[Never Slowroll]  There's nothing clever about showing your hand
as late as possible and tricking the other showdown player into thinking
he's the winner. Slowrolling is hated by poker players for a
reason.\footnote{Tommy Angelo's fantastic book, Elements of Poker, has
a whole section on showdown etiquette, which he calls Sixth Street}
Don't do it. You're better than that.

% FIXME: Cite Angelo's book properly.

\item[Players can play their hand however they want] Anyone playing in
a Poker Tournament can play their hand in any way they want. They
can stay in with 52 on an AAQ board and crack your AK with a runout
of AAQ43 if they want. Don't berate bad players. If you must say
something, a sarcastic ``Nice hand, sir'' or ``Well played, sir'' should
be enough.

\item[Sir or Madam] In the dealing chapter I said that calling
unknown players ``Sir'' or ``Madam'' is a nice touch. It is.

\item[Call your hand honestly at showdown] Don't say ``straight''
or ``flush'' at showdown if your hand isn't a straight or a flush.
Call your hand honestly or don't say anything at all and let
Cards Speak. When I'm showing down what I expect to be the winner,
I announce my hand and expect to pull in a nice pot. If my hand is
iffy I go for the cards-speak option and give the calling player (who
is likely to be the winner) the chance of misreading my hand and
folding the winner. I never mis-call my hand.

\end{description}
