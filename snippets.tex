\chapter{Extra snippets}

% Last updated: 20190609

This chapter contains some short sections about individual aspects
of poker. Some of the ideas turn up elsewhere in the book but are
repeated here to reinforce the concept.

Each snippet is a single idea about poker expressed in a few
short paragraphs. Read each one and see if you agree with the idea.
Write some snippets of your own on one of the blank pages, or
in your own notes.

Re-read these snippets every few months. Perhaps some of the
ideas didn't mean anything to you the first time around, but
then you came across the situation it's talking about in
a game.

\section{The honest raising position}

I believe that the small blind is the most honest raising
position in Poker. It's a terrible position to play from
in all the betting streets, even preflop you're not
acting last. If someone's raising from the small blind
position at a full table, give them credit for a good hand.
If they're a known maniac or very short on chips you need to
factor that in, but a standard player with standard chips usually
raises from the small blind only with a good hand.

If everyone's folded and there's just the small blind and the
big blind left, a small blind will often try a steal. If there's
more than five players at the table, and there's other people
still in the pot besides the big blind when the action hits the small
blind, expect the small blind's raise to mean genuine strength.

\section{Get it Heads up to the flop}

Phil Gordon recommends a style of play where you never limp into any pot,
you're folding or raising and getting Heads Up on the flop. He's always trying
to do this from position, and like me hates playing from out of position.
Phil prefers getting Heads Up against one other player preflop, and
outplaying that player after the flop. ``Outplaying'' doesn't mean
always winning that one particular pot. It means he uses his poker
smarts to make the best decisions possible, and to win big/lose small.

\section{The one card flash}

Someone winning a pot without showdown who then flashes one card
which is a losing card probably wasn't bluffing. If he wanted to
show a bluff he'd show both cards and remove all doubt. Once
I saw a player bet at a flop of K66, win the pot and flash the
Seven of Spades. I was confident that his unshown card was a
Six.

% Next two snippets are now in the "Bluffs" chapter
%% \section{If you're never bluffed, you're calling too much}

%% Everyone hates getting bluffed. Some players hate it so much that
%% they'll call bets on the river with just about anything, bottom pair
%% or even Queen high! If you play like this you're giving your chips
%% away really fast. If you don't have a good sense that the bettor is
%% out of line, and your own hand really can't beat anything, simply fold.

%% \section{If your bluffs always work, you're not bluffing enough}

%% This is the previous snippet, from the bluffer's perspective. If every
%% time you bluff you win the pot, you're not bluffing often enough.
%% Absolute rocks can pull off a bluff once every five rounds but
%% that bluffing frequency is still too low.

%% There's a theoretical proper frequency of bluffing. If your bluffs
%% always work you're not doing it often enough and if your bluffs always
%% fail you're doing it too often or you're leaking obvious tells.

%% You \textbf{should} be caught bluffing now and again.
%% Otherwise you're not bluffing enough. You
%% \textbf{should} be paying off winners now and again.
%% Otherwise you're folding to bets too much.
%% Don't bluff in hopeless spots and don't call every value bet, but
%% realise and accept the fact that you won't get your decisions right
%% all the time.

%% When your bluffs are always failing, your calls are never winning,
%% and your nut-hand bets are never getting called
%% you know you're in a really tough game.

\section{If you make a backdoor hand, overbet it}

Sometimes in poker you'll make a big hand through good luck on the turn and
river. You have a flush draw with low cards but instead hit
runner-runner straight. You have middle pair and hit runner-runner
flush. In these circumstances you finish with a huge hand that's very
hard for your opponents to put you on.

Overbet this hand and expect a big payoff. When you show down your hand,
the players will think you're a luckbox. That's fine. Don't remind them
that you were on a different draw and got lucky. Let them call you names
and keep getting paid off with your big hands.

\section{With the nut hand on a Big Board, overbet it}

A Big Board is a Board where it's very easy to have a flush
or a straight. Four diamonds is a Big Board. Four
straight cards with just one gap is a Big Board (JT876).
A double paired board is a Big Board where it's easy to have a full
house (KK944).

The dream situation in poker is super hand versus a great hand.
One of my favourites is nut straight against straight. I hold J9
on a board of 3678T, I push all in and get called right away by
98. This is a fantastic situation and works well when the board
has four to a straight. You should hope that someone else has
a straight and will call you thinking they'll get a split pot.

This betting also works when you have the high full house on a
double paired board. If you have KT on a board of KKJ55 it's
very tough for someone holding the 5 to fold, so punish him the
maximum. Sure, if you bet smaller you'd get a crying call from
someone with a Jack or even Ace high, but I think it's best
to go for the big payoff here. If you bet small it might look
suspicious and you just get called, if you bet medium it looks
like decent value and you just get called so bet huge and put the
other player to a really tough decision.

\section{Differentiate between big bets and small bets}

There's a huge difference in Poker between big bets and small bets,
big calls and small calls. If a player calls a raise holding trash
that's usually a mistake, unless the raise was small and the player
can hit big if he spikes the right flop. I've called little bets
with junk like T5 lots of times, got a flop of 955 and doubled my entire
stack against the preflop raiser who held QQ. People say ``how could
you call that preflop raise with Ten-Five''? and the point is that
there was huge stack value if I got the perfect flop.

%% Active players are forever probing away at pots with little bets, calls
%% and raises, when they make the big bets and calls they've got a made hand.
%% Daniel Negreanu calls this style ``Smallball'' and describes it in his
%% new book ``Power Holdem Strategy''.

\section{Bet and raise to get information}

You get more information about the relative strength of your hand
by betting and raising, not by calling. When facing a bet, calling is
fine when you have the nuts or a good draw, but the rest of the time
you should be looking at raising or folding.

\section{A bet at two or more players is less likely to be a bluff}

When there's three or more players still in a pot, a bet into that
pot is likely to be an honest bet and not a bluff. It's harder to bluff
two or more players than a single player, and with more players still in
the pot it's more likely that some of the players have connected with
the flop.

One hand I'll never forget had seven players by the turn, where I made
a King high flush. I bet 1,000 into the pot of 1,400 and every player
made the call! I suspected the Ace of the flush suit was drawing against
me, but I couldn't understand what all the other callers had.

The river paired the board, but I felt I was still best and I bet
1,500 into the 8,400 pot, a small bet that gave me wiggle room to get
out if I got raised by a possible full house. I got two players calling
this bet, and my King high flush won the pot.

Getting six callers on my turn bet was a great outcome for me in this hand.

\section{Respect the Check Raise}

Unless you've got a super read on him that suggests he's bluffing,
expect a check raiser to be holding a super strong hand. Check raisers
usually hold at least two pair and often have sets. If you haven't got
a draw to a great hand yourself, just fold. And normally you won't have
a good draw, because if you had a good draw you should be checking the
flop and taking the free turn card.

I estimate that less than 5 percent of sizable check raises in pub poker
are bluffs.

\section{The Halfway Hand}

The halfway starting hand in Holdem is Q7 (sometimes given the name
``computer hand'', I think because it runs well in computer simulations).
50\% of starting hands are better than Q7 (all the pocket pairs,
Ace-any, King-any and QJ,QT,Q9,Q8) and 50\% of hands are worse. I've
found this little snippet of information quite useful, in
deciding whether to call straddles and also in Heads Up play
against no-look maniacs or short-stacked opponents. Remember that Q7
is the 50\% starting hand in Holdem.

Some poker books have a ranking of holdings-- these holdings are in
the top 10\%, add these others to get the top 20\% and so on. These
are worth looking at to get a feel for which hands are better than
others, but don't need to be memorised.

\section{Raise with the Obvious Split Nuts}

Some players just call a river bet when they hold a Nut hand on a
wet board where the winning hand is an obvious
one-card straight. From the action it looks likely that the bettor
also has the straight and this should be a split pot.

The board is KQJ4T so any Ace makes a nut straight. A Bettor
bets and a player with an Ace just calls.

This is very bad play by the caller. There's no certainty the bettor also has
the Obvious Split Nuts. The bettor to KQJ4T sometimes has T9, sometimes
he has QQ, sometimes he has KQ, sometimes he's running an imaginative
bluff.

If you've got the Nuts, raise it on the river. You've got a no-risk
chance to win more chips. Now is not the time to show off to the table
that you have a super read on the Bettor's hand and you \textbf{know}
that raising is no use because the Bettor also has the Nuts and he will
instantly call or re-raise your raise.

%% FIXME: Add a bit about tournament penalty for not raising on the
%% river with the Nuts in last position.

%% FIXME: writing was OK but wanted the chapter to finish earlier for
%% page break purposes.

%% \section{Switching gears}

%% The players in the top games are always looking to switch gears:
%% to play their hands differently over time and not give their
%% observant opponents too much of a clue on their game. After
%% checking with a flush draw on one hand and hitting their flush,
%% they might decide to raise with a flush draw the next time they
%% have one (and an opponent might think ``he hasn't got the
%% flush draw, because he checks his flush draws, I saw him do that
%% on that flush he made three hands ago).

%% The basic change of gear is from tight to loose or loose to tight.
%% Try to stay aggressive, there's little value changing from aggressive
%% to passive, though if your table has Maniacs and you have lots of chips
%% you can consider becoming a Rock for a while. Stay smart. Don't change
%% your poker play from smart to stupid.

%% As a general rule, it's best to be playing the opposite of how the
%% other players are playing. If the table's playing loose, you should
%% play tighter. If the table's passive (as many pub poker tables are
%% preflop) you should be putting the raises in and stealing blinds
%% and punishing the limpers.


