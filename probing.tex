\chapter{Probing}

% Last updated: 20190618

There's a situation that turns up again and again in poker.
It's heads up on the river and you put in a good sized final bet.
Your enemy is wondering if he should call or fold (his own hand
isn't nut enough to raise). He looks you in the eye and
asks the question ``Will you show if I fold?''

\section{Show if I fold?}

This question is an information probe by your enemy and you've got
to be ready for it. Your enemy will be watching you closely
to see how calm and relaxed you are in answering his question.
There's a few ways you can play this.

\begin{itemize}
  \item Don't look him in the eye, don't say a thing. Continue to
    stare at the river card, or wherever your favourite staring place
    on the table is. This is the ``Man of Mystery'' approach. Your
    staring is saying ``Poker's not a social game, you're not my
    mate, and if you want to see my hand you'll have to call that
    bet. I don't even want to talk to you. Figure it out for
    yourself.'' This risks raising the table tension and makes you
    look like the bad guy, even though \textbf{you're} the one being
    probed.

  \item Have a prepared answer that you use every time. I've got a
    simple five word answer that I repeat to myself, like a meditative
    prayer, before a tournament. Here it is: ``I'll show if you
    call''. It's a great answer. It's a little bit smart-arsed and it
    tells the prober ``I'm not playing your silly showdown probing
    game. You've seen my actions on every street, you figure it out,
    buddy.'' This answer puts the ball back in the prober's court and
    makes \textbf{him} look like the bad guy.

  In the earliest Doyle Brunson Puggy Pearson Amarillo Slim Texas
  backroom days, the player considering a call was allowed to reveal
  his hand, and get a reaction from the river bettor. In Super System
  2, Brunson bemoans that he's not allowed to do that anymore.
  % FIXME Quote from Super System appropriately here
  Once players weren't allowed to show their hands, they tried
  honestly verbalising their hands. That's been stamped out of poker
  too.

  \item ``Sure''. If I'm genuinely interested in the ponderer's hand,
    I'll show if he folds but only if he folds face up. If he
    surrenders and I get to see what hand he was considering calling
    with, I'll show him what I had too, and he'll be able to sleep
    tonight knowing if he made a good fold or a bad fold.

    If however, the ponderer folds face down, I won't show. You don't
    get something for nothing. This goes all the way back to the
    kiddies game ``I'll show you mine if you show me yours''.

  \item You could, if you want, engage the guy in conversation and
    discuss the hand, especially the hand you think he has. You're not
    allowed to reveal your own exact hand but you can say stuff like
    ``I can beat top pair'' or ``Aces no good''. You run the risk, if
    you engage your questioner like this, of giving him exactly the
    reaction he's looking for and helping him make either the right
    fold or the right call.

    Like anything in poker, you'll improve at this chat the more you
    do it, but I personally think that this is a no-win game that you
    should avoid. Poker's about putting the other player to the tough
    decision. If you don't engage him meaningfully on the river,
    you'll keep it as tough for him as you possibly can.
    
\end{itemize}

\section{What do you want me to do?}

Another question you get asked a lot, at any stage of the betting, is
``What do you want me to do?''. This is another question worth having
a smarmy canned answer to. Here's two suggestions, use them if you
like or work on your own. ``I want you to stop asking me stupid
questions''. ``I want you to show me your hole cards'' (a stupid
action that would void his hand). One of my mates reminded
me of the following zinger ``I can't help you as I'm in a hand right
now''. Just like the answer to ``Show me if I fold'', you've put the
ball back in his court and made him look like the bad guy.

In the same way as your poker face and betting motions should be the
same regardless of your hand strength, so should your probe
responses. If you answer the same way, every time, you'll give off
as little reliable information as possible, and win the
Information War.

\section{Non-verbal probing}

You're not always probed verbally. Sometimes the ponderer
will make a move to fold his cards, but not let go of them. He's looking
to see your reaction to his fold (if you look relieved, he might reverse
his fold and call your bet, if you look annoyed, because you had a power
hand and wanted him to call to win you a bigger pot, he will continue
with his fold and you win the pot as it is).

Another approach is for the ponderer to count out the chips to make
the call and to hold them in his hand above the table, but not
actually put them down and make the call (it's not a call until a chip
hits the table). This is the same probe-for-a-reaction game, this time
he's looking for your reaction to him calling not your reaction to him
folding.

If I feel I'm being non-verbally probed too much, I will sometimes
simply shut my eyes, or continue to stare at my stare point on the
table. Having a stare point is good poker. He won't get a read on
you if you're not even watching him.
