\chapter{Putting players on a hand}

% Last updated: 20181011

Now you're going to use your dealing skills to follow the action in
every hand of poker you play, and many hands you don't. I'm going
to present a list of questions you should ask yourself as
a hand of poker progresses. The questions marked with a single question
mark are questions of fact which you'll know the answer to from
following the hand. The questions ending in two question marks, which
I'll call conjectures, you need to guess at.

Answering these questions will help you put opponents on hands and
find mistakes in their game. You'll be able to classify their
playing style and take advantage of it.

When putting players on hands, start with a range of hands
and try and narrow that range based on new information on each
betting round.

\section{Preflop betting round}

What are the blinds? How many players are dealt to? Who raised preflop?
What hands does he have?? Who called him? What hands do they have??
Who has better position? Who has the better cards?? How big is the pot?

\section{After the flop}

What's the best possible hand on the flop? What's the best probable
hand on the flop?? Is the leader still ahead?? Did the caller
like the flop?? What possible draws are there on this flop?

How much did the leader bet, and how much is that relative to the pot?
What kind of bet is this?? How many chips does the leader have left?
What should the caller do here?? What does the caller do here? How
big is the pot?

\section{Short pause}

By now this hand has finished 2 of 4 betting streets and I've already
thrown over 20 questions at you. Most of them (with a single
question mark) are the factual questions, which you
should be able to answer accurately. Don't worry if your answer to How
big is the pot? is off by 50 or 100 chips. You can see the pot size,
positions, remaining stacks and community cards at each stage of the
action.

The conjectures (with the double question mark)
are all about putting the players on their hands. This is
a pot you're not even in, and you're using the time to practise hand reading
skills. The better you read a hand, the better poker player you'll be.
Hand reading is more important when you're still involved, but you can get
a feel for the other players by watching them in every pot they're in.

There's some more advanced conjectures you can try. What kind of player
is the raiser?? What kind of player is the caller?? See ``Four Player Types''
in the chapter ``Classifying your Opponents''.

\section{After the turn}

What's the best possible hand on the turn? Has the best possible hand
changed since the flop? What's the best probable hand on the turn??
Has that changed since the flop? Did the turn help the caller?? Did
the turn complete any obvious draws?

How much did the leader bet, relative to the pot? What pot odds does
the caller have? What hands does the leader have?? What hands does the
caller have?? Who is going to win this hand?? How big is the pot?

\section{After the river}

What's the best possible hand on the river? Does it make sense for the
caller to have made that hand?? Did the river complete obvious draws?
How much did the leader bet, relative
to the pot? Should the caller call, fold or raise?? What hand did
the leader show? What hand did the caller show?

\section{Replay the Hand}

Once you've seen the hands shown down by the players, congratulate yourself
silently if their hands were close to your guesses. Often you'll only
see the winning hand, the losing caller will muck after he sees
he's lost. Now rewind the hand all the way back to the Preflop round,
and replay the betting action on every street, with the knowledge of
what cards they were holding. See if the bets made by the players make
sense from their cards. Were the bets sized well? Was the leader too
aggressive? How would you have played the hand differently?

Were there any mistakes made by the players in his hand? Try to find
one snippet of information from this hand for each player.

\section{Putting it all together}

First hand of a tournament, blinds are 25/50 and everyone has 5,000 in chips.

\subsection*{Preflop}

\subsubsection*{The Action}
Under the gun, an office worker with a bit of grey hair in his forties,
raises to 250. I'm next to act and have J3 offsuit and fold.
No need to get fancy on the first hand of the night. Other players fold,
but the Button, a younger office worker aged about 27, calls. Both blinds fold.
Heads Up to the flop.

\subsubsection*{Q\&A}
Now I'll ask and answer some questions, and guess some conjectures. \\
What are the blinds? 25/50 \\
How many players dealt to? Eight \\
Who raised preflop? Under the Gun \\
What hands does he have?? AA, KK, QQ, JJ, TT, AK, AQ, AJ, KQs \\
Who called him? The Button \\
What hands does he have?? AQ to A8, suited Broadway cards, 22-99 \\
Who has better position? The Button \\
Who has the better preflop cards?? Under the Gun \\
How big is the pot? 575, we'll call it 600 for simplicity \\
How big are the remaining stacks? 4,750 each player.

\subsubsection*{Remarks}
I'm assigning strong hands to the Under the Gun raiser and
weaker ones to the Button.
If the Button was superstrong (AK, AA, QQ, KK) I'd expect a reraise
preflop, to be sure to get the blinds out of this pot and be Heads Up
against the raiser. If Under the Gun had smaller pairs like 22 to 99,
he'd probably just limp and try to flop a set, especially on
the first hand of night when the stack size is still 100 big bets.
The raising amount of 250 (5x) is on the high side and could be
a hand like JJ or AJ/AT. A lot of players hate playing JJ and raise
abnormally highly with it, happy just to win the blinds or get
just one caller. It's a hard hand to play postflop and lots
of players are uncomfortable with it.

\subsection*{Flop}

Preflop betting is complete and it's time for the flop.
The flop comes \Kc\eigd\fived.

\subsubsection*{The Action}
I watch Under the Gun for a tell but don't pick
up on anything. He thinks for a few seconds and bets 300. The Button calls
quickly and automatically. My J3 offsuit completely missed this
flop (as it misses most flops) and I have no regrets for folding it.

\subsubsection*{Q\&A}
Time for some post-flop round questions and conjectures. \\
What's the best possible hand on the flop? KKK \\
What's the best probable hand on the flop?? AK, KQ, KdJd, QQ, JJ \\
Is the preflop raiser still ahead?? Yes \\
Did the caller like the flop?? Yes \\
What draws are possible? Diamond flush draw, 67 open ended straight draw \\
How much did the leader bet? 300, half the pot. \\
What kind of bet is this?? Cbet with probable best hand \\
How many chips does the leader have left? 4,450 \\
What should the caller do here? Raise or fold, call if trapping with set \\
What does the caller do here? Calls again \\
How big is the pot? 1,200 \\
What hands does the leader have? AA, AK, QQ, JJ, KQs \\
What hands does the caller have?? K-x, A8, 88, 55, two Diamonds.

\subsubsection*{Remarks}
I'm narrowing down the hands the players might have based on the betting
on this round. For the leader I've discarded KK because I doubt he'd bet
half the pot and risk losing the Button on such a power hand. Sets love
to check-raise and the flop bet risks winning a small pot right now. I doubt AQ
or AJ would bet here. His hand is probably AK or KQs, with the chance
of AA, QQ, or JJ. If he holds QQ or JJ, a flop bet is smart because
he still has a chance to have the best hand and if he's behind he'd like
to find out quickly. Since he got called, he wouldn't be feeling too good
if he does hold QQ on a K85 board, because a calling opponent could
well have a King. If he's got QQ or JJ his continuation bet has now
been called and he'll probably want to slow down the action in case
he's up against a King.

The caller's hand I expect is K-x where x is 9 or better, A8 (middle pair,
top kicker, the caller would like to take a turn and see if he improves),
88 or 55 (slowplaying a set, with the leader betting into him). A diamond
draw or 76 is the only drawing hand he'd be playing.

Could the caller have flopped two pair? He called a good raise preflop,
and a two pair hand on this flop would mean he had one of K8, K5 or 85.
I don't think any of those hands call preflop. This flop is widely spaced
enough that two pair really looks unlikely.

Remember to quickly consider the two pair chances on every flop, and
discount the chance of someone having two pair on a wide flop, and
consider them a lot better on a narrow flop like QJ9. This shouldn't
take you too long at the table.

\subsection*{Turn}

The turn is the 4 of Spades, making the board \Kc\eigd\fived\fours.

\subsubsection*{The Action}
The 4 looks like a blank. It does complete the straight draw for
the holding 76 so it's not a complete brick.

The leader bets 500 and is again called automatically by
the Button.

\subsubsection*{Q\&A}

This pot is building up. More questions to ponder. \\
What's the best possible hand on the turn? 87654 straight \\
What's the best probable hand on the turn?? Still AK. I don't think any
two pair combinations make sense here, could either player be playing
K8, K5, K4, 85, 84 or 54 based on the action so far? I don't think so. I
don't think 76 makes sense preflop. \\
How much did the leader bet? 500 into a pot of 1,200, 40\% of the pot \\
Did the caller like the turn?? Not unless he holds 76 \\
How big is the pot? 2,200 \\
What hands does the leader have? AA, AK, KQs. I've narrowed out QQ and JJ,
since he continues to bet on the turn, even in the face of the Button's flat
call on the flop. \\
What hands does the caller have? K-x, 88, 55, Diamond flush draw.

\subsection*{River}
The river is the 8 of Spades, pairing the board and making
it \Kc\eigd\fived\fours\eigs.

\subsubsection*{The Action}
The leader pauses then checks. The Button bets 800, which the leader calls.
The Button shows \Kd\Jc\ and the leader shows \As\Ks. The Button
checks the board for a second to make sure it's not a split pot, but
sure enough the Ace kicks and the leader wins a nice pot.

\subsubsection*{Remarks}
The betting at the end where the leader checked then called is nice. He
pretends to be scared of the repeat 8, and induces a biggish bet from the
Button. The Button should have simply checked the river, his KJ has some
showdown value but the 800 bet is only getting called by better hands.

\subsection*{Quick Replay}

Now that I know the hands were AK and KJ, I'll do a quick replay
of the hand. AK suited was a nice hand to raise. KJ call from the
Button was OK, I'd have made the same call.

The flop of K85 was good for the Button, but he could be in kicker
trouble. A raise of the flop bet of 300 to 700 would either win the
pot there when he's ahead or save him money when he's behind. Instead
the Button just calls on the flop and also calls on the turn. The
Button hasn't put any pressure on Under the Gun in the whole hand.

\subsection*{Patterns}

Now that the hand has played out and I've been paying attention the whole
way, let me pin a couple of mental notes on each player.

Leader: Will raise confidently with AK under the gun and will Cbet when
he hits. Bets the turn when it's likely he's ahead. Sizes bets well, about
half the pot each time. Can induce bets on the river. Tight/Aggressive so far.

Button: Calling station, doesn't pressure opponent with top pair
average kicker. Can make a play at a pot on the river if checked to.
Check to induce worth trying against him.

Keep your notes accurate and factual. The Button isn't a fish,
and he's not someone who calls with a wide range of hands (King-Jack
is fine in a fast tournament). He is one of those
pub players who didn't raise to put pressure back on the leader, and
took a bad stab at the pot on the river. There are a lot of these
players in pub poker, it seems a natural way to play poker. Now had he
raised he would've been giving away chips, but as I said earlier he
gave away more chips with his flat calling and his river bluff. A
raise on the flop would have been more aggressive play.

\section{Hard work}

This is a tremendous amount of observation and work, in a pretty simple
hand where AK raised, KJ called and AK held up on a board of K8548. But
by tracking the hand the whole way, you've practised your hand reading skills
and picked up on the playing styles of two of your opponents and spotted
some mistakes in the play of the Button that you might be able to
take advantage of soon.

After a few nights of working like this, you'll find that keeping track
of the pot size, the remaining stacks, and the possible and probable
best hands on the flop, turn and river become automatic. That's great: you
free up your brain to concentrate on better and better hand reading. Your
skills at answering the conjectures will get better and better and your
hand reading will become more and more accurate.

A lot of the work in putting the players on hands uses a process
of elimination. I start with a range of hands each player could have,
and I narrow it down based on each betting round. A lot of the narrowing
down comes from thinking along the lines ``He wouldn't do that if he
had that hand'', such as when I eliminated KK from the leader's hand
on the K85 flop.

\section{Keep your work to yourself}

It's important to keep your thoughts to yourself. You're working
hard on your Opposition Research, every hand, and other players aren't.
Sometimes other players want to discuss a hand and they start by asking
very simple questions of fact about events that happened less than
2 minutes ago! ``What happened preflop? Did anybody bet on the turn?''.
These players are trying to replay the hand, but they weren't following
the action closely enough to be able to do it properly.

Players often want to dish out advice, especially if someone has had
pocket Aces cracked. This advice is usually worthless, although you
can get an insight into how the other players think. I rarely offer
advice on how a hand should be played, I just look for the mistakes
that people made during a hand and think about how I can exploit
those mistakes myself.

\subsection*{Thanks for the Help}

Now and then a player at the table talks to me about how he's seeing my
game and some tells that he's spotted. This is talk I'm very happy to
hear. One good Vietnamese guy once picked off three of my bluffs in a short
space of time. I was betting preflop in position, Cbluffing the flop,
picking up a straight or flush draw on the turn, checking behind on the turn,
missing the river, bluffing when checked to and being called every time.
The Vietnamese guy knew from this pattern that I didn't have top pair
and didn't have much of a hand yet. I'd have been better off making a semibluff
on the turn and taking the pot down.

\section{Take a Guess}

I once called a decent raise from the Button when playing in the small
blind holding K5. I'd played against the Button a lot, and thought
his Button raise had some high card backing behind it, probably AJ or
KQ suited. The flop came 553, I checked, a middle player holding 66
bet 2,000 into an 800 pot, everyone else including the Button got
out of the way and I won the final chips of the guy with 66, who wants
to learn a bit about not overbetting the pot with low overpairs. Turn
and river were 9 and Q.

``Nice flop for King Five'' said the Button as I was stacking the pot.
``Yeah, sure was, what did you have, Ace Jack?'' I asked.
``Can you see my cards? Stop peeking at my cards!!'' he said.

My guess on his starting hand was correct!

