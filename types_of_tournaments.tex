\chapter{Types of Pub Tournaments}

% Last updated: 20200420

Even in the field of No Limit Holdem Poker Tournaments,
with blinds increasing after a fixed amount of time, there's still a
lot of variety in tournament conditions. This chapter
looks at the different types of Pub Poker No Limit Holdem Tournaments
and also different arrangements players make to win money apart
from the traditional first/second/third/minors prizepool payouts.

\section{Freezeouts}

A traditional No Limit Holdem Tournament is called a \textbf{Freezeout}.
There's a single, one-off entry fee, which is the most you'll have to
spend to play in this tournament. This is what your parents
think a poker tournament is. Everyone should start with the same number
of chips, although in many Pub tournaments there are Niggles
that give bonus chips to players who've arrived early or have spent
money at the venue. The players compete for the
money in the Prizepool and the top 10\% + 1 of the field win money.

Freezeout example: 30 players enter a \$22 tournament with a \$500
guarantee. \$20 of each buyin goes into the prizepool, making for a \$600
prizepool tonight. The TD keeps a rake of \$2 x 30 = \$60 and is also
getting some money from the venue; which is a business arrangement
between him and the venue. Tonight there are 4 prizes paid.

\begin{tabbing}
First Prize  \=: \= \$280 \\
Second       \>: \> \$180 \\
Third        \>: \> \$100 \\
Fourth       \>: \> \$40
\end{tabbing}

% FIXME: Better LaTeX lining up of the lines above. using a Box
% of some kind?? Not sure.. it's not too bad really.

Notice that the gaps in prizemoney go up as you climb the ladder
from bottom to top. Third gets \$60 more than Fourth, but Second gets
\$80 more than Third. And First gets \$100 more than Second. This is
typical of how a payout structure works.

First place pays a bit under 50\% of the prizepool. In a
big field, with a lot more players to be paid, first place can
be as low as 30\% of the total prizemoney, in a smaller game with
fewer people to be paid, First can be as high as 50\%.
Traditionally the biggest
percentage you can win is in a 10-player sit and go where first prize
is 60\%, second prize is 30\% and everyone else gets nothing and
the house keeps 10\%. Some Satellites are Winner-takes-all; for
example 10 people play a \$60 game with the first and only prize
being a seat for a Main Event valued at \$550.

% FIXME: check this and check some resources on the internet for this

\subsection{Niggles}

There are some Niggles in many Pub Poker Tournaments that operators
have introduced, that make even a simple Freezeout an
uneven playing field. Most of these niggles work by giving some
of the players bonus starting chips if they meet certain
conditions. Here are main ones you'll encounter.


\begin{description}

\item[Spend Cards] Many venues encourage players to buy beer and food
there. When you register you are given a card that you get stamped
when you buy beer and double-stamped when you order a bistro meal. If
you collect four stamps on your card you exchange it after the first
break for extra tournament chips, such as a 5,000 chip in a 20,000
stack tournament.

I hate spend cards. If a Pub has good beer or food I'll spend my money
there on the basis of value-for-money, I don't want to be nudged into
it just to get more Tournament Chips!

Poker players are notoriously tight spenders of money.
Many don't drink during a game because that affects their concentration.
There's not much in it for a venue having a bunch of people using
their floorspace if players aren't eating, drinking and spending like
the other patrons. Spend cards encourage this.

Pub Trivia Quiz players spend like drunken sailors. They're catching
up socially with their trivia mates, they'll get a few drinks and a
bistro meal and play in tonight's free Trivia, which is the only
reason they're in the pub in the first place. Almost everyone who
starts the quiz will stay till the end, unlike a Poker tournament
where people bust out and quickly leave. I haven't seen any numbers,
but surely a trivia quiz delivers much more bang for buck for the
Pub, unless the people playing the poker games are also Poker Machine
junkies and quickly put \$200 through the machines during the 15 minute
breaks. There's no need for Spend Cards in Trivia Quizzes, because the
players are spending up big at the venue and enjoying themselves already.
There are, however, two small niggles I've seen in Trivia Quizzes.

In the first niggle, the organisers want you on their email list, and
email you three tough questions that will be asked in next week's
game. You then go and look up the answers on Google and you go to the
game with a three point advantage on other teams who aren't on the
email list and will get some of those tough questions wrong. A tough
question is one you'd be unlikely to know, such as ``What's the
population of Canada to the nearest Million''?

The second niggle is in game format. Some Trivia Quizzes have adopted
the Big Final Question format from the TV Show Jeopardy! where you can
gamble all, none or any of your points and the winner isn't decided
until the Final Question points have been marked. I prefer a ``Most
points during the quiz wins'' format myself. The Big Final Question
definitely keeps all trivia players at the venue until the end of the
quiz because even if you're 8 points behind you can still dream of a
Big Final Question all-in double up.

Returning to the world of pub poker, Pubs can get in a lot of trouble
if they're promoting extra drinking. Happy hour is just about the only
promotion allowed. Spend cards could be seen to be encouraging
extra drinking and if someone was involved in a drunk incident
that went to court they could well have grounds to blame a venue for
promoting excess consumption of alcohol through poker
spend cards. To my knowledge this hasn't happened in Australia yet
but venues want to be careful.

\item[Early Bird Bonus] A lot of tournaments advertise a Registration
time a full hour ahead of the start time. 6:30pm registration, 7:30pm
start; it sounds like a formal dinner with pre-dinner drinks served
before being seated. Some tournaments reward players for being
at the venue before the tournament starts with extra
chips. The bonus can be awarded for being the full hour early, half an
hour early, or sometimes just on time and seated for the first hand.

As everyone is playing their early hands, the TD walks around
giving an extra big value chip to the early-registered players. The
late players are now playing a 20,000 stack game against some
25,000 stacks and they're already behind without losing a single pot.

The Early Bird Bonus is a genius way to again encourage extra
spending by the players. You're at the pub, the tournament doesn't
start for forty minutes, what are you going to do? Ooooh there's poker
machines in that room. This is one of the most clever Niggles that
there is.

\item[Member's Discount] Membership discounts have been part
of food and drinks prices for years now, and the concept
is being extended at some places to midsized Poker buyins
(\$30-\$70). I've played in a tournament with the buyin
\$35 for Members, \$40 for Guests and another where it was \$50
for Members and \$55 for Guests. The second was promoted as ``10\%
discount for Members'' but it's actually a 9\% discount for Members
(they save \$5 off the \$55 Guest price, which is a one-eleventh saving)
or a 10\% surcharge for Guests (they pay an extra \$5 on top of the
Member's \$50 price, a one-tenth surcharge). Yes, I'm a Maths Geek.

Member's Discount doesn't bother me too much, though the
cynic in me believes that any Guest surcharge money
goes straight into the organiser's pocket and not the prizepool.

\item[Must-be-a-Member of the Club] Some venues (in my experience,
the giant RSL clubs) enforce that you need to be a Member of the Club
to play in their Poker tournaments. In 2018 I turned up for a \$55
tournament a long drive from home to find that I needed to also cough
up \$20 for a club Membership, at a place I have no interest in visiting
more than once a year. Of course I only found this out at registration
time. Another RSL did this to me (with a cheaper club Membership) on a
Sunday in 2024.

This looks unethical to me. I bet there's no RSL club that
restricts Poker machines to paid-up Members, but some do it for
their Poker tournaments. Sure you have to be a Member to use
the gym or the snooker tables, but just to play in a poker
tournament??

% FIXME: The 5km exclusion might disappear through NSW legislation
% in which case the next paragraph will need to go

As always a Five Kilometre exclusion circle applies to registered
clubs, which are typically any place with poker machines that isn't an
out-and-out pub. RSL Clubs, Leagues Clubs and Bowling Clubs all have
the Five kilometre exclusion circle. This is a NSW State Government
restriction to limit problem gambling on Poker machines, it isn't a
pub poker niggle. Make sure you're legally allowed to be at the club
and you're not sneaking in; you don't want to win a big prize and have
your payment refused.

If I play at a venue more than twice in six months, and I like the
venue and the games and expect to return there regularly, I'll join as
a Member.

I'm a member of so many RSL clubs and bowling clubs that the
membership cards can't all fit in my wallet. Before going to a game I
make sure that I have the correct membership card with me.

\item[Prizepool not fully cash] Some prizepools include tickets
to other games, or bar money, as part of the prizepool. I once won a game
where the \$200 first prize was only \$120 in cash, \$25 in bar money
and \$55 to a season final on the second Saturday of the following month,
which I had no interest in playing. That game was a freeroll so I bit
my tongue and accepted the prize as it was; in a freeroll the organisers
can pay you entirely in Bistro Meal Vouchers if they want. Luckily for
me the venue was local so I was able to use my bar money easily
enough. Still, my \$200 first prize was only 60\% in cash.

Unless you're playing in a Big Game satellite, where the first prize
is an entry into a big Casino game like an Aussie Millions Event or a
Star Event, you should have the option of taking your prizemoney
entirely in cash.

\item[SMS spamming] Some tournament directors feel they have a
right to your mobile phone number. They don't.
I give my mobile number out, when asked, but always feel weak
for doing so. I then get SMS messages from the venue
for a few weeks afterwards telling me about upcoming games. Some of
these messages I welcome, some I don't. Your attitude to privacy will
be looser or stricter than mine. If you don't want to give out your
mobile number, stand up for your right to privacy and say ``I'm here
to play tonight's poker tournament'' and don't give it out.

When you do get SMS spams, there is an option to send STOP to a number
and you won't get those spams anymore.

\item[Facebook sharing bonus] Tournaments advertised on Facebook sometimes
offer additional starting chips if you've shared the information
about that tournament on your own Facebook, helping that post go
a bit viral. They used to offer a chip bonus for
checking-in to the venue through Facebook, nowadays you need to share
the information. I've seen players hold their mobiles up to the TD at
registration time to prove they've met the Facebook conditions and
they qualify for tonight's bonus starting chips.

I don't promote the poker aspects of my life on Facebook so I ignore
appeals to share poker tournament information with my ``friends''. I
will, however, recommend the Facebook group APW Informer, which has
good information about the bigger pub games at the \$100-\$350 buyin
level. If I need a smaller nightly pub game, I'll consult the NPL or
APL websites. As well as the Facebook group, APW Informer have a
website at http://apwinformer.com.au

\item[Gold coin doublestack] Some Freeroll games give you a double stack
for a payment of \$2. The TD can't charge any more than that or
players will start complaining that it's not a Freeroll anymore.
It's like an add-on that's available at the start. That token
money goes straight into the TD's pocket and not into the prizepool;
it's hardly worth complaining about (it's \$2 after all) and just adds
\$60 or so to the TD's profit for running the game.

\item[Winner's Jackpot Round] I've seen games where the winner of a
tournament gets a bonus no-poker-skill jackpot draw. The Ace, two,
three, four, five, six, seven, eight and nine of spades are shuffled
around face down on the table and put in a three by three tic-tac-toe
grid. The winner of the tournament is asked to guess where the Ace of
Spades is, a lucky draw 1 in 9 chance. The TD flips over the other
cards, one by one, and excitement builds as each flipped card isn't
the Ace of Spades. It gets similar to one of my least favourite TV
game shows, ``Deal or No Deal'', except there's no bank offers and you
win the jackpot or nothing.

I think games that run this like to advertise to their players that
they can win \$4,000 tonight but really they can only win \$1,000 at
poker and have a 1 in 9 shot at the other \$3,000 which is separate to
the poker game.

Some pubs run a weekly promotion called Joker Poker where one person
gets to pick a card from 53 cards pinned onto a display board. If the
person picks the Joker they win a jackpot prize. If they miss, that
card is now face up and next week's player has a better chance of
picking that Joker.

Here's the stupidest thing I've done in pub poker that didn't cost me
money. I once went to a Sydney pub at 5pm on a Friday because two days
earlier I'd seen on their ``what's on'' blackboard the words ``Friday
5pm: Joker Poker''. I thought they were running a poker tournament,
but it was the ``pick the Joker from the board'' lucky draw game.

\end{description}

Here is a summary of Niggles. Your pain factor will be different to
mine.

\begin{tabular}{|l|l|l|} \hline
NAME    &  FREQUENCY  & PAIN FACTOR\\ \hline
Spend Cards & Common  & 8/10\\ \hline
Early Birds & Common & 7/10\\ \hline
Member's Discount & Rare  & 3/10\\ \hline
Must be a Member & Rare  & 9/10\\ \hline
Prizepool not fully cash & Occasional & 9/10\\ \hline
SMS spamming & Common & 4/10\\ \hline
Facebook bonus & Rare & 2/10 \\ \hline
Gold coin doublestack & Rare & 1/10 \\ \hline
Winner's Jackpot Round & Rare & 10/10 \\ \hline
\end{tabular}

There's sure to be other Niggles I haven't come across yet,
especially in the Freeroll world where a venue is putting up the
prizemoney and they feel more than ever that they have the right
to some return from the players. I have a recurring
nightmare that right now there's a secret society of Poker organisers
who are meeting and coming up with devilish new Niggles to annoy us with.
In 2024 my nightmare came true, as I saw Winner's Jackpot Round
introduced.

\section{Rebuys, Lifelines and Add-ons}

A \textbf{Rebuy}\footnote{Geeky writer's note: Rebuy is now such a
  common word in poker that it has lost its hyphen. I spelled it
  re-buy for years but now it is simply rebuy. Buyin also lost a
  hyphen. Add-on still has it.}
is typically a second chance, pay-the-tournament-buyin-again
option that a player can use if he busts out in the early rounds
(typically, in the first three or four rounds, before the first
break). Some tournaments call this a \textbf{Lifeline} which is a
great name for it.

Sometimes multiple rebuys are allowed, though most Pub tournaments
with a rebuy use the simpler one-only Lifeline rebuy. You
need to be completely bust before you rebuy.

Casinos sometimes use rebuy tournaments as daytime entertainment
before their big tournament series, sometimes free entry events
but with very short stacks and multiple rebuys allowed. A correct
strategy for this tournament can be for players at the table
to rebuy as much as possible; so when the rebuy period ends there's
a lot of chips at that table and players from that table will
have chip mountains when tables merge and they go to
other tables, whose players were more timid in the rebuy
period.\footnote{Daniel Negreanu made 46 rebuys and two add-ons
  in a \$1000 rebuy event at the 2006 WSOP. More recently he has
  supported WSOP rebuy tournaments to be Lifeline rebuy only. I once
  made 19 rebuys in a \$10 tournament and needed to finish at least
  third to actually win money. I came second and won \$50 overall.} You
won't find tournaments like this in the Pub Poker.

If you're playing a tournament with a lifeline, you can change
your early level strategy and take extra chances on early
double ups, knowing that if you bust out your night isn't over and
you can buy a second shot at the tournament. You can bet and raise
on your flush draws and straight draws much more aggressively.

Some tournaments give you a special non-playing ``lifeline chip'',
which must be surrendered (together with the rebuy money) when you
make your rebuy. This is a convenient way for the Tournament
Director to make sure players rebuy only once, and to track the number
of rebuys made.

An \textbf{Add-on} is a final ``buy tournament chips with cash''
option. After the final big break, before play resumes, players still
in the tournament can make one final chip purchase. You should get
quite a lot of chips for your add-on; often equal to the tournament
starting stack.

Add-ons might look like Spend Cards where you're not even
getting any food or drink for your chips, but there are important
differences. The money from your Add-on is going into the Tournament
Prizepool, not the venue's profit. You get a lot more chips for
the price of your Add-on. They are transparently part of the
Tournament conditions which you will be aware of when you register.

You should always take the Add-on. Factor that into your expectations
at the start of the night; if you're playing a \$15 game with a \$10
rebuy and a \$10 add-on, expect this tournament to cost you maybe
\$35 but at least \$25. You'll save some money if you don't
need the lifeline, but \textbf{always} take the add-on.

You might dream in the early levels that you're going so well
tonight and you're so far in front that you can save yourself
the cost of the add-on. That's bad thinking and you should always
invest in the extra chips. Other players will be buying the add-on
so you need those add-on chips to stay in front.

Even if you've got 20,000 chips and everyone else has 5,000 chips at
your table, you should still take a 5,000 chip add-on so you're staying
as far ahead as you can. After the break you'll have a 2.5 to 1 chip
advantage (25k to 10k) compared to a 2 to 1 chip advantage (20k to 10k).
If this game was a freezeout you'd keep your 4 to 1 chip advantage
(20k to 5k) but sadly anyone who adds on has just caught up a lot to you
without even having to win a pot.

I've won two tournaments where I'd done great in the first session,
still taken the add-on and stayed alive in the tournament only because
I'd taken the add-on. If I hadn't taken the add-on I would've been
kicked out on my second bad beat but I still had chips
and went on to win. Of course in many more tournaments
I've paid for an add-on and busted out and cost myself a bit more by
taking that add-on, but for the price of twenty wasted add-ons, I've won two
tournaments. That's a bargain. One last time: \textbf{always take the add-on.}

\subsection{Trustworthy Tournament Directors}

In rebuy and add-on games, there's extra cash being taken by
the Tournament Directors after the initial buyin. You want
a trustworthy TD and a high confidence that this extra cash is
going into the prizepool. You can count the number of players
at the start of a tournament and know pretty well what the prizepool
will be in a straight Freezeout, but in a Rebuy/Add-on tournament
you can't keep track of how many players rebought and how
many Added-on. A well-run tournament will display this
information on the tournament clock screen; a badly run tournament
will have you worried that the TD is on the take.

\section{Bounty Tournaments}

A Bounty Tournament is a tournament with a traditional prizepool
plus a secondary prizepool of Bounties. The bounty
is a small prize you get when you knock a player out.
Each player gets a non-playing special Bounty chip.
When you knock a player out, he gives you his bounty chip.
He himself cashes out any bounty chips he's won from knocking
other people out earlier in the tournament. Each bounty chip is worth
some small, fun amount like \$5 and it's a little extra work for the
Tournament Director cashing them in for busted players.

One minor annoyance of Bounties is when you win a massive pot
from a player, crippling him, but he had a few more chips than
you and stays on in the tournament, and immediately loses the
next hand to someone else and that third player gets the
Bounty chip. You've got all his tournament chips, but someone
else gets the cash bounty chip.

One good thing about Bounties is there is less ``checking it down''
from active players against an all-in player; because the active
players are in competition to try to win the all-in player's
Bounty chip even if the tournament chip amount of the all-in player
isn't much.

I like Bounties, though the field size needs to be 40 or more to
make them worthwhile. A \$27 game of \$20 prizepool, \$5 bounty
and \$2 rake is a nicely priced game.

\subsection{Jackpot Bounty Players}

Some tournaments designate a player or small group of players
to be the Bounty Players for that night; if one of them wins
the tournament they will score an extra jackpot on top of
the prize money involved. This adds a lucky draw element to
poker and adds extra excitement to the Bounty Players chosen.

The Tournament Director announces who the night's bounty players
are soon after the tournament begins. It's never been me.

\subsection{Mystery Bounty Player}

Sometimes the Tournament Director doesn't announce who tonight's
Bounty Player is, until he gets knocked out. Everyone playing in the
tournament hopes that they're tonight's mystery player and they'll
win bonus money if they win the tournament.

\subsection{Bounty Knockouts}

I haven't played in a tournament with a Bounty Knockout but I
mention it here for completeness. A Bounty Knockout is a price
put on a player's head, if you knock that particular player out
you get a prize, on top of his remaining chips. It's like a
Bounty Chip Tournament but the bounty is only one or at most
a few players not everyone in the field. The Bounty Knockout player
could be last week's winner, or a well-known regular chosen
by the TD, or just randomly chosen from the field. The Bounty Knockout
players have a price on their head.

I like to tell people in tournaments with any kind of Bounty
that there's a Bounty on my head and that Bounty is the first
place prize. I'll be coming first or second in tonight's game (that's
how good I am) so if you knock me out, you'll get the first prize.

\section{Last Longer Bet}

A Last Longer Bet is a side arrangement between a few regulars
where they put in extra money into a private prizepool and the player
in that group who busts last in the tournament (the last man standing)
gets that money, winner takes all. It's a tournament inside a tournament.
Some good TDs will run Last Longer Bets and hold the cash.

I'll normally take part in a Last Longer Bet if I know there's one
going. They can be bunch-of-mates arrangements and
they don't have to let me in on it if they don't want to. \$20 in my
experience is the ideal Last Longer Bet amount per entrant.

I love Last Longer bets. Whenever you win a tournament, you get a
nagging annoying feeling that some of your prize money went to second
place and third place and the minor placings and the tournament wasn't
winner-takes-all. There's no minor placings in a Last Longer Bet. If
you last the longest, you get it all.

\newpage

\section{Swaps}

% FIXED
% This section is an A4 standalone page, stop fiddling with it and please
% do not make further changes to this \section{Swaps}
% This means YOU, Stuart!!

Swaps are an arrangement between two people where they
agree to share part of the prizemoney should
either make the money tonight. I've done regular swaps of 10\% with a mate and
done well out of them; once he paid me
\$170 after he made \$1700 in a \$110 tournament and another time
he refused my 10\% swap offer in a \$50 game that I
went on to win for \$2000.\footnote{He'd
just done a 10\% swap with another mate so he had enough swap action
already, otherwise he would have taken up my offer. The
guy he swapped with that night was a chump and that's
overstating his poker ability. I remind my swapmate about this once
a month, minimum.} Instead of giving \$200 to my mate, I kept the whole
\$2000.

The swap amount is on your total prizemoney. If your buyin is
\$110 and you get back \$1700, you pay out on the full
\$1700 not your profit of \$1590. That's because both you and your
swapmate paid the same tournament buyin.
In this tournament you finish with \$1530 (up \$1420) and your swapmate
finishes with \$170 (up \$60).

If you both make the money, you both pay out on the swap, the
higher finisher will pay some money to the lower finisher but a lesser
amount than if the lower finisher didn't cash. If you get heads up
with your swapmate for first and second prize, you can cancel the swap
and negotiate a traditional prizemoney split. Extremely unlikely.

Keep your swaps private and don't swap more than 20\% of your action.
Swap an even amount with your friend. (5\% for 5\%, 10\% for 10\%).
If he's a bad player and he'll never win and you sometimes will, don't
swap with him at all and find better poker friends.

If you make it to the big money, discuss with your swapmate before you
do any prizemoney splits. He's getting part of your money tonight and
wants you to win as much as possible.

In May 2019 I cashed in a huge \$175 tournament with a first
prize of over \$30,000! 123 players got paid (there were four Day Ones
and over 1,230 people overall entered). I finished 105th for \$500, my
Swapmate ran deep and finished 20th for \$900. We'd agreed to a 5\% swap.
5\% of my cash was \$25, but 5\% of his was \$45, so he gave me \$20.
% Amusingly my swapmate was drawn at my table at the start of Day 2 of
% this huge event (330 day two runners, 37 tables!) and stayed alive in
% the tournament well before the money by rivering me, which he will
% remind me of for months. If he'd missed the cash and I'd finished my
% 105th, I would have paid him \$25, but his deep run made me \$20, so
% getting rivered made me \$45! He finished with \$880 and I
% finished with \$520. We're both happy.

Because the first prize was so huge, I only wanted to do a 5\% swap.
For a \$2000 first prize or so I normally swap 10\%. If nothing else,
the maths is easier. For a 5\% swap, divide by 10 and then halve it.

Here's my table of swap guidelines. Ballpark, single night small field
games have a first prize of 15-20 x buyin, daylong tournament bigger
field games will pay a first prize of 30-40 x buyin.
These are my guidelines only. Work out your comfort zone and make the
swaps that are right for you.

\begin{tabular}{|l|l|l|l|} \hline
BUYIN      & FIRST PRIZE    & SWAP \% & COMMENT\\ \hline
Free       & \$0-\$100      & 0\%     & Don't play games this small \\ \hline
\$10       & \$200-\$500    & 0\%     & Not worth making swaps \\ \hline
\$22-\$55  & \$500-\$2000   & 10\%    & Swaps are fun at this level \\ \hline
\$80-\$500 & \$3000-\$20000 & 5\% or 2x5\% & You can swap 5\% with two mates \\ \hline
\$1000+    & \$40000+ & ?? & Don't play games this big \\ \hline
\end{tabular}

\section{Guarantees}

Some tournaments are now offering a minimum guaranteed prizepool. You
don't want to go out of your way to a \$22 poker tournament on the
other side of town, get there and find out that there's 4 players tonight
and first prize is \$50 and second prize is \$30. You'd rather have
done something else with your time than go to a nothing Holdem Tournament.

Guarantees protect a player against this disappointment. A venue
advertising a guarantee for a game will make up the shortfall if the
player buyins don't cover the guarantee. Our four players of a \$22
poker tournament with a \$500 guarantee could be playing for prizes
of \$250, \$150 and \$100 even though it's only cost them \$22 each.

Naturally venues won't be happy making up shortfalls like this and
their guarantee will usually be comfortably covered by player numbers.

Some places try to weasel out of their guarantees on quiet nights.
For a few months the APL website carried this disclaimer:
``Any Guarantees listed above are indicative only and all effort
is made to ensure that they are correct. APL will not cover
any guarantee listed incorrectly, please check with your local
host or venue to confirm the guarantee if your decision to
play is based solely on the guarantee.''

Some venues advertise their games by saying how much bonus
guarantee money there can be. Here's an SMS from late December 2024,
word for word but with the venue changed.

% Pendle Inn -> The Localpub
``Monday night poker is on tonight at The Localpub. Massive value, only 2
tables last week and \$1000 was paid out, don't miss out on easy
\$\$\$ tonight. 7pm Start Late Rego till 9pm. \$25 Entry. \$1000
Guaranteed Prizepool''.

The guarantee actually works against players
boosting a poorly attended game through word of
mouth. They're more likely to keep this game a secret and turn up to a
poorly attended game with a better chance of winning that free money
in a smaller field. Extra money in the prizepool that's put in by force
from the guarantee and not paid for by the players is called an
\textbf{overlay}.

One group I play with sends out a Text message advertising a big
guarantee; but it's the sum total of the guarantees of their games
across their venues for that night. \$8,000 Guaranteed tonight; a
\$6,000 Guarantee at a 6:30pm game and a \$2,000 Guarantee at
a 7:00pm game across town. I guess if you bust out of the 6:30pm game
very early you can still make the (slightly) later game.

\section{Tuesday night Pub Poker}

Here's a listing of Sydney games in the National Poker League
for a Tuesday night in October, 2018. It should give you an
idea of the choice available and the different options available.
Of the 11 games, only 5 of them were classical Freezeouts.
I don't know which Niggles applied at any of the games.


\begin{tabular}{|l|l|l|l|} \hline
Suburb & Buyin & Rebuy/AddOn & Guarantee \\ \hline
Fairfield        & \$25 & \$25+\$50 & NONE \\ \hline
Hoxton Park       & \$10 & \$10+\$10 & \$500 \\ \hline
George Street CBD & \$20 & FREEZEOUT & \$400 \\ \hline
Chester Hill & Free & \$10+\$20 & \$500 \\ \hline
Collaroy & \$25 & FREEZEOUT & NONE \\ \hline
Guildford & \$28 & FREEZEOUT & \$1000 \\ \hline
Windsor & \$22 & FREEZEOUT & \$500 \\ \hline
Seven Hills & \$20 & \$10 rebuy & \$500 \\ \hline
Hurstville & \$15 & \$10+\$10 & \$1200 \\ \hline
Dundas & \$6 & FREEZEOUT & \$500 \\ \hline
Pennant Hills & \$8 & \$10+\$10 & \$500 \\ \hline
\end{tabular}

A good range of games there, that shows how much option you
have in just the National Poker League on a Tuesday night.
I don't like the look of the Fairfield Add-on being priced at \$50
so I won't play that game. The big Guildford guarantee looks
good, so does the Hurstville one.

I didn't show it in the table, but there was no variation in start
time; all games started at 7:30pm, the bigger games had late buyin
up until 9:00pm. This is typical of weeknight poker
tournaments. You give up your night to play poker,
you want to get to the venue a bit after 7:00pm and hope to still
be there when the tournament finishes after 10:30pm.

Out of these games, I'd play the Guildford game, unless I had a big
rebuy tournament on the horizon and felt like some rebuy practice, in
which case I'd play Hurstville. You get better players and a better
game with the bigger buyins; so even though at Dundas I could turn a
tiny \$6 into a \$200 first prize, I wouldn't like the game and I'd be
up against \$6 players who had no skin in the game
and I wouldn't feel an investment in the game either. The Windsor game
looks OK but it's too far for me to drive.

\section{Raked cash games are illegal}

In the APW era (2009-2011), there was a popular
cash-game-in-disguise format that some venues ran
called \textbf{Chip-Chops}, or \textbf{Timed Tournaments}.
You got chips to the value of your buyin (less a 10\% buyin rake), the blinds
never went up and after a fixed amount of time the tournament was
declared over and everyone got cash back to the value of their chips.

Rake wasn't taken out of each pot, but it was on each buyin.
It cost you \$22 to get \$20 worth of chips, \$55 to get
\$50 in chips and so on. The small Chip-Chops had blinds of
50 cents/1 dollar and the bigger Chip-Chops had blinds of
1 dollar/1 dollar or 1 dollar/2 dollars. Most Chip-Chops went for 90
minutes, sometimes the big Chip-Chops went for 2 hours.

Just about the only practical difference between a
Chip-Chop and a real cash game is that in a real Cash game
you can cash out early; you can if you like play one hand,
triple up and make it a very early night and cash out.
You can't do that in a Chip-Chop, you have to sit through
the whole length of the tournament before you could cash out your
remaining chips for real cash.

Busted players could rebuy or simply leave the tournament,
at their choice.

These games were put to an end by the NSW Office of Liquor,
Gaming and Racing (``OLGR''), in April 2011, at the instigation of The
Star Casino, who pay a lot of money to the State Government to have an
exclusive live cash poker license in New South Wales.

In 2020 the OLGR has been renamed to simply
``Liquor and Gaming NSW'', it looks like Racing has been spun off to a
different department. The rules on poker are in
the PDF document ``Fact Sheet FS3001: Poker Tournaments'' from
the website https://www.liquorandgaming.nsw.gov.au.

Paraphrasing slightly to keep this excerpt to less than four paragraphs,
FS3001 says ``As the chips in a Chip-Chop tournament
have monetary value, the game is being played for stakes. As a result,
the game would be unlawful if (a) the dealer is not a participant in the game
(b) a person other than a participant in the game receives a payment
or benefit from the playing of the game (c) a payment is made for the
right to participate in the game or to enter premises where the game
is played''

There are extremely heavy penalities for playing in an illegal game,
fines of thousands of dollars and even jail time!
Organisers risk 2 years imprisonment and players risk 12 months.

A buyin rake of \$55 for \$50 in chips obviously
makes the game unlawful from item (b): a person other than a
participant in the game receives a payment or benefit from the playing
of the game.

An early version of this book had a chapter on Chip-Chops and
Chip-Chop Strategy, which I've taken out. Chip-Chops are illegal and I
won't condone breaking the law. This book is entirely about
Tournament poker.

In 2020 some pubs are now running cash games again, but mindful
of the Liquor and Gaming NSW rules there's no rake of any kind and
players can join and leave the game at any time.

None of this book is about Online Poker, but while I'm at it,
let me remind you that playing Poker online from Australia
\textbf{is also illegal}.

\section{Should Pub Poker be regulated?}

No. Philosophically and politically I'm against Nanny State
Government regulations. Though neither should Pub Poker
be a Wild West free-for-all where operators subtly rip the players
off. Normally in regulation situations I prefer a ``Code of Conduct''
where competing operators agree on principles and police each other,
rather than heavy-handed Government intervention. This works well
in the Media industry.

At the casino level, the Poker Tournament Director's Association is a
great industry body that improves the integrity of the game. Casinos
that don't have a lot of experience running Poker Tournaments can get
the rules from the Poker TDA and run a pretty good poker tournament. A
similar initiative in the Australian Pub Poker scene would do a lot
towards ironing out occasional problems such as inconsistent rulings,
non-payment of guarantees and other quibbles that arise from time to
time.


