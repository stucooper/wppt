\chapter{Flop Strategy}

% Last updated: 20190623

Holdem is a game of four betting streets; preflop, flop, turn and river.
Although I've split each streets into its own chapter,
keep in mind that good play in earlier streets set you up for good
play in later streets.

The flop is the defining moment of a Holdem hand. Three community
cards are revealed at once and you've seen 5/7 of your final hand,
and 3/7 of your opponent's final hand.

\section{Look for flop reaction tells}

As the flop is revealed, don't look at it. Instead look
at the faces of other players who are looking
at the flop, and look for tells in their reactions to the flop.
The flop will still be there a few seconds later for you to look at
yourself.

If you find yourself looking for flop reactions from another player
and instead of looking at the flop that player is looking back at you,
give him a wink. He's doing exactly the same thing as you, looking for flop
reactions. Expect him to be a cut above your average Pub Poker player.

Players who quickly glance at their chips after seeing the flop
like their hand and are considering betting or check raising.
Players who stare at the flop for a few seconds weren't helped by it.

You might also notice players sit up straight after seeing the flop
or players look away as though they don't like what they see. Both
of these tells indicate that a player does in fact like the flop.
If a player stays slumped and bored, he's probably lost interest
in the hand and will fold to any kind of action. This is straight
out of Mike Caro's famous ``Book of Poker Tells'' which has plenty
of photos of players reacting to flops.

% FIXME: Cite Book of Poker Tells properly

You won't always find tells watching the flop reactions of the other
players. But you'll \textbf{never} find any tells if you're not
looking for them! Watch the other players, not the flop.

\subsection{Which player should you look at?}

There might be too many people taking a flop for you to watch for
reactions from every player. If there's a preflop raiser, look
for his reaction. If there's a whole heap of limpers,
look at the downstream players. The upstream players you can look at
for chip betting tells.

\subsection{Preflop folders}

Look for any disappointed reactions from preflop folders. Players
out of the hand preflop have no reason to deceive and their
reactions will usually be honest. Some players out of the hand
will actually reach for their cards from the muck, and would like
to show their also-out-of-the-hand buddy the T3 that they folded
when the flop comes J33. If you pick up on this you should consider
it unlikely that an active player has a 3. If you hold AJ in this
hand you should expect it to be good. I'd probably bet and take this
pot, because I don't fancy being outdrawn by a Q or K on the turn.
Less observant players at the table still in the hand didn't notice
the folder fishing into the muck and are still scared I might
have the 3 with my bet. In my experience this is a very reliable tell.

\section{A poker rhyme}

One of the very first poker books I read was about Limit Holdem and
had the following little rhyme for how to act on the flop.

``Bet with the best, Good draw to invest, Fold all the rest''.

%% The book was amusingly called ``Winning at Poker'' and had a big splash
%% logo ``WINNING AT POKER!'' on the cover. It included the priceless
%% information that holding KK, a flop of KK4 was a ``green light'' flop for you,
%% and it was safe for you to bet this flop.

Here's the same information in just four words: ``Hit, Fit or Fold''.

There are some exceptions with position plays but this is sound
advice (and even sounder in Limit Cash Poker). After the flop
you've seen 5/7 of your final hand. You've only got two
more cards to come to improve your hand.

The chances of making runner runner flush are tiny ( 5\%, 1 in 20 )
so don't stick around if the only thing going for you is runner
runner flush potential. Keep this 1 in 20 odds statistic in your
mind, to stop yourself chasing this longshot draw.
Don't criticise a player if he stayed in a hand for runner
runner flush. If that was the only thing happening in his hand
(he had no pair or overcards or straight chances), he's a walking
ATM.\footnote{I once won a big sit and go when two tough players
busted to a new player who hit runner-runner flush on them
in a megapot that he had no business being in. Great luck for me.}

If I have three to a straight and three to a flush I'll sometimes
take a turn card and see if it makes an Open Ended Straight Draw or
flush draw. If I play \tend\nined\ and get a flop of \Ah\eigd\twoc\ I'll take a
turn card if it's not too expensive. If I get a J or 7 I've got
that Straight Draw and if I get a Diamond I get a flush draw.
Sometimes I'll get the \Jd\ or \sevd\ and have both straight
and flush outs on the river. This play is loose and doesn't
come off very often, but if I can hit a backdoor flush or straight
like this once in a while the payoff is enormous.

When I do make a backdoor hand like this, my opponents at the table
think I'm an idiot. I don't remind them that they kept me in
the hand cheaply, or that I had straight outs along with my flush
outs. I just let them think I'm a luck player as I enjoy stacking a
mountain of chips.

\section{The one-time two pair call}

Most of the time you call a bet on the flop on a draw, you'd like to
see the river as well. I'll sometimes call a flop bet to see if the
turn improves my hand to two pairs or trips right away.
If I miss and face a decent bet on the turn, I'll fold.
Often the bettor will be a bit wary of my flop call and might
give me a free river so I get to see both cards for the price of the
flop call. The following action turns up in a lot of Holdem hands:
FLOP Bet/Call TURN: Check/Check. It's only the really big pots where
every street is bet.

Say I hold 98 and the flop is K96. I have middle pair not-great
kicker and face a half-pot bet. I've got chips behind me so I make
the call. I'm hoping for a 9 or 8 straight away. If I get it I'll
probably go for a big check raise to take down the pot on the turn. If
the bettor has K5 he may be a bit scared of my flop call and give
me a free river. If he's got AK he will bet the turn as well but he'll
be in a lot of trouble if I've improved on the turn.

Sometimes I'll backdoor a sneaky straight with this move. The turn comes
a 7 and I'm up and down with 9876. If I make it on the river,
again the table will think I'm stupid, calling top pair with just the
middle pair, and getting runner runner for a miracle straight.

Flop play, turn play and river play are easiest from last position, not
first position. This is another reason why you have to play stronger hands
from early position. When you have a marginal hand in early position
it's hard to control the pot size and the amount of chips
you'll need to commit to this hand. From late position you can control
the pot size and use the extra information available to you from
earlier players' actions. You've got a better idea of where you're at
and a better chance of making the pot the size you want.

\section{Orphan flops}

An orphan flop is one that's missed everybody and is just waiting
for someone to adopt it by betting at it and winning it. Say there's a raise
and three callers. You'd expect the players in this pot to
have hole cards in the 9,T,J,Q,K,A range, or small
pocket pairs 22 to 99 to see if they
can hit a set. Anyone with pocket pairs TT to AA would have
re-raised preflop.

The flop comes 622. This flop is unlikely to help anyone.
A2 suited might have taken a flop but nobody's playing 72, 92, 32 to the
preflop raise. Bet about a half to two thirds the size of the pot,
and expect to take it down. If you get flat called you could
be in trouble (if I had the 2 I'd flat call).
But a lot of the time you can pick up small pot after small pot
with bets like this.

Picking up all these small pots is one of the big ideas behind
Doyle Brunson's classic NLH Cash Game chapter in Super System 2.
% FIXME: Cite SS2 properly
By always betting, betting, betting, Doyle keeps an aggressive image,
and can get hugely rewarded when he does hit a big hand and people
pay him off. Doyle also suggests big semi-bluffing, using the
profit from all those small pots to take a stab at a huge pot
with a good straight draw or flush draw. Even if he's called
he's still 30\% to win a massive pot. But sometimes he'll be
pushing a huge bet into a pot holding middle set, and he'll be
95\% to win that massive pot against top pair.

Brunson explains how his aggressive approach pays rewards
in getting his good hands paid off with his famous quote:
``I get a lot of action because I give a lot of action''.

Doyle Brunson obviously has magnificent feel for poker. He can
sense which are the orphan pots and which ones connected with opponents hole cards.
Next time you see what looks to be an orphan pot, put in a bet at it.
You'll probably pick it up. The table might resent this
naked steal, and someone might fight back at you two hands
later when you hold 66 on a rainbow flop of T62.

Never show your cards if you've successfully stolen an orphan pot.
You're just making it harder for yourself to steal the next
orphan pot that comes along. If the flop was 226 and you steal it
and someone asks what you had, say something reasonably midstrength
like ``Ace Six'' or ``Pocket Fives''. You're representing a feeler bet
instead of an orphan pot steal.

You'll feel great inside when you pick up the orphan pot. No need
to show the table that you're smart enough to identify orphan
pots and adopt them. We'll keep that our little secret. You'll want to
steal orphan pots again and again. Every time garbage comes on the flop,
think to yourself ``OK, this pot's going to be mine''.

Stealing these pots is an excellent bluff. Even if a player
suspects you're just stealing it, he probably won't call without
a hand because if you do have a hand this time he'll be facing
another bet from you on the turn and a third bet from you on the
river. So you're not just scaring him with the bet on the flop.
You're also scaring him with the threat of future bets on later
streets.

I like to bet very close to the pot when stealing an orphan flop.
If there's 5 limpers in a pot making the pot 500, I'll bet 400 to 600
to steal the orphan pot. Often people will fold pretty quickly to that
bet which is just how I like it, and the final player will see all these
folds in front of him and consider that he's not getting the right odds
to call my bet with just two overcards, and he'll fold as well.

400 to 600 is still a relatively cheap bluff, if I get re-raised I'll
Hollywood and fold. The raise tells me that this wasn't an orphan flop
after all. But it's a big bet in relation to the pot, so it
gives me excellent chances of taking it.

If someone pushes all in I'll have to fold because there's no way my
hand will win a showdown. Sometimes someone will be
slowplaying JJ and come over the top on a nothing flop of 226. I'd always
raise preflop with JJ and I'd hate there to be 5 people seeing the flop.
You're just asking to be outdrawn playing like this. Yes, you can get
winning action from A6 on the flop of 226 but you'll also get
losing action from A2 if you've let him take an unraised flop
through slowplay.

With JJ, I like to raise preflop, get one or two callers, get a Ten
high flop and get it all in against someone with AT or QT, who'll be
way behind and only a 20\% chance of winning. If I'm not deep stacked,
I'll occasionally lose all my chips to sets on these flops. It happens.

If I hit an orphan flop of 622 (I have the 2) I'll
bet it, and I also bet if I have the 6, since I have
what could be the best hand and there's too many overcards to improve
other players on the turn. So some of my bets at orphan flops are
Pure Bluffs, some are Value Bets with the likely best hand and a few
bets are monsters. I bet a lot at Orphan flops.

\section{Did the flop help your opponents?}

You can immediately see how the flop helped your hand, or failed
to help your hand. If you have 66, and the flop contains
a 6, that flop certainly helped your hand. If the flop is KJT that
flop missed you completely and you need to fold to any betting action.

After thinking how the flop helped your hand, think about how
it helped your opponents hands. Some flops look like they can't
help anyone (the Orphan Flops). Some flops look so
good that they've just got to have helped other players still in
the hand. Most flops lie somewhere in between.

You win big pots in poker when you have a strong hand and the board
has helped your opponent make a second-strong hand. Two pairs
versus a straight is a classic example. Once I held 89, saw a flop
of KT7, and I was check-raised all in by the small blind when a 6 hit
the turn. I beat him to the pot and he showed T6 for two pairs and
was a long way behind my straight. He stayed that way on the river.

See if the flop helped the other players in this pot, and merge
that information in with what you think their hand holdings are.
If you thought from the preflop action that a player held a high
pocket pair, JJ, QQ, KK or AA but he checks a flop of 67T maybe
you're wrong and he has a hand like AK, AQ, AJ, KQ which hasn't connected
at all with that flop. You could put out a feeler bet and see if
you can get him to fold. If he has TT he's slowplaying top set and
will probably check-raise here, allowing you to get out of the hand
cheaply. Remember what each player did preflop and see how they
follow it up with their new bets on each subsequent betting round.

Keep your bets pretty small if you're making a position steal. Your own
hand can't win a showdown, so you're hoping to pick up the pot
by stealing it from someone you hope has AQ and missed. A bet
of half-the-pot or just over should win you this pot. If me and
my opponent both have 5,000 behind and there's 1,000 in the pot
and he checked, I'd bet 600 at it. If he's got the set of Tens or
the overpair he'll probably raise here and it's only cost me 600
and I can fold. If I bet 900 and get raised that's
cost me 900. There's no need to bet too much on what is basically
a steal. 40\% to 70\% of the pot should be your range here.

Sizing your bets properly is one of the real arts of No Limit poker.
When you bet or raise, you get to decide how much you're betting
or raising. I'll have more to say on bet sizing later.

\section{Flop Texture}

Take a quick look at the flop and think about the starting cards this
flop fits well with. Everyone thinks about how well the flop helps
their own hands- good players pause and consider how well it helps others.

Flop 1: \Ah\tenh\sevs

Set for AA, TT, 77. Flush draw for anyone holding two hearts.
Gutshot straight draw for KQ, QJ, KJ. Two pair for AT, A7, T7.
Open ended straight draw for 89. Top pair top kicker for AK. Top pair
good kicker for AQ, AJ.

There were 5 limpers who took this flop.

Just about every hand helped by this flop you'd take a flop with.
Probably only T7 and A7 (from early position) would be folded preflop.
There's a few people in this hand, and there's no way this is an orphan
flop. There's too much straight and flush action possible, with two
cards still to come. The winning hand by the time all the cards are
out could be be at least a straight.

Flop 2: \Kh\tred\tres

This is almost an orphan flop but that King is a concern. Notice
there's no straight draws or flush draws possible on this rainbow flop.
If someone flat calls a bet they're either slowplaying a 3
(and why not slowplay? nobody will make a straight or flush on the next card)
or they have K-good kicker. If I had King-bad kicker here I'd raise to see
where I was at- and if I'm re-raised big I'll fold.

Flop 3: \Kh\Kd\treh. I hold \eigc\eigd.

KK3 is a better flop for 88 than K33 was. With the K33 there's a good
chance someone holds a King, and a small chance someone holds a 3,
and both holdings beat the 88. With this KK3 flop there's a small chance
someone has a King, and a good chance someone has a 3, and 88 only loses
to someone holding the King.

Almost everyone will slowplay a K on a flop of KK3. But there's not much
for an opponent to improve to that's second best. If an 8 rolls off on
the turn the King will get a lot of action from the 88, and lose to it.
You're almost better off betting holding KJ on a flop of KK3, and hope
someone with A3 loses their head.

I'd sometimes give a free card here but I'd never give a free river as well.
It's not masterful deception to check, check holding KJ on a board of
KK342 and someone beats my big river bet to the pot holding A5. He hit
a hugely improbable hand ( he was 2\% to win on the flop ) but by
not betting at all I've kept him in the hand and I've hanged myself.
It's better to win a small pot with a flop bet than it is to lose
my whole stack to an improbable hand that would never have got there
had I bet anything at all on the flop.

Flop 4: \tenh\nined\eigd.

This is an action flop with great draws to flushes and straights. Depending
on the betting action I could fold \Ks\Kc\ to this flop. I could already be
a mile behind a flopped straight or a flopped two pair.  And even when
I'm ahead, there's a lot of cards that can outdraw me.

% FIXME: how many outs for someone with JdTd? 16 twice??

There's too many ways for an opponent's hand to improve on this flop.
There's no law in poker that says you have to play your big starting
hand through to the bitter end no matter what the flop is. Wait for a
better spot if this flop has missed you.

\section{Raise to find out where you're at}

If someone bets at you on the flop and you have a mid-strength hand
and just call, you don't really know if you're ahead or not. Phil
Hellmuth, in ``Play Poker Like the Pros'', teaches a technique of
raising the flop bet and seeing what the original bettor does in
reaction to your raise. He will either fold to your raise, re-raise
yet again (giving you the information that you've paid for that you're
well behind) or just call your raise and possibly give you a free river
card by slowing down and not betting the turn.
Whatever action occurs, your raise has helped you get a better feel for
whether you're ahead or not.

%% In Limit Poker there's a move called the ``Free Card Play''. You have
%% a decent draw and you'd like to see your hand through to at least the
%% river. Someone bets at you on the flop, and you raise. That player
%% usually calls and just checks after the turn. If the turn hasn't
%% made your hand, you check behind and get to see the river for free.
%% Because the betting in Limit doubles on the turn, in a \$5/\$10 Limit
%% game a flop raise costs you \$10, but a flop call followed by a turn call
%% costs you \$15. Saving money when you're behind is smart Limit play.

This raise is a new kind of bet. It's not a best hand bet, it's not a
bluff and it's not a semibluff. It's a raise to get more information
on where you stand. It's an information raise.

Sometimes I raise and hope to be reraised a lot, so I can fold. I know that this
will be the cheapest way out of the pot for me. Getting out cheaply, for
a reasonable amount of chips, is a lot better than just calling the flop bet,
calling the turn bet and calling the river bet. If I'm behind in this hand
I want to find out now.

If you find yourself throwing chips into the pot without much thought
as to what you're doing and why, try asking yourself what kind of
bet or raise you're making. Say ``best hand raise'', ``information raise'',
``bluff'' or ``semibluff'' to yourself. Your inner voice has identified
the kind of bet that you're making, and you'll play the hand out
according to your plan better.

I encourage you to spend a lot of time understanding this idea. I've
found it very useful. Sometimes it costs you pots because you re-open
the betting and the original bettor can raise you enough to get you
off the hand, and if you'd stayed in you would have hit two pair or
trips on the turn. But much more often it will save you money by helping
you lose small on a pot you could have lost a lot more on. And sometimes
it'll show you that you're ahead and help you play the turn and the
river properly.

Most of Hellmuth's book is written on Limit Poker, where you can always
call down to the river with a hand, whereas in No Limit you can get
raised your entire stack. I still find ``raise to get
information'' a great play in No Limit. Sometimes you need to
be able to get away from the hand, if the information you receive
is ``you're a mile behind''. In a Limit game you could call all the
way to the river and make completely sure you're losing this hand, in
a No Limit game you need to trust your reads.

``Raise to find out where you're at'' sounds a risky strategy,
but really it's a lot better than just calling, calling and calling.
That's why when you face a bet, you should consider \\
(1) Raising \\
(2) Folding \\
(3) Calling \\
in that order. Calling is usually the worst of your options.

One poker phrase you might have heard of that encourage aggressive play is
``If it's good enough to call with, it's good enough to raise with''.

\section{The Draw Shove}

In middle and late stages of tournaments there's often an all-in bet
on a wet flop, from a player who has flopped a straight or flush draw.
This bet is called a \textbf{semibluff} and it's a powerful move.
It can win the pot outright if nobody calls and it can win a huge pot
if someone calls but the draw gets there on the turn or the river.
This gives the semibluffer ``two ways to win''. The combination
of 30\% chance of the draw coming in and a 25\% chance (or so) of
everyone folding to the flop shove makes this play a winner.

The Percentage chance that people will fold to your bet is
called Fold Equity. It's different against different opponents
and on different boards and on different bet sizes. If your opponent has
the Nuts, you'll never get him to fold, if your opponent completely
missed the flop, you'll easily get him to fold.

Draw Shoves turn up a lot in Pub Poker so I'll cover them in detail.

\subsection{Making the Draw Shove}

Have a think about who your opponent is in the hand and if he'll
fold to your draw shove. If you're up against a Payoff Wizard,
you might be better off just calling bets and being paid only if
your draw comes in, but getting paid the maximum when it does. It's
great in poker getting paid off big on the river. In this circumstance
you don't have Fold Equity (Payoff Wizards always call) but you've got
maximum Implied Odds.

Don't just shove with draws, shove with Top/Top and Sets on occasion.
Remember the Board is wet which means that your opponent will sometimes
have the draw himself and call on that basis, but it also means that
sometimes he'll put you on a Draw Shove and call very weakly when
he's got middle pair or something and you're a massive favourite and
you don't even have to worry about the draw coming in for him! This
is a great feeling in poker when you've basically Value Shoved and
got called.

\subsection{Defending the Draw Shove}

Firstly you need to have flopped a decent hand to defend against
the Draw Shove, Top-Top, Two Pair or a Set. Often a draw shover
has a Nut Flush Draw with Ax suited, and if you don't have an Ace
yourself he has an overcard Ace that he can win with if an Ace
falls on the turn or the river (giving him 12 outs not 9).

Before calling a Draw Shove, take some time and figure out if the
bet is indeed a Draw Shove. It could be a Value Shove from a player
scared of the Wet Board and trying to price out draws. You don't
want to call Value Shoves when you're virtually no chance against
two pair or a set and you haven't even got the Primary Draw yourself.
The ``have a real hand'' stipulation in the previous paragraph should
protect you against this.

One interesting hand you can call a Draw Shove with is a better draw.
The flop is JT9 and you've got QJ for top pair, open-ended straight
draw and you're up against J8 who has top pair, open-ended straight
draw but you've got the better kicker, and the better straight draw,
and if the draws miss by the board running out brick/brick you end up
winning on a small difference in kicker value. If you hold an Ace-high
flush draw and someone draw shoves at you, you normally have to call
and hope that your draw gets there. When the hands get turned over,
the hand the draw shover hates to see is a hand that's ahead of him
already and also has the draw covered.

Sometimes you correctly call the Draw Shove and the other player's
draw gets there. It happens. That's poker. You feel terrific when
you see his cards turned over and you realise you've made a great
call and you're the 70\% favourite and once the draw gets there
you're feeling dreadful and you're out of the Tournament. Nevermind,
you played the hand very well (so did the Draw Shover;
it's possible for two players to both play a hand of poker very well
and it's up to the Card Gods to determine the winner).


