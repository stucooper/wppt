\chapter{River strategy}

% Last updated: 20181016

Once the river is shown, the Board is complete. There's no more
drawing or protecting your hand. Your chance of winning the pot
is now either 100\% or 0\% and unless you've got the Nuts you
can't be entirely sure that your chance is 100\%.

Betting, Raising, Calling and Folding on the last round of betting
is a balance of probabilities. Using the information you've seen
so far, you make your best decision.

\section{One pair is not a Big Pot Hand}

Now that all the Board cards are out, you can see just how big
your final hand is. A big preflop hand like \Ah\Kh\ might have
shrunk to nothing on a board of \Jc\eigh\fivec\ninec\tenc.

One pair, even an overpair of \Ah\As, is not a Big Pot hand once
the final Board is out. It loses to two pair, trips, straights,
flushes and full houses. Don't make a big pot on the end with
just a one pair hand.

\section{Rewind the hand}

During the hand you've been putting your opponent on a likely
hand, and narrowing that range based on the action on every street---
preflop, flop and turn. Review my hand rewinding technique in the
Chapter ``Putting players on a hand''. Does the action on the river
fit with what you expect? Often a bluff just smells wrong, because
it doesn't make sense with the earlier streets of betting.

\section{Did the river change the lead?}

A lot of poker hands are a battle between a made hand and a drawing
hand. Have a look at the river and see if it completed any obvious
draws. If you've been betting AQ all the way on a QT9 board and the
runout is QT954 then you're likely still ahead. However if the river
is a K or an 8 it completes an obvious straight draw for Jx.

\section{Value Betting}

If your best judgement is that you've had the lead since the flop
and you've still got the lead, because the river was a Brick
that didn't complete any draws, you should normally keep value betting.
If the Board has been wet the whole time, better made hands than
yours should have raised earlier to avoid losing the pot when the
draws come in--- they'd want to protect their hand.

This logic doesn't work for all players--- some players are happy
to take the chance of being outdrawn, and bet the river only once
the draws miss and they're confident their hand is good. I don't
like that style of play in a Fast Tournament or from a
Risk/Reward standpoint. You can't ``play it safe'' in No Limit Poker,
you have to get your money in when there's still cards to come, in
circumstances where you can get outdrawn.
