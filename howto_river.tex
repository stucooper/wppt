\chapter{River strategy}

% Last updated: 20190628

Once the river is shown, the Board is complete. There's no more
drawing or protecting your hand. Your chance of winning the pot
is now either 100\% or 0\% and unless you've got the Nuts you
can't be entirely sure that your chance is 100\%.

Betting, Raising, Calling and Folding on the last round of betting
is a balance of probabilities. Using the information you've seen
so far, you make your best decision.

\section{One pair is not a Big Pot Hand}

Now that all the Board cards are out, you can see just how big
your final hand is. A big preflop hand like \Ah\Kh\ might have
shrunk to nothing on a board of \Jc\eigh\fivec\ninec\tenc.

One pair, even an overpair of \Ah\As, is not a Big Pot hand once
the final Board is out. It loses to two pair, trips, straights,
flushes and full houses. Don't make a big pot on the end with
just a one pair hand.

\section{Rewind the hand}

During the hand you've been putting your opponent on a likely
hand, and narrowing that range based on the action on every street---
preflop, flop and turn. Review my hand rewinding technique in the
Chapter ``Putting players on a hand''. Does the action on the river
fit with what you expect? Often a bluff just smells wrong, because
it doesn't make sense with the earlier streets of betting.

\section{Did the river change the lead?}

A lot of poker hands are a battle between a made hand and a drawing
hand. Have a look at the river and see if it completed any obvious
draws. If you've been betting AQ all the way on a QT9 board and the
runout is QT954 then you're likely still ahead. However if the river
is a K or an 8 it completes an obvious straight draw for Jx.

\section{Value Betting}

If your best judgement is that you've had the lead since the flop
and you've still got the lead, because the river was a Brick
that didn't complete any draws, you should normally keep value betting.
If the Board has been wet the whole time, better made hands than
yours should have raised earlier to avoid losing the pot when the
draws come in--- they'd want to protect their hand.

This logic doesn't work for all players--- some players are happy
to take the chance of being outdrawn, and bet the river only once
the draws miss and they're confident their hand is good. I don't
like that style of play in a Fast Tournament or from a
Risk/Reward standpoint. You can't ``play it safe'' in No Limit Poker,
you have to get your money in when there's still cards to come, in
circumstances where you can get outdrawn.

\section{Value Raising}

I've seen hundreds of Pub Poker hands that go like this next
one. There's a standard preflop raise and one out-of-position caller,
the flop comes K99. Check-bet-call on the flop. The in-position
bettor, who was the preflop raiser, has AA, AK, KQ, QQ or KJ. The turn
is a 3 and the betting is also check-bet-call; standard bet
sizing. The river is a 5 and on the final board of K9935 the betting
goes check-bet-call again. The called player shows KQ and the caller
shows T9 suited and wins with trips. ``I knew I had you beat'' says the
player with T9 and rakes in a very nice pot.

\textbf{What???} The player with T9 ``knew I had you beat'' but just
called on the river and didn't raise? He's missed three chances to
check-raise in this hand, where he's flopped a miracle. I like the
check-call on the flop and the check-call on the turn is also good (no
straights or flushes developing so no real need to protect this hand,
only a King on the river will kill us) but raise it on the river!
You'll flop trips like this maybe twice a tournament, and when you've
got an obviously winning hand you have to make it pay off for you as
much as you can. Put in a value raise on the river and win the maximum
every time you have the winner.

Now T9 is not a gobroke hand on K9935, but the preflop raiser doesn't
have KK or K9 because he bet the flop and he's very unlikely to have
A9, Q9 or J9. When you've got the most likely winner, raise it on
the river. The in-position player with KQ will correctly figure you
don't have AK because of the preflop call and will be drawn into
giving you more chips on the river because KQ is a pretty good holding
on a board of K9935.

Average pub players play sets quite well but continually underplay
trips. Holding 77, our out-of-position caller will get a raise in on a
runout of KT725 but holding T9 he'll win the minimum on our runout of
K9935.

Don't be afraid to raise with trips.

Every time someone flat calls on the river with a powerhouse, and
says ``I knew I had you beat'' it makes me want to scream. Please
don't make me scream.

\section{Bet to stop a Bluff}

A lot of poker hands go like this: check-bet-call on the flop
then check-check on the turn. On the turn street, last-to-act
is checking behind for one of a few reasons. He could be on a draw
and there's no point putting more chips in when his flop bet got
called; he'll take the free card and see if the river makes his draw.
He could have a midstrength hand and he doesn't want to lose more chips
those times where he's behind; he'll see a free river and then based
on the river betting he'll consider a value bet. Or he could be
strong and he's giving a free card because he's unlikely to be lose
this hand and he'll sell it for a value bet on the river.

Once the river card is out, first-to-act can bet on the river
to stop last-to-act bluffing. If first-to-act checks, last-to-act
will bluff busted draws and value bet good hands; meaning first-to-act
will sometimes incorrectly fold to the bluffs and incorrectly pay
off the winners. If first-to-act bets, this will often
stop last-to-act bluffing his busted draws (bluff raising is
very hard to do). First-to-act donk betting like this is like
the Blocking Bet on the turn. The Turn Blocking Bet tried to set
the price for this betting round, the River Blocking Bet tries
to block the last-to-act player from bluffing.

Because the turn betting went check-check, the donk bet
on the river looks very much like a bet wanting to be called,
from a hand who now knows from the quiet turn betting that it's
the winner. If this donk bet still gets raised, first-to-act
should almost always fold.
