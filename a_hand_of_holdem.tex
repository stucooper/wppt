\chapter{A Hand of Texas Holdem}

% Last updated: 20200424

A quick refresher on a hand of Texas Holdem poker and the terminology
used.

\begin{itemize}

\item Two forced bets (the \textbf{Small Blind} and the \textbf{Big
  Blind}) are posted by   the players to the left and second-left of
  the dealer. The Big Blind   is normally twice the amount of the
  Small Blind. These bets create the initial pot for players to
  fight over. The Big Blind is the minimum bet size on all betting streets.
  
\item The dealer deals a card to each player; starting at the small
  blind and finishing with himself, followed by a second card to each
  player. These two cards are \textbf{Hole Cards} and are used by each
  player to form their best five card poker hand in combination with
  the \textbf{Community cards}.
  
\item A betting round takes places; beginning with the player to the
  left of the Big Blind (this position is called Under-The-Gun). This
  player must match the Big Blind amount, or raise, or fold. When the
  turn returns to the Big Blind, he may check (a bet of zero) if there
  has been no raises, or he may raise. If there has been a raise, the
  Big Blind must match the amount, raise it, or fold.

  In any betting round, a player can bet and if no other player
  matches his bet (the other players all fold) then the hand is over
  and the bettor, the last man standing, is awarded the pot.
  If all bets are matched and there are two or more players remaining,
  the betting round is over and play proceeds to the next phase.
  
\item The dealer burns the top card of the deck and flips over the
  next three cards, producing the first three community cards, called
  \textbf{The Flop}.

\item There is another betting round.

\item The dealer burns the top card of the deck and flips over the
  next card, which becomes the fourth community card, \textbf{The
    Turn}.

\item There is another betting round.

\item The dealer burns the top card of the deck and flips over the
  next card, which becomes the fifth and final community card, \textbf{The
    Turn}.

\item There is a final betting round.

\item If there are still two or more players active after the final
  betting round, players show their hole cards in a \textbf{Showdown}.
  The player who makes the highest five-card poker hand, using his
  hole cards in combination with the five Community Cards, is awareded
  the pot.

\end{itemize}

I've simplified this a little. I haven't explained how pots work when
there's an all-in player. Also not every hand of poker has a Small Blind
(that player could have busted out the hand before). But this short
explanation will do us for now on how a hand of Texas Holdem Poker is
played.
