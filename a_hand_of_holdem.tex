\chapter{A Hand of Texas Holdem}

% Last updated: 20200505

A quick refresher on a hand of Texas Holdem poker and the terminology
used. In this book I've invented some terms for key concepts:
Flopstreet, Turnstreet, Finalstreet, Turnboard and Finalboard. These
are explained here.

\begin{itemize}

\item Two forced bets (the \textbf{Small Blind} and the \textbf{Big
  Blind}) are posted by the players to the left and second-left of
  the dealer. The Big Blind is normally twice the amount of the
  Small Blind. These bets create the initial pot for players to
  fight over. The Big Blind is the minimum bet size on all betting streets.
  
\item The dealer deals a card to each player; starting at the small
  blind and finishing with himself, followed by a second card to each
  player. Each player gets a unique two cards, secret to himself,
  called \textbf{Hole Cards}. The cards are dealt clockwise, and
  betting rounds proceed clockwise as well.
  
\item A betting round takes place, \textbf{Pre-flop betting}. Betting
  begins with the player to the left of the Big Blind (this position
  is called Under-The-Gun). This player must match the Big Blind
  amount, or raise, or fold. When the
  turn returns to the Big Blind, he may check (a bet of zero) if there
  have been no raises (because he has already bet that amount blind
  even before being dealt his cards) or he may raise. If there have
  been raises, the Big Blind must match the amount, re-raise it, or
  fold.

  In any betting round, a player can bet and if no other player
  matches his bet (the other players all fold) then the hand is over
  and the bettor, the last man standing, is awarded the pot.

  If all bets are matched and there are two or more players remaining,
  the betting round is over and play proceeds to the next phase.
  The chips from the betting round are collected and added to the pot.

\item The dealer burns the top card of the deck and flips over the
  next three cards, producing the first three of the \textbf{Community
  Cards}, called \textbf{The Flop}.

\item There is another betting round, \textbf{The Flopsteet.} Betting
  begins with the first
  active player to the left of the dealer. In this betting round there
  is no Blind-bet to match, and the first player can check.

\item The dealer burns the top card of the deck and flips over the
  next card, which becomes the fourth community card, \textbf{The
    Turn}. The four community cards form \textbf{The Turnboard.}

\item There is another betting round, \textbf{The Turnstreet.} Again
  this is a fresh betting round. The first player can check.

\item The dealer burns the top card of the deck and flips over the
  next card, which becomes the fifth and final community card, \textbf{The
    River}. The five community cards make \textbf{The Finalboard.}

\item There is a final fresh betting round, \textbf{The Finalstreet.}

\item If there are still two or more players active after the final
  betting round, players show their hole cards in a \textbf{Showdown}.
  The player who makes the highest five-card poker hand, using any or
  all of his hole cards in combination with the five Community Cards,
  is awarded the pot.

  It's possible for two or more players to make the same highest
  hand. When this happens, the pot is split equally between the
  winning players.

\end{itemize}

I've simplified this a little. I haven't explained how pots work when
there's an all-in player. I haven't told you what it means to burn a
card, because you know that already. Not every hand of poker has
a Small Blind (that player could have busted out the hand before). I
haven't explaned how Showdowns work. But this short explanation will
do us for now on how a hand of Texas Holdem Poker is played.

The dealer has a lot to do in this procedure; handling the cards and
bringing out the flop, turn and river at the appropriate times. The
other players are called into action during the four betting
rounds. Their decisions in the hand are whether to check, bet, fold,
call or raise. \textbf{Poker is all about betting}.
