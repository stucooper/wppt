\chapter{Stereotypes}

% Last updated: 20181027

It's not polite in society to stereotype people. In poker though,
it can be useful to take a guess at somebody's playing style.
In a fast tournament you might only play a few hands against
a particular player, and you might not have time to see
what a person's playing style is before one of you is out
of the game. So you can take a guess at someone's style based
on a stereotype. If that person then plays differently to your guess
you need to adjust your thinking immediately, but a lot of the
time your guess of a player's default style will be correct.

The following stereotypes I've found pretty useful, but use them
with care.

\section{Women}

Most women I've played against are very Tight/Passive players. They
are especially tight preflop and won't reraise unless they have
Pocket Kings or Pocket Aces. They don't bluff very often but they'll
often call when aggressive players try to bluff them off hands.

\section{Drunks}

Someone who's had a few drinks has usually lost patience with a poker
tournament.  He'll play poor starting cards and overcommit to pots, sometimes
trying outrageous bluffs against a strong hand.

Keep your composure against a drunk. Don't try bluffs against him, but you
can win a lot of chips from him when you've got a hand. Expect to win
a huge pot against a drunk by showing down a strong winning hand.

\section{Uni students}

Expect younger players to play a pretty good game, and be mindful of pot
odds and the mathematics of poker. Often at this time of their lives
they don't have a lot of spare cash. Many students have a game that's
good enough to play in \$330 casino tournaments but they don't have
the bankroll. Cheaper pub games are an easier investment for them,
in both time and money.

\section{Old players}

Old players are often tight. Many have played limit games most of their
poker lives, and haven't adjusted their betting well to No Limit. Some
will habitually bet small, it's worthwhile picking up on this because
you'll be getting good pot odds if you're drawing against this player.

\section{Online kids}

Sometimes a young guy who mostly plays online will play in a pub game.
He'll play a similarly good game to the Uni students, but he'll be
more impatient and fearlessly aggressive with his chips.

There's no love lost between the online guys and the old players. Sometimes
an old player will collect the stack of an online kid who refused to
give him any respect.

\section{White collar workers}

Some of the players at Monday to Friday after work games will be white
collar workers, relaxing with a few drinks and a poker game after work.
They'll be wearing collared shirts and sometimes ties, and be employed
in Finance or IT or some such profession.

They don't play much poker, maybe once or twice a week, and they haven't
given much thought or study to the game. They're often playing for the
social side of the game and they're usually nice people to chat with.

Look to value bet a lot against these players. They'll call quite often
with hands that are outkicked.

%% FIXME: this writing was OK, but I wanted to keep the chapter to 2 pages.
%% \section{Loser}

%% Once in a while you'll encounter a person who you can tell is a loser.
%% He might be grumbling about his pokies losses or something bad in his
%% life. He's actually looking to reinforce his bad-luck life by losing
%% in this tournament.

%% Losers often call all-in bets with very few outs and exit tournaments
%% when their middle pair doesn't improve or their top pair is always outkicked.

%% Always be polite to losers. Shake their hand and say ``well played'' as they
%% slink away from the tables.


