\chapter{Dealing}

% Last updated: 20191008

As Pub Tournaments are self-dealt, you need to know how
to run a hand of poker. These skills
help you in your poker playing. You're following
the action, keeping track of the pot size, watching
the bets and understanding what they mean.

Several poker professionals have come from dealing
backgrounds\footnote{Including Mike ``The Mouth'' Matusow, who has also
spent time in prison for dealing things that weren't playing
cards} and USA Casino employee tournaments are hotly contested by
the dealers, who aren't allowed to play in their
own casino at any other time. Some of my toughest Pub opponents
are dealers and floor managers at The Star Casino. Star dealers
aren't allowed to play there at any time, and often chit-chat about
their home games or their trips to Crown Casino in Melbourne.

\section{The shuffle}

Split the deck roughly in two and riffle the cards together.
Do this at least three times, and if you've fumbled the cards,
do it an extra time.

Make sure the cut card is with the player on your right (the cutoff).
All Pub games cut to the right. Some games used to cut to
the left, but some people thought that it'd be
easier for a pair of cheats to sit next to each other and cut in just the
right place to give a strong hand to the small blind, whereas
if it were the cutoff making the cut it'd be hard to give a strong
hand to the 6th or 7th player dealt to. It's a minor distinction,
but such a simple difference makes the deal fairer for everyone.

Don't do any cuts yourself, just the three or four riffles. Some dealers
cut themselves, then pass to the cutoff who re-cuts. The effect of this
is that the cutoff's cut seems to cancel out the dealer's original cut.
So don't cut your own cards, just riffle them, and let the cutoff cut them
for you.

\section{Preflop deal}

Before dealing make sure the big blind and small blind (if any) are
posted. If the players on these stacks are absent, announce to the table
the amount you're posting from the unattended stack and post from
that stack yourself. Always tell the seated players what you're doing and why,
before touching chips that aren't yours.

Make sure you know if the small blind or big blind have taken change
already. At 200/400 blinds, a Big Blind will often put a 500 chip in
and take 100 from the small blind. Now there's a 100 chip on the
small blind and a 500 chip on the big blind but they're both correctly
in for 200 and 400.

If it gets raised to 1200, when action returns to the big blind it
looks like it's only 700 more to him, since he has a 500 chip in front
of him. But he's already taken change, he's in for 400, so it's 800 more
to him.

Announce how much more it is to call: ``you're in for 400,
800 more to call''.

You've now got the deck back from the cutoff, with the cut card on
the bottom. Stand. You can now reach a bit further
and you've drawn some attention to yourself.

Start at the small blind and deal to the stacks, finishing with yourself
on the Button. If you're at a big oval table and an awful dealing position like
one of the end seats, deal the cards to the other side as far as you comfortably
can and ask other players to pass the cards for you.

Once all stacks have two hole cards, put the deck down and put the Dealer
Button on it. Protect your own hole cards with some small value chips or
your card protector if you use one.

One thing I can't understand about pub poker players is that even by
this late stage some players haven't collected their hole cards together.
Their cards are still in the middle of the table and they're at risk
of being hit by early players quickly folding.  Push uncollected
cards closer (if you can reach them) or ask someone to
pass them right to the player.

Now it's time to fold the cards at the unattended stacks. Don't wait for
it to be their turn in the betting. Sometimes players needing a toilet
stop or a drink will get up and leave the table after checking their hole
cards. Never wait for them to come back, or call out to them across the
room, just fold them. If someone isn't seated at the arrival of their
second hole card, they don't get to play this hand\footnote{I once
lost in the first round of a \$250 buyin game against a player who
shouldn't have got to play his hand. My AK got an AKT flop, the late
arriver had KK}.

% FIXME: check PokerTDA or Robert's Rules or Daniel Negreanu on
% late arrivers

After everyone has two hole cards, and absent stacks are mucked, you can
sit down.

\subsection*{The first burn card is on top of the deck}

Occasionally a card will flip over in the deal. Keep dealing to the
next player, and after everyone else has two cards, give the flipped player
a second face down card and get the flipped one back and place it on
top of the deck. Some dealers do this face up, I'll usually say
``first burn card: Jack of Spades'' and put it back face down.

Put the burn card back on top of the deck. Now you won't forget
to re-burn it when you flop. I've seen a lot of errors when players
keep the exposed card somewhere else, then it comes to the flop and
they accidentally burn a second card, because everyone is so used
to burning every time they flop. The smart thing to do is simply
put the burn card back on top of the deck,
where the first burn card always is.

\subsection*{Misdeals}

If two cards flip over preflop, it's a misdeal and you're probably
trying to pitch them too far. Get help from the other players in
passing the cards. If cards people are mucking become exposed, it's
not a misdeal. Announce to the table the cards that were mucked
(``six of hearts and two of clubs are out of play'') and place
them face down in the muck.

If either of the first two cards dealt is exposed (the first card dealt to
the small blind and the big blind), it's a misdeal.

If you missed dealing to any of the stacks, it's a misdeal.
Remember to start at the small blind and finish with yourself, dealing
to every stack. Deal to the stacks, not to the players.

If any player has one card or three cards, it's a misdeal.

\section{Four areas of the table}

When you're dealing you'll need to maintain four separate areas of the table.
These are the pot, the muck, the community cards (which include the burn cards)
and the betting from the current round. Keep these four areas of the table
visibly separate and you'll minimise the chance of errors.

\section{Protect the muck}

If someone throws their cards into the muck, that's where they stay.
A lot of players fold out of turn preflop and then realise
they're out of turn and try to retrieve their cards from the muck
so they can fold a second time, this time in turn. Yuck! Keep the
folded cards in the muck and ask the player to act in turn next time.

Often in pub games, inattentive big blinds will instantly fold,
especially if the dealer hasn't asked for the blinds to be in place
before dealing. You will be making sure the blinds are in place before
you deal the hole cards, so the big blind really has no excuse
if he still folds like this. Once a hand is folded it stays folded.

\section{Preflop betting}

Remember to fold the cards dealt to unattended stacks.
Now it's time to get the action moving. Indicate to the under the gun
player that it's his action and what the amount to call is. I usually
say words like ``Eight players, 200 to call'' and move it around
the table, keeping up an auction patter. ``200 to call, fold, call,
first raise to 600, 600 to call, 600...400 more to you Sir''. When the
action arrives to me on the Button, I announce ``action on the dealer''
and only then do I look at my own hole cards and act. Because I've
been announcing the action, when it's my turn to act I've got it all
fresh in my head and I can act reasonably quickly. Make sure people
keep their bets in front of them and they're easily identifiable as their
bets.

% Throwing a bet into the pot so its hard to check the size of the bet
% is called ``Splashing the Pot''. Mike McDermott (Matt Damon) asks
% Teddy KGB (John Malkovich) to stop doing that in ``Rounders''.

Once the action is complete, get the chips pushed together into the
centre to form the pot. Do this after each betting round, so that the
new bets don't get confused with the previous bets, which now live
in the centre as part of the pot.

Just before performing the flop,I announce how many players I dealt
to and how many players are still active on the flop. ``Runners''
is well understood slang for the number of active players. If it
gets down to two runners, you can announce ``heads-up''.

My final act preflop is to check that the pot size is correct. Say blinds
are 100-200, it's been raised to 600 and there's 3 people taking the flop,
including the big blind but not the small blind who tossed it in. That's
a pot of 3 times 600 plus 100 from the small blind, making 1,900.

\section{Announce the flop}

After you turn over the flop, announce what the cards are, from
highest to lowest. If there's more than one card of the same suit,
announce that as well. If the suits are all different, call it rainbow.
So \tenh\Kh\tens\ is ``King Ten Ten, two hearts'', while \Ac\tend\sixh\ is
``Ace Ten Six, rainbow''. \Qc\Ac\fourc\ is ``Ace Queen Four, all clubs''.

Although in this book I present every flop high-to-low
(except the ones just above) don't rearrange the flop order of the
physical cards on the table. Keep them in the order that they came
out, left-to-right, but do announce the three flop cards in the order
highest to lowest.

Announce what the turn and river cards are, rank and suit, as you reveal them.
Never say anything about what hands are possible using the
board cards. Players have to read the board themselves,
it's a fundamental and key Holdem skill. Put the turn card to the
right of the flop and the river card to the right of the turn. That
way a player trying to rewind the hand can see what the flop was, what
the turn was and what the river was.

\section{Later betting rounds}

As always, follow the action, and keep a track of the pot size.

The minimum bet in later rounds is the big blind, no matter what the
pot size is or what the preflop raise was.

Check with the table before you produce the turn card and check
again before you produce the river. This will avoid mistakes in prematurely
exposing community cards before the betting action for that round
is complete. Because you're following the action so well, you will
rarely make a mistake here. In the rare cases that you do, here are
the PokerTDA rules for handling premature exposure of cards:

\begin{description}

\item[Premature Flop] Wait for preflop betting to properly finish.
Leave the flop burn card as the burn. Return the premature board
cards to the deck and reshuffle the deck. Re-deal the flop
(without another burn) from the newly shuffled deck.

\item[Premature Turn]  Wait for flop betting to properly finish.
The premature turn card is put to the side.
Another card is burned, and the normal river card is used as the new turn card.
After action on the turn, the premature turn card is returned to the deck
and a reshuffle takes place. A river card is then dealt without another
burn.

\item[Premature River] Wait for turn betting to properly finish.
The third burn card stays as is. Return the premature river to the
deck and reshuffle and deal the river without a new burn card.

\end{description}

Premature exposure of cards can cause heated arguments at the table.
Take extra care that you don't do this. It's much better to be too slow
and have players prodding you that they're waiting for the turn or
river, than it is to expose these cards before the previous action
is complete. If in doubt, ask ``Am I good to go?'' and the players still
in the pot will let you know if betting is complete and you can
produce the next community card.

\section{Burn cards}

Never mix the muck cards with the burn cards. Always keep them
separate. That way everyone can see three burn cards--- one preflop,
one pre-turn and one pre-river. I keep the pre-flop burn card close to
the first card of the flop, the pre-turn burn card next to the turn and
the pre-river burn card next to the river. There's a nice visual effect
where the community cards now total eight cards; five board cards and
three burn cards.

% FIXME: do a picture of this if it typesets well

Never burn a card until it's time to produce the next phase
of the Board. Think of flopping as ``Burn and flop'', producing the
turn as ``Burn and Turn'' and producing the river as ``Burn and river''.
It's a two-part action each time. If you've seen Rounders, you can
hear John Malkovich saying ``burrrn and turrrn'' in his crazy accent.

The whole point of burn cards is to
defend against cheats who try marking the cards through bending or
fingernail marks or other card marking tricks.
If a cheat knows through card marking that the next card off the deck
is an Ace, he gets an advantage, but the next card off the deck
becomes a burn card and doesn't feature on the turn or the river.
If the cheat knew that his marked card would be the turn or the river
and be part of the community cards he'd have a huge advantage.

Players don't get to see the back of the first flop card (called
the ``Window'' card) until preflop betting is complete. Players
don't get to see the back of the turn card or river card until
they're produced. They can stare at the deck during the hand
if they like, but the card they're staring at is the next burn card
and won't be part of the board.

Don't try to burn/turn or burn/river one-handed. You risk flipping the
burn card over or revealing it to closely-watching players. Pick up
the entire deck (obviously including the cut card at the bottom of it
all) and burn face downwards then produce the next community card.

\section{The Showdown}

The called player (the last aggressor) shows first.
If nothing's happened after a few seconds, ask the called player to show.
If a river round of betting has been checked around, ask for
players to show in betting order, so if only the Big Blind and the Button
are in the hand, ask the Big Blind to show first.

Both cards need to be revealed for a showdown. Call the best
hand and award the pot. When calling the best hand I push the community
cards used in that hand up a fraction, to show how much of the community
a winning hand is using.

When calling a winning hand less than a flush, call all five cards.
This will help make sure that you don't miss any accidental split pots.
QJ verus KJ on a board of 77JA3 is actually a split, both players are
playing ``Jacks and Sevens, Ace kicker''.

If there's a main pot and side pots, award the side pots first.
I say the words ``playing for the side pot'' and award the side
pot, then keep the winning hand face up and invite the all-in
player in the main pot to reveal his hand. The all-in player
for the main pot may fold and concede the main pot (and his
tournament life) to the side pot winner.

Some side pot winners muck their cards as soon as
they've won the side pot. This is a grey area.
You want to protect the integrity of the muck, and you've reminded
the side-pot winner that he's first playing for a side-pot with the words
``playing for the side pot''.  If the mucker's hand has been
shown down, during the awarding of the side pot, the whole table knows
what hand he's been playing and he can contest the main pot with his
shown hand. But if he's won the side pot with an uncalled bet, and then mucks
his cards unseen before the main pot showdown, it's a very grey situation.
Call the Tournament Director for a ruling.

Once the pot's been awarded, keep the cut card (you're the
cutoff for the next deal) and pass the cards to the next dealer.
He'll shuffle and pass the deck to you for a cut. Cut about half the
deck onto the cut card, then put the bottom half of that deck on
the top.

\section{All-in showdowns}

Once players are all in and there's only one active player left
with chips, ask for the cards to be turned over. Say the words
``No more action''. Make absolutely sure that there is no more action.

Always deal out the turn and river even when it seems someone
is drawing dead. It's surprising the outs that can turn up in
seemingly dead situations. \Ac\As\  versus \tenc\nines\ on a flop
of \Ah\treh\fiveh\ ? The turn and the river can both be hearts and
both players play the flush on the board and it's a split. A runner
runner low straight could sneak along for another split.

Even \Qc\Qd\ versus \Ac\Jd\ on a board of \Qh\sixc\sixd\sixh ? If the
river is the final six, then \Ac\Jd\ wins with \sixc\sixd\sixh\sixs\Ac versus
\sixc\sixd\sixh\sixs\Qh. Improbable boards can and do happen,
so \textbf{always} deal it out.

\section{Keep your thoughts to yourself}

As I said before, don't make any comment about the community
cards and the possible hands people might have. Reading the board
and figuring out what the good hands are is a required skill for
players in Texas Holdem. In other forms of poker, notably the Australian
game Manila, the dealer announces the best possible hand with every
community card shown, but in Holdem the dealer stays quiet.

In fact, keep all opinions on the play to yourself. If someone
bets minimum 100 into a 12,000 pot don't say ``that's a mighty small bet''
just announce it as a bet of 100. Likewise if someone bets 8,000 into
a 1,000 pot just announce the size.

Never cheer for one of the players in the pot to win or lose.
Regardless of your feelings towards the players, the dealer needs to
be impartial and be seen to be impartial.

\section{Absolute Beginners}

Occasionally you'll be dealing to an absolute beginner. Some won't be
keeping their cards on the table and might try showing them to friends
for advice on how to act. They'll never know when the action is on them.

Be polite to the new player, everyone has to start somewhere. Let him
know whenever the action is on him and what his options are. If he
faces a call for 600 on the river, I'll say ``Action to you, Sir, you
can call for 600, fold, or raise for 1,200 or more''.

Calling all players ``Sir'' or ``Madam'' is a nice touch.

\section{Handling Disputes}

Always call a Tournament Director (TD) over to assist. As a playing
dealer, you have an interest in the outcome of a hand, even
if you've already folded and you're not in this hand, you have
an interest in how you go in the Tournament.

Allow the players to have their say first and if you think you've
got anything to add to it from the dealer's point of view, have
your say when you get the chance.

\section{Count down the deck}

Every now and then you'll be dealing in a hand where it's going to be
active on the river, it's a big pot, there won't be a dispute and you
sense that the final betting street will take a while. In these
circumstances, once you've produced the river card, you can count
down the deck.

Counting down the deck is counting the rest of the cards to make
sure that there's still 52 cards in the deck, nobody has
taken a card off the table or hidden it up his sleeve or any
cheating method like that.

A deck of cards has 52 cards plus the cut card on the bottom.
You've announced (and remembered during the hand) how many players
were dealt to. There hasn't been any flipped up card or card
off the table or premature burning/reshuffling mistakes, so far
this deal has been exemplary.

Once you've produced the river card and it's time for the slow
river betting street to take place, you count the rest of the
cards in the deck, starting from the next number of cards.
Each player was dealt 2 hole cards, and the five community
cards plus three burn cards always makes eight cards. So if
I'm counting down the deck after eight players were dealt
to preflop, the number of cards already accounted for is
16 (two times eight) plus 8 (community + burns) = twenty four.
One at a time I quickly count the rest of the deck in front
of me (face down, obviously) and whisper to myself
``twenty five twenty six twenty seven..'' until I hit fifty two
and then the next card is the cut card and all the cards
are in the deck.

Only count down the deck in those situations where there's
a good chance of a long round of river betting and you're
not holding up the next deal. It's very quick and easy to
do.

\section{Non-playing dealers}

Towards the end of a tournament you might find someone offering
to deal for the table. If that person's a Tournament Director
take up this offer. If the person's a
regular who you know to be a competent dealer, also take up
the offer. If the person's an unknown who's busted out of the
tournament, and wasn't a good dealer earlier on, politely
refuse. You'll hurt his feelings a lot less by refusing outright
at the start compared to sitting through 5 badly dealt hands
and then asking him to stop dealing. If the person
offering to deal seems chummy with one of the other players
still in the tournament, I'd be inclined to refuse his offer
to deal. There's probably nothing going on, but you've got
enough to worry about in playing your best poker without
having doubts that the dealer isn't impartial.

It's fastest for the table to agree to a non-playing dealer
doing his own cuts.

\section{Dealing isn't easy}

Dealing's a tough job. Even without the shuffling and the handling
of the cards it's a tough job. You could write an entire book
just about poker dealing\footnote{People already have. \textbf{The
Professional Poker Dealer's Handbook} by Dan Paymar, Donna Harris
and Mason Malmuth. Two Plus Two Publishing, 1998}

Casino dealers get weeks of full time training before they can deal
Blackjack, and many more before they can run a poker table.

Use commonsense and stay calm at all times. Dealing's a
thankless task and people only notice you as a dealer when you're
doing it badly.

% Here's a grumpy player complaining
% about Jade Tavern's ``Night of Sin'' featuring lingerie waitresses dealing:

% ``With respect to the ladies concerned,
% they are some of the worst dealers I've ever laid
% eyes on. Start at the small blind, deal to the stacks
% and finish at the Button, for Christ's sake. I've
% honestly had better dealing at a PokerDome Sit and Go.
% Following the betting action can also be a fine way of
% knowing that the table is ready for the turn card, girls.''

% That was me, on PokerNetwork, whingeing about some ordinary dealers. Even
% lovely naked breasts can't take my mind off bad dealing.
