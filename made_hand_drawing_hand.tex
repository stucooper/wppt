\chapter{Made Hand versus Drawing Hand}

% Last updated: 20200507

This chapter looks at the Old Poker Saying, ``Poker
is a battle between a Made Hand and a Drawing Hand''.

Hands like Top Pair/Top Kicker, Two Pairs and Sets
are usually the best Five card poker hand anyone can
make after the flop. But by the time that the turn and river come out and
there's the Finalboard of Five Community Cards for players to use, Straights
and Flushes are often possible.

For the purposes of this chapter, I'm going to label some hands on the
poker ladder; ignoring the top hand (Straight Flush, very rare) and
the bottom hand (High Card). Quads and Full Houses are \textbf{Monster
Hands}, Flushes and Straights are \textbf{Power Hands} and Three of a
Kind, Two Pairs and Pairs are \textbf{Made Hands}.

A \textbf{Drawing Hand} is a four-card hand that can become a power hand
when the right fifth card comes on the turn or the river. A drawing hand
is four-to-a-flush or four-to-a-straight.

A \textbf{Complete Hand} in poker is a Five-Card poker hand where
you're using all of your five cards to make the hand. There's no
kicker in your five-card hand. The Complete hands are Full Houses,
Flushes and Straights. Quads is by this definition not a Complete Hand
but if you make Quads using one of your hole cards you're just about
unbeatable so we'll treat it as a complete hand also.

% FIXME: do this in a table?

In a Made Hand versus Drawing Hand matchup, the Made Hand bets on the
flop and the turn, and the Drawing Hand calls trying to hit a
Complete Hand. Most of the time the Made Hand is a big favourite;
often a 65\% favourite on the flop and a 80\% favourite on the
turn. I'll look at the bet sizing implications of these numbers.

\section{Evaluate the Flop}

When the flop comes, have a look at how draw-heavy it is. A flop of
\Ks\eigh\trec\ has no straight or flush draws possible. This flop is
called a \textbf{dry} flop, K83 Rainbow is desert-dry. A flop with
straight and flush possibilities, like \tend\nined\sixc\ is a
\textbf{wet} flop. Some flops are so wet that it's possible for
someone to already have a Power hand of a Flush or a Straight. On most
flops a flush draw or an open-ended straight draw is possible.

As the turn and the river are added to the Board, the drawing hand
either gets there, turning into a Power Hand, or it misses and ends up
worthless. The player who flopped the Made hand will stay in the lead
when the drawing hand misses and win the pot. The Made hand player
loses the pot (and possibly quite a lot more chips on the riverstreet)
when the drawing hand gets there.

The trick in No Limit Poker, when you're the player with the Made Hand
on the flop, is to charge the drawing player an amount that he will
call, but that is still a losing call for him. If
you do that properly, the calling player is making a mistake,
\textbf{whether he makes his draw or not.} Because of implied odds,
the amount of money that a drawing player can win isn't just the pot
he has to call his bet into; he could win more money on the betting on
the river.

One challenge for the Made Hand Bettor is that unless he has the best
possible Made Hand of Top Set, he doesn't know for sure that his
Enemy has a drawing hand or a better Made Hand or a worse Made
Hand. Poker isn't always Made Hand versus Drawing Hand; sometimes it's
Made Hand versus Better Made Hand; in which case the pot can get very
big very quickly. The most explosive confrontation like this is set
over set. A Made Hand bettor is trying to find a middle ground in
charging drawing hands too much to try to hit their draws, while
trying to discover if the enemy has a drawing hand at all or might
even have a Made Hand himself.

The highest of the flopped Made Hands (Sets and Two Pair) are themselves
drawing to the Monster Hands of Quads and Full Houses. A flopped Set
becomes a Monster about 30\% of the time, flopped Two Pair becomes a
Monster about 16\% of the time.

\subsection{Two-Way Hands}

A few Holdem holdings flop both Made Hand and Drawing Hand aspects.
These are called Two-Way hands; they have two things going for them.
They've got a decent Made hand, which could be best right now.
If they're behind to a better Made hand, they have the chance to make
a Power Hand.

The best of the Two-Way Hands is Pair With Flush Draw, such as
\Kd\tend\ on the flop \Kc\nined\eigd\ . This holding has a flush
draw but also has Top Pair/Decent kicker already as a Made Hand.

Flush Draws with overcards can improve to either a Made hand by
pairing the card or a Power hand by making the flush. \Ad\sevd\ on
\Kc\nined\eigd\ can improve to a pair of Aces with an Ace on the turn
or the river.

Two overcard straight draws can likewise improve to top pair or the
straight. \Jh\tenc\ on a flop of \eigh\sevc\tres\ improves to a Monster
Nut straight with a Nine or a Made top pair with a Jack or a Ten.

Next in the Two-Way Hands is Pair with Straight Draw like
\tend\nined\ on the flop \tenc\eigh\sevs\ . In practice this hand
isn't as good as Pair With Flush Draw; because if it improves to two
pair it often improves someone else to their straight.

The last of the Two-Way Hands is Overpair with Straight
Draw. \Js\Jc\ on a flop of \tenc\nineh\eigs\ can beat any Top Pair
and has flopped an Open-Ended straight draw and should be the winner
if the runout brings a Queen or a Ten. Like Pair with Straight Draw,
this hand doesn't want to improve as a Made Hand. If you improve to
trip Jacks, someone could well have a straight on a perfectly
connected Turnboard of \tenc\nineh\eigs\Jh\ .

Omaha, where each player gets Four hole cards, is full of two-way
and even three-way or four-way hands. KKJT on a KQ9 flop has top
set and top straight and King-high flush draw.

\section{High Card Holdem}

Late in Tournaments, as the blinds get very big and stack sizes diminish
to 15 BB or so, High Cards predominate. Small suited cards and small
connectors just don't hit often enough to justify calling bets with
preflop. They can occasionally be used as re-steal hands, but cracking
hands go down a lot in Final Table poker. Confrontations at the Final
Table are normally Made Hand against Made Hand, and whoever's flopped
the best pair wins.

\section{Exercises}

% Notice I get the chapter number from a LaTeX variable, not hard coded.

\arabic{chapter}.1 (YouTube) In the last two tables in the 2015 WSOP
Main Event, Joe McKeehen busts Justin Schwartz on a
\sixh\tred\twod\ flop, all the money going in on the flop. What hands
did McKeehen and Schwartz have?
