\chapter{Made Hand versus Drawing Hand}

This chapter is a lengthy examination of the Old Poker Saying, ``Poker
is a battle between a Made Hand versus a Drawing Hand''.

Hands like Top Pair/Top Kicker and Two Pairs and Sets turn up on the
Flop and are usually the best Five card poker hand anyone can
make. But by the time that the turn and river come out and there's the
full Five Community Cards for players to use, Straights and Flushes
are often possible.

For the purposes of this chapter, I'm going to label some hands on the
poker ladder; ignoring the top hand (Straight Flush, very rare) and
the bottom hand (High Card). Quads and Full Houses are \textbf{Monster
Hands}, Flushes and Straights are \textbf{Power Hands} and Three of a
Kind, Two Pairs and Pairs are \textbf{Made Hands}.

% FIXME: do this in a table?

\section{Reading the Board}

One of the great attractions of Holdem Poker is every player can see,
from the Final Board, what the Nuts is. If the Board doesn't have a
pair, then nobody has a full house or quads. If the Board doesn't have
at least three of a suit, nobody has a flush. If the Board doesn't
have at least three connected cards in a continuum of five, nobody has
a straight. Players soon learn to recognise when they have the Nuts,
and it's one of the great things about Holdem.

The most popular game before Holdem was Seven Card Stud. In Seven Card
Stud, each player still in the hand at the final betting street has their
own unique seven card hand, with four of those cards visible to the
other player. A player's final hand looks like
\back\back\nines\Qc\Jc\tred\back\ . With three hidden cards, it's possible
for this player to have a full house, quads, a Club flush or a Straight.
Indeed, with three hidden cards, it's always possible for a player to have
Quads or a Full House.

In an earlier game, Five Card Stud, you got to see too much of a
player's hand. Each player still in at the end have five unique cards
but four of them were visible. A player's final hand in Five Card Stud
looks like \back\nines\Qc\Jc\tred\ . With only five cards in his hand,
it's hard to make any kind of poker hand at all higher than a pair.

In Draw poker games, players have five cards to themselves and they're
all private. Here's a player's hand after the final draw in Draw
Poker: \back\back\back\back\back\ . If he drew one card in his draw,
he might've had two pair looking for a full house, four cards to a
flush, four cards to a straight, trips and he's only drawing one card
not two for deception, any kind of hand really. Draw poker gives you
not much information at all.

Holdem poker gives you just the right amount of information. Draw
Poker has no information. Five Card Stud has Too Much Information and
with only Five Cards the hands made are boringly low. Seven Card Stud
has a good mix of downcards and upcards but with three private cards
not two it's still not enough for some people. In my opinion Seven
Card Stud is a fantastic game and deserves to be played a lot more
than it is. Holdem hits the perfect sweet spot of five upcards and two
downcards per player. With the five upcards shared between all
players, it's possible for a player to know he's got the unbeatable
Nuts, regardless of any other player's downcards.

\section{Evaluate the Flop}

When the flop comes, have a look at how draw-heavy it is. A flop of
\Ks\nineh\trec\ has no straight or flush draws possible. This flop is
called a \textbf{dry} flop, K93 Rainbow is desert-dry. A flop with
straight and flush possibilities, like \Td\nined\sixh\ is a
\textbf{wet} flop. Some flops are so wet that it's possible for
someone to already have a Power hand of a Flush or a Straight. Most
flops give someone a flush draw or a straight draw.

As the turn and the river are added to the Board, the drawing hands
either get there, turning into Power Hands, or they miss and end up
worthless. A player who has flopped a Made hand will stay in the lead
when the drawing hands miss, and win the pot, and lose the pot (and
possibly a decent sized call on the riverstreet) when the
drawing hands get there.

The trick in No Limit Poker, when you're the player with the Made Hand
on the flop, is to charge the drawing player an amount that he will
call, but is a -EV bet for him, and as much -EV as you can make it. If
you do that properly, the calling player is making a mistake,
\textbf{whether he makes his draw or not.}

One challenge for the Made Hand Bettor is that unless he has the best
possible Made Hand of Top Set, he doesn't know for sure that his
Enemy has a drawing hand or a better Made Hand or a worse Made
Hand. Poker isn't always Made Hand versus Drawing Hand; sometimes it's
Made Hand versus Better Made Hand; in which case the pot can get very
big very quickly. The most explosive confrontation like this is set
over set. A Made Hand bettor is trying to find a middle ground in
charging drawing hands too much to try to hit their draws, see if the
enemy in the pot has a drawing hand to begin with or might have a
better or worse drawing hand, and to get a good sense for where he
stands in the pot.

\section{Exercises}

% Notice I get the chapter number from a LaTeX variable, not hard coded.

% FIXME: check numbers and names in .1 exercise
\arabic{chapter}.1 (YouTube) At the final table of the 2018 WSOP Main
Event, JoeMcKeenan busts UnpleasantGuy on a 632 flop, all the money
going in on the flop. What hands did McKeenan and UnpleasantGuy have?
