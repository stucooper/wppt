\chapter{Made Hand versus Drawing Hand}

This chapter is a lengthy examination of the Old Poker Saying, ``Poker
is a battle between a Made Hand versus a Drawing Hand''.

Hands like Top Pair/Top Kicker and Two Pairs and Sets turn up on the
Flop and are usually the best Five card poker hand anyone can
make so far. But by the time that the turn and river come out and
there's the full Five Community Cards for players to use, Straights
and Flushes are often possible.

For the purposes of this chapter, I'm going to label some hands on the
poker ladder; ignoring the top hand (Straight Flush, very rare) and
the bottom hand (High Card). Quads and Full Houses are \textbf{Monster
Hands}, Flushes and Straights are \textbf{Power Hands} and Three of a
Kind, Two Pairs and Pairs are \textbf{Made Hands}.

% FIXME: do this in a table?

\section{Evaluate the Flop}

When the flop comes, have a look at how draw-heavy it is. A flop of
\Ks\nineh\trec\ has no straight or flush draws possible. This flop is
called a \textbf{dry} flop, K93 Rainbow is desert-dry. A flop with
straight and flush possibilities, like \tend\nined\sixh\ is a
\textbf{wet} flop. Some flops are so wet that it's possible for
someone to already have a Power hand of a Flush or a Straight. Most
flops give someone a flush draw or a straight draw.

As the turn and the river are added to the Board, the drawing hands
either get there, turning into Power Hands, or they miss and end up
worthless. A player who has flopped a Made hand will stay in the lead
when the drawing hands miss, and win the pot, and lose the pot (and
possibly a decent sized call on the riverstreet) when the
drawing hands get there.

The trick in No Limit Poker, when you're the player with the Made Hand
on the flop, is to charge the drawing player an amount that he will
call, but is still a losing bet for him. If
you do that properly, the calling player is making a mistake,
\textbf{whether he makes his draw or not.} Because of implied odds,
the amount of money that a drawing player can win isn't just the pot
he has to call his bet into; he could win more money on the betting on
the river.

One challenge for the Made Hand Bettor is that unless he has the best
possible Made Hand of Top Set, he doesn't know for sure that his
Enemy has a drawing hand or a better Made Hand or a worse Made
Hand. Poker isn't always Made Hand versus Drawing Hand; sometimes it's
Made Hand versus Better Made Hand; in which case the pot can get very
big very quickly. The most explosive confrontation like this is set
over set. A Made Hand bettor is trying to find a middle ground in
charging drawing hands too much to try to hit their draws, while
trying to discover if the enemy has a drawing hand at all or might
even have a Made Hand himself.

A few Holdem holdings flop both Made Hand and Drawing Hand aspects.
The best of these is Pair With Flush Draw, such as
\Kd\tend\ on the flop \Kc\nined\eigd\ . This holding has a flush
draw but also has Top Pair/Decent kicker already as a Made Hand. These
hands, that swing both ways, are great hands that sometimes win
unimproved and even when behind to better Made Hands they have the
chance to make Power Hands. Omaha, where each player gets Four hole
cards, is full of hands that swing both ways.

% FIXME: Add pair with straight draw; overpair with gutshot
% FIXME: and are there others I can think of?

\section{High Card Holdem}

Late in Tournaments, as the blinds get very big and stack sizes diminish
to 15 BB or so, High Cards predominate. Small suited cards and small
connectors just don't hit often enough to justify calling bets with
preflop. They can occasionally be used as re-steal hands, but cracking
hands go down a lot in Final Table poker. Confrontations at the Final
Table are normally Made Hand against Made Hand, and whoever's flopped
the best pair wins.

\section{Exercises}

% Notice I get the chapter number from a LaTeX variable, not hard coded.

\arabic{chapter}.1 (YouTube) At the final table of the 2015 WSOP Main
Event, Joe McKeehen busts Justin Schwartz on a \sixh\tred\twod\ flop,
all the money going in on the flop. What hands did McKeehen and
Schwartz have?
