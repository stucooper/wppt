\chapter{Preflop strategy}

% Last updated: 20181029

The decision on whether to see a flop is your most 
frequent in poker. In almost every hand you have to look at your
holding and decide if you want to stay in the hand.\footnote{Sometimes
you're in the Big Blind and every player folds preflop and you pick
up the Small Blind without having to make any decision. This is called
a walk} Many times I've
busted out of tournaments with marginal hole
cards, which I could have safely folded preflop
for little or no chip loss. My biggest
mistake was playing the hand in the first place.

Many Holdem books have starting hand tables, suggesting
hands to call or raise with (all others, fold). These
charts have adjustments for your position in the betting,
the number of players dealt to and whether the table is tight or loose.

I won't reproduce starting charts myself, you can easily find some
if you want. I will advise you to fold A9, A8 and lower from early
position, unless they're suited. These hands play poorly from
early position because they miss a lot of flops and when you do
pair the Ace you easily lose on kicker to other players in the pot.

\section{Table character}

Just as a player can be loose or tight, and passive or
aggressive, so can a table. If 6 limpers see a flop,
that table is loose/passive. Many Pub Poker tables are like this
in the early stages. If it's raised preflop to
5 big bets, and there's one caller only, the table is
tight/aggressive. If 5 people call that raise, the table
is loose/aggressive (it's unlikely 5 players have the hole
cards to justify the call). If only 2 limpers and the big blind
take the flop, the table is tight/passive, a Rock Garden.

Once you've figured out the character of the table, consider doing
the opposite. At a loose table, play tight. At a tight table, play
loose and go for more blind steals. There's no fun playing \Js\sevs\ in a 
field of 5 players, you'll still lose. Play \Qc\Qd\ or \Ac\Kh\ and put in 
the raises and win a nice pot.

At an aggressive table, you can play passively and let the
maniacs do the betting for you. Have you ever seen pots with three
players in them, with two guys betting and raising and a third man
who's just calling, calling, calling? It's the calling player
who ends up winning the pot with a very strong made hand (such
as a full house or high flush). Sometimes the third man is so inconspicuous
that the two maniacs forget he's in the hand and act out of turn.
The third man is happy to be invisible and let others build his pot
for him.

This happens on a flop like T88, the maniacs have JT and 
KT and the third man has 87. It's an easy trapping situation for him.

At a passive table, be aggressive. Steal the blinds, make the position
plays. Passive tables are by far the most common type of Pub Poker table preflop.
If six limpers are seeing every flop this isn't poker,
this is lucky draw. Best fit wins, the cards will decide the winner. The flop comes
J44 and whoever has the 4 wins that hand. The next flop comes Q72 and
someone with QT beats someone with Q6. Stop playing lucky
draw poker.

At a tight table, play a bit looser. You're up against better quality
starting hands, so you'll usually get paid off when you hit strongly.
Make sure your hand has cracking potential- don't start playing
\Jc\tres\ because I suggested playing loose. Trash is still trash. But
hands like \tenh\ninec\ and \eigs\sixs\ have great cracking potential.

\section{Cracking hands}

Cracking hands are hands that have potential to make both straights
and flushes. \nines\eigs\ and \sixh\fourh\ are good examples. You don't want
to pay too much to see a flop with these hands, because they
usually miss completely and have to be folded. Late in a tournament
when the blinds are so big that players have 20BB stacks, cracking
hands go down in value--- the tournament becomes High Card Holdem.

They also get you into kicker problems a lot: you don't want to call a big
flop bet with \nines\eigs\ on a flop of \ninec\fived\tred\ because the 
big bettor usually has \Ac\nined\ or \Qs\Qc. The value in your 
cracking hands is when they make really big hands.

You really want to hit hard with your cracking hands: two pair or
trips or strong draws. Remember the strength of your starting hand
is its ability to make straights or flushes. If you just flop top
pair with them, you'll often lose on kicker or lose to an overpair.

Any two cards flop two pair or better 5\% of the time. So if you have
a chip mountain and you think you have live cards against a payoff
wizard you can take a flop with just about anything, provided the
preflop call is less than 5\% of your stack.

In cash poker, you want a trash call to be less than 5\% of his stack,
but in a tournament you have to bust players, and the value of
busting a player out of the game is more than the value of the
extra chips you win from the pot. If I have 30 big bets or more and
someone is all-in for 3 big bets, and there's no other callers, I'll
usually look them up with any two cards. I don't like doubling players
up, but there's value in knocking other players out. If I bust him
I move closer to the first prize, if he doubles up I've still got
a 24 BB to 6BB advantage over him.

Tight players will get very upset if you pull in a huge pot
with marginal hole cards. They may start playing looser themselves,
which they won't be so good at and is again giving you the advantage.

\section{Game plan for this hand}

When you take a flop with a cracking hand, have a plan for your hand, based
on what you expect the other player to be holding and what your
position is and what your cards are. Here's my thought process
in a hand where I call a raise out of position with \eigc\sixc.

``John's raised from the Button and I don't think he's on a blind
steal because I've never seen him blind steal ever. Because he's
got a starting hand he likes, there's no point reraising him here. He's
either got Ace-good, King-good suited or a pocket pair 99 or better.
I'll take a flop with my \eigc\sixc\ , since I have a mountain of
chips right now and John should pay me off if I hit, but unless I 
hit pretty hard then I won't go any further with this hand. John
never C-bluffs if he misses, so if the flop comes rags and he checks
behind me I'll take a shot at it on the turn if the turn is also low.
He'll probably fold Ace-good if he hasn't paired or made a flush
draw by then''.

This little game-plan for this particular hand has set me
up to play it properly. It's a lot easier to make plans
against only one player Heads Up like this because I only have
to observe and use the information on one player. With more players
in the pot there's too much going on, and too high a chance someone
has hit a good hand. If I play a cracking hand multiway it will
normally be to try and hit an unexpected big winner with it, I won't
be trying many bluffs or steals against two other players.

With this game plan in place, here's how I'd play the following
five flops from my out of position big blind, with my \eigc\sixc.

\subsection*{Flop 1}

\begin{cards}
\crdAh\crdKc\crdtenh%
\end{cards}

Check/Fold. I won't start trying to find great ways to bluff,
or convince myself that John is scared of the flush
draw. This flop completely missed me, and unless John also
checks behind, I can safely assume based on his preflop raise
and the flop composition that this is a good flop for him.
Even if John checks the flop behind me, I'm looking
to Check/Fold on the turn.

% so that flop2 appears above the description of Flop 2!!!
\newpage

\subsection*{Flop 2}

\begin{cards}
\crdKs\crdeigh\crdsixd%
\end{cards}

I hope to check-raise a big amount. Hopefully the King connects
with his hand: I'd love him to have AK here. The flop is rainbow
so I don't have to worry about him hitting a flush. A check raise
here should win me the pot. If he checks behind, I'll bet half
the pot on the turn, whatever card the turn is. If he's got KK
in this hand, I'm going broke.

\subsection*{Flop 3}

\begin{cards}
\crdsixd\crdtwos\crdtwoc%
\end{cards}

I have top pair here but I'm still worried that John might have
an overpair. If I check it however he's a good chance to hit
a better pair on the turn if he has got something like AJ or KQ.
I'll bet out at him here and see what happens. If he raises I'll
put him on the overpair and fold, if he calls then I'll check the
turn unless I improve with another 6 or 8. If I do improve on the turn,
I'll look to get all my chips in.

\subsection*{Flop 4}

\begin{cards}
\crdsevh\crdtwoc\crdfivec%
\end{cards}

On yeah, 5678 open ended straight draw and the flush draw. I'll just check here and
see if John bets at it. Hopefully he'll bet at it with JJ and 
will pay me off if I hit. If his bet is so big that he's pot
committed I'll just call. If his bet is smaller and looks like
a feeler bet I'll check-raise him all in. I'm actually the
favourite on this flop against just about anything, so if I
can take down a nice pot on the flop, I'll do it.

\subsection*{Flop 5}

\begin{cards}
\crdJh\crdnineh\crdninec%
\end{cards}

Check/fold. I don't have any piece of this, and I'm out of
position the whole way. I don't think a bet here will win me
the pot. A weak bet here will just be giving away some chips
and a strong bet here might well be giving away a lot of chips.
I took a chance with my preflop call but I missed the flop.
I'll stick to the original plan of ``if I don't connect strongly with
the flop, I'll get out of this hand''.

If John checks behind, I'll probably check/fold the
turn unless the turn gives me a chance of backdoor straight
or backdoor flush. Even if I make my straight or flush I could
still lose to a J9 fullhouse. I'm treading very carefully in
this hand, the flop is coordinated against me.


