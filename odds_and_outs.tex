\chapter{Odds and Outs}

% Last updated: 20181012

Now that you've seen how expected value works, let's look at how it
works in Holdem. There's a bit of Maths in this chapter but nothing you
didn't learn at school, just some multiplication, division and a
few percentages.

\section{Percentages}

If your chance of winning the hand is greater than 50\%, you're the favourite.
If your chance is 80\%, you'll win the hand 4 times out of 5. If your chance
is 20\%, you'll win 1 time out of 5.

If you've watched some poker on the Internet, you'll see winning percentages
written next to each player's hand. When you're playing the game yourself you
don't know the other players' cards, but you can take a good shot at estimating
your own chances of winning.

\section{Outs}

A card that improves your hand to a better hand is called an out. I have \fourd\tred\ and
I call a raise from the big blind, two of us take the flop. The big blind has \As\Js.
The flop comes \Ad\tens\eigd. My four high won't win anything, but I can improve to a flush
if a diamond comes on the turn or the river. There's 13 diamonds in the deck of cards, I've
seen 4 of them, there's 9 remaining. I have 9 outs after the flop to improve to a flush.

Now the turn comes \twoc\, giving a board of \Ad\tens\eigd\twoc. I didn't make my flush
yet, but I've picked up a chance at a straight. If a 5 comes on the river I'll make the
best hand. There's four fives in the deck, so I now have 9 flush outs + 4 straight outs,
which looks like 13 winning cards for me. But one of my flush outs is the 5 of diamonds,
and I can't count the 5 of diamonds twice. So I have 9 flush outs + 3 extra straight outs
= 12 outs after the turn.

% FIXME: list all the outs

Now see what happens when I have the same cards and the same flop and turn, but this
time the big blind has \As\Ah. On the flop of \Ad\tens\eigd\, the \tend\ isn't a winning
card for me anymore, it makes my flush but it makes the big blind a full house and I'm
drawing dead and no river card saves me. When the turn comes \twoc\, the board is now
\Ad\tens\eigd\twoc\ and the \twod\ isn't an out for me, again it will make the
big blind a full house. So this time I have 8 outs after the flop and 7 flush outs +
3 straight outs = 10 outs after the turn.

Notice in counting my outs I'm assuming all the diamonds and all the fives are still in
the deck. In reality this isn't the case, some of the players who folded preflop would
have folded a diamond and maybe there's a diamond or a five as one of the burn cards.
It doesn't change the maths much so you can usually ignore this. One time this does
count is when you're in a big pot with three or more players, often there can be
two players on flush draws and that means that you're less chance of making your flush
and even if you do make a flush you lose to a third guy's higher flush.

\section{Rule of Four, Rule of Two}

If you're on a draw, you can find your winning percentages by counting your outs and
using the Rule of Four or Rule of Two. After the flop, your percentage chance
of improving your hand is your number of outs times 4 (Rule of Four). After the turn, your
percentage change of improving your hand is your number of outs times 2 (Rule of Two).

So going back to my \fourd\tred\ against \As\Js\, when I flop the flush draw
I have 9 outs and my chance of improving to a flush by the river is 9 x 4 = 36\%, about
1 in 3. After the turn, I have 12 outs to a flush or a straight and now my
chances of improving are 12 x 2 = 24\%, about 1 in 4. Even though I've picked up
three more outs by the turn, my percentage has gone down because now
there's just one more card to come.

The Rule of Four and Rule of Two work because the number of unseen cards is
close to 50. If you're on a flush draw with 9 outs, on the turn your
chance is 9/46 which is close to 9/50. 9/46 is 0.195 while 9/50 is 0.18,
these numbers are close. On the turn, your chance is 9/45 which is close to
9/50. 9/45 is 0.20 while 9/50 is 0.18.

\section{Pot odds}

Now you know your winning percentages you can tell if a call is worth it or not.
If you're a 24\% chance after the turn bit you have to call a bet of 1,000 into
a pot of 1,700, you're not getting a high enough return. You're a 1 in 4 chance,
but you're only winning 1.7 times your money. You want 3 to 1 or better. If
instead you have to call 1,000 into a pot of 3,200, go for it. 3 out of 4 times
you miss and it costs you 3,000, but 1 in 4 times you win and win 3,200.

In tournaments you sometimes make a call where the pot odds don't justify it,
because chips are so important. In cash games you should pay a lot of
attention to the percentages and the pot odds.

\section{Implied odds}

Pot odds tell you what you'll win from the pot right now if you improve your hand.
But there's a final betting round after the river, and if you hit your hand you should
win extra chips on the river. Of course if your opponent is already all in before you
make your call, there's no more chips to be won. But usually you can count on
winning at least the amount of the call extra on the river, so you can factor
that into your calling decision.

\section{Flop percentages}

If you have seperate cards, half of the time you hit no pair, the other half of the
time you hit one pair and very occasionally (5\%) you hit two pairs.

If you have a pocket pair, you hit your set about 12\% of the time, 1 in 8.

If you haven't hit the card you want on the flop, the chances of doing it
on either the turn or the river is about half as much as your chances
were of doing it
on the flop. So if you miss your 12\% chance of hitting a set on the flop, you're
roughly a 6\% chance of doing it on the turn or river. The Rule of Four says you
have 2 outs after the turn, so your chance of improving is 4 x 2 = 8\%.

This is why in an AA versus KK confrontation, the AA is an 80\% favourite.
KK needs to hit the third K to win, and that will happen 20\% of the time.

The chance that both AA and KK will hit a set on the flop is 1\%.

You probably know already that AK versus JJ is close to a 50/50 coinflip.
The percentages are actually AK 48\% and JJ 52\% but those numbers are so
close it's almost 50/50. Basically AK will end up with a pair on the flop 50\%
of the time and a pair by the river about 70\% of the time. But JJ will make
a set roughly 20\% of the time and that set will normally be the winner. So the AK winning percentage is
the chance that AK will hit a pair minus the chance that JJ won't hit a set.
The JJ winning percentage is the chance AK won't hit a pair plus the chance
JJ will hit a set.


