\chapter{Odds and Outs}

% Last updated: 20190628

Now that you've seen how expected value works, let's look at how it
works in Holdem. There's a bit of Maths in this chapter but nothing you
didn't learn at school, just some multiplication, division and a
few percentages.

\section{Percentages}

If your chance of winning the hand is greater than 50\%, you're the
favourite. If your chance is 80\%, you'll win the hand 4 times out of
5. If your chance is 20\%, you'll win 1 time out of 5.

If you've watched some poker on the Internet, you'll see winning
percentages written next to each player's hand. When you're playing
the game yourself you don't know the other players' cards, but you can
take a good shot at estimating your own chances of winning.

\section{Outs}

A card that improves your hand to a better hand is called an out. I
have \fourd\tred\ and I call a raise from the big blind, two of us
take the flop. The big blind has \As\Js. The flop comes
\Ad\tens\eigd. My four high won't win anything, but I can improve to a
flush if a diamond comes on the turn or the river. There's 13 diamonds
in the deck of cards, I've seen 4 of them, there's 9 remaining. I have
9 outs after the flop to improve to a flush.

Now the turn comes \twoc\, giving a board of \Ad\tens\eigd\twoc. I
didn't make my flush yet, but I've picked up a chance at a
straight. If a 5 comes on the river I'll make the best hand. There's
four fives in the deck, so I now have 9 flush outs + 4 straight outs,
which looks like 13 winning cards for me. But one of my flush outs is
the 5 of diamonds, and I can't count the 5 of diamonds twice. So I
have 9 flush outs + 3 extra straight outs = 12 outs after the turn.

% FIXME: list all the outs

Now see what happens when I have the same cards and the same flop and
turn, but this time the big blind has \As\Ah. On the flop of
\Ad\tens\eigd\, the \tend\ isn't a winning card for me anymore, it
makes my flush but it makes the big blind a full house and I'm drawing
dead and no river card saves me. When the turn comes \twoc\, the board
is now \Ad\tens\eigd\twoc\ and the \twod\ isn't an out for me, again
it will make the big blind a full house. So this time I have 8 outs
after the flop and 7 flush outs + 3 straight outs = 10 outs after the
turn.

Notice in counting my outs I'm assuming all the diamonds and all the
fives are still in the deck. In reality this isn't the case, some of
the players who folded preflop would have folded a diamond and maybe
there's a diamond or a five as one of the burn cards. This actually
doesn't matter because all that counts is that, in the shuffled-deck
at the time the cards were dealt, the card that is in the position of
the river card is one of your outs. The chances of that card being a
winner for you were determined by the shuffle and not by what anyone
else has folded.

One time you do have to worry about fewer flush or straight outs being
available on the river is in multiway pots where three or more players
are still in the pot; there can be two players on flush draws and that
means that you're less chance of making your flush and even if you do
make a flush you can lose to a higher flush. In a heads-up
pot, where you've got the drawing hand against a made hand, just go
with the maths of full outs on the river.

\subsection{The Monty Hall Problem}

There's a famous Maths/Logic fallacy called ``The Monty Hall
Problem'', named after a US quiz show host. A quiz contestant gets to
pick a door from A, B or C; to see if he wins the big prize. Two doors
have booby prizes behind them, the third door has the big prize. She picks
door B. Monty, who knows which door has the big prize behind it, says
``just as well you didn't pick Door A, Barbara, because that one's a
loser'' and he opens door A to show that door has one of the two booby
prizes behind it. ``But here on the Monty Hall show, we'll give you the
chance to change your mind. Do you want to stick with Door B, or
change to Door C?''

A lot of people say it doesn't matter if she switches or stay; because
now she knows Door A isn't the prize door it's 50-50 between doors B
and C. This thinking is wrong. She should actually switch. At the time
Barbara chose Door B, her chance of picking it right was 1 in 3; and
that chance hasn't changed even now she knows Door A was a loser.

Imagine if there were 100 doors with 1 winning door and Monty opened
up 98 losing doors, and then gave her the chance to
switch. She'd switch then, because there's a 99\% chance the prize
door is the hidden door and just a 1\% chance she got the door right
on her initial pick. In the three door game, there's a 66\% chance
Door C in the winner and a 33\% chance Door B is the winner.

The TV show ``Deal or No Deal'' is full of Monty Hall style choice
decisions. I used to enjoy watching it, but I gave up on it because I
was screaming at the TV and tearing my hair out in frustration that
the contestants didn't know Monty Hall fallacies. Nowadays I prefer the
knowledge quiz show ``The Chase'' a lot more.

\section{Rule of Four, Rule of Two}

If you're on a draw, you can find your winning percentages by counting
your outs and using the Rule of Four or Rule of Two. After the flop,
your percentage chance of improving your hand (with two community
cards still to come) is your number of outs
times 4 (Rule of Four). After the turn, with just the River to come,
your percentage change of improving your hand is your number of outs
times 2 (Rule of Two).

So going back to my \fourd\tred\ against \As\Js\, when I flop the
flush draw I have 9 outs and my chance of improving to a flush by the
river is 9 x 4 = 36\%, about 1 in 3. After the turn, I have 12 outs to
a flush or a straight and now my chances of improving are 12 x 2 =
24\%, about 1 in 4. Even though I've picked up three more outs by the
turn, my percentage has gone down because now there's just one more
card to come.

The Rule of Four and Rule of Two work because the number of unseen
cards is close to 50. If you're on a flush draw with 9 outs, on the
turn your chance is 9/46 which is close to 9/50. 9/46 is 0.195 while
9/50 is 0.18, these numbers are close. On the turn, your chance is
9/45 which is close to 9/50. 9/45 is 0.20 while 9/50 is 0.18.

The Maths is tricky and I won't bore you with it here, but
Runner-Runner flush on the flop is a 5\% chance. Runner-runner
straight is about a 1\% chance.

\section{Pot odds}

Now you know your winning percentages you can tell if a call is worth
it or not. If you're a 24\% chance after the turn but you have to call
a bet of 1,000 into a pot of 1,700, you're not getting a high enough
return. You're a 1 in 4 chance, but you're only winning 1.7 times your
money. You want 3 to 1 or better. You need to win three times your
money. Forget about the fact that some of that money in the pot came
from your stack in earlier betting rounds. There's 1,700 in pot and
you have to call 1,000 to stay in. You should fold.

% FIXME: add EV calculation here

Now look at the case where you have to call 1,000
into a pot of 3,200 with a 24\% chance. This time you are winning
three times your money, if you hit on the river. Make the call. 3 out
of 4 times you miss and it costs you 1,000 each time, for 3,000 total,
but 1 in 4 times you win and win 3,200.

% FIXME: add EV calculation here

In tournaments you sometimes make a call where the pot odds don't
justify it, because chips are so important. In cash games you should
pay a lot of attention to the percentages and the pot odds.

\section{Implied odds}

Pot odds tell you what you'll win from the pot right now if you
improve your hand. But there's a final betting round after the river,
and if you hit your hand you should win extra chips on the river. Of
course if your opponent is already all in before you make your call,
there's no more chips to be won. But usually you can count on winning
at least the amount of the call extra on the river, so you can factor
that into your calling decision.

\section{Flop percentages}

When you have separate cards, half of the time you hit no pair, the
other half of the time you hit one pair and very occasionally (5\%)
you hit two pairs.

When you have a pocket pair, you flop a set about 12\% of the time, 1
in 8.

If you haven't hit the card you want on the flop, the chances of doing
it on either the turn or the river is about half as much as your
chances were of doing it on the flop. So if you miss your 12\% chance
of hitting a set on the flop, you're roughly a 6\% chance of doing it
on the turn or river. The Rule of Four says you have 2 outs after the
turn, so your chance of improving is 4 x 2 = 8\%.

This is why in an AA versus KK confrontation, the AA is an 80\%
favourite. KK needs to hit the third K to win, and that will happen
20\% of the time.

The chance that both AA and KK will hit a set on the flop is 1\%.

You probably know already that AK versus JJ is close to a 50/50 coinflip.
The percentages are actually AK 48\% and JJ 52\% but those numbers are so
close it's almost 50/50. Basically AK will end up with a pair on the
flop 50\% of the time and a pair by the river about 70\% of the
time. But JJ will make a set roughly 20\% of the time and that set
will normally be the winner. So the AK winning percentage is the
chance that AK will hit a pair minus the chance that JJ won't hit a
set. The JJ winning percentage is the chance AK won't hit a pair plus
the chance JJ will hit a set.


