\chapter{Expected Value in Poker}

The last two chapters have been a thorough look at two key ideas in
poker; expected value and counting outs. Now I'm going to join
those concepts together and show you how to use them in your betting
decisions and give you some suggestions on Bet Sizing. Before we get
to Bet Sizing, let's look at a Poker game where you don't get
to choose the size of your bet, Limit Holdem.

\section{Limit Holdem}

Elsewhere in the book I call this game ``Fixed Limit Holdem'' to
really get the point across that I'm talking about Limit Holdem.
In this section I call the game it's traditional name, Limit
Holdem.

Limit Holdem was my first Holdem game and I still love it. Hardly
anybody wants to play it anymore in Australia. In my early days of
Poker I played a few days of Limit at the famous Bay 101 Casino
in San Jose, Silicon Valley, California. I often daydream about
playing there again.

A big game of Limit Holdem took place in The Star in 2019, by
agreement between the players. The Star can only run No-Limit and
Pot-Limit games, but if they players wink-wink agree to only bet and
raise in Fixed amounts, and to cap the betting at the third raise, it's
a Limit game. The limits in that 2019 game were \$100/\$200.

% FIXME: was it in fact Limit Omaha?

In Limit you often get many bets and calls in the pot by the
turnstreet. In a \$10/\$20 game, there's often \$150 or more in the
pot with one card to come. Three players limped the
\$10 preflop, the button raised to \$20 and the small blind,
big blind and the three limpers all called. Six players took the flop,
the pot was \$120. The button bet \$10 on the flop, three players
dropped out but two players call the \$10. The pot is \$150 and the turn is
shown. The betting amount now doubles to \$20.

In this example there's \$150 in the pot on the turn. The Leader has
top two pair and is considering betting. The Chaser has a flush draw
and an 18\% chance of improving to the winning hand on the river.
This is a classic Made Hand versus Drawing Hand confrontation.
The third player still in the hand has top pair no kicker and will
fold to any bet and is drawing dead.

If the flush gets there on the river, the Leader will call a \$20 bet
from the Chaser and lose the pot plus that \$20 call. If the flush draw
misses, the Leader will bet \$20 and the Chaser will fold.

\subsection{If the Leader Bets the Turn}

This is the easiest bet to make in Poker; just bet it and hope that
your hand holds up and enjoy the money when the pot gets pushed to
you.

EV = P(win) x (amount won) - P(loss) x (amount lost) \\
   = 0.82   x (\$170)      - 0.18    x (\$190) \\
   = \$139.40 - \$34.20 \\
   = \$105.20

\subsection{If the Leader Checks the Turn}

If the Leader is a timid player and won't bet the turn
he wins less.

EV = P(win) x (amount won) - P(loss) x (amount lost) \\
   = 0.82   x (\$150)      - 0.18    x (\$170) \\
   = \$123.00 - \$30.60 \\
   = \$92.40

Now let's look at the EVs for the Chaser.

\subsection{When the Chaser calls the Turn}

The Leader has bet the \$20 on the turn. I said that was the easiest
bet in poker. The Chaser has the easiest call in Poker.

EV = P(win) x (amount won) - P(loss) x (amount lost) \\
   = 0.18 x \$190 - 0.82 x (\$20) \\
   = \$34.20 - \$16.40 \\
   = \$17.80

\subsection{When the Chaser gets a free river}

If the Leader wass timid and didn't bet the turn, the Chaser gets a
free river. He has a 18\% chance of winning the \$150 pot plus a \$20
crying call on the river

EV = P(win) x (amount won) - P(loss) x (amount lost) \\
   = 0.18 x (\$170) - 0.82 x (\$0) \\
   = \$30.60 - \$0.00 \\
   = \$30.60


These EVs are big postive amounts for both the Leader and the
Chaser. How is this possible? It's possible because of the four
players who were in the pot earlier but have since folded. Three
players paid \$20 to see the flop and dropped out, a fourth player
paid another \$10 to see the turn but is now drawing dead. There's
\$90 of dead money in that \$150 pot, and the Leader and the Chaser
are fighting over that \$90 plus each other's \$30 so-far-invested.

A hand of poker involves a series of betting decisions. The Leader and
the Chaser have made two betting decisions already; each with their
own EV equation and each being either a good bet or an EV Mistake. But
money in the pot belongs to the pot, and with the action on the turn,
both the bet and the call are good bets.

\section{Stack to Pot Ration: SPR}

In most Limit Holdem hands, the focus is on winning the nice big pot
as it stands. If you can win an extra \$20 on the river that's a
bonus, and it's unusual to win an extra \$40 and almost unheard of to
win an extra \$60. With a pot of \$150 and the chance of winning an
extra \$20 on the river, the main goal is to win the \$150.

% FIXME: expand on this once you have time.
