\chapter{Expected Value in Poker}

The last two chapters have been a thorough look at two key ideas in
poker; expected value and counting outs. Now we're going to join
those concepts together and see how to use them in your betting
decisions. First, a look at a Poker game where you fold, check, bet,
call and raise, but by a fixed amount each time: Fixed Limit Holdem.

\section{Fixed Limit Holdem}

I grew up with Fixed Limit and still love it. Nobody wants to play it
anymore in Australia. In my early days of Poker I played a few days of
Fixed Limit at the famous Bay 101 Casino in San Jose, Silicon Valley,
California. I often daydream about playing there again (it's moved a
few miles since the time I played there in 2000, but I'll be able to
find it.)

In Fixed Limit you often get many bets and calls in the pot by the
turnstreet. In a \$10/\$20 game, there's often \$140 in the pot with
one card to come. The betting amount by now is a fixed \$20 but that's
the only amount you can bet. Let's look at the EV for the Leader,
with top two pair, betting against a Chaser who has a Flush Draw.
If the flush gets there on the river, the Leader will check/call a Bet
of \$20, and lose that extra money. If the flush doesn't come on the
river, the Leader will be \$20 and the Chaser will fold and the leader
will win the pot of \$180, which is a \$160 win on his turn bet of
\$20.

EV = P(win) x (amount won) - P(loss) x (amount lost) \\
   = 0.82   x (\$160)      - 0.18    x (\$180) \\
   = \$128 - \$32.40 \\
   = \$95.6

That's the easiest bet to make in Poker; just bet it and hope that
your hand holds up and enjoy the money when the pot gets pushed to
you. If the Leader is a timid player and won't bet the turnstreet
he wins less. In this next EV equation, the Leader doesn't bet the
turn and will check/call the river and lose a bigger pot if the flush
card comes in. If the flush misses; the betting will go check/check.

EV = P(win) x (amount won) - P(loss) x (amount lost) \\
   = 0.82   x (\$140)      - 0.18    x (\$160) \\
   = \$114.80 - \$28.80 \\
   = \$86.00

So if the Leader is timid and doesn't bet the turn, he'll win \$86 on
average, but if he bets the turn, he'll win \$95.60 on average. Let's
look at the EVs for the Chaser.

If the Leader knows he's good and bets \$20 on the turn, the Chaser
has to call \$20 for a chance of winning the \$160 in the pot plus
another \$20 when the Leader crying calls the river.

EV = P(win) x (amount won) - P(loss) x (amount lost) \\
   = 0.18 x \$180 - 0.82 x (\$20) \\
   = \$32.40 - \$16.40 \\
   = \$16.00

If the Leader is timid and doesn't bet the turn, the Chaser gets a
free river. He has a 18\% chance of winning the \$140 pot plus a \$20
crying call on the river

EV = P(win) x (amount won) - P(loss) x (amount lost) \\
   = 0.18 x (\$160) - 0.82 x (\$0) \\
   = \$28.80 - \$0.00 \\
   = \$28.80
