\chapter{Expected Value in Poker}

The last two chapters have been a thorough look at two key ideas in
poker; expected value and counting outs. Now we're going to join
those concepts together and see how to use them in your betting
decisions. First, a look at a Poker game where you fold, check, bet,
call and raise, but by a fixed amount each time: Fixed Limit Holdem.

\section{Fixed Limit Holdem}

I grew up with Fixed Limit and still love it. Nobody wants to play it
anymore in Australia. In my early days of Poker I played a few days of
Fixed Limit at the famous Bay 101 Casino in San Jose, Silicon Valley,
California. I often daydream about playing there again (it's moved a
few miles since the time I played there in 2000, but I'll be able to
find it.)

In Fixed Limit you often get many bets and calls in the pot by the
turnstreet. In a \$10/\$20 game, there's often \$150 or more in the
pot with one card to come. Three players limped to the
\$10 big blind preflop, the button raised to \$20 and the small blind,
big blind and the three limpers all called. Six players took the flop,
the pot was \$120. The button bet \$10 on the flop, three players
dropped out but two players call the \$10. The pot is \$150 and the turn is
shown. The betting amount now doubles to a fixed \$20.

Let's look at the EV for the Leader, with top two
pair, betting against a Chaser who has a Flush Draw. The third player
still in the hand has top pair no kicker and will fold to a bet.
If the Chaser hits his flush on the river, the Leader will check/call
a Bet of \$20, and lose that extra money along with the pot. If the
flush doesn't come on the river, the Leader will bet \$20 and the
Chaser will fold.

% FIXME: Check the numbers in these equations.

EV = P(win) x (amount won) - P(loss) x (amount lost) \\
   = 0.82   x (\$170)      - 0.18    x (\$190) \\
   = \$128 - \$32.40 \\
   = \$95.6

This is the easiest bet to make in Poker; just bet it and hope that
your hand holds up and enjoy the money when the pot gets pushed to
you. If the Leader is a timid player and won't bet the turnstreet
he wins less. In this next EV equation, the Leader doesn't bet the
turn and will check/call the river and lose a bigger pot if the flush
card comes in. If the flush misses; the betting will go check/check.

EV = P(win) x (amount won) - P(loss) x (amount lost) \\
   = 0.82   x (\$140)      - 0.18    x (\$160) \\
   = \$114.80 - \$28.80 \\
   = \$86.00

So if the Leader is timid and doesn't bet the turn, he'll win \$86 on
average, but if he bets the turn, he'll win \$95.60 on average. Let's
look at the EVs for the Chaser.

If the Leader is bold and bets \$20 on the turn, the Chaser
has to call \$20 for a chance of winning the \$170 in the pot plus
another \$20 when the Leader crying calls the river.

EV = P(win) x (amount won) - P(loss) x (amount lost) \\
   = 0.18 x \$190 - 0.82 x (\$20) \\
   = \$34.20 - \$16.40 \\
   = \$17.80

If the Leader is timid and doesn't bet the turn, the Chaser gets a
free river. He has a 18\% chance of winning the \$150 pot plus a \$20
crying call on the river

EV = P(win) x (amount won) - P(loss) x (amount lost) \\
   = 0.18 x (\$170) - 0.82 x (\$0) \\
   = \$30.60 - \$0.00 \\
   = \$30.60

These EVs are big postive amounts for both the Leader and the
Chaser. How is this possible? It's possible because of the four
players who were in the pot earlier but have since folded. Three
players paid \$20 to see the flop and dropped out, a fourth player
paid another \$10 to see the turn but is now drawing dead. There's
\$90 of dead money in that \$150 pot, and the Leader and the Chaser
are fighting over that \$90 plus each other's \$30 so-far-invested.

