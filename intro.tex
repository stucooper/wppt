\chapter{Introduction}

% Last updated: 20200517

Small buyin No Limit Holdem Pub Poker Tournaments are excellent
entertainment and gambling opportunities for players prepared to do
some work. This book is all about how to play these games.

There are many different kinds of Poker tournament, I'll cover these
soon in the chapter ``Types of Pub Tournaments''. For now a
working definition of a small buyin
Pub Poker Tournament is a game with a single once-only entry fee of
between \$10 and \$40 dollars, which will be played by 10 to 50
players and will be finished in less than 4 hours.

You've played a few of these games already. Maybe you've done well at
some, made the final table a few times, maybe won one or two.
Or perhaps you keep finishing about halfway through, and just can't
seem to get the right cards at the right time. If you carefully read
this book and work on your game, your results will improve and you'll
enjoy your poker more.

\section*{Standard Poker Books}

% \section*{name} : an unnumbered section that won't appear in the contents

Most of the standard classic No Limit Holdem Poker Tournament books
focus on big stack poker tournaments with slow blind increases.
I'm thinking of books by Dan Harrington, David Sklansky, Lee Nelson
and T J Cloutier. Slow tournaments are about patience, endurance,
trapping and shifting gears. Pub poker is a much
shorter, faster game, so a different approach
is required. I encourage you to read the
classic authors, and also to play in slow deepstack tournaments,
but keep in mind that a different set of skills is required to succeed
in the fast Pub games. This book concentrates on these fast
tournament skills.

Just as the advice of Harrington and the others isn't always right
for fast Pub Poker games, my advice isn't good
for big buyin deepstack games, and is even worse for
cash games. If you want to succeed in the big games you'll have to do a
lot more work and reading than what you'll find in this book.

The best book on how to play Fast No Limit Holdem Tournaments
(besides this one) is The Poker Tournament Formula by Arnold Snyder.
I look at this fantastic book more in the chapter ``Position Plays''.
Make sure you get ``The Poker Tournament Formula'' and not the followup,
``The Poker Tournament Formula 2'', which is about slower big buyin games.
Get PTF2 later, but start with the original which is targetted to fast
NLH tournaments which is exactly what a single night pub tournament is.

You can win your way into big buyin games through fast games,
so fast games skills can help get you cheap entry into the big games.
Also even slow games become fast games once the blinds get too big,
so you'll need fast game skills even in games that start slow.

\section*{Enjoy your poker}

As well as presenting strategies to help you succeed
in pub tournaments,
I also write about how you can enjoy your poker
more. Poker can be a very enjoyable game if you approach
it with the right attitude, but I see many players at the table
not having a good time. I don't think there's much
point playing if you're not enjoying it.

Usually the grouches make the game bad for the other
players as well. If you have a good time at the table, you'll make the
game better for the others as well, at the same time as you're taking
their chips.

In the same way as you should look for games where you'll have
a good chance of playing well and winning, you should
also look for games where you'll enjoy yourself.
If you can find somewhere where you like
the organisers, the players, the venue and the other things
you can do at the venue when you're out of the action,
you'll have some great poker experiences.

\section*{Gambling}

Poker is a gambling
game: a game with a skill component certainly but still
a gambling game. You can go all-in for your tournament life
as a 70\% favourite and easily lose. There are plenty of problem gamblers
in Australia, mostly on poker machines in the Pubs and Clubs
(the same Pubs and Clubs you'll be playing Holdem Poker
in) but also horses, greyhounds, sports, lotteries and even Keno.

Poker's a seductive and addictive game,
and tournaments can be played from all stakes from free to \$10 to \$100
to \$1000 to \$10,000 to \$100,000. Make sure you're keeping it fun
and enjoying your poker. Make sure you're in control at all times,
and if it stops being fun, stop doing it. Keep your life in balance.

Some of the better pub poker players I know are actually
self-excluded from The Star Casino in Sydney. They've had
their problems with cash poker or the bigger buyin poker games or
House games like Blackjack, Baccarat and Roulette. They've
made the choice to stay out of the Casino and play smaller poker games,
where they can relax and enjoy themselves. That's a really smart
move, and I think well of these players for making that choice.

\section*{Disclaimer}

This book is written with all care taken but no responsibility accepted.
Following my suggestions does not guarantee success.
There's plenty of different ways to play poker, think about poker
and write about poker. You might disagree with some of my advice.
That's fine. At the very least, this book should help you think
more clearly about your poker.

I've lived with this book for many years and revised every chapter
carefully at least twice. I can honestly say that this is the best
poker book that I'm capable of producing.

\section*{Pub Trivia}

In my University years (1989-1992) my Mother, Brother and I used to
play in a lot of Pub Trivia nights, in the Newcastle area. Some of
them had an entry fee and guaranteed prizemoney, other games were for
the benefit of schools or charities, with donated prizes.

We took our Trivia seriously, and had some good wins and
great nights out. A lot of the other teams weren't taking the games
seriously and never did as well as we did.

After a while we'd recognize other trivia regulars. They were also
taking the games seriously and playing a few nights a week and going
from venue to venue to find the best Trivia nights. Just like we
were.

Before Trivia there was a scene of Pool Competitions. Pool
competitions and Trivia are still around, but these days the main Pub
entertainment is Poker Tournaments. You can imagine paid Pub
entertainment evolving over the years from Pool Competitions to Trivia
Nights to Poker Tournaments.

I still play Pub Trivia and I'm very good at it, but these days
I only play about once every two months.

You'll see familiar faces at your pub poker games,
regular players who are taking the game seriously. Alongside them
will be more social players, who might be regulars at the venue
and some regular players who just aren't very good. They're playing
poker regularly, but they're not working on their game and they're
just going through the motions the way they always have and always
will.

I once played the first Million Dollar winner of Australian ``Who
Wants to be a Millionaire'' at Pub Poker! He was very much a beginner,
and called a river bet on a board of JTT98 holding A4. I would've like
to have played him for his whole million dollars!

\section*{Reading this book}

I've arranged this book so it can be read from start to finish, but jump
around between sections if that works better for you. It's your book too.
Sections II and III, ``Playing your hand'' and ``Essential Skills'' form
the bulk of the playing advice, so make sure you read those, together
with ``Putting players on a hand'', which I've put just after the
Dealing chapter, as the skills I teach in hand reading are related
to good dealing skills.

\section*{What you won't find in the book}

This book is all about how to play No Limit Texas Holdem Fast
Tournaments in a non-Casino, Pubs-and-Clubs setting. There's one or
two ideas from Cash poker, Fixed Limit poker and Online poker in the
book, and a tiny bit of Omaha and the tiniest bit of Seven Card Stud.
But this is \textbf{not} a book about how to play poker online, or how
to play Cash poker or Omaha or Stud.

There's no advice given on Tournament Direction or how to run Home
Games. This is a book for pub poker players, written by a pub poker
player.

Or perhaps an ex-player. I mostly play Cash Pot Limit Omaha these days;
a thrilling action game where you can win or lose hundred of dollars
in three minutes, compared to a Poker Tournament where you lose most of
the time and only occasionally make a big score. Pot Limit Omaha
is like Crystal Meth compared to No Limit Texas Holdem's Cigarette
Smoking. Don't smoke and stay well away from Ice, but give Omaha a go
sometime.

I've got big Omaha games on Friday nights, Saturday nights and Sunday
nights so even without a pub poker game I still get about twenty hours
of poker action a week.


\section*{Acknowledgements}

I've written this book entirely on my own and self-published it, so I don't
have too many people to thank. I did my own proofreading, so any mistakes
are certainly mine.

A lot of my early pub poker was at the legendary \textbf{Jade Tavern},
which for years had a day and night commitment to poker.
It was a great venue to play in, largely because of the enthusiasm
of \textbf{Paul Walker}. Sadly Jade is no longer operating and
its loss is still keenly felt. The premises is now a Karaoke bar.

Among my poker friends I'll single out \textbf{Simon Rea} for his friendship,
encouragement and continuing help in my poker career.

Lastly, \textbf{Al Markarian} did a super job in producing Australian
Poker Weekly, which ran every single week for two and a half years
until Online Poker's Black Friday. It was a pleasure to write for
it. I've re-used one of my APW articles as a chapter in this book.

\section*{Contacting Me}

The best way to contact me is by email. My email address
is stuart.cooper@gmail.com.

\section*{Writing this book}

If you think \textit{playing} poker is hard (and it is) then you should try
\textit{writing} a poker book; which is about four times harder
again. This book nearly drove me mad; I've written other books before
and this one has been by far the hardest book to write, to proofread
and to finish. I'm very satisfied that this book is finished. I've
enjoyed working on the book, and it's helped my game a lot and I'm
proud of the result.

A lot of the final work on the book was done during the 2020
Coronavirus COVID-19 shutdown, when nobody could play pub poker,
Casino poker or even home game poker. With a lot of time on my hands,
working on the book was a good reminder of happier poker times. The
COVID-19 shutdown was the worst Bad Beat I've ever seen in poker.

This was a hard book to stop writing. I'm always working on my game
and coming up with new ideas to use in my poker, both from experience
at the tables and from reading books and the Internet. However the
information in this book is complete and well worth
publishing. If I come up with many more new poker ideas I'll write
a second book.

I probably spend an hour studying poker for each tournament I play.
You might not have as much time to study poker as I do, but you'll
find like most things in life the more work you put in, the
more you'll get out of it.

Good luck at the tables.

\section*{Dedication}

This book is dedicated to the memory my father,
Michael Douglas Cooper (1933-2017).



%% \section*{About the author}

% FIXME: Back cover copy?

%% Stuart Cooper wrote the ``Doctor Straight'' column, primarily
%% on Pub Poker Tournaments, in the Australian Poker Weekly newspaper
%% for the entirety of the newspaper's two-and-a-half year run.
%% In the 131 issues of the paper he produced 105 columns.

%% Stuart has three major tournament wins (50\$ buyin, 100+ players)
%% to his credit and many small nightly tournament wins.

%% He lives and plays poker in Sydney.
