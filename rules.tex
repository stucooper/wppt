\chapter{Rules}

% Last updated: 201200428

In 1933 Don Bradman sat and passed the cricket umpire's exam. I don't
think he ever wanted to umpire; he was more interested in making
sure that he knew all the rules and could use them to his advantage;
and that opposition captains couldn't put anything past him. A very
canny player.

The two best sources of poker rules are both found online.
One is called ``Robert's Rules of Poker'' by Bob Ciaffone
who's also written poker strategy books. This covers
poker in general and is excellent for home games. For tournaments,
you should also consult the Poker Tournament Director's Association,
at http://www.pokertda.com. I refer to Poker TDA rules in a few
places in this chapter.

% A sit and go is a tournament. Indeed a lot of venues call them 
% ``single table tournaments''. Maybe ``single table tournaments'' sounds grand.
% Non-Casinos in Australia are legally allowed to run only
% tournaments, so they try to call everything
% a tourmanent. New poker players find the term ``sit and go'' 
% confusing (I know I did at first).

% A Chip Chop is also a tournament, a Timed Tournament. Tournament rules
% still apply- in particular if players are all-in with no more betting and
% there's still community cards to come, hands must be shown.

Here are some misunderstood and poorly applied rules.

\section{Preflop Min Reraise}

Blinds are 50-100, and the under the gun player
raises to 500. If someone wants to reraise, the minimum
he needs to raise to is 900, not 1,000. 

This is because the initial raise was from a call of 100 to an
amount of 500, which is a raise of 400. The next raise needs
to be at least 400, so is 500 + 400 = 900 or more.

Imagine that the small blind and big blind didn't have to be posted
before the cards were dealt, but had to be placed as part of the
preflop betting round, and announced. The small blind says
``I bet fifty'', the big blind says ``Raise to a hundred''.
UTG says ``Reraise to five hundred''. UTG's 500 bet isn't a bet
of 500 from a start of 0, it's a raise of 100 to 500, making it a bet
of 500 but a raise of 400. The next raise needs to be at least 400
again.

The small and big blind are posted before the cards are dealt to
speed up the game and make sure they've been properly paid; so
people have forgotten over time that there is already a blind raise
in the pot.

After the flop, turn and river, when bets are done into a new
betting round, if someone bets 500 it's a straight bet of 500
and the raise amount is 1,000 or more.

I love making preflop min reraises like this to smoke out
the ``experts'' who insist the reraiser has to make it 1,000
or more. A small argument usually develops which annoys the
experts when they're shown to be wrong. Anyone who stands
up for the min-reraise to 900 is usually a thoughtful player
who knows about rules and has a good amount of poker experience.

Strategy-wise the min re-raise is a cheap isolation tactic
that often gets you heads up with the bettor, and saves you
a few chips when the bettor has a super strong hand and
re-re-raises you out of the pot preflop.

\section{Action out of turn}

This is the plague of the pub poker scene. Before
someone gets to act, a later player announces a bet (even ``all in'')
out of turn, sometimes as an angle shoot to bully the earlier player
into checking. There are only two possible (but very different)
rulings on the out-of-turn's action:

Ruling A: The bet is ruled void and the chips are returned to the out-of-turn
player. If all the players before him check, the out-of-turn player
must also check. If a player before him bets, then free
action is available to the out-of-turn player who can now call/raise/fold
as usual.

Ruling B: The bet is valid. If an earlier player bets more, out-of-turn's chips
remain in the pot even if he decides to fold. 

Ruling A is the most common and would be the expected ruling in most
Pub games/mob rules games. Ruling B has been adopted
by Crown Casino for their cash games, so may become more prevalent
in the future. I've played at some venues that enforce Ruling B, and
I personally prefer that interpretation.

Under Ruling A, late-acting players can angle-shoot to get free
cards, usually when they're on a draw. Say the flop is 89K, two hearts.
An out of turn player holds A9 offsuit (Ace of hearts) and can announce
all-in to bully everyone into checking around and give him a free
turn card which can either give him two pair AA99, trip 999 or a draw
to the nut flush.

Late acting players already have enough advantages in poker without 
being able to bully free cards like this. I'm hoping ruling B can 
gain ground, but unless you know otherwise, expect ruling A to apply.

Poker TDA rules lean towards ruling B, but only if the in-turn players
check or fold to the out of turn player. ``Action out of turn may be binding
and will be binding if the action to that player has not changed. A
check, call or fold is not considered action changing''.

% FIXME: carefully re-read Robert's Rules on this

\section{The full-bet rule}

This is a rule you often see completely ignored in pub games. 
This is part of the price you pay for playing
in pub tournaments- they're cheap to play and you play against
bad opponents, but rules sometimes get ignored or incorrectly applied.

%% Even a major online site once applied this rule incorrectly in
%% a cash game, and to their credit refunded the player disadvantaged
%% when he pointed it out to customer support at the site concerned.

Here's how the full bet rule works. A player bets. Another player 
raises all-in for more than that bet, but less than what a minimum raise is. 
The all-in raiser doesn't have enough chips to put in a full raise. 
Other players call. When action returns to the original better, he 
cannot reraise, he can only fold or call the all-in player's bet.

To put some numbers in it, imagine Susan Smallblind has 10,000,
Bill Bigblind has 4,000 and Danny Dealer has 17,000. After the flop, 
Susan bets 2,200. Bill goes all in for 4,000 which is less
than a full raise. Danny calls the 4,000 and action is now back on Susan.
Susan can now either fold or just call the 4,000. She can't go all-in 
for her 10,000.

On the other hand, if Susan had bet 2,000, Bill goes all in for 4,000
and Danny calls the 4,000 Susan can now push all in for 10,000 since
Bill's all-in was a full raise, re-opening the action to Susan.

Because No Limit means ``you can bet all your chips at any time'', and
that's nearly always true, application of the full bet rule is sketchy
at best.

Here's a one sentence rule that sums up the full-bet rule:
``If it wasn't a full raise, betting is not re-opened''.

Poker TDA rules: ``In no-limit and pot-limit, an all-in bet of less
than a full raise does not reopen betting to a player who has already
acted''.

Make sure you understand ``re-open'' and ``already acted''. It's only
Susan who can't raise in this betting round. Danny can still go all-in for
his 17,000 when it's his turn.

\section{Single chip throwout}

Blinds are 50-100 and first to act casually throws out a 500 chip.
What does this mean? A raise to 500? A call of 100? A minimum raise
to 200? What's going on here?

This area of poker really needs tidying up, in the same way that
string betting has been tidied up out of the game. In casinos, a
single chip throwout constitutes a call.

I found this out the hard way when I played in a 10/20 limit game
and the first hand I got was KK in the big blind. Action got around
to me with action at \$20 with 4 runners and I wanted to re-raise. I'd
just bought in for \$400 so I had two big piles of red \$5 chips
for \$200 and 8 little green \$25 chips for the other \$200. Without
announcing anything I tossed a green \$25 chip in next to my two red
\$5 chips making my \$10 big blind, and it was counted as a call.

I ended up winning the pot with my pocket Kings, but by getting my
raising wrong with the single chip toss-in and no verbal announcement,
I cost myself another \$30 in the pot when the other three players
in the pot would have called my preflop re-raise.

I've played thousands of hands of poker since my single-chip toss-in
cost me \$30 in a pot. I have not made the single chip throwout mistake 
again in any of those hands. You learn your lessons well when they cost 
you money.

In some Pub games, the guy who threw out the 500 chip would be asked
what he wanted to do. He could potentially angle-shoot the mood
of the table to his thrown-out chip before deciding if he was
raising or calling.

If two or more chips are thrown out without announcement, it stands
as the value of the chips thrown in, or a call if it's not a full
raise. So if the call is 800 and someone chucks in two 500 chips,
that's a call and 200 change is given.

The safest and simplest thing to do is announce what you're doing
at every action. ``Call 600. Raise to 1,500. Fold''. Try not to
say ``Calling'' because it sounds an awful lot like ``All in''.
If you do say ``Calling'', say the amount such as ``Calling 600''.
I sometimes say ``Play'' or ``Still in'' when making a call.

Poker TDA rules give many sentences to this issue, in a section
called ``Oversized Chip''. ``A single oversized chip will be considered
a call if the player does not announce a raise. If a player puts an
oversized chip into the pot and states raise but does not state the
amount, the raise will be the maximum allowable up to the size of that
chip. After the flop, an initial bet of a single oversized chip
without comment will constitute the size of the bet. To make a raise
with a single oversized chip a verbal declaration must be made
before the chip hits the table surface''.

%% \section{Straddles}

%% A straddle is a raise made by the under the gun player at
%% the time the blinds are put out. He hasn't seen his cards yet,
%% or even been dealt them. For example, small blind posts
%% 50, big blind posts 100 and the under the gun player chucks in 300.
%% This is called raising ``no look'' or ``blind'' or ``in the dark''.

%% This move is seen more often in cash games that have been running
%% for a while, and some big stack players think there's not much
%% action with blinds at \$2/\$3 so let's chuck \$10
%% in no-look to shake up this pot. It can turn up in tournaments,
%% when the straddler wants to project an image of a loose gambler,
%% which is exactly the image you'll be projecting if you play
%% the way I suggest in this book. You'll be in enough showdowns anyhow
%% that you won't need to straddle yourself, but if you're at an ultratight
%% table or you're the monster chip leader, a straddle to three times
%% the blind can be fun. If there's some new players at the table they'll
%% be confused by the straddle, and confusing players new to the game
%% can be a good strategy.

%% Straddles do not re-open the action to the straddler. Once a few
%% players have called the 300, and nobody's raised, he can't re-raise
%% his own 300 bet (even if by now he has looked at his cards
%% and found pocket Kings!). In some cash games, the straddler can re-raise
%% preflop here. This game is said to have ``live straddles''. You won't
%% find such a game in a Pub tournament, always expect straddles to be dead.
%% Ask for a ruling from the Tournament Director if you want.

%% Poker TDA rules do not cover straddles. As I said, they're pretty
%% unusual in tournaments.

\section{Players joining the table}

A player joining the table when moved there by the tournament director
gets dealt cards in any position except the small blind. 
He can have his first hand on the Dealer Button.

If a player has turned up late, and he is joining his stack which has been
blinded out in his absence, he is able to take play his first hand in
any position including the small blind. His stack would have been blinded 
for the big blind the hand before.

There is always a big blind in every hand of poker. If the player who
would have been small blind has been busted or moved to another table,
a big blind only is posted and there is no small blind for that hand.
Next hand, the Button will be on the empty position, and the player
who posted the sole big blind will now post a small blind.

\section{Button in Heads Up play}

The Button is the dealer and has the small blind. He acts first preflop
but acts second on the flop, turn and river. When play moves from
three handed to two handed (third place busts out) you can't be big blind
two hands in a row, so if you were big blind when three handed, you become
small blind and the Button for the first hands of Heads Up. I used to
get this situation wrong, but Poker TDA rules corrected me.

\section{Respect the Tournament Director}

You've got to respect the Tournament Directors. They're honest
people trying to run a good night of poker. They don't get paid very much,
many are Uni students with a love of poker making a little bit of money
on the night. Most of them are very likeable people with at least a 
few months' experience and give good rulings.

If you get some bad rulings and don't like the game, don't kick up a fuss.
Play out your tournament and don't play there again. Take your poker
custom elsewhere.

