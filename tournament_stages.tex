\chapter{Tournament Stages}

% Last updated: 20190530

Broadly there are three stages to a Fast Pub Poker Tournament;
Early, Middle and Late.

\begin{description}

\item[Early] A lot of players think that the Early Stage of a
Tournament is where you fold a lot, to establish a tight image,
that will be to your benefit later on when you need to start
stealing blinds and bluffing more.

There's some merit to this line of thought, but for me every stage
of a Poker Tournament is about winning chips. You can make
your Position Plays from the very first hand of the Tournament,
establishing your Maniac image at a time when losing a few pots
doesn't cost you much.

During the early stages when stacks are still 40BB or more, you can
play the Cracking Hands; suited connectors, one-gappers and weird
low card hands. If you can get an early payoff and double up with
one of these hands, you'll be in a great position both
Chipwise and Imagewise.

If you play No Limit Holdem Cash games, the Early stage of a Tournament
is where your cash skills are most appropriate. Later on, as stacks
are shorter, you have to use Tournament skills more and Cash skills less.

\item[Middle] The middle stages of a Fast Tournament are from the
fourth blind level to the Final Table. A few players (maybe 10\%)
have busted out already and stack sizes are now very different
at the table.

One traditional piece of Tournament Wisdom is to target the Small Stacks.
You threaten them with elimination if they call you all-in and lose.
Again this line of thought has merit but unless you're up against
a very passive player, a small stack will sooner or later take a stand.

Layne Flack says the correct approach is to target the average stacks.
They've got enough chips that they can afford to make a good fold,
but not so many chips that they'll call you and still be OK if they lose.
Target the average stacks.

\item[Late] The first thing you do at the final table is to act cool
and look like you belong. You make final tables all the time, this is
nothing special. The other players will all be doing the same
thing, everyone's a champion. This is a harmless bit of personality
bluff that all players do. You don't want to look shaking and
nervous.

Stack sizes are probably pretty short by this time, and there's
actually a fair bit of luck involved in the seat draw. If you're under
the gun or just past it, your blinds are coming up very soon, before
you've got a chance to see what the other final table players are
like. You'll probably have played with four of them so far during the
tournament, but three or four of the other players will be new to
you.

Try playing ``High Card Holdem'' for a few hands. Avoid those 98
suited connectors that were good to you earlier in the
tournament. Fold J8 suited. You need a hand with great Hot and Cold
Strength to take a stand with.

If you do get a good seat draw and don't have to pay blinds for four
or five hands, you've got the double advantage of other players
possibly busting out even before you have to pay a blind, plus late
position that you can use for blind stealing if everyone folds before
you.

Take a bit more time with your preflop decisions at the final
table. You want the other players to know you're a thoughtful preflop
player who is getting dealt close-to-good hands that are almost worth
taking a flop with; even though you're holding onto Jack-Four offsuit
for eight seconds before folding it.

Always re-raise with JJ, QQ, KK or AA. You don't flop sets with them
often and they should be the best hand right now (AA of course
is). If you flat call and try to trap, either you'll be beaten by the
flop and lose unexpectedly or the flop will miss your enemy and he'll
find an easy fold to your flop bet. Your dream that you'll hit your
set and your enemy will still like that flop enough to lose all his
chips to you is very very small. You're much better off re-raising
preflop.

\end{description}
