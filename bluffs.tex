\chapter{Bluffs}

% Last updated: 20190205

People who haven't played a lot of poker think that the game
is all about Bluffs and Tells. While they're an important
part of the game, they're not the be-all and end-all.

I've covered a lot of Position bluffs that you should be making
regularly throughout a Fast Tournament. This chapter covers
a few other bluffs that you can make yourself or look out for
from other players.

\section{Entitlement Bluff}

Sometimes there's a wet flop but the Turn and River give no help
to the obvious draws. JdTd5c2h6h didn't complete any straights
or the Diamond flush.

% FIXME: LaTeX cards for the board above

A player who flopped a good draw feels entitled to winning
this pot, and will call a bet on the flop, a bet on the turn
and try bluffing on the river when his draw misses. I call
this bluff an Entitlement Bluff.

On this board, JdTd5c2h6h, whoever was the leader on the flop
will still be in the lead on the river. The flop was wet enough
that if the caller had a better made hand, he should have raised
earlier in the hand, as he himself would be afraid that the bettor
was betting a drawing hand and might draw out.

An Entitlement Bluff is usually easy to spot and call down; it
just smells wrong. It doesn't make sense from the three street action
``he's behind on the flop, he's behind on the turn, the river
didn't change anything and now he's betting?? That's got to be
a bluff- I call!''.

I've made some pretty bad Entitlement Bluffs and easily called
other people's Entitlement Bluffs. When I'm the leader on
an unchanged Board and I'm first to act, I'll often check the
river to induce an Entitlement Bluff. I can't get ``Three Streets
of Value'' by betting it myself, my best play is to check/call here.

If I'm last to act as the leader, I'll call an Entitlement Bluff
if he's made one, and Value Bet or Check Behind if he's checked.
I'll value bet if I have a good two pair hand or a set; if I've
got just a single pair (even top/top) I'll often check behind
because drawing hands can sometimes sneak two pair (in this example,
5d6d misses the flush but spikes a winning and callable two pair
on the river). A value bet in this circumstance risks being one
of those bets that only gets called when I'm losing.

\section{Scary Board Bluff}

A Scary Board is one where there's four-to-a-flush or four-to-a-straight.
A board showing Two Pairs is also Scary, on that board there's
four-to-a-fullhouse.  On Scary Boards it's obvious that the
right one-card hand makes a powerful hand. Normally this is a
cooler situation and betting slows down with check-check, but it's
also an opportunity to make a daring big bluff called the Scary Board Bluff.

Scary Board Bluffs can get Two pair or even Sets to fold, if they're
big enough.

\section{Re-Steal}

You won't be the only player trying to steal the blinds, especially
as the blinds get big and on final table play. Other players will
be trying to steal your chips when you're the Small Blind or the
Big Blind.

When you're in the Blinds, you get to act in late position for
the first round of betting only. If you consider a bettor is
trying to steal your blinds, you can reraise and try to
get him to fold.

This bet will be pretty big and if you get it wrong you can
lose a ton of chips if not your Tournament life. But if you get
it right you'll win a lot of chips and quieten down other
players at the table who now know it's not a pushover to steal
your blinds.

This move is High Risk/High Reward.

If the bet has come from early position, it's probably not a Blind Steal,
unless you've got a read on this player as being pretty wild. A bet
from the Button or Cut-off after a bunch of folds is more likely to
be a Steal and your Re-Steal is more likely to succeed. Wild players
get dealt good starting cards, and they get dealt good starting cards
on the Button and the Cut-off. If your Re-steal gets snapped off and
you lose to a powerful hand, never mind, at least you went down fighting.
