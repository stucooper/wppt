\chapter{Bluffs}

%CHAPTERLENGTH: 4
%proofread 20250420

People who haven't played a lot of poker think that the game
is all about bluffs and tells. While they're an important
part of the game, they're not the be-all and end-all.

I've covered a lot of position bluffs that you should be making
regularly throughout a fast tournament. One of them was
blind stealing, another important one is calling in position and
taking the pot with a pre-flop call in position and a bet on the flop
or the turn. Make sure you understand these moves thorougly. This
chapter covers a few other bluffs that you can make yourself or look
out for from the other players.

\section{Bluffs to make}
% FIXME: Change the wording to be Single-Card Double-Card etc.

\subsection{Scary Board Bluff}

A \textbf{Scary Board} is one where there's four-to-a-flush or
four-to-a-straight. A board showing two pairs is also scary, that
board is four-to-a-fullhouse.  On scary boards it's obvious that the
right one-card hand makes a powerful hand. Normally this is a
cooler situation and betting slows down with check-check, but it's
also an opportunity to make a daring big bluff called the scary board
bluff.

Scary board bluffs can get two pair and even sets to fold, if they're
big enough. Before making your scary board bluff, rewind the hand and
see if you having the single scary card makes sense. If you've called
a big bet preflop it's unlikely you've got a magic 3 on a scary board
of A4T52. But on a scary board of \Ad\nined\sixs\fourd\tred\ you could
easily have the \Kd\ or \Qd\ in your hand.

\subsection{Blind Re-Steal}

You won't be the only player trying to steal the blinds, especially
as the blinds get big and on final table play. Other players will
be trying to steal your Blinds.

When you're in the Blinds, you get to act in late position for
the first round of betting only. If you consider a bettor is
trying to steal your blinds, you can reraise and try to
get him to fold.

This bet will be pretty big and if you get it wrong you can
lose a ton of chips if not your Tournament life. But if you get
it right you'll win a lot of chips and quieten down other
players at the table who can see you're no pushover.

This move is High Risk/High Reward.

When considering a re-steal, consider two factors of the player who
raised preflop. What do you know about him, and what position did he
raise from? If the raise came from early position, it's probably not a
Blind Steal, unless you've got a read on this player as being pretty
wild. A bet from the Button or Cut-off after a bunch of folds is more
likely to be a Steal and your Re-Steal is more likely to succeed. Wild
players get dealt good starting cards, and they get dealt good
starting cards on the Button and the Cut-off. If your Re-steal gets
snapped off and you lose to a powerful hand, be proud, at least you
went down fighting.

\subsection{Flop Check-Raise from Nowhere}

This is one of my favourite bluffs, but don't overuse it or you'll be
giving away a lot of chips and you won't be able to make any more
bluffs for a long time. You're under the gun (with something flop
worthy like T9s) and you limp-call a raise from the button. It's
a heads-up pot, the flop comes Q75 rainbow, you check the button bets
and you raise him.

Check-raises on the flop typically come from sets, good two-pair hands
or two-way hands like pair/flush-draw that are happy to play an
all-in pot. You're gambling that the in-position flop bettor doesn't
have a real hand here, he's just following through his preflop raise
with a Cbet.

On a Q75 flop, a lot of preflop raising hands will get out of the way
to your check-raise. Unless the bettor has a board beating hand like
AQ, KK or AA he'll probably get out of your way; even AK will probably
wait for a better flop.

If your check-raise is called it's time to slow down unless the board
runs out scary and you can consider the scary board bluff. If the
enemy folds to your check-raise from nowhere, \textbf{never show your
bluff}.

This bluff is similar to the previous bluff, blind re-steal. In both
bluffs you're out of position and putting in a re-raise. If you like
you can think of flop check-raise from nowhere as a blind re-steal
move, just made on the flopstreet, not pre-flop.

\subsection{Yes I hit my flush}

I sometimes take one off on a flop with a straight draw, but there's
also a flush possible. If my straight hits, especially in the
non-flush suit, bingo and I should win a fantastic pot. But if the
flush comes in, I can represent that I was on the flush draw and I'm
winning the pot with the flush.

Flushes look scarier than straights and it seems more natural for a
caller to be chasing the flush rather than the straight. If the leader
is likely to have a good one-pair hand he won't himself have the flush
and I might be able to take this pot with the ``Yes I hit my flush''
bluff.

\textbf{At the tables} I've got \nineh\eigc\ and I'm heads-up to the
flop of \Ad\tend\sevc\ . The turn is \tred\ , bringing the flush. I
check and the leader checks behind. The river is a bricky \Qc\, and on
the final board of \Ad\tend\sevc\tred\Qc\ I put in a good sized
bluff and the enemy folds.

Being first to act helped here because my turnstreet check could have
been a check looking for a check-raise with my now-made flush. Since
the enemy checked behind, my river bet looks like a made flush now
trying to get whatever value it can on the river.


\section{Bluffs to call}
\subsection{Entitlement Bluff}

Sometimes there's a wet flop but the turn and river give no help
to the obvious draws. The Finalboard \Jd\tend\fivec\twoh\sixs\ started
pretty wet but the runout didn't complete the original straights or the
diamond flush. A flopped Made Hand would expect to still be the winner
here.

A player who flopped a good draw feels entitled to winning
this pot, and will call a bet on the flop, a bet on the turn
and try bluffing on the river when his draw misses. I call
this bluff an entitlement bluff.

The key to understanding entitlement bluffs is that if the Enemy had a
better made hand than you, he would have raised on either the flop or
the turn, because he doesn't know that you're not on the draw, and he
needs to protect his strong made hand against getting outdrawn. If the
Enemy has just been calling on a wet flop and wet turnboard, the most
likely reason is he was drawing himself.

An entitlement bluff is usually easy to sense and call down; it
just smells wrong. It doesn't make sense from the three street action
``he's behind on the flop, he's behind on the turn, the river
didn't change anything and now he's betting?? That's got to be
a bluff - I call!''.

I've made some terrible entitlement bluffs and easily called
other people's entitlement bluffs. When I'm the leader on
an unchanged board and I'm first to act, I'll often check the
river to induce an entitlement bluff. I can't get three streets
of value by betting it myself, my best play is to check/call here.

If I'm last to act as the leader, I'll call an entitlement bluff
if he's made one, and value bet or check behind if he's checked.
I'll value bet if I have a good two pair hand or a set; if I've
got just a single pair (even top/top) I'll often check behind
because drawing hands can sometimes sneak two pair (in this example,
\sixd\fived\ misses the flush but spikes a call-worthy two pair
on the river). A value bet in this circumstance risks being one
of those bets that only gets called when I'm losing.

\subsection{Only way to win bluff}

Sometimes your poker sense tells you that you're ahead, but the enemy
doesn't have much, not with those bricks on the turn and river that
didn't bring in any flushes or straights and didn't bring any
overcards to your top pair.

You're first-to-act but if you put in a decent bet you'll just win the
pot as it is, the enemy will fold. But if you check, you give the
enemy the chance to bluff on the river; it's his only way to win this
pot, there's no way he's winning a showdown.

This move is called ``check to induce'', you're checking the
finalstreet to give the enemy a chance to bluff at it. It can win you
quite a few more chips; chips that you would't have won with a
standard value bet yourself.

Unless your hand is so strong that you're nut or gobroke, you just
check/call on the finalstreet. If you check/raise, your raise is only
getting called (or even reraised) by a surprising but not impossible
better hand.

\section{Theoretical Bluffing Frequency}

In poker you don't want to be scared of having your bluffs called. It
feels a bit embarassing and some of the players might have a laugh at
you, but if you're never caught bluffing then you're not bluffing
enough. If you only bluff 1 in 20 hands, and your bluffs always work,
you could probably bluff 1 in 10 hands and still have all your bluffs
work. While you feel great when your bluff gets through, you miss
other chances to pick up pots.

Some players are scared of folding on the riverstreet and being shown
a bluff by the enemy. That's even more humiliating than being caught
bluffing yourself. Poker's a tough game. Do your best to take your ego
out of it and make the best finalstreet decisions you can.

There's a completely foolproof way of never being bluffed. Call every
bet on the river. Be the ultimate calling station. You'll find
yourself out of the tournament very quickly as you get value bet to
death. But you can never be bluffed out of a pot.

On the other hand, if your bluffs are rarely working, you're bluffing
too often or in bad positions or you could even be giving off tells
that the other players are picking up on before they call you.

You \textbf{should} be caught bluffing now and then, because if your
bluffs always work you might not be bluffing often enough. You
\textbf{should} be paying off value bets sometimes, because if you
don't you've probably folded to some bets that were in fact bluffs.
Accept the fact that sometimes your bluffs won't work, sometimes
you'll be paying off winners and sometimes you'll be getting bluffed.

When your bluffs are always failing, your calls are never winning,
and your nut-hand bets are never getting called
you know you're in a really tough game.
