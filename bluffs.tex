\chapter{Bluffs}

% Last updated: 20181109

People who haven't played a lot of poker think that the game
is all about Bluffs and Tells. While they're an important
part of the game, they're not the be-all and end-all. 

I've covered a lot of Position bluffs that you should be making
regularly throughout a Fast Tournament. This chapter covers
a few other bluffs that you can make yourself or look out for
from other players.

\section{Entitlement Bluff}

Sometimes there's a wet flop but the Turn and River give no help
to the obvious draws. JdTd5c2h6h didn't complete any straights
or the Diamond flush.

% FIXME: LaTeX cards for the board above

A player who flopped a good draw feels entitled to winning
this pot, and will call a bet on the flop, a bet on the turn
and try bluffing on the river when his draw misses. I call
this bluff an Entitlement Bluff.

On this board, JdTd5c2h6h, whoever was the leader on the flop
will still be in the lead on the river. The flop was wet enough
that if the caller had a better made hand, he should have raised
earlier in the hand, as he himself would be afraid that the bettor
was betting a drawing hand and might draw out.

An Entitlement Bluff is usually easy to spot and call down; it
just smells wrong. It doesn't make sense from the three street action
"he's behind on the flop, he's behind on the turn, the river
didn't change anything and now he's betting?? That's got to be
a bluff- I call!".

I've made some pretty bad Entitlement Bluffs and easily called
other people's Entitlement Bluffs. When I'm the leader on
an unchanged Board and I'm first to act, I'll often check the
river to induce an Entitlement Bluff. I can't get "Three Streets
of Value" by betting it myself, my best play is to check/call here.

If I'm last to act as the leader, I'll call an Entitlement Bluff
if he's made one, and Value Bet or Check Behind if he's checked.
I'll value bet if I have a good two pair hand or a set; if I've
got just a single pair (even top/top) I'll often check behind
because drawing hands can sometimes sneak two pair (in this example,
5d6d misses the flush but spikes a winning and callable two pair
on the river). A value bet in this circumstance risks being one
of those bets that only gets called when I'm losing.

\section{Scary Board Bluff}

A Scary Board is one where there's four-to-a-flush or four-to-a-straight.
A board showing Two Pairs is also Scary, on that board there's
four-to-a-fullhouse.  On these boards it's easy for a one-card hand
to make a powerful hand. Normally this is a cooler situation and
betting slows down with check-check, but it's also an opportunity
to make a daring big bluff called the Scary Board Bluff.
