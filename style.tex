\chapter{Style}

% Last updated: 20190611

Before we go any further I need to say a few things about
some style choices used in this book.

\section*{Eight Player Tables}

Unless specified otherwise, hands in this book assume an
8-player table. A lot of Fast Pub Poker Tournaments use
round eight-player tables with foldup legs and a drink holder
in the table for each seat. These tables are easy for a single TD
to fold into a large disc and roll away as the tournament
field shrinks and tables are combined. A final table of Nine players
is assumed.

\section*{Suited and Offsuit holdings}

When an offsuit holding is shown, Hearts/Clubs are used, to make
it easier to see on paper that this is an offsuit holding. A pocket
pair is an offsuit holding. Obviously six offsuit combinations are
possible (Hearts/Clubs, Hearts/Diamonds, Hearts/Spades, Spades/Clubs,
Spades/Diamonds, Diamonds/Clubs). This book uses a red suit/black suit
pairing so it's easier on your eyes and your brain will quickly pick
up that your holding is offsuit. Same-suited
holdings are shown in the suit of Diamonds.

I shoved with \Ah\tenc\ and had no chance against \Ad\Kd\ and \Jh\Jc\ .

\section*{Stack sizes}

Stack sizes are often expressed as a multiple of the Big Bet/Big Blind
(BB), rounded intelligently. With a stack of 26,500 with blinds at
1,000/2,000 then you've got 13BB. A 39,200 stack at the same blind
level is 40BB. A lowercase k is used as an abbreviation for thousand,
the two stacks shown are 26.5k and 39k stacks with blinds at 1k/2k.

\section*{High to low flop}

The flop is always shown from high card down to low card, left-to-right,
regardless of the true order. You should think of the flop as a
simultaneous three card reveal, so really there is no true order. The turn
and river are always shown to the right of the flop.

When dealing, I announce the flop, high to low, and say ``two spades''
if there's two of that suit, ``all diamonds'' if there's three of that
suit, and ``Rainbow'' if the suits are all different.

% FIXME: add two flops to show this

\section*{Example hands}

Most of the example hands are from my own play, or hands I've
witnessed myself. A few of the examples are from the WSOP Main Event
of 2005, which was a memorable poker tournament for many reasons.

\begin{itemize}
  \item It was won by Australia's Joe Hachem. This created a lot of
    media interest in Australia which led to the Pub Poker boom.
  \item The ESPN coverage of the event, with commentary from
    Lon McEachern and Norman Chad, is both extensive and
    excellent. Excerpts from this coverage can be easily found on
    YouTube. I own about five hours of 2005 Main Event coverage on
    DVD, which I watched in my early poker years, and I remember a lot
    of the featured hands very well.
  \item Dan Harrington uses four examples from this tournament in his
    book Harrington on Holdem Volume 3: The Workbook. If the hands are
    good enough for Dan, they're good enough for me.
  \item The start stack of the tournament was 10,000 in
    chips. Although the blind levels were very slow (two hours)
    players were short in chips early on in the tournament.
\end{itemize}

I've found I learn more from watching WSOP hands than I do
from World Poker Tour hands. I have WPT Season 1 on DVD and don't feel
a strong urge to re-watch any. World Poker Tour normally films just the
final table so you're already watching shove and call final table
poker. You can learn a lot of Final Table play from WPT, but for early
stage and middle stage play, WSOP hands will teach you more.
