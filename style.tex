\chapter{Style}

% Last updated: 20200426

Some words on the style choices in this book.

\section*{Eight Player Tables}

Hands in this book assume an Eight-player table. A lot of Fast Pub
Poker Tournaments use round Eight-player tables with foldup legs and a
drink holder in the table for each seat. These tables are easy for a
single TD to fold into a large disc and roll away as the tournament
field shrinks and tables are combined. A final table of Nine players
is assumed.

\section*{Suited and Offsuit holdings}

When offsuit hole cards is shown, I use Hearts/Clubs, to make
it easy to see on paper that this is an offsuit holding. If the
offsuit holding is a Pocket Pair I use Spades/Clubs, which will
attract your attention and help you see that a pocket pair is in play.
Same-suit holdings are shown in the suit of Diamonds.

I shoved with \Ah\tenc\ but had no chance against \Ad\Kd\ and \Jc\Js\ .

%% \section*{Stack sizes}

%% Stack sizes are often expressed as a multiple of the Big Bet/Big Blind
%% (BB), rounded intelligently. With a stack of 26,500 with blinds at
%% 1,000/2,000 then you've got 13BB. A 39,200 stack at the same blind
%% level is 40BB. A lowercase k is used as an abbreviation for thousand,
%% the two stacks shown are 26.5k and 39k stacks with blinds at 1k/2k.

\section*{High to low flop}

The flop is always shown from high card down to low card,
regardless of the card order. The flop is a simultaneous three card
reveal, so the order is unimportant. The turn is shown to the right of
the flop and the river is shown to the right of the turn.

The Board for Joe Hachem's 2005 WSOP Final Hand was
\sixh\fived\fourd\As\fourc\ .

\section*{Example hands}

Most of the example hands are from my own play, or hands I've
witnessed myself. A few of them are from the WSOP Main Event
of 2005, which was a memorable poker tournament for many reasons.

\begin{itemize}
  \item It was won by Australia's Joe Hachem. This created a lot of
    media interest in Australia which led to the Pub Poker boom.
  \item The ESPN coverage of the event, with commentary from
    Lon McEachern and Norman Chad, is extensive and
    excellent. Excerpts from this coverage can be easily found on
    YouTube. I own about five hours of 2005 Main Event coverage on
    DVD, which I watched in my early poker years, and I remember a lot
    of the featured hands very well.
  \item Dan Harrington uses four examples from this tournament in his
    book Harrington on Holdem Volume 3: The Workbook. If the hands are
    good enough for Dan, they're good enough for me.
  \item The start stack of the tournament was 10,000 in
    chips. Although blind levels were very long (two hours)
    players quickly became short on chips once they lost some pots.
  \item There were a lot of inexperienced players in the tournament,
    chasing their poker dream after the previous year's win of
    Internet qualifier Chris Moneymaker.
\end{itemize}

%% I've found I learn more from watching WSOP hands than I do
%% from World Poker Tour hands. I have WPT Season 1 on DVD and don't feel
%% a strong urge to re-watch it. World Poker Tour normally films just the
%% final table so you're already watching shove and call final table
%% poker. You can learn a lot of Final Table play from WPT, but for early
%% stage and middle stage play, watching WSOP hands will teach you more.

\section*{Exercises}

% FINALFIXME: Check the number of exercises is 20

There are 20 Exercises in this book, scattered at the end of some of
the chapters. Have a go at them using a pen and notepaper. There's no
pass or fail, they just test your understanding of the material. All
of the answers are provided at the back. Some of the exercises need
you to search on Google or YouTube; a few need Maths and one needs a
calculator. The Exercises are numbered by their chapter.

\arabic{chapter}.1 (Google search) Which poker player won the World
Series Main Event in the 1980s, recovering from having just one chip
left, which led to the phrase that all you need to win a Poker
Tournament is ``a Chip and a Chair''?
