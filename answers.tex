\chapter{Answers to Exercises}

The exercises are at the end of the chapters concerned.

% FINALFIXME: Check that the answer numbers match up with the exercise
% numbers.. they might jump around depending how I re-order the
% chapters. If I was a total LaTeX genius I could probably automate
% this but seems a lot more trouble than it's worth.

3.1 Jack ``Treetop'' Strauss won the 1982 WSOP Main Event,
having at one stage just a chip and a chair. He'd
pushed all his chips in and lost to a player who had him
covered. But he hadn't announced ``all-in'' and found when standing up
to go that he had an extra 500 chip under a napkin. He was allowed to
keep playing and went on to win the Tournament.

A TD In a small tournament once gave me a free re-entry for knowing
this piece of poker history.

8.1 (a) EV = P(win) x (amount won) - P(loss) x (amount lost) \\
= (18/38) x 10 - (20/38) x 10 \\
= (180/38 - 200/38) \\
= (-20/38) \\
= -0.53 = -53 cents

(b) This is the same house edge as pick the exact number, the bets are
equally fair. It seems like a fairer bet because you win it more
often, but the house cut of 5.3\% is exactly the same.

8.2 EV = P(win) x (amount won) - P(loss) x (amount lost) \\
-0.5  = (1/4)  x (amount won) - (3/4) x \$10 \\
(x4 both sides) -2 = (amount won) - 30 \\
(+30 both sides) 28 = (amount won) \\

So a \$10 Exact Number Keno bet with an EV of -5\% should return
\$38.

8.3 P(loss) = (20/38) = 0.52 \\
P(five consecutive losses)  = 0.52 x 0.52 x 0.52 x 0.52 x 0.52 \\
= 0.038 = 0.04 \\

The gambler has a 4\% chance (1 in 25) of losing five bets in a row
and going bust. 24 in 25 times he wins his \$10, but when he loses it
costs him \$310 (\$10+\$20+\$40+\$80+\$160).

9.1 \Ad\Jd\ beats \Ah\Kh\ on a flop of \Ac\Kc\fourd\ with the
following runouts: (i) runner-runner flush (except \Kd\ which
make a full house for \Ah\Kh\ ) (ii) runner-runner straight (QT or TQ)
(iii) runner-runner JJ for trip Jacks.

A split pot occurs with runner-runner fours, on the final board of
AK444 we both make Fours full of Aces.

9.2 You have to hit one of six cards on the turn and then have that
same rank repeat on the river. 6 outs on the turn x 2 outs on the
river = 0.12 x 0.04 = 0.0048 = less than 1 percent. 1 in 200.

% FIXME: Add an answer for Question 9.3:  How often does a flopped set
% become a Full House?

9.3 Assume a top set 99 on a 987 flop. There are 7 outs to improve on
the turn (the final 9, one of three 888 or one of three 777). By the
rule of 2 that's 14\% on the next card.

The turn doesn't pair the board, it's a 4 say. Now going to the river,
with the turnboard 9874, we have the same 7 outs we had on the flop,
plus the 4 can pair, we have three new 444 outs. That's 10 outs going
to the river which by the Rule of 20 is 20\%.

If we hit on the turn, we can hit again (not that we need to) on the
river with one of 6 outs.

Hit on turn = 7 outs = 7/46 = 0.15 = 15\% \
Miss on turn but hit on the river = 85\% x 10/45 = 0.172 = 17\%

You improve to a full house 31\% of the time.

Miss on turn/Miss on river = 40 ins x 36 ins = 0.80 x 0.72 = 0.576

38.1 On the flop of 632, Joe McKeehen had 66 and Schwartz had 33; top
set over middle set. Nobody at the table was sad to see Schwartz go.
