\chapter{Turn strategy}

% Last updated: 20200427

The turn is probably the easiest betting street to play in Holdem.
If you play preflop and the flop properly, you should have no
problem with the turn.

If you're on a good draw, you'll want to stick around to
the river if you've missed it on the turn. A big bet from your opponent
might get you off the hand (always check the pot odds), but if the price
is right you should call and hope you'll hit on the river.

At this point in the hand there's only one more board card to come,
so if you're on a draw you're now
using Rule of Two outs, not Rule of Four. Your chances of making
a Flush or Straight are now about 1 in 5, so you're normally going
against the odds when you call a normal sized turn bet. Have a look
at the Implied odds; you should be able to win more from your opponent
when you hit, so as long as the turn bet you call is about one-third pot,
you're normally correct to call.

% FIXME: add a section called second barrell? about whether to keep
% betting if you were the flop bettor and got called.

\section{Pot Commitment}

One important concept to understand in tournament poker is pot
commitment. If you've put 30\% or more of your chips into any pot,
you'll probably want to see this hand through to showdown. For
example you're down to 1,000 chips early on with 25/50 blinds and you've
called a raise preflop to 300. Would you fold your last 700 to
an all-in bet on the flop, when you'd be calling that 700 into a
pot of 1,300?

You don't want to call too much of your stack preflop and
pot commit yourself preflop on speculative hands. They just don't
hit very often.

Keep that 30\% figure in your head. It works both
for you and your opponents. If your opponent is in for 30\% of his
chips already, most of the time you won't be able to bluff him off
this pot. He's taking it all the way to the showdown.

If you're pot committed anyhow, and you expect to be calling a turn
bet no matter what, you can semi-bluff on the turn and go all-in first.
Most of the time in poker it helps to have last position but occasionally
it's good to have first-mover advantage and be the first to go all-in.
This may sometimes get a player
off the hand, and even if he calls you've still got your outs.

If you hit a hand that's vulnerable to a lot of river cards, bet the
turn heavily. An example is a small flush that could easily lose
to a four-flush. If you have \tend\nined\ to a board of \Ac\eigd\tred\fived\
and someone's betting at you, they could have \Ad\Js\ and
sneak a better flush on the river. Now your might get called and
the river brings the fourth diamond and you lose. That's poker. Poker
is not a safe game. It's very rare that someone calls you and is
drawing dead. Betting and losing to a bad river is so much better then
checking, giving the river for free, and losing and paying off on the
river or folding to a scary river card and being shown a bluff.

\section{Made hands make their money on the turn}

If you hold a made hand on the turn (such as top/top,
two pairs or trips) and you're up against a drawing hand such as a flush
draw or straight draw, the turn is the betting street where you make your
money. Get the drawing hand to call a bet on the turn, because he won't
call a big bet on the river if his draw misses.\footnote{He might
bluff though, so consider check-to-induce on the river. But still bet
the turn. If you bet the turn, then check the river, someone who
missed their draw might try a daring Entitlement Bluff on the river,
winning you more chips where a third Value Bet would simply be folded
to.}

Many players are too passive on the turn, and give free cards when
they're ahead which allows their foes
to draw out against them. Charge the drawing players to draw out on you,
because if their draws miss you can't get many more chips out of them
now that all Board cards are out.

You want to bet enough so that the player on a draw is making a big
mistake by calling, but not so much that the drawing player folds.
Yes you win the pot and you never get outdrawn if the drawing player
folds, but you also win a smaller pot with your boss hand. That's
bad from a risk/reward standpoint.

\section{Defending the Cooler Card}

A Cooler Card is a board card that cools down betting action by making
a big Single-Card hand possible. The board now shows
four cards of a suit or four to a straight.
Two pair hands and even sets don't look like winners
anymore and can't be bet strongly on this Board.

If the board is \tenh\nineh\sixh\trec\fourh\, the highest single heart wins.
If the board is \Jc\tens\nineh\tred\eigc\, a single Queen wins,
and a Seven will otherwise win.

Another cooler is a second board pair, which now makes a Single-Card
Full House possible. On \Qc\fived\fiveh\fourh\Qh, a single Queen makes a
fullhouse and so does a single Five.

When a cooler arrives it's hard for you to bet big unless you've got
the obvious Nut hand yourself. If your opponent doesn't have the big
hand himself, the betting will often go check/check on the river
and your hand still wins.

Betting your hand on the turn, getting more money into the pot before
the cooler comes, is a great way to lock in some profit before the
cooler comes and the betting dries up.

\section{Hitting the one-time two pair}

If I've made a sneaky one-time two-pair on the turn I'll be trying to
take this pot down now. Two pair is a pretty vulnerable hand, and
often loses to better two pair or standard top pair with a counterfeit.
Say I've got 98 on a board of K968.  I've spiked two pair on the turn,
having called a flop bet from a probable top pair/good kicker hand.
Indeed the flop bettor has KJ. The following rivers improve him to the
winner: \\
K (2 outs) KKKJ9 beats KK99J \\
J (3 outs) KKJJ9 beats 9988K \\
6 (3 outs) KK669 beats 9988K \\

That's 8 river outs--- as good as an open ended straight draw. I'm a 1 in 6
chance to lose this one, so I should be betting it while I'm ahead.

You might not want to play the one-time two pair move much at all.
It doesn't come off all that often and you can steadily lose decent
amounts of chips pot after pot chasing it. I think it's
a great weapon to have in your arsenal though. Perhaps you
can decide to try it only when you have a dominant chip stack at
the table, and you can take a few extra chances in return for the
possibility of busting your opponents. Maybe you'll use it on average
one in three times, when you figure you have a good read on
the other player in the pot and can get him to pay you off heavily
if you hit. Maybe you'll use it when you're winning plenty of pots
and you'd like to keep the good times going.

Sometimes in a multiway pot, the betting lead changes on the turn.
This is usually a reliable sign that someone's draw just hit. Unless
you still have a good draw to a nut hand, get out of this hand now.
The new bettor expects callers (he's in a multiway pot and against
other players who've been betting and calling) and he'd like to
get your chips too.

The turn is your last street for check-raising. If you flopped a set
and you're being bet at from position, consider flat calling the
flop bet and check-raising the turn. Hopefully your opponent
is a two-bullet bettor. By doing your check-raise on the turn
you're winning a much bigger pot and your opponent might be
forced to call, if he's pot committed.

Slowplay your sets a lot more on uncoordinated flops. 33 on K93 is good,
but 66 on 986 shouldn't be giving a free turn--- you're just inviting
straights and flushes to hit.

\section{Bet your full house on the turn}

If you're lucky enough to have a full house by the turn, bet it.
Probably about one-third to one-half the pot. If there's straight
and flush chances out there, you'll get calls from players on
these draws while they're drawing dead. Free chips!! This is great for when
they miss, because if you'd checked the turn then you're not charging them and
they certainly won't call bets on the river if their draw busts.
And they miss their 8-9 out draw 80\% of the time, so get some money
from them now, because this could be your last chance.

But it's even better when they hit! They'll beat you to the pot, because
they've just made the very hand they were drawing to! They'll guess
you had trips on the turn and you were betting to get people off draws.
I find this play very exciting. Phil Gordon's
Little Green Book paid for itself just by teaching me this betting pattern.

If the board is super dry (you hold K4 on a board
of KK48) just check and hope something good happens
to your opponent on the river. Maybe they hold J8 and an 8 comes
on the river and they'll lose their head. Superdry Boards
are pretty rare in Holdem, there's usually a flush draw or straight draw
for someone to draw to after the turn.

Here's a hand I busted a player with, which involved betting my full
house on the turn. Only the small blind (sb) and big blind took the flop,
I was the big blind and acted in position on every betting
street.

Deepstack, sb: 8,500 bb (me): 22,000 blinds 300/600.

sb completes to 600, I have \Qc\Jh\ and check.
Pot 1,200. sb: 7,900. me: 21,400.

Flop: \Qs\Qh\Jd

Oh yeah. Flopped full house and there's straight chances there. I hope
the small blind improves to second best. My hand is so massive I'll give
a free card. Check/check.

Pot 1,200. sb: 7,900. me: 21,400.

Turn: \fours\ sb check, I bet 600. Time to start building the pot. To
my great relief, the small blind calls. Pot: 2,400

Pot: 2,400. sb: 7,300. me: 20,800.

The turn doesn't complete any straights but at least there's
two spades now and he could have an outside flush draw. Maybe he's
got the Jack or perhaps the 4 paired him. I'm not really thinking
too hard about what he has at this stage because I'm still rubbing my
hands with glee at the QQJ flop!

River: \fourc. sb bets 1,200, I Hollywood and raise 5,000, sb announces all-in
and I beat him to the pot.

Small blind shows Q7 for full house QQQ44. I show QJ for the better
full house QQQJJ.

Pot: 17,000. sb: 0, busted. me: 30,500.

This is a horror hand for the small blind the whole way and he's probably
going broke no matter how we bet this. He might have been looking for
a check raise on the flop but I was slowplaying my QQQJJ flopped full house.
Nice flat call on the turn, he thinks I'm trying to steal it and he'll take
the chance against me rivering a straight or something. On the river his
1,200 bet is screaming for a call, instead I raise. This hand, which until
the river has been bet like a 300/600 Limit poker hand, goes ballistic.

By the end he holds Q7 to a board of QQJ44--- the only hands that beat
him are QJ and 44. All indications are he's winning or getting a split
against Qx. If he'd suspected I had Q-better kicker he might have
raised my turn bet but I would have just flat-called and still busted
him on that river.

The lesson in this hand is that even Heads Up when you have a lock on
the board your opponent can still find a way to improve to a gobroke
hand. He's got the nightmare hand of the Dreammare.
If you've got the dream hand, don't be afraid to bet it big. You're
hoping you're up against the nightmare hand, which will pay you off
the maximum.

\subsection*{A rare circumstance}

Betting full houses on the turn this way can cost you a massive pot in
the following circumstance. You bet holding K9 on a board
of KK98, looking for action from straight draws. Someone
holding 66 easily folds, someone with JT calls and the river comes
a 6. If you'd slow played it you could have won all the chips from
the player holding 66 who's just improved to a go-broke losing full
house.

This circumstance is extremely rare. The player with 66 is only 4\% to
improve on the river, which is 1 in 25. It doesn't happen often enough
to make checking the turn a worthwhile play. Simply bet
the turn and make whatever money you can from any drawing hands
out there.

\section{The Blocking Bet}

If I'm on a draw and first to act, I sometimes put in a smallish
turn bet, to try to avoid having to call a bigger bet which
I'd expect to happen if I check. For example I've had to call a bet
of 300 on the flop from the Button, who I think has Top Pair
Good Kicker. The pot is now 1,000 and if I check the turn I expect to
face a bet of 600-800. Instead of checking I bet 450, and hope the Button
will just call the 450 and not raise to 900 or more.

This bet is called a Blocking Bet and is an attempt to set the betting
amount for this hand. Sometimes I spook the Button and make him think I've
suddenly hit my hand. I called his flop bet and now I'm leading out
on the turn. Maybe I've got a set or two pairs and I'm starting to
milk him. If the Button was Cbluffing
the flop, he'll often fold to my Blocking Bet. Just because I put him
on Top Pair Good Kicker doesn't mean that's what he actually has!

I'll also sometimes make the same small bet from out of position
when I have a monster hand. So a smallish bet on the turn from me
isn't always a Blocking bet. The late player will almost always call
my small ``Blocking'' bet, because folding looks so weak, and an
advanced player might even bluff raise me if he thinks I'm making
a blocking bet. If he does that when I've got a monster, he's giving
me lots of chips. If I don't have a good hand, he can take the pot.
I don't need to see the river anyway, for that price, I'm long odds to
make any kind of hand.

If a player does fold to my small blocking value bet, it's unlikely
I could have made much on this hand anyhow, even if I'd checked
he wouldn't have ended up with a strong hand no matter what
the river card was.

You can also use a Blocking Bet on the river, as part of a
bet/fold pattern (if your bet is raised, you're highly
confident you're losing, few players are good enough in
Pub Poker to raise on the river as a big bluff). It's not
easy to fold to a raise after you were the bettor, but it could
be cheaper than a check/call, in that you're setting the betting
price for this street.

Some books and commentators call a bet from first-to-act a \textbf{donk.}

% FIXME does the next paragraph belong in this section?
% In a fast tournament players hit chip desperation earlier, and there's
% not the need for trapping that there is in deepstack games. Once I had
% JJ on the Button on the first hand of a sit n go, 2,000 start stack.
% I raised it to 500 and got two callers. The flop came Qh Qs Jh. There's
% the low full house for me. The big blind shoved all in, and there was
% a call before it got around to me. Big blind showed Ah5h for the heart
% draw (drawing dead) and the caller had 77 black, just in case the big blind
% was shoving on a draw I guess. That's a triple up on the first hand
% for me and I didn't even get to bet the flop, just call two other
% player's all-ins!

