\chapter{Tells}

% Last updated: 20200425

A lot of tells are Flop Reaction Tells. I cover these in
the chapter ``Flop Strategy''. Review the section there first, then
read on for some more tells.

\section{Unusual actions}

Always be suspicious when someone does something unusual.
Once I was in a hand with a player I nicknamed Sheiky, as
he looks a bit like WSOP and High Stakes Poker star Sean Sheikhan.
There was a flop of \Kc\nined\sevd\, and Sheiky started counting on his
fingers, and looked as though he was counting odds and the number
of outs he had to make a hand. Now \Kc\nined\sevd\ is not an especially
draw heavy flop and not a hard flop to count odds from. Probably
\Qd\tens\ would be the hardest hand to count odds for on this flop but
it's not that hard. Why was Sheiky counting on his fingers on this
flop?

To trick me into thinking he was on a draw. He was strong in
this hand. Unfortunately I fell for his act, and doubled him up.
He had \ninec\nineh\, and had flopped a set, and was a mile in front (I turned
a straight draw myself but didn't get lucky on the river). This
was very bad play on my part - I saw him do something unusual, which
he'd never done before, and I didn't see through his obvious
``I must be on a draw'' act.

\subsection{Theory of tells}

An unusual, out of character act can often indicate
strength. Someone who starts moving or talking is up to something.
Probably he's strong and he's trying to get your chips when he's
a huge favourite. When someone's doing something different, beware.

If a guy who looks like Sean Sheikhan is at your table and starts
counting on his fingers after the flop, fold any non-nut hand. He's
got a strong made hand; top-two pair or a set.

\section{Time Tells}

If a player takes a lot of time and seems to be agonising over a call
but then raises or pushes all in, he's usually got the goods.
This is a ``weak means strong'' tell.
Unless you're very strong yourself, you should get out
of the way. Try not to get too upset at the staller for
his time-wasting antics. At least his act has given you the
information you needed to make the right decision.

I once raised preflop in a big tournament with the very flop-worthy hand
AJ. The small blind agonised over his decision for at least ninety
seconds then shoved all in. I pretended to agonise over my decision
for a polite fifteen seconds, and folded. The small blind didn't show
his hand, but it was QQ, KK, AA or AK; all of which have me crushed.
Two rounds later the same
guy complained that he was card dead and wasn't getting any hands
in this level. Well, apart from that monster he got in the small blind
which he pretended was a tough decision for ninety seconds.

If a player shoves chips in quickly or aggressively it may be a bluff,
a ``strong means weak'' tell.

\section{What does the player reach for first?}

Sometimes I bet the flop on a continuation bluff, and I'd really like
to win the pot now. After I make the flop bet, I like to see what
my opponent first reaches for. If he never touches his cards, and his
hand moves first to his chips to make his call, he probably likes his
hand and won't be folding to a turn bet either. I should
give up on this pot without further chip damage.

A quick, automatic call of a regular bet is often a hand on a
draw. A player on a draw thinks mostly of calling, a little about raising
and not much about folding, so it's a quick call. But a slowish call
where the hands never touch the cards and only ever move to the chips
is often a strong hand that isn't thinking about folding. Your enemy's
thought process is ``call or raise'' here, which puts him on a strong
hand.

\section{How are the cards handled?}

A player who holds his cards on their sides and taps them on
the table is likely to fold them. He's taking one last look
at his pocket hand, which he now considers no good and is about
to fold. His hand is so poor that he can relax now, he no longer
needs to protect his hand against accidental exposure.
On the other hand, if a player is still sneaking looks at his
well protected hand, he likes it and he's still playing the hand.

A lot of the Mike Caro tells are about how a player
throws in his chips. I find the way a player
handles his cards is also revealing. Is he still protecting
his cards, or is he holding them up and bouncing them around?

Just as three elements of poker are cards, chips and position,
three elements of tells are cards, chips and alertness position
of the player.

\section{The Speech}

If a player starts a decent speech and pretends he's scared of various
hands, or has to leave early or wants to be home soon, look out.
Nobody talks like this at the tables unless he's trying to trick
you into a losing call. If he really wanted to go home, he'd play
badly and he'd be home soon enough. A long speech counts as an unusual
action, most of the time people play poker very quietly indeed. As I
said in the earlier section, beware of unusual actions.

\section{The Poker Face}

As well as looking and acting on other people's tells and telegraphs,
you want to avoid giving tells yourself. This is where it helps
to have a ``poker face''. Good acting in poker isn't about doing
a silly ``strong when weak'' tell such as saying ``Ohhh I really
didn't want to see that King'' on a \Kh\Qd\eigc\ flop when holding \Ks\Kc.
Good acting in poker is giving away nothing whatsoever.

The best Poker face I've seen belongs to \textbf{Patrik
Antonius}. Phil Ivey and Tom Dwan also have great poker faces.

\section{The best card reader ever}

One night I was sitting opposite a player at an Eight-seater round
table. If he bet and won a pot without having to showdown, he'd
fold his cards unshown but he threw them into the muck using a lofted toss
that allowed me to quickly see what his hand was.

I only ever got to see his cards when he folded like this. I pretended
to be a card-reading genius who could tell the hand he'd won the pot
with. One time he hit his flush on the turn and took the pot and I
quickly saw his folded cards were QT of spades.

``You hit the flush there'' I said. ``You bet 600 so I'd
say you had the Queen high flush. You might've had Queen Jack of Spades,
hmmmm no Queen Jack doesn't fit...yes Queen Ten of Spades is the hand
you had.''

He looked at me with eyes as wide as dinner plates, and insisted on buying
me a drink.

Since I only ever got to see his cards when he'd already folded, the tell
was pretty useless, but it was fun being thought of as a Daniel
Negreanu-quality card reader.

%% Another time I saw someone raise 600 preflop, take the pot and then
%% he politely showed TT. Two hands later he raised 700 preflop, a bit
%% more than the 600, and again won the blinds. Since the 700 was a bit
%% more than 600, I thought his hand was one better than Tens, and
%% sure enough he showed JJ.

\section{Flush card re-check}

When a three-flush comes on the Board (either on the flop or with
a third suited card on the turn) players who don't have the flush
quickly re-check their hole cards, to see if they have a high card
of that suit and can draw to a four-flush on the next Board card.

If you're playing \Ad\tend\ your brain remembers ``I'm playing Ace-Ten
suited in Diamonds'' but if you're playing \Ad\tens\ your brain remembers
``I'm playing Ace-Ten offsuit'' because on most Boards you're not going to
make a flush. However if the flop comes \Kd\eigd\sevd, you're suddenly
very interested in whether your Ace was the Ace of Diamonds or an Ace of
a different suit. So you re-check your hole cards. Whereas if you were
playing suited cards, you remember that and you also remember which suit
you're hoping to hit.

So players who re-check their cards when the third flush card hits the
board haven't already made a flush.
