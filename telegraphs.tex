\chapter{Telegraphs}

% Last updated: 20200507

A Telegraph is a tell from a player who hasn't acted yet; it's not his
turn but he's seen his cards and he knows what he wants to do in this
betting round. Players must wait for their turn to act, but many will
give away their intention to fold or call if you take a short look at
what they're doing with their cards and chips. Telegraphs are
especially useful in the Preflop betting round.

I've put the Telegraphs chapter before the Tells chapter,
because people are a lot more familiar with tells than
telegraphs. Looking for telegraphs can really help your preflop game,
so get in the habit to taking a quick look downstream before
making your preflop decision.

Here's a simple way to not giving away telegraphs yourself preflop.
Don't look at your hole cards until it's your turn to act. It's that easy.
You don't know what you're going to do this hand, because
you haven't even seen your own cards yet! This frees up your brain
and you can concentrate on watching the action of the earlier players
and the downstream players as you look for their tells and telegraphs.
They might be holding their cards three-quarters of the way to the
muck already so it's not a long toss for them when it is their
turn, alternatively they could be sitting up straight watching
the action very closely, which is a good indication that they're
going to be playing and they really like their starting hand.

If you've looked at your cards early and find \sevh\trec\ offsuit,
you'll lose interest in the action and this hand. If you've looked
early and found \Kc\Ks\ you'll be very interested in what other people
do preflop. This is an ``alertness'' tell. If you haven't looked at
your cards at all, you don't know what you want to do yet and anyone
peeking at you for a tell won't get any information at all.

The only time I look at my cards early is if I need a bathroom break,
in those circumstances I will occasionally fold preflop out of turn
and then go for my break.

Try and take the same amount of time for each preflop decision.
Avoid insta-folding and also avoid the insta-call. If you like,
count to three in your head after seeing your cards before you act.
Don't count any higher or you will be slowing down the game. Three
seconds is enough to give all your actions an appearance of
thoughtfulness and consideration.

Take the trouble to adopt these suggestions. Never look at your hole
cards before you have to, and look for telegraphs from other players
who have looked at their cards early. Your preflop game will
improve markedly. Keep these excellent preflop habits
for the rest of your poker career.

You might worry that by waiting until the last moment to check your
cards, you give your opponents a chance to get tells on you.
You need to be comfortable with people staring at you, it's part of
poker. You want to be practised in not giving away tells. People
staring at you in preflop betting are actually hopeful that you'll do
what they want you to do, and scared that you'll do what you don't
want them to do. A limper hoping to see a flop is hoping you'll fold
or also limp, and scared that you'll raise. A blind stealer is hoping
you'll fold, scared that you'll call and scared that you'll
raise. An early position player with a monster hand like \Kc\Ks\ or
\Ad\Kd\ is hoping you'll raise so he can come over the top, happy with
a call though he'll have more callers in the pot to beat him, and
happy if you fold.

So, most of the time people study you preflop they're simply
hoping that you'll do what they want you to do. They're not picking up
superb poker tells on your actions.

After the flop, the best telegraph to look out for is a player
holding some chips in his hand. If he's doing this, he's trying
to discourage you from a bet. Sometimes you can reach for chips
yourself and see what his hands do; if he reaches for chips also
he's discouraging the bet. I've won a lot of pots by betting
with nothing into a player who shows this telegraph. Keep a
lookout for it.

