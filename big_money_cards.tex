\chapter{Big money Cards}

% Last updated: 20181012

There's two big money cards in some Holdem hands. When these cards come,
on the turn or river, big pots occur and someone wins a pile of chips.

\section{The Full House of Flush}

It's pretty common for a Holdem flop to be non-rainbow. Most of the time
there'll be two of a suit on a flop. Players with four to a flush will
stick around in a hand, hoping to make a flush on the turn or river.

I'll put up a sample hand. You have black 99 and the flop is
9h 5s 3h. Your opponent has Kh4h. He's hoping to make his flush.

On the turn there's one Full House of Flush card, the 5h. Let's say
the turn card misses pairing the board and the flush draw, it's
the Ts. The river now has two Full House of Flush cards, the 5h and
the Th.

If one of these huge cards come on the river, bet it as big as you can.

Remember if you'd hit your full house on the turn ( 3d say ), still bet
this street and get the flush draw to commit more chips when he's drawing
dead.

\section{The Straight of Two Pairs}

Straights often collide against two pairs. Both players are playing
decent connecting cards, one player has hit the straight and his straight
making card makes two pairs for the other player.

If I've got QT and a board of 589J, there's lots of chances for a foe
to have a playable hand that's hit two pair. J8, J9 and 98 are all playable
hands that have hit two pairs.

Straights are worth a feeler bet on the turn, in case the turn card was
the Straight of Two Pairs. If it is, the two pair hand might raise
you. You might even be able to get all your chips
in now. It's worth doing it now, because often the river will be a cooler
card, making it obvious that there's now an easy straight on the board,
and your opponent shuts down and won't call a big bet. As always, the
time to make money with a made hand is the turn betting street.

A made straight versus two pairs with one card to come is a 92\% favourite.
Don't fret about being bad beated on the river, this will only happen
1 in 11 times.

If you go for low straights, sometimes the Ace is the Straight of Two pairs,
and a very well disguised one. One of my favourite hands saw me
cracking AQ, when my 54 flopped up and down on a Q32 flop and turned
an Ace, which was the Straight of Two Pairs.
