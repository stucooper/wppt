\chapter{Australian Poker Weekly}

% Last updated: 20181029

Australian Poker Weekly was a newspaper devoted to poker in Australia
that came out every Friday. I wrote in it from the second issue,
and always got some good tips from the other writers plus some information 
on what games are happening this week from the tournament listings. 

I usually wrote 500 to 600 words a week on interesting hands and 
tournament results. A few of my columns were about hands I misplayed 
and the mistakes that I made, but I also get to write up some good 
wins and tactics here and there. 

My articles for Australian Poker Weekly were so good that I'm 
looking at collecting them and publishing them as a second book.
For now, here's a long article from Issue 4 of February 2009
that was very well received by readers.

\section{Considering a Big Laydown}

It's early January and I'm playing in a small weeknight \$10 NPL game, 
close to where I live. The game is small (typically 20 to 35 runners) with 
\$500 in guaranteed prizemoney, with the top 3 finishers getting \$250, \$150 
and \$100. I'm defending champion, having won the same tournament 
the week before.

It's the sixth hand of the night, blinds are still 25/50 but 
my 2,500 start stack is now at 1,650. I'm the shortest stack at the 
table with most players at the starting stack. I'm in the big blind and 
in a hand with 2 limpers and the small blind completing, I look down to 
find two red Aces and I raise to 500 straight. The two limpers get out 
of the way, the small blind calls.

I always like playing AA in a raised pot against one opponent only. 
Often this ensures he has a hand that plays badly against AA, such as 
an underpair or AK or AT suited. Without a preflop raise, good 
cracking hands as weak as 54 suited can see a flop, and spike a huge 
hand on a flop like J55.

In this hand, I've built the pot to 1,100 and I'm up against one 
opponent, the small blind, who's out of position for the remaining 
betting rounds. I couldn't be in better shape, unless everyone had 
folded and I got a 150 profit straight away.

The flop comes down JT3 rainbow. The small blind does a funny motion 
where he gets about 650 in chips in his hand, advances it into the 
table, withdraws his hand and his chips and finally checks. This is 
no regulation insta-check, and I read this as a ``weak means strong'' 
tell. I think the small blind has hit this flop hard.

Flops containing both a Jack and a Ten are likely to be good flops 
for calling hands, except missed underpairs like 22 to 99. I bet 
500 after the small blind's strange action, and he check raises me 
the minimum to 1,000.

Whoa. I have a really strong feeling that I'm behind here, but can 
I get away from AA? The only 2 pair holding he could have that makes 
sense is JT--- he's not calling 500 preflop with T3 or J3. Set hands 
are JJ, TT and 33. He could have a worse overpair like QQ or KK, 
but I'd expect a reraise preflop with those holdings. 

Possibly he has AJ or KJ for top pair, good kicker, but I'd 
expect a leading bet with such a hand, not a check-raise. Maybe he 
has 89 or KQ for an open ended straight draw, but I think he'd just 
call my 500 flop bet, and not check-raise. If he's check-raising 
as a semibluff, why not check-raise me all in?

I now have 3 indications of small-blind strength: his preflop call 
of a very decent raise, his strange chip motions on the flop and his 
check-raise the minimum amount. The guy looks a straightforward player 
who wouldn't go overboard just with top pair top kicker. If I fold 
to this check-raise, I'll still have 650, very short
against all the stacks but at least keeping me in the tournament.

The small blind himself has had 2 indications of strength from 
me--- a big preflop raise and a solid flop continuation bet of just 
under half the pot and just under half my stack. Really he's not hoping 
or expecting me to fold with his check raise.

Reasoning that ``it's a little \$10 NPL tournament'' I reraise my 
last 150 in chips and the small blind calls and shows JJ for top set. 
I'm now a 92\% chance to be out of this tournament and that's just where 
I am two cards later.

You can play freerolls and \$10 games all week and not see a dent 
in your wallet, but if you're not playing good poker in them you're 
just picking up cards and shifting chips. Many of the players will 
be taking a \$10 game lightly--- most of your advantage in the game 
is treating it seriously and playing the best poker that you can. 
Look for tells, put people on hands, size your bets well and 
play proper poker. Treat any game you play with respect.

If it helps, imagine that your superior poker skills are giving 
you that \$150 second prize with a shot at first. A stupid poker 
mistake on your part won't cost you your \$10 buyin, it'll cost 
you that \$150 second prize that should be yours against this 
field of beginners.

A key skill in poker is trusting your reads and acting on them. 
It's great to spot tells, but if you don't have the guts to follow 
through on them you're wasting effort even looking for them. It's 
great to spot weakness tells and make a hero call against a big bluff. 
It's much harder to make hero folds after spotting strength tells, 
especially when you have a strong hand yourself.
In this hand, my reading skills were great, but I lacked the guts to act
on my read and throw away my powerful AA.

In No Limit, where a single hand can eliminate you or cripple you, the key
question is ``Am I beaten?''. If the answer is ``Yes'', and you don't have
the outs to continue in the hand, the correct play is to fold.

Imagine the respect I'd have got from the table if I'd looked at the 
small blind after that check-raise, said ``both your cards are 
Jacks, I fold'' and thrown Aces into the muck face up!. I think the 
small blind would have turned over his Jacks to show I'd made a great 
laydown. The feeling of making a great laydown like that,
and the amazement from the other players would have been awesome, almost as
good as pocketing that \$150 second prize that should be mine.

